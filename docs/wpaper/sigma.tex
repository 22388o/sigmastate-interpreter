\def\shownotes{1}
\def\notesinmargins{0}
\def\lncs{1}
\def\shortVersion{0}


\documentclass[11pt]{article}
\usepackage{fullpage}

\usepackage{amsmath,color,xcolor,hyperref,graphicx,wrapfig}


\ifnum\shownotes=1
\ifnum\notesinmargins=1
\newcommand{\authnote}[2]{\marginpar{\parbox{\marginparwidth}{\tiny %
  \textsf{#1 {\textcolor{blue}{notes: #2}}}}}%
  \textcolor{blue}{\textbf{\dag}}}
\else
\newcommand{\authnote}[2]{
  \textsf{#1 \textcolor{blue}{: #2}}}
\fi
\else
\newcommand{\authnote}[2]{}
\fi

\newcommand{\lnote}[1]{{\authnote{\textcolor{orange}{Leo notes}}{#1}}}
\newcommand{\dnote}[1]{{\authnote{\textcolor{red}{Dima notes}}{#1}}}
\newcommand{\knote}[1]{{\authnote{\textcolor{green}{kushti notes}}{#1}}}

\newcommand{\ret}{\mathsf{ret}}
\newcommand{\new}{\mathsf{new}}
\newcommand{\hnew}{h_\mathsf{new}}
\newcommand{\old}{\mathsf{old}}
\newcommand{\op}{\mathsf{op}}
\newcommand{\verifier}{\mathcal{V}}
\newcommand{\prover}{\mathcal{P}}
\newcommand{\key}{\mathsf{key}}
\newcommand{\nextkey}{\mathsf{nextKey}}
\newcommand{\node}{\mathsf{t}}
\newcommand{\parent}{\mathsf{p}}
\newcommand{\leaf}{\mathsf{f}}
\newcommand{\vl}{\mathsf{value}}
\newcommand{\balance}{\mathsf{balance}}
\newcommand{\lft}{\mathsf{left}}
\newcommand{\rgt}{\mathsf{right}}
\newcommand{\lbl}{\mathsf{label}}
\newcommand{\direction}{\mathsf{d}}
\newcommand{\oppositedirection}{\bar{\mathsf{d}}}
\newcommand{\found}{\mathsf{found}}
\newcommand{\mypar}[1]{\smallskip\noindent\textbf{#1.}\ \ \ }
\newcommand{\ignore}[1]{}

\newcommand{\langname}{$\Sigma$-State}

\begin{document}

\title{\langname, an Alternative to Bitcoin Script}

%\author{Leonid Reyzin\inst{1,3}, Dmitry Meshkov\inst{2}, Alexander Chepurnoy\inst{2}}

%\institute{Boston University \\\email{reyzin@bu.edu} \and
%IOHK Research \\\email{\{dmitry.meshkov,alex.chepurnoy\}@iohk.io} \and
%Waves platform \\\email{sasha@wavesplatform.com}}
%\fi

\maketitle


\begin{abstract}

We report on design and implementation of \langname, a more powerful, protocol-friendly and at the same more secure language than the  Script language used in Bitcoin. We provide some examples of scripts in the language.


\end{abstract}


\section{Introduction}

Since very early days, Bitcoin~\cite{Nak08} is not just about simple money transfers: it has a scripting language named Bitcoin Script. The language is a primitive stack-based language without loops. A program in this language is protecting a transaction output~(usually, from a party who does not know some secret information). To spend the output, a spending transaction must provide a script in the same language, such as concatenation of both scripts is to be interpreted in {\em true} value on top of the stack. Combined with a maximum size of a 10 kilobytes, the absence of loops provides a guarantee of strictly finite validation time. However, some denial-of-service attacks were found against script validation time. \knote{links}

Smart contracts platform Ethereum is allowing arbitrary program to be run on top of blockchain. \knote{continue}

Our paper is introducing a new language, which is more powerful than Bitcoin Script, and more friendly to cryptographic protocols and applications. It is also providing better guarantees for validation to be finished within target timeout. \knote{continue}


The idea behind the language is that a subset of zero-knowledge protocols known as $\Sigma$-protocols (sigma protocols) could be combined via $\land$ and $\lor$ connectives forming complex statements like ``prove me a knowledge of discrete logarithm of (publicly known) $x_1$ or knowledge of Pedersen commitment $x_2$ preimage''. We make an observation that sigma protocol statements and theit conjectures are isomorphic to boolean values, and we can add arbitrary boolean predicates to a statement provable via a $\Sigma$-protocol, if both prover and verifier are able to evaluate the predicates in exactly the same way. This is the case if predicates are evaluated deterministically in the same way by both the prover and the verifier, and inputs for the predicates are the same on both sides. We use predicates over state of blockchain system during script validation~(which happens when a transaction tries to spend an output protected by the script). To avoid inefficient processing and impossibility for a light client to validate a transaction, this state is very lean but nevertheless it is richer than in Bitcoin. Like in Bitcoin, we use current height of the blockchain, but also a spending transaction with outputs it creates and outputs it tries to spend. Unlike Bitcoin, we allow outputs to contain more fields than amount and protecting script, in form of additional registers an outut can have. In addition to $\land$ and $\lor$ connectives, we have different operations over typed arguments. We filter out trivially wrong expressions, like $2 + 2 > true$, thanks to a type system used. 

In blockchain systems, there is a strict need to tackle the problem of denial-of-service attack carrying by crafting scripts which are too costly to validate. For example, if it is needed for more time to validate a script than average delay between blocks on commodity hardware, network could be obviously attacked, with nodes stuck in processing, and also increased number of forks as result. 


\subsection{Related Work}

\knote{Possible candidates: Simplicity, Plutus, Typecoin, Rholang, FSM-DSL. Good list is available at https://arxiv.org/pdf/1801.00687.pdf, pg. 11. }

\subsection{Organization of the Paper}

\section{Language Design}

We assume that a {\em prover} and a {verifier} have a shared context. As we are designing a language for cryptocurrencies, this context is about current state of the blockchain (such as height $h$ of a best block in the blockchain, a spending transaction with outputs it spends and newly created outputs, etc). Please note that the language can be repurposed for other areas where shared context is possible, but this is out of scope of the paper.

\subsection{Notation}

We use $dlog(x)$ to denote a statement ``prove a knowledge of such $w$ that $x = g^w$''. Proving is to be done in zero knowledge~(with no presenting the secret $w$). Statement $dlog(x_1) \land dlog(x_2)$ means ``prove a knowledge of both $w_1$ and $w_2$, such as $x_1 = g^{w_1}, x_2 = g^{w_2}$'', similarly, $dlog(x_1) \lor dlog(x_2)$ is about a proof knowledge of whether $w_1$ or $w_2$. 

\subsection{UTXO model}

We assume a transactional model close to Bitcoin's. That is, a transaction spends unspent outputs from previous transaction written into the blockchain, and creates new outputs. An output is associated with arbitrary amount of money, and also a protecting script. In Bitcoin, there is a special kind of transaction, so-called {\em coinbase transaction}, which can create some amount of money out of thin air~(to reward a block generator). In Bitcoin transaction fees money flow is not captured in outputs, a transaction fee is just a difference of amounts of outputs being spent and outputs being created. In Appendix~\ref{apx:unified} we provide a way to avoid coinbase transactions, and also express fees explicitly as outputs. In Appendix~\ref{apx:account} we discuss how to make a language for a cryptocurrency with account-based transactions~(such as Waves).

\subsection{General Idea}

By using a $\Sigma$-protocol, a prover can prove a knowledge of secret information corresponding to publicly known values, in zero-knowledge, for some relations between secret and public values. Unlike generic proof systems, $\Sigma$-protocols are efficient~(for both the prover and the verifier). For a cryptocurrency setting, the pros of this class of protocols are generic transformation from an interactive protocol to non-interactive one~(by using Fiat-Shamir transformation), and also composability: we can combine statements provable with $\Sigma$-protocols via $\land$, $\lor$ and k-out-of-N conjectures, and the compound statement is also provable via a $\Sigma$-protocol. An observation which is lying in the foundation of our work is that we can view a (potentially complex) statement provable via a $\Sigma$-protocol as a logic formula. For example, the statement $dlog(x_1) \lor dlog(x_2)$ could be viewed as a logic formula consisting of two boolean values. Then we can enrich the language of $\Sigma$-protocols (which describes relations between prover's secrets and their publicly known images) with deterministic predicates over a context shared between the prover and the verifier. Still, we are using only $\land$, $\lor$ and k-out-of-N conjectures. By evaluating predicates over the context into concrete boolean values and then eliminating them, both (honest) the prover and verifier do agree on the same reduction procedure output, which could be one of the following: boolean value (true or false) or a statement provable via a $\Sigma$-protocol. However, in the light of denial-of-service attacks found against scripting capabilities of Bitcoin and Ethereum \knote{todo: links}, we need to limit possible complexity of the reduction procedure. For that, we have two measures against possible overload issues. In the first place, we have to use only context which is efficiently computable. For example, we can not have predicates over transactional history, as it is linearly growing with time, and also could not be hold by a light client. In opposite, we can use height of a block which contains the transaction spending the output of interest, access to this information is constant-time and available to light clients. If a validation state~(which is similar to UTXO set in Bitcoin for Bitcoin-like cryptocurrencies) is authenticated~(like proposed in the paper \knote{cite AVL paper}), we can construct a predicate for existence of an unspent output (with some conditions to be met), then the prover is providing a Merkle proof for an output satisfying the predicate, and even a light client can validate the predicate efficiently. In the second place, we put a limit on size of an initial logic formula which protects an output, and also we take care that the formula will not become too big due rewritings (as some transformations in our proposal could actually increase a size of the formula) and number of transformations is below an upper limit.

\subsection{Model}

An output to spend is protected by a logical expression, which we are also calling a {\em guarding expression}. An expression is a mix of predicates over publicly known context, and statements provable via sigma protocols. As a simplest example, consider the following statement:

\begin{equation}
\label{eq:example1}
dlog(x_1) \lor ((h > 5) \land dlog(x_2))
\end{equation}

which is to be read as ``proof of knowledge of a secret with public image $x_1$ is always enough, also, if height of a block containing spending transaction is more than 5, knowledge of a secret with public image $x_2$ is also enough''.

Both the prover and the verifier are first reducing the statement by substituting shared context variables and evaluating parts of the expression which are possible to evaluate. Four outcomes of the reduction process are possible: {\em true}, {\em false}, failure to reduce~(if statement is invalid, or transformations of it are taking too much time or result in unreasonably big statement), which is equivalent to {\em false}; or statement which contains only cryptographic statements. For the example, if $h = 10$, the reduced statement is $dlog(x_1) \lor dlog(x_2)$. In this case, the prover is generating a proof, and verifier is checking validity of the proof, accepting or rejecting it. We use cryptographic statements which are provable via {\em sigma protocols}~(\knote{links}). These protocols are efficient zero-knowledge \knote{special honest verifier ZK actually} proof-of-knowledge protocols which are composable via $\land$, $\lor$ and k-out-of-N conjectures, also any sigma protocol has a standard way to be converted into a signature by using the Fiat-Shamir transformation. 

In addition to secret information, which knowledge is to be proven in zero-knowledge, we allow prover to enhance context with custom variables. For example, for the statement:

$$dlog(x) \land (blake2b256(c) = C)$$

where $C$ is some constant~\footnote{we avoid provide a value for the constant due to column size limit}, and $blake2b256$ is operation which calculates hash value for function Blake2b256~(\knote{link}). The prover then needs to prove knowledge of discrete logarithm of $x$ and also to present value $c$ such as Blake2b256 on $c$ evaluates to $C$. Even if verifier does not know $c$ before being presented a proof, we can not count $c$ as secret, as it could be replayed by an eavesdropper.   


\subsection{Language Details} 
\label{sec:lang-details}

Here we provide details on building blocks of the language.  
Both the prover and the verifier are doing the same first few steps, and they both are using the same deterministic interpreter. In the first place, the interpreter is parsing incoming expression~(in typical case of validating a transaction within a block, it is encoded in a binary form), building a tree from the expression, and checking that the expression is well-formed according to typing rules. We describe types and typing rules in Section~\ref{sec:types}. The interpreter then is reducing the expression by applying rewriting rules to the tree as described in the Section~\ref{sec:rewriting}. The possible result of the reduction is whether an abort, or a successfully reduced expression, which could be whether a boolean value or a statement provable via a $\Sigma$-protocol. For the latter case, the prover and the verifier are doing different jobs: the prover is proving knowledge of secrets associated with the statement, as described in Section~\ref{sec:proving}; the verifier is checking a proof generated by the prover against the statement, as described in the Section~\ref{sec:verifying}.   

The interpreter is also checking that the expression is not exceeding by number of sub-expressions and their cumulative complexity some predefined limit~(which is the same for the prover and the verifier, e.g a constant of a blockchain system, or changed by miners in predictable and controllable fashion, like gas limit per block in Ethereum). As complexity is going beyond the limit, the interpreter aborts immediately. 


\subsection{Types}
\label{sec:types}

All the operations as well as operands have types. For example, addition operation ``+'' takes two integers and returns an integer. Comparison operation ``$>$'' takes two integers and returns a boolean value. We consider statements provable via $\Sigma$-protocols, like $dlog(x)$ as instances of the boolean type. The interpreter checks that all the nodes in the tree have children of appropriate types, and the whole guarding expression~(the root node of the tree) has a boolean type and rejects the expression if the conditions are not met. 

We have following types in the language: 

\begin{itemize}
    \item{signed integer}
    \item{boolean}
    \item{byte array}
    \item{collection}
    \item{avl+ tree verification data}
    \item{group element}
    \item{box}
\end{itemize}

\knote{todo: improve description, also, add unsigned integer?}

\ignore{
\begin{center}
    \begin{tabular}{| l | l | l | l | l |}
    \hline
    Operation & bytes & ints & prop & bool \\ \hline
    $=$ & + & + & + & + \\ 
	$\neq$ & + & + & + & +\\ 
	$+$ & + & + & - & - \\    
	$-$ & - & + & - & - \\
	$>$ & - & + & - & - \\
	$\ge$ & - & + & - & -\\
	$<$ & - & + & - & -\\
	$\le$ & - & + & - & -\\
	$\oplus$ & + & - & - & + \\
	$\lor$ & - & - & - & + \\
	$\land$ & - & - & - & + \\
	$blake2b256$ & + & - & - & -\\
	$dlog$ & - & - & - & -\\
	$dh$ & - & - & - & -\\
    \hline
    \end{tabular}
\end{center}
}


\subsection{Rewriting a Tree}
\label{sec:rewriting}

By parsing a statement, interpreter first builds a tree from a formula. For the example~\ref{eq:example1} the tree would be as following:

\knote{draw the tree}.

If the tree is well-formed according to the typing rules, and also has complexity no more than an allowed limit, interpreter is going to rewrite it, in potentially many steps. For every step, the interpreter tries, going from bottom to top of the tree, to replace nodes which are ready to be transformed~(operands are in place, and the transformation itself is known). The next step is about the same transformations in the same bottom-top order to be done. The process finishes in one of the following cases:

\begin{itemize}
    \item{abort: } happens if transformation process by its complexity exceeds an upper bound. The complexity estimation works as follows. The interpreter remembers initial cumulative complexity $C_0$, and for each replacement in the tree it calculates cumulative complexity of a subtree to be inserted instead of a subtree to be removed $\Delta C$, and adds $\Delta C$ to current cumulative complexity $C$, where $C = C_0$ initially.  
    \item{success: } we finish with this status if during last step there are no any transformations done.  
\end{itemize} 

If the transformations are finishing with abortion, the result is considered as $false$ (we recall that the interpreter is giving a single boolean result finally, whether an output could be spent or not). Otherwise, if the tree could not be transformed anymore the interpreter is looking into it. If the tree is about just a single node which contains boolean constant value~($true$ or $false$), the value is the result of the interpretation. If the tree contains statements not provable via $\Sigma$-protocols, the interpreter finishes with $false$. Otherwise, if the tree contains only statements provable via $\Sigma$-protocols, the interpreter is continuing to work, and here execution is different for the prover and the verifier. The prover is generating a proof for the statement by using its secrets, as described in Section~\ref{sec:proving}, and the verifier checks the statement against the proof~(provided in a spending transaction), as described in Section~\ref{sec:verifying}. The verifier outputs $true$ if the proof is valid for the statement, $false$ otherwise. We are skipping for now the question how Fiat-Shamir transformation works in our setting, and how a message, which is used in a non-interactive protocol is formed; details are given further in Appendix~\ref{apx:tx-format}. 


\subsection{Proving}
\label{sec:proving}

\knote{Raw text from Leo's email, rewrite}

Prover steps:
    
    1. bottom-up: mark every node real or simulated, according to the following rule. DLogNode -- if you know the secret,
     then real, else simulated. $\lor$: if at least one child real, then real; else simulated. $\land$: if at least one child
     simulated, then simulated; else real. Note that all descendants of a simulated node will be later simulated, even
     if they were marked as real. This is what the next step will do.
    
     Root should end up real according to this rule -- else you won't be able to carry out the proof in the end.
    
    2. top-down: mark every child of a simulated node "simulated." If two or more more children of a real $\lor$ are real,
     mark all but one simulated.
    
    3. top-down: compute a challenge for every simulated child of every $\lor$ and $\land$, according to the following rules.
     If $\lor$, then every simulated child gets a fresh random challenge. If $\land$ (which means $\land$ itself is simulated, and
     all its children are), then every child gets the same challenge as the $\land$.
    
    4. bottom-up: For every simulated leaf, simulate a response and a commitment (i.e., second and first prover message)
     according to the Schnorr simulator. For every real leaf, compute the commitment (i.e., first prover message) according
     to the Schnorr protocol. For every $\lor$/$\land$ node, let the commitment be the union (as a set) of commitments below it.
    
    5. Compute the Schnorr challenge as the hash of the commitment of the root (plus other inputs -- probably the tree
     being proven and the message).
    
    6. top-down: compute the challenge for every real child of every real $\lor$ and $\land$, as follows. If $\lor$, then the
     challenge for the one real child of $\lor$ is equal to the XOR of the challenge of $\lor$ and the challenges for all the
     simulated children of $\lor$. If $\land$, then the challenge for every real child of $\land$ is equal to the the challenge of
     the $\land$. Note that simulated $\land$ and $\lor$ have only simulated descendants, so no need to recurse down from them.

\subsection{Verifying}
\label{sec:verifying}

Verifier steps:
    1. Place received challenges "e" and responses "z" into every leaf.

    2. Bottom-up: compute commitments at every leaf according to $a = g^z/h^e$. At every $\lor$ and $\land$ node, compute
      the commitment as the union of the children's commitments. At every $\lor$ node, compute the challenge as the XOR of
      the children's challenges. At every $\land$ node, verify that the children's challenges are all equal. (Note that
      there is an opportunity for small savings here, because we don't need to send all the challenges for a $\land$ --
      but let's save that optimization for later).

    3. Check that the root challenge is equal to the hash of the root commitment and other inputs.

\subsection{Threshold Signatures}

\subsection{Context}

Shared context of a blockchain system could be expressed in different ways, based on desired expressiveness, efficiency, planned use cases and so on. In this paper we focus on context for Ergo blockchain. The main priority for this blockchain is maximum efficiency of transaction validation process, safety, and friendliness to light clients. Considering this, we require that context should contain only spending transaction along with outputs it spends, and limited number of last block headers. Thus even a client which does not have all the headers of the blockchain~(for example, the client could be working in a light-SPV mode, where the client is holding only sublinear part of the headers-chain) is able to validate a transaction, by being shown it~(as well as outputs the transaction spends along with Merkle proofs for them).


\section{Examples}

In this section we provide some examples of useful contracts. We focus on examples which are impossible or harder to realize in Bitcoin.

\subsection{Crowdfunding}
\label{sec:crowdfunding}

We provide very simple solution to crowdfunding here. In the example, a crowdfunding project associated with public key $x_P$ is considered successful if it is possible to collect for the projects unspent outputs with total value not less than $to\_raise$ before height $timeout$. The project backer creates an output protected by following statement: 

$$(height \ge timeout \land dlog(x_B)) \lor (height < timeout \land dlog(x_P) \land tx.has\_output(amount \ge to\_raise \land script = dlog(x_P)))$$

Then the project can collect biggest outputs with total value not less than $to\_raise$ with a single transaction~(it is possible to collect up to ~22,000 outputs in Bitcoin, which is enough even for a big crowdfunding campaign). For remaining small outputs over $to\_raise$, it is possible to construct follow-up transactions. 

Please note the extension which is not a part of Bitcoin: we use the condition on a spending transaction, namely, we require the transaction to have an output with value not less than required and also with particular statement protecting the output.

\subsection{Money With Scheduled Maintenance Payments}

\knote{description}

$$user\_statement \lor (height \ge (self.height + period) \land tx.has\_output(amount \ge (self.amount-cost) \land script = self.script))$$

We highlight impossibility of such a statement in the Bitcoin Script: similarly to the statement in the previous section~\ref{sec:crowdfunding}, we use a condition on a spending transaction. Another feature missed in the Bitcoin Script is that we also use an output to spend in the execution context. In particular, in the example above we require a spending transaction to have an output which has the same statement as an output it spends. 

\subsection{Ring Signature}
\label{sec:ring}

Linear-sized ring signature is very straightforward in the language. Assume a ring consists of $m$ public keys $x_1, \dots, x_m$. If we want an output to be spent by a ring signature associated with the ring, the output is to be protected by the following statement:

$$dlog(x_1) \lor \dots \lor dlog(x_m)$$  

Please note that proving of the statement is to be done in zero-knowledge, so it is not possible to know which key signed, the only fact to conclude is that some key from the ring signed output spending. 

\subsection{Complex Signature Schemes}

We can build more complex signature schemes than possible in Bitcoin. One particular example was provided in the previous Section~\ref{sec:ring}. Another example is a scheme where at least one out of (Alice, Bob), and at least one out of (Charles, Diana) are needed to sign, and it is not to be known who signed an output spending. The corresponding statement involving public keys $x_A, x_B, x_C, x_D$ of Alice, Bob, Charles and Diana respectively would be following:

$$(dlog(x_A) \lor dlog(x_B)) \land (dlog(x_C) \lor dlog(x_D))$$

\subsection{Simple Tumbler}
\label{sec:tumbler}

\knote{does the example makes sense? check other tumbler papers. also, update the scripts, now approach is more generic than using tx.outbytes}

Privacy is a tough problem for cryptocurrency users. Bitcoin is a pseudonymous cryptocurrency, so no real identities attached to a transaction. However, it is possible to reconstruct transactional graph for all the transactions ever entered into the Bitcoin blockchain, and often restore real identities by using auxiliary databases. \knote{link} To improve privacy, tumblers are being used. A tumbler is a scheme which us getting some money transfers as inputs, produces output money transfers, and has a property of unlinkability: it is not possible to draw a link from an input to an output. Thus a tumbler user is hiding herself amongst a ring of users sending inputs to the scheme. Privacy then depends on a ring size. For maximum privacy, there exists a cryptocurrency ZCash with inbuilt tumbler based on zkSnarks\knote{links}, where a user is hiding among all the users in the system. However, ZCash requires trusted setup, and transaction validation is relatively slow~(10 ms). In other cryptocurrencies users are usually using external services varying in security and efficiency.

Here we are describing simple tumbler service. It is very efficient and requires no trusted dealer. Its disadvantage is that from observing blockchain transactions it is possible to conclude that users are using a tumbler.

Assume Alice with public key $x_A$ and Bob with public key $x_B$ want to relocate funds to keys $y_1$ and $y_2$ respectively, with a property than external observer looking into blockchain is not capable to know beyond the fact that money flows from $x_A$ to whether $y_1$ or $y_2$~(and the same for $x_B$).

First, Alice and Bob communicate off-chain to construct collectively outputs of the final transaction~(payments to $y_1$ and $y_2$). Each of them then is calculating a hash value from outputs bytes $h$~(we assume that Blake2b-256 hash function is used). Then each of them is creating an output to spend~(possibly, in a dedicated transaction) with such a condition for Alice and Bob, respectively:

$blake2b256(tx.outbytes) = h \lor dlog(x_A)$ 

$blake2b256(tx.outbytes) = h \lor dlog(x_B)$

Then it is possible to make a refund at any moment of time~(right condition in the $\lor$ conjectures), and before a refund any of the them can construct a transaction which is spending the outputs, and it is impossible to construct an alternative transaction, for which hash of the output bytes is $h$, but bytes are different~(if chosen hash function is collision-resistant).

\subsection{Oracle Example}


\section{Safety Guarantees}
\label{sec:safety}

We need to be sure that an adversary can not produce such a statement for which the Verifier spends more time than it is safe to spend.
\knote{links to verifier dilemma, orphan rates etc} 

\subsection{Cost Model}
\label{sec:cost-model}

\subsection{Denial-of-Service Attacks}

\section{Extensibility}

It is hard to predict which functions would be useful for blockchain applications.

\section{Implementation and Evaluation}

\section{Further Work}

\section{Conclusion}

We show in this paper...




\bibliographystyle{alpha}
\bibliography{sigma}



\appendix

\section{Tree Nodes and Transformation Rules}

\section{Cost Table}

\section{Roadmap}

\section{Outputs vs Accounts}

\section{Unified Transactions}
\label{apx:unified}

\section{An Example of a Concrete Transaction Format}
\label{apx:tx-format}

\knote{Describe Ergo transaction format here. Malleability problems to be discussed here.}

\section{A Language For An Account-Based Cryptocurrency}
\label{apx:account}



\end{document}