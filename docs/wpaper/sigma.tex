\documentclass[11pt]{llncs}

\def\shownotes{1}
\def\notesinmargins{0}

\usepackage{fullpage}

\usepackage{amsmath,color,xcolor,hyperref,graphicx,wrapfig,listings}


\ifnum\shownotes=1
\ifnum\notesinmargins=1
\newcommand{\authnote}[2]{\marginpar{\parbox{\marginparwidth}{\tiny %
  \textsf{#1 {\textcolor{blue}{notes: #2}}}}}%
  \textcolor{blue}{\textbf{\dag}}}
\else
\newcommand{\authnote}[2]{
  \textsf{#1 \textcolor{blue}{: #2}}}
\fi
\else
\newcommand{\authnote}[2]{}
\fi

\newcommand{\lnote}[1]{{\authnote{\textcolor{orange}{Leo notes}}{#1}}}
\newcommand{\knote}[1]{{\authnote{\textcolor{green}{kushti notes}}{#1}}}
\newcommand{\mnote}[1]{{\authnote{\textcolor{blue}{morphic notes}}{#1}}}

\newcommand{\ret}{\mathsf{ret}}
\newcommand{\new}{\mathsf{new}}
\newcommand{\hnew}{h_\mathsf{new}}
\newcommand{\old}{\mathsf{old}}
\newcommand{\op}{\mathsf{op}}
\newcommand{\verifier}{\mathcal{V}}
\newcommand{\prover}{\mathcal{P}}
\newcommand{\key}{\mathsf{key}}
\newcommand{\nextkey}{\mathsf{nextKey}}
\newcommand{\node}{\mathsf{t}}
\newcommand{\parent}{\mathsf{p}}
\newcommand{\leaf}{\mathsf{f}}
\newcommand{\vl}{\mathsf{value}}
\newcommand{\balance}{\mathsf{balance}}
\newcommand{\lft}{\mathsf{left}}
\newcommand{\rgt}{\mathsf{right}}
\newcommand{\lbl}{\mathsf{label}}
\newcommand{\direction}{\mathsf{d}}
\newcommand{\oppositedirection}{\bar{\mathsf{d}}}
\newcommand{\found}{\mathsf{found}}
\newcommand{\mypar}[1]{\smallskip\noindent\textbf{#1.}\ \ \ }
\newcommand{\ignore}[1]{}

\newcommand{\langname}{$\Sigma$-State}
\newcommand{\lst}[1]{\text{\lstinline[basicstyle={\ttfamily}]$#1$}}

\begin{document}

\title{\langname, an Alternative to Bitcoin Script}

%\author{Leonid Reyzin\inst{1,3}, Dmitry Meshkov\inst{2}, Alexander Chepurnoy\inst{2}}

%\institute{Boston University \\\email{reyzin@bu.edu} \and
%IOHK Research \\\email{\{dmitry.meshkov,alex.chepurnoy\}@iohk.io} \and


\maketitle


\begin{abstract}
We report on design and implementation of \langname, a more powerful, protocol-friendly and at the same more secure language than the  Script language used in Bitcoin. We provide some examples of scripts in the language.
\end{abstract}



\section{Introduction}

Since very early days, Bitcoin~\cite{Nak08} is not just about simple money transfers: it has a scripting language named Bitcoin Script. The language is a primitive stack-based language without loops. A program in this language is protecting a transaction output~(usually, from a party who does not know some secret information). To spend the output, a spending transaction must provide a script in the same language, such as concatenation of both scripts is to be interpreted in {\em true} value on top of the stack. Combined with a maximum size of a 10 kilobytes, the absence of loops provides a guarantee of strictly finite validation time. However, some denial-of-service attacks were found against script validation time. \knote{links}

Smart contracts platform Ethereum is allowing arbitrary program to be run on top of blockchain. \knote{continue}

Our paper is introducing a new language, which is more powerful than Bitcoin Script, and more friendly to cryptographic protocols and applications. It is also providing better guarantees for validation to be finished within target timeout. \knote{continue}


The idea behind the language is that a subset of zero-knowledge protocols known as $\Sigma$-protocols (sigma protocols) could be combined via $\land$ and $\lor$ connectives forming complex statements like ``prove me a knowledge of discrete logarithm of (publicly known) $x_1$ or knowledge of Pedersen commitment $x_2$ preimage''. We make an observation that sigma protocol statements and theit conjectures are isomorphic to boolean values, and we can add arbitrary boolean predicates to a statement provable via a $\Sigma$-protocol, if both prover and verifier are able to evaluate the predicates in exactly the same way. This is the case if predicates are evaluated deterministically in the same way by both the prover and the verifier, and inputs for the predicates are the same on both sides. We use predicates over state of blockchain system during script validation~(which happens when a transaction tries to spend an output protected by the script). To avoid inefficient processing and impossibility for a light client to validate a transaction, this state is very lean but nevertheless it is richer than in Bitcoin. Like in Bitcoin, we use current height of the blockchain, but also a spending transaction with outputs it creates and outputs it tries to spend. Unlike Bitcoin, we allow outputs to contain more fields than amount and protecting script, in form of additional registers an outut can have. In addition to $\land$ and $\lor$ connectives, we have different operations over typed arguments. We filter out trivially wrong expressions, like $2 + 2 > true$, thanks to a type system used. 

In blockchain systems, there is a strict need to tackle the problem of denial-of-service attack carrying by crafting scripts which are too costly to validate. For example, if it is needed for more time to validate a script than average delay between blocks on commodity hardware, network could be obviously attacked, with nodes stuck in processing, and also increased number of forks as result. 



\subsection{Organization of the Paper}


\section{Related Work}

\knote{A good survey is available at https://arxiv.org/pdf/1801.00687.pdf, pg. 11}

\begin{itemize}
    \item{Simplicity}
    \item{Plutus}
    \item{TypeCoin}
    \item{Rholang}
\end{itemize}


\section{Language Design}

We assume that a {\em prover} and a {verifier} have a shared context. As we are designing a language for cryptocurrencies, this context is about current state of the blockchain (such as height $h$ of a best block in the blockchain, a spending transaction with outputs it spends and newly created outputs, etc). Please note that the language can be repurposed for other areas where shared context is possible, but this is out of scope of the paper.

\subsection{Notation}

We use $dlog(x)$ to denote a statement ``prove a knowledge of such $w$ that $x = g^w$''. Proving is to be done in zero knowledge~(with no presenting the secret $w$). Statement $dlog(x_1) \land dlog(x_2)$ means ``prove a knowledge of both $w_1$ and $w_2$, such as $x_1 = g^{w_1}, x_2 = g^{w_2}$'', similarly, $dlog(x_1) \lor dlog(x_2)$ is about a proof knowledge of whether $w_1$ or $w_2$. 

\subsection{UTXO model}

We assume a transactional model close to Bitcoin's. That is, a transaction spends unspent outputs from previous transaction written into the blockchain, and creates new outputs. An output is associated with arbitrary amount of money, and also a protecting script. In Bitcoin, there is a special kind of transaction, so-called {\em coinbase transaction}, which can create some amount of money out of thin air~(to reward a block generator). In Bitcoin transaction fees money flow is not captured in outputs, a transaction fee is just a difference of amounts of outputs being spent and outputs being created. In Appendix~\ref{apx:unified} we provide a way to avoid coinbase transactions, and also express fees explicitly as outputs. In Appendix~\ref{apx:account} we discuss how to make a language for a cryptocurrency with account-based transactions~(such as Waves).

\subsection{General Idea}

By using a $\Sigma$-protocol, a prover can prove a knowledge of secret information corresponding to publicly known values, in zero-knowledge, for some relations between secret and public values. Unlike generic proof systems, $\Sigma$-protocols are efficient~(for both the prover and the verifier). For a cryptocurrency setting, the pros of this class of protocols are generic transformation from an interactive protocol to non-interactive one~(by using Fiat-Shamir transformation), and also composability: we can combine statements provable with $\Sigma$-protocols via $\land$, $\lor$ and k-out-of-N conjectures, and the compound statement is also provable via a $\Sigma$-protocol. An observation which is lying in the foundation of our work is that we can view a (potentially complex) statement provable via a $\Sigma$-protocol as a logic formula. For example, the statement $dlog(x_1) \lor dlog(x_2)$ could be viewed as a logic formula consisting of two boolean values. Then we can enrich the language of $\Sigma$-protocols (which describes relations between prover's secrets and their publicly known images) with deterministic predicates over a context shared between the prover and the verifier. Still, we are using only $\land$, $\lor$ and k-out-of-N conjectures. By evaluating predicates over the context into concrete boolean values and then eliminating them, both (honest) the prover and verifier do agree on the same reduction procedure output, which could be one of the following: boolean value (true or false) or a statement provable via a $\Sigma$-protocol. However, in the light of denial-of-service attacks found against scripting capabilities of Bitcoin and Ethereum \knote{todo: links}, we need to limit possible complexity of the reduction procedure. For that, we have two measures against possible overload issues. In the first place, we have to use only context which is efficiently computable. For example, we can not have predicates over transactional history, as it is linearly growing with time, and also could not be hold by a light client. In opposite, we can use height of a block which contains the transaction spending the output of interest, access to this information is constant-time and available to light clients. If a validation state~(which is similar to UTXO set in Bitcoin for Bitcoin-like cryptocurrencies) is authenticated~(like proposed in the paper \knote{cite AVL paper}), we can construct a predicate for existence of an unspent output (with some conditions to be met), then the prover is providing a Merkle proof for an output satisfying the predicate, and even a light client can validate the predicate efficiently. In the second place, we put a limit on size of an initial logic formula which protects an output, and also we take care that the formula will not become too big due rewritings (as some transformations in our proposal could actually increase a size of the formula) and number of transformations is below an upper limit.

\subsection{Model}

An output to spend is protected by a logical expression, which we are also calling a {\em guarding expression}. An expression is a mix of predicates over publicly known context, and statements provable via sigma protocols. As a simplest example, consider the following statement:

\begin{equation}
\label{eq:example1}
dlog(x_1) \lor ((h > 5) \land dlog(x_2))
\end{equation}

which is to be read as ``proof of knowledge of a secret with public image $x_1$ is always enough, also, if height of a block containing spending transaction is more than 5, knowledge of a secret with public image $x_2$ is also enough''.

Both the prover and the verifier are first reducing the statement by substituting shared context variables and evaluating parts of the expression which are possible to evaluate. Four outcomes of the reduction process are possible: {\em true}, {\em false}, failure to reduce~(if statement is invalid, or transformations of it are taking too much time or result in unreasonably big statement), which is equivalent to {\em false}; or statement which contains only cryptographic statements. For the example, if $h = 10$, the reduced statement is $dlog(x_1) \lor dlog(x_2)$. In this case, the prover is generating a proof, and verifier is checking validity of the proof, accepting or rejecting it. We use cryptographic statements which are provable via {\em sigma protocols}~(\knote{links}). These protocols are efficient zero-knowledge \knote{special honest verifier ZK actually} proof-of-knowledge protocols which are composable via $\land$, $\lor$ and k-out-of-N conjectures, also any sigma protocol has a standard way to be converted into a signature by using the Fiat-Shamir transformation. 

In addition to secret information, which knowledge is to be proven in zero-knowledge, we allow prover to enhance context with custom variables. For example, for the statement:

$$dlog(x) \land (blake2b256(c) = C)$$

where $C$ is some constant~\footnote{we avoid provide a value for the constant due to column size limit}, and $blake2b256$ is operation which calculates hash value for function Blake2b256~(\knote{link}). The prover then needs to prove knowledge of discrete logarithm of $x$ and also to present value $c$ such as Blake2b256 on $c$ evaluates to $C$. Even if verifier does not know $c$ before being presented a proof, we can not count $c$ as secret, as it could be replayed by an eavesdropper.   



\subsection{Language Details} 
\label{sec:lang-details}

Here we provide details on building blocks of the language.  
Both the prover and the verifier are doing the same first few steps, and they both are using the same deterministic interpreter. In the first place, the interpreter is parsing incoming expression~(in typical case of validating a transaction within a block, it is encoded in a binary form), building a tree from the expression, and checking that the expression is well-formed according to typing rules. We describe types and typing rules in Section~\ref{sec:types}. The interpreter then is reducing the expression by applying rewriting rules to the tree as described in the Section~\ref{sec:rewriting}. The possible result of the reduction is whether an abort, or a successfully reduced expression, which could be whether a boolean value or a statement provable via a $\Sigma$-protocol. For the latter case, the prover and the verifier are doing different jobs: the prover is proving knowledge of secrets associated with the statement, as described in Section~\ref{sec:proving}; the verifier is checking a proof generated by the prover against the statement, as described in the Section~\ref{sec:verifying}.   

The interpreter is also checking that the expression is not exceeding by number of sub-expressions and their cumulative complexity some predefined limit~(which is the same for the prover and the verifier, e.g a constant of a blockchain system, or changed by miners in predictable and controllable fashion, like gas limit per block in Ethereum). As complexity is going beyond the limit, the interpreter aborts immediately. 


\subsection{Types}
\label{sec:types}

All the operations as well as operands have types. For example, addition operation ``+'' may take two Int and returns an Int value. Comparison operation ``$>$'' takes two Int and returns a boolean value. Some operations may be overloaded so that the same symbol is reused for operations acting on different types, e.g. $+: (Int,Int) \to Int$, $+: (Long, Long) \to Long$ etc.
We consider statements provable via $\Sigma$-protocols, like $dlog(x)$ as instances of the boolean type. The interpreter checks that all the nodes in the tree have children of appropriate types, and the whole guarding expression~(the root node of the tree) has a boolean type and rejects the expression if the conditions are not met.

We have following types in the language:

\begin{itemize}
    \item{signed integer types of 8, 16, 32 and 64-bits size}
    \item{big integer type of arbitrary size}
    \item{boolean}
    \item{avl+ tree verification data}
    \item{cryptography group element}
    \item{box}
    \item{tuple of values of different types}
    \item{array of values of the same type}
    \item{optional value}
\end{itemize}

\knote{todo: improve description, also, add unsigned integer?}

\ignore{
\begin{center}
    \begin{tabular}{| l | l | l | l | l |}
    \hline
    Operation & bytes & ints & prop & bool \\ \hline
    $=$ & + & + & + & + \\ 
	$\neq$ & + & + & + & +\\ 
	$+$ & + & + & - & - \\    
	$-$ & - & + & - & - \\
	$>$ & - & + & - & - \\
	$\ge$ & - & + & - & -\\
	$<$ & - & + & - & -\\
	$\le$ & - & + & - & -\\
	$\oplus$ & + & - & - & + \\
	$\lor$ & - & - & - & + \\
	$\land$ & - & - & - & + \\
	$blake2b256$ & + & - & - & -\\
	$dlog$ & - & - & - & -\\
	$dh$ & - & - & - & -\\
    \hline
    \end{tabular}
\end{center}
}


\subsection{Rewriting a Tree}
\label{sec:rewriting}

By parsing a statement, interpreter first builds a tree from a formula. For the example~\ref{eq:example1} the tree would be as following:

\knote{draw the tree}.

If the tree is well-formed according to the typing rules, and also has complexity no more than an allowed limit, interpreter is going to rewrite it, in potentially many steps. For every step, the interpreter tries, going from bottom to top of the tree, to replace nodes which are ready to be transformed~(operands are in place, and the transformation itself is known). The next step is about the same transformations in the same bottom-top order to be done. The process finishes in one of the following cases:

\begin{itemize}
    \item{abort: } happens if transformation process by its complexity exceeds an upper bound. The complexity estimation works as follows. The interpreter remembers initial cumulative complexity $C_0$, and for each replacement in the tree it calculates cumulative complexity of a subtree to be inserted instead of a subtree to be removed $\Delta C$, and adds $\Delta C$ to current cumulative complexity $C$, where $C = C_0$ initially.  
    \item{success: } we finish with this status if during last step there are no any transformations done.  
\end{itemize} 

If the transformations are finishing with abortion, the result is considered as $false$ (we recall that the interpreter is giving a single boolean result finally, whether an output could be spent or not). Otherwise, if the tree could not be transformed anymore the interpreter is looking into it. If the tree is about just a single node which contains boolean constant value~($true$ or $false$), the value is the result of the interpretation. If the tree contains statements not provable via $\Sigma$-protocols, the interpreter finishes with $false$. Otherwise, if the tree contains only statements provable via $\Sigma$-protocols, the interpreter is continuing to work, and here execution is different for the prover and the verifier. The prover is generating a proof for the statement by using its secrets, as described in Section~\ref{sec:proving}, and the verifier checks the statement against the proof~(provided in a spending transaction), as described in Section~\ref{sec:verifying}. The verifier outputs $true$ if the proof is valid for the statement, $false$ otherwise. We are skipping for now the question how Fiat-Shamir transformation works in our setting, and how a message, which is used in a non-interactive protocol is formed; details are given further in Appendix~\ref{apx:tx-format}. 

\section{Sigma prrotocols for an arbitrary And/Or/Threshold composition}
\newcommand{\andnode}{\ensuremath{\mathsf{AND}}}
\newcommand{\ornode}{\ensuremath{\mathsf{OR}}}
\newcommand{\tnode}{\ensuremath{\mathsf{THRESHOLD}}}
\newcommand{\GF}{\ensuremath{\mathrm{GF}}}
\lnote{Somewhere we should introduce basic notation/terminology for $\Sigma$ protocols. In particular, we should say that we mean ``non-interactive'' $\Sigma$-protocol throughout, where the second message is computed as hash of the first. Somewhere we should define ``commitment'', ``challenge'', and ``response''; moreover, we shoild be clear that  restrict ourselves to $\Sigma$-protocols that require the sending of only ``challenge'' and ``response'' (we can handle protocols that require ``commitment'' and ``response'' instead, but we will be less communication-efficient than optimal, probably --- this require more thinking) }

In this section, we consider the following setting. Let a $\Sigma$-protocol be described by a tree, as follows. The leaves of the tree are atomic $\Sigma$-protocols, for which prover, verifier, and simulator code is available. They do not have to be the same, but must all have the same challenge bit-length $t$. Each non-leaf node is one of three types: $\andnode$, $\ornode$, or $\tnode(k)$, where $k\ge 1$ is an integer parameter that is no greater than the number of the node's children (and the number of children is less than $2^t$). The meaning of the proof corresponding $\tnode(k)$ is ``the prover knows witnesses for at least $k$ children of this node''. Semantically, $\andnode$ and $\ornode$ are simply special cases of \tnode: the meaning  of the proof corresponding $\andnode$ (respectively, $\ornode)$ is ``the prover knows witnesses for all children (respectively, at least one child) of this node''. However, $\andnode$ and $\ornode$ are implemented differently from $\tnode$ for efficiency.

We show the steps of the prover and verifier given such a tree. While $\andnode$, $\ornode$, and $\tnode$ composition of $\Sigma$-protocols has been addressed in the literature before \lnote{need citations}, prover and verifier algorithms have been described only for a single node, in terms of prover, verifier, and simulator algorithms for its children. Recursively constructing code for an entire tree by dynamically constructing prover, verifier, and simulator code for each node is inefficient. Here we take a different approach by explicitly describing how and in what order the protocol messages for each node need to be computed and verified.

Our description uses two kinds of tree traversal: bottom-up (also known as post-order) and top-down (also known as preorder). In a bottom-up traversal, for each node, operations are recursively applied to the children of the node before being applied to the node itself. In a top-down traversal, for each node, operations are applied to the node itself before being recursively applied to each of its children.

\subsection{Proving}
\label{sec:proving}

For each node in the tree, the prover will maintain a flag indicating whether it is real or simulated. Each node will also have a $t$-bit challenge. The leaves will additionally have two protocol values: a challenge and a response. The pseudocode below explains how these values are computed. \lnote{we should probably mention that ``random'' below means pseudo-random, too}

The prover first has to decide which nodes will have real proofs (for which witnesses are required) and which will be simulated. For example, in an $\ornode$ proof, only one of the children will be real. The prover will do so based on witnesses that are available, in three steps:

\begin{enumerate}
    \item This step will mark as ``real'' every node for which the prover can produce a real proof. This step may mark as ``real'' more nodes than necessary if the prover has more than the minimal necessary number of witnesses (for example, more than one child of an $\ornode$).   In a bottom-up traversal of the tree, do the following: 
        \begin{itemize}
            \item If the node is a leaf, mark it ``real'' if the witness for it is available; else mark it ``simulated''
            \item If the node is $\ornode$, mark it ``real'' if at least one child is marked real; else mark it ``simulated''
            \item If the node is $\andnode$, mark it ``real'' if all of its children are marked real; else mark it ``simulated''
            \item If the node is $\tnode(k)$, mark it ``real'' if at least $k$ of its children are marked real; else mark it ``simulated''
        \end{itemize}
    
    \item If the root of the tree is marked ``simulated'' then the prover does not have enough witnesses to perform the proof. Abort.
    
    \item This step will change some ``real'' nodes to ``simulated'' to make sure each node has the right number of simulated children.
          In a top-down traversal of the tree, do the following:
        \begin{itemize}
            \item
            If the $\ornode$ node is marked ``real,'' mark all but one of its children ``simulated''
            (the node is guaranteed, by the previous step, to have at least one ``real'' child).
            Which particular node is left ``real'' is not important and is subject to optimized choice.
            \item
            If the node is $\tnode(k)$ marked ``real,'' mark all but $k$ of its children ``simulated''
            (the node is guaranteed, by the previous step, to have at least $k$ ``real'' children).
            Which particular ones are left ``real'' is not important.
            \item
            If the node is marked ``simulated'', mark all of its children ``simulated''
        \end{itemize}
\end{enumerate}

\noindent
Now the prover will compute protocol values for every node, as follows:

\begin{enumerate}
\setcounter{enumi}{3}
    \item In a top-down traversal of the tree, compute the challenges for simulated children of every node, as follows:
    \begin{itemize}
        \item If the node is marked ``real'', then each of its simulated children gets  a fresh uniformly random challenge. (Note that a real $\andnode$ node has no simulated children, so this step applies only to real $\ornode$ and $\tnode$ nodes.)
        \item If the node is marked ``simulated'', let $e_0$ be the challenge computed for it.  All of its children are simulated, and thus we compute challenges for all of them, as follows:
        \begin{itemize}
            \item If the node is $\andnode$,  then all of its children get $e_0$ as the challenge
            \item If the node is $\ornode$, then each of its children except one gets a fresh uniformly random challenge in $\{0,1\}^t$. The remaining child gets a challenge computed as an XOR of the challenges of all the other children and $e_0$.
            \item If the node is $\tnode(k)$, assume the children are numbered from $1$ to $n$. Pick  $n-k$ fresh uniformly random values $a_1, \dots, a_{n-k}$ from $\{0,1\}^t$ and let $a_0=e_0$. Viewing $1, 2, \dots, n$ and $a_0, \dots, a_{n-k}$ as elements of $\GF(2^t)$, evaluate the polynomial $Q(x) = \sum {a_i x^i}$ over $\GF(2^t)$ at points $1, 2, \dots, n$ to get challenges for child $1, 2, \dots, n$, respectively.
        \end{itemize}
    \end{itemize}
    
    \item For every leaf marked ``simulated'', use the simulator of the $\Sigma$-protocol for that leaf to compute the commitment and the response, given the challenge that is already stored in the leaf
    
    \item For every leaf marked ``real'', use the first prover step of the $\Sigma$-protocol for that leaf to compute the commitment
    
    \item Convert the tree to a string $s$ for input to the Fiat-Shamir hash function. The conversion should be such that the tree can be unambiguously parsed and restored given the string. For each non-leaf node, the string should contain its type ($\andnode$, $\ornode$, or $\tnode(k)$). For each leaf node, the string should contain the $\Sigma$-protocol statement being proven and the commitment. The string should not contain information on whether a node is marked ``real'' or ``simulated'', and should not contain challenges, responses, or the real/simulated flag for any node.
    
    \item Compute the challenge for the root of the tree as the Fiat-Shamir hash of $s$ (and, if applicable,  the associated data, such as the message being signed). 
    
    \item Perform a top-down traversal of only the portion of the tree marked ``real'' in order to compute the challenge for every node marked ``real'' and, additionally, the response for every leaf marked ``real'' (note that nodes marked ``simulated'' have all their descendants marked ``simulated'' and all the challenges for these descendants already computed, so there is no need to recurse down ``simulated'' nodes).
    
    \begin{itemize}
        \item If the node is a non-leaf marked ``real'' whose challenge is $e_0$, proceed as follows:
    
        \begin{itemize}
            \item If the node is $\andnode$, let each of its children have the challenge $e_0$
            \item If the node is $\ornode$, it has only one child marked ``real''. Let this child have the challenge equal to the XOR of the challenges of all the other children and $e_0$
            \item If the node is $\tnode(k)$, number its children from $1$ to $n$. Let $i_1, \dots, i_{n-k}$ be the indices of the children marked ``simulated'' and $e_1, \dots,  e_{n-k}$ be their corresponding challenges. Let $i_0 = 0$. Viewing $0, 1, 2, \dots, n$ and $e_0, \dots, e_{n-k}$ as elements of $\GF(2^t)$, find (via polynomial interpolation) the lowest-degree polynomial $Q(x)=\sum_{i=0}^{n-k} a_i x^i $ over $\GF(2^t)$ that is equal to $e_j$ at $i_j$ for each $j$ from $0$ to $n-k$ (this polynomial will have $n-k+1$ coefficients, and the lowest coefficient will be $e_0$). For child number $i$ of the node, if the child is marked ``real'', compute its challenge as $Q(i)$ (if the child is marked ``simulated", its challenge is already $Q(i)$, by construction of $Q$).
        \end{itemize}
    
        \item If the node is a leaf marked ``real'', compute its response according to the second prover step of the $\Sigma$-protocol given the commitment, challenge, and witness
        
    \end{itemize}
    
    \item Output the proof consisting of the following information:
    \begin{itemize}
        \item The challenge of the root node
        \item For every $\ornode$ node: the challenges of all of its children but the rightmost
        \item For every $\tnode$ node: the coefficients $a_1 \dots a_{n-k}$ of the polynomial $Q(x)$ (excluding $a_0$)
        \item The responses for every leaf node
    \end{itemize}
\end{enumerate}

\subsection{Verifying}
\label{sec:verifying}

For each node in the tree, the verifier will compute a $t$-bit challenge given the information in the proof. The leaves will additionally have a response, which the verifier obtains directly from the proof, and a commitment, computed by the prover. \mnote{It is not clear how commitment computed by the prover is available to the verifier, because it is not stored in the proof}
Note that, unlike the prover, the verifier has no way of knowing which nodes are real and which are simulated (that is, in fact, one of the security guarantees of the protocol).
The pseudocode below details the verifier's steps.

\begin{enumerate}
\item Place into the root the root challenge received in the proof. 
\item In a top-down traversal of the tree, compute the challenges for the children of every non-leaf node. Let $e_0$ be the challenge in the node. Proceed as follows:
        \begin{itemize}
            \item If the node is $\andnode$,  then all of its children get $e_0$ as the challenge
            \item If the node is $\ornode$, then each of its children except rightmost one gets the challenge given in the proof for that node. The rightmost child gets a challenge computed as an XOR of the challenges of all the other children and $e_0$.
            \item If the node is $\tnode(k)$, let the number of its children be $n$. Assume the children are numbered from $1$ to $n$. Let $a_0=e_0$ and take the values $a_1, \dots, a_{n-k}$ from the proof. Viewing $1, 2, \dots, n$ and $a_0, \dots, a_{n-k}$ as elements of $\GF(2^t)$, evaluate the polynomial $Q(x) = \sum {a_i x^i}$ over $\GF(2^t)$ at points $1, 2, \dots, n$ to get challenges for child $1, 2, \dots, n$, respectively. 
        \end{itemize}

\item For every leaf node, compute the commitment from the challenge computed in the previous step and the response given in the proof, per the verifier algorithm of the leaf's $\Sigma$-protocol. If the verifier algorithm of the $\Sigma$-protocol for any of the leaves rejects, then reject the entire proof.

\item Convert the tree to a string $s$ for input to the Fiat-Shamir hash function, using the same conversion as the prover

\item Accept the proof if the challenge at the root of the tree is equal to the Fiat-Shamir hash of $s$ (and, if applicable,  the associated data). Reject otherwise.
\end{enumerate}


\subsection{Context}

Shared context of a blockchain system could be expressed in different ways, based on desired expressiveness, efficiency, planned use cases and so on. In this paper we focus on context for Ergo blockchain. The main priority for this blockchain is maximum efficiency of transaction validation process, safety, and friendliness to light clients. Considering this, we require that context should contain only spending transaction along with outputs it spends, and limited number of last block headers. Thus even a client which does not have all the headers of the blockchain~(for example, the client could be working in a light-SPV mode, where the client is holding only sublinear part of the headers-chain) is able to validate a transaction, by being shown it~(as well as outputs the transaction spends along with Merkle proofs for them).

\knote{brief context description, link to an appendix with details}


\section{Examples}

In this section we provide some examples of useful guarding scripts. We focus on examples which are impossible or much harder to express in Bitcoin Script.

\subsection{Crowdfunding}
\label{sec:crowdfunding}

We provide simple solution to crowdfunding here. In the example, a crowdfunding project associated with public key $x_P$ is considered successful if it can collect unspent outputs with total value not less than $to\_raise$ before height $timeout$. A project backer creates an output protected by the following statement: 

\begin{equation*}
\begin{split}
(height \ge timeout \land dlog(x_B)) \lor \\
(height & < timeout \land dlog(x_P)\\
& \land Exists(Outputs, 20,\\ 
& \quad \quad \quad \quad ExtractAmount(TaggedBox(20)) \ge to\_raise \land \\ 
& \quad \quad \quad \quad ExtractScriptBytes(TaggedBox(20)) = ToBytes(dlog(x_P))))
\end{split}
\end{equation*}

Then the project can collect biggest outputs with total value not less than $to\_raise$ with a single transaction~(it is possible to collect up to ~22,000 outputs in Bitcoin, which is enough even for a big crowdfunding campaign). For remaining small outputs over $to\_raise$, it is possible to construct follow-up transactions. 

Please note why such a guarding expression is not possible in Bitcoin: we use the condition on a spending transaction, namely, we require the transaction to have an output with value not less than required, and also with a particular statement protecting the output.

\subsection{Money With Scheduled Maintenance Payments}

\knote{description}

\begin{equation*}
\begin{split}
user\_statement \lor \\ 
(height & \ge (ExtractHeight(self) + period) \land \\
    & Exists(Outputs, 20, \\
    & \quad \quad \quad \quad ExtractAmount(TaggedBox(20)) \ge (ExtractAmount(self) - cost) \land \\ 
    & \quad \quad \quad \quad ExtractScriptBytes(TaggedBox(20)) = ExtractScriptBytes(self)))
\end{split}
\end{equation*}

We highlight impossibility of such a statement in the Bitcoin Script: similarly to the statement in the previous section~\ref{sec:crowdfunding}, we use a condition on a spending transaction. Another feature missed in the Bitcoin Script is that we also use an output to spend in the execution context. In particular, in the example above we require a spending transaction to have an output which has the same statement as an output it spends. 

\subsection{Ring Signature}
\label{sec:ring}

Linear-sized ring signature is very straightforward in the language. Assume a ring consists of $m$ public keys $x_1, \dots, x_m$. If one wants an output to be spent by a ring signature associated with the ring, the output is to be protected by the following statement:

$$dlog(x_1) \lor \dots \lor dlog(x_m)$$  

Please note that proving of the statement is to be done in zero-knowledge, so it is not possible to know which key signed, the only fact to conclude is that some key from the ring signed output spending. 

\subsection{Complex Signature Schemes}

We can build more complex signature schemes than possible in Bitcoin. One particular example was provided in the previous Section~\ref{sec:ring}. Another example is a scheme where at least one out of (Alice, Bob), and at least one out of (Charles, Diana) are needed to sign, and it is not to be known who signed an output spending. The corresponding statement involving public keys $x_A, x_B, x_C, x_D$ of Alice, Bob, Charles and Diana respectively would be following:

$$(dlog(x_A) \lor dlog(x_B)) \land (dlog(x_C) \lor dlog(x_D))$$

\subsection{Simple Tumbler}
\label{sec:tumbler}

\knote{does the example makes sense? check other tumbler papers. also, update the scripts, now approach is more generic than using tx.outbytes}

Privacy is a tough problem for cryptocurrency users. Bitcoin is a pseudonymous cryptocurrency, so no real identities attached to a transaction. However, it is possible to reconstruct transactional graph for all the transactions ever entered into the Bitcoin blockchain, and often restore real identities by using auxiliary databases. \knote{link} To improve privacy, tumblers are being used. A tumbler is a scheme which us getting some money transfers as inputs, produces output money transfers, and has a property of unlinkability: it is not possible to draw a link from an input to an output. Thus a tumbler user is hiding herself amongst a ring of users sending inputs to the scheme. Privacy then depends on a ring size. For maximum privacy, there exists a cryptocurrency ZCash with inbuilt tumbler based on zkSnarks\knote{links}, where a user is hiding among all the users in the system. However, ZCash requires trusted setup, and transaction validation is relatively slow~(10 ms). In other cryptocurrencies users are usually using external services varying in security and efficiency.

Here we are describing simple tumbler service. It is very efficient and requires no trusted dealer. Its disadvantage is that from observing blockchain transactions it is possible to conclude that users are using a tumbler.

Assume Alice with public key $x_A$ and Bob with public key $x_B$ want to relocate funds to keys $y_1$ and $y_2$ respectively, with a property than external observer looking into blockchain is not capable to know beyond the fact that money flows from $x_A$ to whether $y_1$ or $y_2$~(and the same for $x_B$).

First, Alice and Bob communicate off-chain to construct collectively outputs of the final transaction~(payments to $y_1$ and $y_2$). Each of them then is calculating a hash value from outputs bytes $h$~(we assume that Blake2b-256 hash function is used). Then each of them is creating an output to spend~(possibly, in a dedicated transaction) with such a condition for Alice and Bob, respectively:

$blake2b256(tx.outbytes) = h \lor dlog(x_A)$ 

$blake2b256(tx.outbytes) = h \lor dlog(x_B)$

Then it is possible to make a refund at any moment of time~(right condition in the $\lor$ conjectures), and before a refund any of the them can construct a transaction which is spending the outputs, and it is impossible to construct an alternative transaction, for which hash of the output bytes is $h$, but bytes are different~(if chosen hash function is collision-resistant).

\subsection{Oracle Example}

\knote{Text below is just copied from code comments, polish it}

A trusted weather station is publishing temperature data on blockchain. Alice and Bob are making a contract based on the data:
they have locked coins in such way that if output from the station shows that temperature announced by the oracle is > 15 degrees, money are going to Alice, otherwise to Bob.
    
We consider that for validating transaction only limited number of last headers and the spending transaction should be enough, in addition to outputs being spent by the transaction. Thus there is no need for knowledge of an arbitrary output. To show the coin of the weather service in the spending transaction, outputs from Alice and Bob are referencing to the coin by using Merkle proofs against UTXO set root hash of the latest known block.
    
A tricky moment is how Alice and Bob can be sure that a coin is indeed created by the service, having just the coin (and also service's public key x = $g^w$, where service's secret w is not known.
    
For that, we consider that the service creates a coin with registers of following semantics (R0 and R1 are standard):
    
R1 - coin amount
R2 - protecting script, which is the pubkey of the service, $x = g^w$
R3 - temperature data, number
R4 - $a = g^r$, where r is secret random nonce
R5 - $z = r + ew mod q$
R6 - timestamp
    
Then Alice and Bob are requiring from the coin that the following equation holds: $g^z = a * x^e$, where $e = hash(R3 ++ R6)$
    
Thus Alice, for example, is created a coin with the following statement (we skip timeouts for simplicity):
"the coin is spendable by presenting a proof of Alice's private key knowledge if against UTXO set root hash for
the last known block there is a coin along with a Merkle proof, for which following requirements hold:
$R2 = dlog(x) /\ g^(R5) = R4 * x^(hash(R3 ++ R6)) /\ (R3) > 15$. Similarly, the coin is spendable by a proof of
knowledge of the Bob's private key if all the same conditions are met but $(R3) <= 15$.".
    
The Bob can create a box with the same guarding conditions. However, if Alice's box is already in the state, then Bob can stick to it by using the trick from the "along with a brother" test.


\section{Safety Guarantees}
\label{sec:safety}

We need to be sure that an adversary can not produce such a statement for which the Verifier spends more time than it is safe to spend.
\knote{links to verifier dilemma, orphan rates etc} 

\subsection{Cost Model}
\label{sec:cost-model}

\subsection{Denial-of-Service Attacks}

\section{Extensibility}

It is hard to predict which functions would be useful for blockchain applications.

\section{Implementation and Evaluation}

\section{Further Work}

\section{Conclusion}

We show in this paper...




\bibliographystyle{splncs03}
\bibliography{sigma.bib}



\appendix

\section{Types}

\[\begin{tabular}{@{}l c l l}

      $\mathcal{T} \ni \tau$			& ::= 	    &\
         \lst{Byte} $\mid$ \lst{Short} $\mid$
         \lst{Int} $\mid$ \lst{Long} $\mid$ \lst{BigInt}  & numeric types     \\
      &	$\mid$	& \lst{Boolean} 			& type of logical values true/false   \\
      &	$\mid$	& \lst{Unit} 				& type with single element   \\
      &	$\mid$	& \lst{GroupElement} 		& element of the cryptographic group  \\
      &	$\mid$	& \lst{Box} 				& value in a box protected by proposition  \\
      &	$\mid$	& \lst{AvlTree} 			&  Authenticated Dynamic Dictionary \\
      &	$\mid$	& $(\tau_1, \dots, \tau_n) $	& binary product type  \\
      & $\mid$  & $\text{\lst{Array}}[\tau]$	& array type       \\
      & $\mid$  & $\text{\lst{Option}}[\tau]$	& optional value (either $Some(\tau)$ or $None$)      \\
      & $\mid$  & \lst{Any}                     & type of any value (common supertype of all the types) \\
%      &   $\mid$  & $(\tau_1 + \tau_2) $					& sum type       \\
%      &   $\mid$  & $(\tau_1 \to \tau_2) $				& function type       \\
      %				& 	     	&										&				\\
%      $Term \ni e$	& ::= 		&   $l \mid x : \tau $							& integer literals and vars     \\
%      & 	$\mid$ 	& 	$\TyLam{x}{\tau}{e} \mid e_1 e_2$ 	& functions  \\
%      & 	$\mid$ 	& 	$k~\Ov{e}$ 							& constructors  \\
%      & 	$\mid$ 	& 	$\delta~\Ov{e}$ 					& primitives  \\
%      & 	$\mid$ 	& 	\lst{case} $e$ \lst{of} \{ $p_i$ $\rightarrow$ $e_i$  \}  & case analysis \\
      %				& 	$\mid$ 	& 	\lst{let} $x = e$ \lst{in} $e'$  	& let binding \\
      %				& 	     	&										&				\\
%      $k$				& ::= 		&   $\Tup{} \mid \Tup{\_, \_}$ 			& unit and pair     	\\
%      & 	$\mid$ 	&	$\Left{\tau_1}{\tau_2}{\text{\lst{\_}}} \mid \Right{\tau_1}{\tau_2}{\text{\lst{\_}}}$			& injections		\\
      %				& 	     	&										&				\\
%      $\delta$		& ::= 		&   \Fst{\_} $\mid$ \Snd{\_}			& projections     	\\
%      & 	$\mid$ 	&	$\_ \oplus \_$						& binary operations	\\
      %				& 	     	&										&				\\
%      $p$				& ::= 		&   $l \mid k~\Ov{x}$ 			& patterns     	\\
      %				& 	     	&										&				\\
%      $v$				& ::= 		&   $l \mid k~\Ov{v} \mid \TyLam{x}{\tau}{e}$ 		& values     	\\
%      & & & \\
\end{tabular}\]

\section{Tree Nodes and Transformation Rules}

\section{Cost Table}

\section{Roadmap}

\section{Outputs vs Accounts}

\section{Unified Transactions}
\label{apx:unified}

\section{An Example of a Concrete Transaction Format}
\label{apx:tx-format}

\knote{Describe Ergo transaction format here. Malleability problems to be discussed here.}

\section{A Language For An Account-Based Cryptocurrency}
\label{apx:account}



\end{document}