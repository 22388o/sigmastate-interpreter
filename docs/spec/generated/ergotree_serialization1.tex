
\subsubsection{\lst{ConcreteCollection} operation (OpCode 131)}
\label{sec:serialization:operation:ConcreteCollection}

 

\noindent
\(\begin{tabularx}{\textwidth}{| l | l | l | X |}
    \hline
    \bf{Slot} & \bf{Format} & \bf{\#bytes} & \bf{Description} \\
    \hline
         $ numItems $ & \lst{VLQ(UShort)} & [1, *] & number of item in a collection of expressions \\
    \hline
           $ elementType $ & \lst{Type} & [1, *] & type of each expression in the collection \\
    \hline
          \multicolumn{4}{l}{\lst{for}~$i=1$~\lst{to}~$numItems$} \\
    \hline
             ~~ $ item_i $ & \lst{Expr} & [1, *] & expression in i-th position \\
    \hline
          \multicolumn{4}{l}{\lst{end for}} \\
\end{tabularx}\)
       

\subsubsection{\lst{ConcreteCollectionBooleanConstant} operation (OpCode 133)}
\label{sec:serialization:operation:ConcreteCollectionBooleanConstant}

 

\noindent
\(\begin{tabularx}{\textwidth}{| l | l | l | X |}
    \hline
    \bf{Slot} & \bf{Format} & \bf{\#bytes} & \bf{Description} \\
    \hline
         $ numBits $ & \lst{VLQ(UShort)} & [1, *] & number of items in a collection of Boolean values \\
    \hline
           $ bits $ & \lst{Bits} & [1, 1024] & Boolean values encoded as as bits (right most byte is zero-padded on the right) \\
    \hline
      
\end{tabularx}\)
       

\subsubsection{\lst{Tuple} operation (OpCode 134)}
\label{sec:serialization:operation:Tuple}

 

\noindent
\(\begin{tabularx}{\textwidth}{| l | l | l | X |}
    \hline
    \bf{Slot} & \bf{Format} & \bf{\#bytes} & \bf{Description} \\
    \hline
         $ numItems $ & \lst{UByte} & 1 & number of items in the tuple \\
    \hline
          \multicolumn{4}{l}{\lst{for}~$i=1$~\lst{to}~$numItems$} \\
    \hline
             ~~ $ item_i $ & \lst{Expr} & [1, *] & tuple's item in i-th position \\
    \hline
          \multicolumn{4}{l}{\lst{end for}} \\
\end{tabularx}\)
       

\subsubsection{\lst{SelectField} operation (OpCode 140)}
\label{sec:serialization:operation:SelectField}

Select tuple field by its 1-based index. E.g. \lst{input._1} is transformed to \lst{SelectField(input, 1)} See~\hyperref[sec:appendix:primops:SelectField]{\lst{selectField}}

\noindent
\(\begin{tabularx}{\textwidth}{| l | l | l | X |}
    \hline
    \bf{Slot} & \bf{Format} & \bf{\#bytes} & \bf{Description} \\
    \hline
         $ input $ & \lst{Expr} & [1, *] & tuple of items \\
    \hline
           $ fieldIndex $ & \lst{Byte} & 1 & index of an item to select \\
    \hline
      
\end{tabularx}\)
       

\subsubsection{\lst{LT} operation (OpCode 143)}
\label{sec:serialization:operation:LT}

Returns \lst{true} is the left operand is less then the right operand, \lst{false} otherwise. See~\hyperref[sec:appendix:primops:LT]{\lst{<}}

\noindent
\(\begin{tabularx}{\textwidth}{| l | l | l | X |}
    \hline
    \bf{Slot} & \bf{Format} & \bf{\#bytes} & \bf{Description} \\
    \hline
        \multicolumn{4}{l}{\lst{match}~$ (left, right) $} \\
         
    \multicolumn{4}{l}{~~\lst{otherwise} } \\
    \hline
            ~~~~ $ left $ & \lst{Expr} & [1, *] & left operand \\
    \hline
          ~~~~ $ right $ & \lst{Expr} & [1, *] & right operand \\
    \hline
          \multicolumn{4}{l}{\lst{end match}} \\
\end{tabularx}\)
       

\subsubsection{\lst{LE} operation (OpCode 144)}
\label{sec:serialization:operation:LE}

Returns \lst{true} is the left operand is less then or equal to the right operand, \lst{false} otherwise. See~\hyperref[sec:appendix:primops:LE]{\lst{<=}}

\noindent
\(\begin{tabularx}{\textwidth}{| l | l | l | X |}
    \hline
    \bf{Slot} & \bf{Format} & \bf{\#bytes} & \bf{Description} \\
    \hline
        \multicolumn{4}{l}{\lst{match}~$ (left, right) $} \\
         
    \multicolumn{4}{l}{~~\lst{otherwise} } \\
    \hline
            ~~~~ $ left $ & \lst{Expr} & [1, *] & left operand \\
    \hline
          ~~~~ $ right $ & \lst{Expr} & [1, *] & right operand \\
    \hline
          \multicolumn{4}{l}{\lst{end match}} \\
\end{tabularx}\)
       

\subsubsection{\lst{GT} operation (OpCode 145)}
\label{sec:serialization:operation:GT}

Returns \lst{true} is the left operand is greater then the right operand, \lst{false} otherwise. See~\hyperref[sec:appendix:primops:GT]{\lst{>}}

\noindent
\(\begin{tabularx}{\textwidth}{| l | l | l | X |}
    \hline
    \bf{Slot} & \bf{Format} & \bf{\#bytes} & \bf{Description} \\
    \hline
        \multicolumn{4}{l}{\lst{match}~$ (left, right) $} \\
         
    \multicolumn{4}{l}{~~\lst{otherwise} } \\
    \hline
            ~~~~ $ left $ & \lst{Expr} & [1, *] & left operand \\
    \hline
          ~~~~ $ right $ & \lst{Expr} & [1, *] & right operand \\
    \hline
          \multicolumn{4}{l}{\lst{end match}} \\
\end{tabularx}\)
       

\subsubsection{\lst{GE} operation (OpCode 146)}
\label{sec:serialization:operation:GE}

Returns \lst{true} is the left operand is greater then or equal to the right operand, \lst{false} otherwise. See~\hyperref[sec:appendix:primops:GE]{\lst{>=}}

\noindent
\(\begin{tabularx}{\textwidth}{| l | l | l | X |}
    \hline
    \bf{Slot} & \bf{Format} & \bf{\#bytes} & \bf{Description} \\
    \hline
        \multicolumn{4}{l}{\lst{match}~$ (left, right) $} \\
         
    \multicolumn{4}{l}{~~\lst{otherwise} } \\
    \hline
            ~~~~ $ left $ & \lst{Expr} & [1, *] & left operand \\
    \hline
          ~~~~ $ right $ & \lst{Expr} & [1, *] & right operand \\
    \hline
          \multicolumn{4}{l}{\lst{end match}} \\
\end{tabularx}\)
       

\subsubsection{\lst{EQ} operation (OpCode 147)}
\label{sec:serialization:operation:EQ}

Compare equality of \lst{left} and \lst{right} arguments See~\hyperref[sec:appendix:primops:EQ]{\lst{==}}

\noindent
\(\begin{tabularx}{\textwidth}{| l | l | l | X |}
    \hline
    \bf{Slot} & \bf{Format} & \bf{\#bytes} & \bf{Description} \\
    \hline
        \multicolumn{4}{l}{\lst{match}~$ (left, right) $} \\
         
    \multicolumn{4}{l}{~~\lst{with}~$(Constant(l, Boolean), Constant(r, Boolean))$ } \\
    \hline
            ~~~~ $ opCode $ & \lst{Byte} & 1 & always contains OpCode 133 \\
    \hline
          ~~~~ $ (l,r) $ & \lst{Bits} & 1 & two higher bits in a byte \\
    \hline
      
    \multicolumn{4}{l}{~~\lst{otherwise} } \\
    \hline
            ~~~~ $ left $ & \lst{Expr} & [1, *] & left operand \\
    \hline
          ~~~~ $ right $ & \lst{Expr} & [1, *] & right operand \\
    \hline
          \multicolumn{4}{l}{\lst{end match}} \\
\end{tabularx}\)
       

\subsubsection{\lst{NEQ} operation (OpCode 148)}
\label{sec:serialization:operation:NEQ}

Compare inequality of \lst{left} and \lst{right} arguments See~\hyperref[sec:appendix:primops:NEQ]{\lst{!=}}

\noindent
\(\begin{tabularx}{\textwidth}{| l | l | l | X |}
    \hline
    \bf{Slot} & \bf{Format} & \bf{\#bytes} & \bf{Description} \\
    \hline
        \multicolumn{4}{l}{\lst{match}~$ (left, right) $} \\
         
    \multicolumn{4}{l}{~~\lst{with}~$(Constant(l, Boolean), Constant(r, Boolean))$ } \\
    \hline
            ~~~~ $ opCode $ & \lst{Byte} & 1 & always contains OpCode 133 \\
    \hline
          ~~~~ $ (l,r) $ & \lst{Bits} & 1 & two higher bits in a byte \\
    \hline
      
    \multicolumn{4}{l}{~~\lst{otherwise} } \\
    \hline
            ~~~~ $ left $ & \lst{Expr} & [1, *] & left operand \\
    \hline
          ~~~~ $ right $ & \lst{Expr} & [1, *] & right operand \\
    \hline
          \multicolumn{4}{l}{\lst{end match}} \\
\end{tabularx}\)
       

\subsubsection{\lst{If} operation (OpCode 149)}
\label{sec:serialization:operation:If}

Compute condition, if true then compute trueBranch else compute falseBranch See~\hyperref[sec:appendix:primops:If]{\lst{if}}

\noindent
\(\begin{tabularx}{\textwidth}{| l | l | l | X |}
    \hline
    \bf{Slot} & \bf{Format} & \bf{\#bytes} & \bf{Description} \\
    \hline
         $ condition $ & \lst{Expr} & [1, *] & condition expression \\
    \hline
           $ trueBranch $ & \lst{Expr} & [1, *] & expression to execute when \lst{condition == true} \\
    \hline
           $ falseBranch $ & \lst{Expr} & [1, *] & expression to execute when \lst{condition == false} \\
    \hline
      
\end{tabularx}\)
       

\subsubsection{\lst{AND} operation (OpCode 150)}
\label{sec:serialization:operation:AND}

Returns true if \emph{all} the elements in collection are \lst{true}. See~\hyperref[sec:appendix:primops:AND]{\lst{allOf}}

\noindent
\(\begin{tabularx}{\textwidth}{| l | l | l | X |}
    \hline
    \bf{Slot} & \bf{Format} & \bf{\#bytes} & \bf{Description} \\
    \hline
         $ conditions $ & \lst{Expr} & [1, *] & a collection of conditions \\
    \hline
      
\end{tabularx}\)
       

\subsubsection{\lst{OR} operation (OpCode 151)}
\label{sec:serialization:operation:OR}

Returns true if \emph{any} the elements in collection are \lst{true}. See~\hyperref[sec:appendix:primops:OR]{\lst{anyOf}}

\noindent
\(\begin{tabularx}{\textwidth}{| l | l | l | X |}
    \hline
    \bf{Slot} & \bf{Format} & \bf{\#bytes} & \bf{Description} \\
    \hline
         $ conditions $ & \lst{Expr} & [1, *] & a collection of conditions \\
    \hline
      
\end{tabularx}\)
       

\subsubsection{\lst{AtLeast} operation (OpCode 152)}
\label{sec:serialization:operation:AtLeast}

 Logical threshold.
 AtLeast has two inputs: integer \lst{bound} and \lst{children} same as in AND/OR.
 The result is true if at least \lst{bound} children are proven.
         See~\hyperref[sec:appendix:primops:AtLeast]{\lst{atLeast}}

\noindent
\(\begin{tabularx}{\textwidth}{| l | l | l | X |}
    \hline
    \bf{Slot} & \bf{Format} & \bf{\#bytes} & \bf{Description} \\
    \hline
         $ bound $ & \lst{Expr} & [1, *] & required minimum of proven children \\
    \hline
           $ children $ & \lst{Expr} & [1, *] & proposition to be proven/validated \\
    \hline
      
\end{tabularx}\)
       

\subsubsection{\lst{Minus} operation (OpCode 153)}
\label{sec:serialization:operation:Minus}

Returns a result of subtracting second numeric operand from the first. See~\hyperref[sec:appendix:primops:Minus]{\lst{-}}

\noindent
\(\begin{tabularx}{\textwidth}{| l | l | l | X |}
    \hline
    \bf{Slot} & \bf{Format} & \bf{\#bytes} & \bf{Description} \\
    \hline
         $ left $ & \lst{Expr} & [1, *] & left operand \\
    \hline
           $ right $ & \lst{Expr} & [1, *] & right operand \\
    \hline
      
\end{tabularx}\)
       

\subsubsection{\lst{Plus} operation (OpCode 154)}
\label{sec:serialization:operation:Plus}

Returns a sum of two numeric operands See~\hyperref[sec:appendix:primops:Plus]{\lst{+}}

\noindent
\(\begin{tabularx}{\textwidth}{| l | l | l | X |}
    \hline
    \bf{Slot} & \bf{Format} & \bf{\#bytes} & \bf{Description} \\
    \hline
         $ left $ & \lst{Expr} & [1, *] & left operand \\
    \hline
           $ right $ & \lst{Expr} & [1, *] & right operand \\
    \hline
      
\end{tabularx}\)
       

\subsubsection{\lst{Xor} operation (OpCode 155)}
\label{sec:serialization:operation:Xor}

Byte-wise XOR of two collections of bytes. Example: \lst{xs | ys}. See~\hyperref[sec:type:SigmaDslBuilder:xor]{\lst{SigmaDslBuilder.xor}}

\noindent
\(\begin{tabularx}{\textwidth}{| l | l | l | X |}
    \hline
    \bf{Slot} & \bf{Format} & \bf{\#bytes} & \bf{Description} \\
    \hline
         $ left $ & \lst{Expr} & [1, *] & left operand \\
    \hline
           $ right $ & \lst{Expr} & [1, *] & right operand \\
    \hline
      
\end{tabularx}\)
       

\subsubsection{\lst{Multiply} operation (OpCode 156)}
\label{sec:serialization:operation:Multiply}

Returns a multiplication of two numeric operands See~\hyperref[sec:appendix:primops:Multiply]{\lst{*}}

\noindent
\(\begin{tabularx}{\textwidth}{| l | l | l | X |}
    \hline
    \bf{Slot} & \bf{Format} & \bf{\#bytes} & \bf{Description} \\
    \hline
         $ left $ & \lst{Expr} & [1, *] & left operand \\
    \hline
           $ right $ & \lst{Expr} & [1, *] & right operand \\
    \hline
      
\end{tabularx}\)
       

\subsubsection{\lst{Division} operation (OpCode 157)}
\label{sec:serialization:operation:Division}

Integer division of the first operand by the second operand. See~\hyperref[sec:appendix:primops:Division]{\lst{/}}

\noindent
\(\begin{tabularx}{\textwidth}{| l | l | l | X |}
    \hline
    \bf{Slot} & \bf{Format} & \bf{\#bytes} & \bf{Description} \\
    \hline
         $ left $ & \lst{Expr} & [1, *] & left operand \\
    \hline
           $ right $ & \lst{Expr} & [1, *] & right operand \\
    \hline
      
\end{tabularx}\)
       

\subsubsection{\lst{Modulo} operation (OpCode 158)}
\label{sec:serialization:operation:Modulo}

Reminder from division of the first operand by the second operand. See~\hyperref[sec:appendix:primops:Modulo]{\lst{\%}}

\noindent
\(\begin{tabularx}{\textwidth}{| l | l | l | X |}
    \hline
    \bf{Slot} & \bf{Format} & \bf{\#bytes} & \bf{Description} \\
    \hline
         $ left $ & \lst{Expr} & [1, *] & left operand \\
    \hline
           $ right $ & \lst{Expr} & [1, *] & right operand \\
    \hline
      
\end{tabularx}\)
       

\subsubsection{\lst{Exponentiate} operation (OpCode 159)}
\label{sec:serialization:operation:Exponentiate}

Exponentiate this \lst{GroupElement} to the given number. Returns this to the power of k See~\hyperref[sec:type:GroupElement:exp]{\lst{GroupElement.exp}}

\noindent
\(\begin{tabularx}{\textwidth}{| l | l | l | X |}
    \hline
    \bf{Slot} & \bf{Format} & \bf{\#bytes} & \bf{Description} \\
    \hline
         $ this $ & \lst{Expr} & [1, *] & this instance \\
    \hline
           $ k $ & \lst{Expr} & [1, *] & The power \\
    \hline
      
\end{tabularx}\)
       

\subsubsection{\lst{MultiplyGroup} operation (OpCode 160)}
\label{sec:serialization:operation:MultiplyGroup}

Group operation. See~\hyperref[sec:type:GroupElement:multiply]{\lst{GroupElement.multiply}}

\noindent
\(\begin{tabularx}{\textwidth}{| l | l | l | X |}
    \hline
    \bf{Slot} & \bf{Format} & \bf{\#bytes} & \bf{Description} \\
    \hline
         $ this $ & \lst{Expr} & [1, *] & this instance \\
    \hline
           $ other $ & \lst{Expr} & [1, *] & other element of the group \\
    \hline
      
\end{tabularx}\)
       

\subsubsection{\lst{Min} operation (OpCode 161)}
\label{sec:serialization:operation:Min}

Minimum value of two operands. See~\hyperref[sec:appendix:primops:Min]{\lst{min}}

\noindent
\(\begin{tabularx}{\textwidth}{| l | l | l | X |}
    \hline
    \bf{Slot} & \bf{Format} & \bf{\#bytes} & \bf{Description} \\
    \hline
         $ left $ & \lst{Expr} & [1, *] & left operand \\
    \hline
           $ right $ & \lst{Expr} & [1, *] & right operand \\
    \hline
      
\end{tabularx}\)
       

\subsubsection{\lst{Max} operation (OpCode 162)}
\label{sec:serialization:operation:Max}

Maximum value of two operands. See~\hyperref[sec:appendix:primops:Max]{\lst{max}}

\noindent
\(\begin{tabularx}{\textwidth}{| l | l | l | X |}
    \hline
    \bf{Slot} & \bf{Format} & \bf{\#bytes} & \bf{Description} \\
    \hline
         $ left $ & \lst{Expr} & [1, *] & left operand \\
    \hline
           $ right $ & \lst{Expr} & [1, *] & right operand \\
    \hline
      
\end{tabularx}\)
       

\subsubsection{\lst{MapCollection} operation (OpCode 173)}
\label{sec:serialization:operation:MapCollection}

 Builds a new collection by applying a function to all elements of this collection.
 Returns a new collection of type \lst{Coll[B]} resulting from applying the given function
 \lst{f} to each element of this collection and collecting the results.
         See~\hyperref[sec:type:SCollection:map]{\lst{SCollection.map}}

\noindent
\(\begin{tabularx}{\textwidth}{| l | l | l | X |}
    \hline
    \bf{Slot} & \bf{Format} & \bf{\#bytes} & \bf{Description} \\
    \hline
         $ this $ & \lst{Expr} & [1, *] & this instance \\
    \hline
           $ f $ & \lst{Expr} & [1, *] & the function to apply to each element \\
    \hline
      
\end{tabularx}\)
       

\subsubsection{\lst{Exists} operation (OpCode 174)}
\label{sec:serialization:operation:Exists}

Tests whether a predicate holds for at least one element of this collection.
Returns \lst{true} if the given predicate \lst{p} is satisfied by at least one element of this collection, otherwise \lst{false}
         See~\hyperref[sec:type:SCollection:exists]{\lst{SCollection.exists}}

\noindent
\(\begin{tabularx}{\textwidth}{| l | l | l | X |}
    \hline
    \bf{Slot} & \bf{Format} & \bf{\#bytes} & \bf{Description} \\
    \hline
         $ this $ & \lst{Expr} & [1, *] & this instance \\
    \hline
           $ p $ & \lst{Expr} & [1, *] & the predicate used to test elements \\
    \hline
      
\end{tabularx}\)
       

\subsubsection{\lst{ForAll} operation (OpCode 175)}
\label{sec:serialization:operation:ForAll}

Tests whether a predicate holds for all elements of this collection.
Returns \lst{true} if this collection is empty or the given predicate \lst{p}
holds for all elements of this collection, otherwise \lst{false}.
         See~\hyperref[sec:type:SCollection:forall]{\lst{SCollection.forall}}

\noindent
\(\begin{tabularx}{\textwidth}{| l | l | l | X |}
    \hline
    \bf{Slot} & \bf{Format} & \bf{\#bytes} & \bf{Description} \\
    \hline
         $ this $ & \lst{Expr} & [1, *] & this instance \\
    \hline
           $ p $ & \lst{Expr} & [1, *] & the predicate used to test elements \\
    \hline
      
\end{tabularx}\)
       

\subsubsection{\lst{Fold} operation (OpCode 176)}
\label{sec:serialization:operation:Fold}

Applies a binary operator to a start value and all elements of this collection, going left to right. See~\hyperref[sec:type:SCollection:fold]{\lst{SCollection.fold}}

\noindent
\(\begin{tabularx}{\textwidth}{| l | l | l | X |}
    \hline
    \bf{Slot} & \bf{Format} & \bf{\#bytes} & \bf{Description} \\
    \hline
         $ this $ & \lst{Expr} & [1, *] & this instance \\
    \hline
           $ zero $ & \lst{Expr} & [1, *] & a starting value \\
    \hline
           $ op $ & \lst{Expr} & [1, *] & the binary operator \\
    \hline
      
\end{tabularx}\)
       

\subsubsection{\lst{SizeOf} operation (OpCode 177)}
\label{sec:serialization:operation:SizeOf}

The size of the collection in elements. See~\hyperref[sec:type:SCollection:size]{\lst{SCollection.size}}

\noindent
\(\begin{tabularx}{\textwidth}{| l | l | l | X |}
    \hline
    \bf{Slot} & \bf{Format} & \bf{\#bytes} & \bf{Description} \\
    \hline
         $ this $ & \lst{Expr} & [1, *] & this instance \\
    \hline
      
\end{tabularx}\)
       

\subsubsection{\lst{ByIndex} operation (OpCode 178)}
\label{sec:serialization:operation:ByIndex}

Return the element of collection if \lst{index} is in range \lst{0 .. size-1} See~\hyperref[sec:type:SCollection:getOrElse]{\lst{SCollection.getOrElse}}

\noindent
\(\begin{tabularx}{\textwidth}{| l | l | l | X |}
    \hline
    \bf{Slot} & \bf{Format} & \bf{\#bytes} & \bf{Description} \\
    \hline
         $ this $ & \lst{Expr} & [1, *] & this instance \\
    \hline
           $ index $ & \lst{Expr} & [1, *] & index of the element of this collection \\
    \hline
          \multicolumn{4}{l}{\lst{optional}~$default$} \\
    \hline
    ~~$tag$ & \lst{Byte} & 1 & 0 - no value; 1 - has value \\
    \hline
    \multicolumn{4}{l}{~~\lst{when}~$tag == 1$} \\
    \hline
             ~~~~ $ default $ & \lst{Expr} & [1, *] & value to return when \lst{index} is out of range \\
    \hline
          \multicolumn{4}{l}{\lst{end optional}} \\
\end{tabularx}\)
       

\subsubsection{\lst{Append} operation (OpCode 179)}
\label{sec:serialization:operation:Append}

Puts the elements of other collection after the elements of this collection (concatenation of 2 collections) See~\hyperref[sec:type:SCollection:append]{\lst{SCollection.append}}

\noindent
\(\begin{tabularx}{\textwidth}{| l | l | l | X |}
    \hline
    \bf{Slot} & \bf{Format} & \bf{\#bytes} & \bf{Description} \\
    \hline
         $ this $ & \lst{Expr} & [1, *] & this instance \\
    \hline
           $ other $ & \lst{Expr} & [1, *] & the collection to append at the end of this \\
    \hline
      
\end{tabularx}\)
       

\subsubsection{\lst{Slice} operation (OpCode 180)}
\label{sec:serialization:operation:Slice}

Selects an interval of elements.  The returned collection is made up
  of all elements \lst{x} which satisfy the invariant:
  \lst{
     from <= indexOf(x) < until
  }
         See~\hyperref[sec:type:SCollection:slice]{\lst{SCollection.slice}}

\noindent
\(\begin{tabularx}{\textwidth}{| l | l | l | X |}
    \hline
    \bf{Slot} & \bf{Format} & \bf{\#bytes} & \bf{Description} \\
    \hline
         $ this $ & \lst{Expr} & [1, *] & this instance \\
    \hline
           $ from $ & \lst{Expr} & [1, *] & the lowest index to include from this collection \\
    \hline
           $ until $ & \lst{Expr} & [1, *] & the lowest index to EXCLUDE from this collection \\
    \hline
      
\end{tabularx}\)
       

\subsubsection{\lst{ExtractAmount} operation (OpCode 193)}
\label{sec:serialization:operation:ExtractAmount}

Monetary value in NanoERGs stored in this box. See~\hyperref[sec:type:Box:value]{\lst{Box.value}}

\noindent
\(\begin{tabularx}{\textwidth}{| l | l | l | X |}
    \hline
    \bf{Slot} & \bf{Format} & \bf{\#bytes} & \bf{Description} \\
    \hline
         $ this $ & \lst{Expr} & [1, *] & this instance \\
    \hline
      
\end{tabularx}\)
       

\subsubsection{\lst{ExtractScriptBytes} operation (OpCode 194)}
\label{sec:serialization:operation:ExtractScriptBytes}

Serialized bytes of the guarding script which should be evaluated to true in order to
 open this box (spend it in a transaction). See~\hyperref[sec:type:Box:propositionBytes]{\lst{Box.propositionBytes}}

\noindent
\(\begin{tabularx}{\textwidth}{| l | l | l | X |}
    \hline
    \bf{Slot} & \bf{Format} & \bf{\#bytes} & \bf{Description} \\
    \hline
         $ this $ & \lst{Expr} & [1, *] & this instance \\
    \hline
      
\end{tabularx}\)
       

\subsubsection{\lst{ExtractBytes} operation (OpCode 195)}
\label{sec:serialization:operation:ExtractBytes}

Serialized bytes of this box's content, including proposition bytes. See~\hyperref[sec:type:Box:bytes]{\lst{Box.bytes}}

\noindent
\(\begin{tabularx}{\textwidth}{| l | l | l | X |}
    \hline
    \bf{Slot} & \bf{Format} & \bf{\#bytes} & \bf{Description} \\
    \hline
         $ this $ & \lst{Expr} & [1, *] & this instance \\
    \hline
      
\end{tabularx}\)
       

\subsubsection{\lst{ExtractBytesWithNoRef} operation (OpCode 196)}
\label{sec:serialization:operation:ExtractBytesWithNoRef}

Serialized bytes of this box's content, excluding transactionId and index of output. See~\hyperref[sec:type:Box:bytesWithoutRef]{\lst{Box.bytesWithoutRef}}

\noindent
\(\begin{tabularx}{\textwidth}{| l | l | l | X |}
    \hline
    \bf{Slot} & \bf{Format} & \bf{\#bytes} & \bf{Description} \\
    \hline
         $ this $ & \lst{Expr} & [1, *] & this instance \\
    \hline
      
\end{tabularx}\)
       

\subsubsection{\lst{ExtractId} operation (OpCode 197)}
\label{sec:serialization:operation:ExtractId}

Blake2b256 hash of this box's content, basically equals to \lst{blake2b256(bytes)} See~\hyperref[sec:type:Box:id]{\lst{Box.id}}

\noindent
\(\begin{tabularx}{\textwidth}{| l | l | l | X |}
    \hline
    \bf{Slot} & \bf{Format} & \bf{\#bytes} & \bf{Description} \\
    \hline
         $ this $ & \lst{Expr} & [1, *] & this instance \\
    \hline
      
\end{tabularx}\)
       

\subsubsection{\lst{ExtractRegisterAs} operation (OpCode 198)}
\label{sec:serialization:operation:ExtractRegisterAs}

 Extracts register by id and type.
 Type param \lst{T} expected type of the register.
 Returns \lst{Some(value)} if the register is defined and has given type and \lst{None} otherwise
         See~\hyperref[sec:type:Box:getReg]{\lst{Box.getReg}}

\noindent
\(\begin{tabularx}{\textwidth}{| l | l | l | X |}
    \hline
    \bf{Slot} & \bf{Format} & \bf{\#bytes} & \bf{Description} \\
    \hline
         $ this $ & \lst{Expr} & [1, *] & this instance \\
    \hline
           $ regId $ & \lst{Byte} & 1 & zero-based identifier of the register. \\
    \hline
           $ type $ & \lst{Type} & [1, *] & expected type of the value in register \\
    \hline
      
\end{tabularx}\)
       

\subsubsection{\lst{ExtractCreationInfo} operation (OpCode 199)}
\label{sec:serialization:operation:ExtractCreationInfo}

 If \lst{tx} is a transaction which generated this box, then \lst{creationInfo._1}
 is a height of the tx's block. The \lst{creationInfo._2} is a serialized bytes of the transaction
 identifier followed by the serialized bytes of the box index in the transaction outputs.
         See~\hyperref[sec:type:Box:creationInfo]{\lst{Box.creationInfo}}

\noindent
\(\begin{tabularx}{\textwidth}{| l | l | l | X |}
    \hline
    \bf{Slot} & \bf{Format} & \bf{\#bytes} & \bf{Description} \\
    \hline
         $ this $ & \lst{Expr} & [1, *] & this instance \\
    \hline
      
\end{tabularx}\)
       

\subsubsection{\lst{CalcBlake2b256} operation (OpCode 203)}
\label{sec:serialization:operation:CalcBlake2b256}

Calculate Blake2b hash from \lst{input} bytes. See~\hyperref[sec:appendix:primops:CalcBlake2b256]{\lst{blake2b256}}

\noindent
\(\begin{tabularx}{\textwidth}{| l | l | l | X |}
    \hline
    \bf{Slot} & \bf{Format} & \bf{\#bytes} & \bf{Description} \\
    \hline
         $ input $ & \lst{Expr} & [1, *] & collection of bytes \\
    \hline
      
\end{tabularx}\)
       

\subsubsection{\lst{CalcSha256} operation (OpCode 204)}
\label{sec:serialization:operation:CalcSha256}

Calculate Sha256 hash from \lst{input} bytes. See~\hyperref[sec:appendix:primops:CalcSha256]{\lst{sha256}}

\noindent
\(\begin{tabularx}{\textwidth}{| l | l | l | X |}
    \hline
    \bf{Slot} & \bf{Format} & \bf{\#bytes} & \bf{Description} \\
    \hline
         $ input $ & \lst{Expr} & [1, *] & collection of bytes \\
    \hline
      
\end{tabularx}\)
       

\subsubsection{\lst{CreateProveDlog} operation (OpCode 205)}
\label{sec:serialization:operation:CreateProveDlog}

ErgoTree operation to create a new \lst{SigmaProp} value representing public key
 of discrete logarithm signature protocol.
         See~\hyperref[sec:appendix:primops:CreateProveDlog]{\lst{proveDlog}}

\noindent
\(\begin{tabularx}{\textwidth}{| l | l | l | X |}
    \hline
    \bf{Slot} & \bf{Format} & \bf{\#bytes} & \bf{Description} \\
    \hline
         $ value $ & \lst{Expr} & [1, *] & element of elliptic curve group \\
    \hline
      
\end{tabularx}\)
       

\subsubsection{\lst{CreateProveDHTuple} operation (OpCode 206)}
\label{sec:serialization:operation:CreateProveDHTuple}

 ErgoTree operation to create a new SigmaProp value representing public key
 of Diffie Hellman signature protocol.
 Common input: (g,h,u,v)
         See~\hyperref[sec:appendix:primops:CreateProveDHTuple]{\lst{proveDHTuple}}

\noindent
\(\begin{tabularx}{\textwidth}{| l | l | l | X |}
    \hline
    \bf{Slot} & \bf{Format} & \bf{\#bytes} & \bf{Description} \\
    \hline
         $ g $ & \lst{Expr} & [1, *] &  \\
    \hline
           $ h $ & \lst{Expr} & [1, *] &  \\
    \hline
           $ u $ & \lst{Expr} & [1, *] &  \\
    \hline
           $ v $ & \lst{Expr} & [1, *] &  \\
    \hline
      
\end{tabularx}\)
       

\subsubsection{\lst{SigmaPropBytes} operation (OpCode 208)}
\label{sec:serialization:operation:SigmaPropBytes}

Serialized bytes of this sigma proposition taken as ErgoTree. See~\hyperref[sec:type:SigmaProp:propBytes]{\lst{SigmaProp.propBytes}}

\noindent
\(\begin{tabularx}{\textwidth}{| l | l | l | X |}
    \hline
    \bf{Slot} & \bf{Format} & \bf{\#bytes} & \bf{Description} \\
    \hline
         $ this $ & \lst{Expr} & [1, *] & this instance \\
    \hline
      
\end{tabularx}\)
       

\subsubsection{\lst{BoolToSigmaProp} operation (OpCode 209)}
\label{sec:serialization:operation:BoolToSigmaProp}

Embedding of \lst{Boolean} values to \lst{SigmaProp} values.
 As an example, this operation allows boolean experessions
 to be used as arguments of \lst{atLeast(..., sigmaProp(boolExpr), ...)} operation.
 During execution results to either \lst{TrueProp} or \lst{FalseProp} values of \lst{SigmaProp} type.
         See~\hyperref[sec:appendix:primops:BoolToSigmaProp]{\lst{sigmaProp}}

\noindent
\(\begin{tabularx}{\textwidth}{| l | l | l | X |}
    \hline
    \bf{Slot} & \bf{Format} & \bf{\#bytes} & \bf{Description} \\
    \hline
         $ condition $ & \lst{Expr} & [1, *] & boolean value to embed in SigmaProp value \\
    \hline
      
\end{tabularx}\)
       

\subsubsection{\lst{DeserializeContext} operation (OpCode 212)}
\label{sec:serialization:operation:DeserializeContext}

Extracts context variable as \lst{Coll[Byte]}, deserializes it to script
 and then executes this script in the current context.
 The original \lst{Coll[Byte]} of the script is available as \lst{getVar[Coll[Byte]](id)}.
 Type parameter \lst{V} result type of the deserialized script.
 Throws an exception if the actual script type doesn't conform to T.
 Returns a result of the script execution in the current context
         See~\hyperref[sec:appendix:primops:DeserializeContext]{\lst{executeFromVar}}

\noindent
\(\begin{tabularx}{\textwidth}{| l | l | l | X |}
    \hline
    \bf{Slot} & \bf{Format} & \bf{\#bytes} & \bf{Description} \\
    \hline
         $ type $ & \lst{Type} & [1, *] & expected type of the deserialized script \\
    \hline
           $ id $ & \lst{Byte} & 1 & identifier of the context variable \\
    \hline
      
\end{tabularx}\)
       

\subsubsection{\lst{DeserializeRegister} operation (OpCode 213)}
\label{sec:serialization:operation:DeserializeRegister}

Extracts SELF register as \lst{Coll[Byte]}, deserializes it to script
 and then executes this script in the current context.
 The original \lst{Coll[Byte]} of the script is available as \lst{SELF.getReg[Coll[Byte]](id)}.
 Type parameter \lst{T} result type of the deserialized script.
 Throws an exception if the actual script type doesn't conform to \lst{T}.
 Returns a result of the script execution in the current context
         See~\hyperref[sec:appendix:primops:DeserializeRegister]{\lst{executeFromSelfReg}}

\noindent
\(\begin{tabularx}{\textwidth}{| l | l | l | X |}
    \hline
    \bf{Slot} & \bf{Format} & \bf{\#bytes} & \bf{Description} \\
    \hline
         $ id $ & \lst{Byte} & 1 & identifier of the register \\
    \hline
           $ type $ & \lst{Type} & [1, *] & expected type of the deserialized script \\
    \hline
          \multicolumn{4}{l}{\lst{optional}~$default$} \\
    \hline
    ~~$tag$ & \lst{Byte} & 1 & 0 - no value; 1 - has value \\
    \hline
    \multicolumn{4}{l}{~~\lst{when}~$tag == 1$} \\
    \hline
             ~~~~ $ default $ & \lst{Expr} & [1, *] & optional default value, if register is not available \\
    \hline
          \multicolumn{4}{l}{\lst{end optional}} \\
\end{tabularx}\)
       

\subsubsection{\lst{ValDef} operation (OpCode 214)}
\label{sec:serialization:operation:ValDef}

 

\noindent
\(\begin{tabularx}{\textwidth}{| l | l | l | X |}
    \hline
    \bf{Slot} & \bf{Format} & \bf{\#bytes} & \bf{Description} \\
    \hline
    % skipped OptionalScope(type arguments, ArrayBuffer())

\end{tabularx}\)
       

\subsubsection{\lst{FunDef} operation (OpCode 215)}
\label{sec:serialization:operation:FunDef}

 

\noindent
\(\begin{tabularx}{\textwidth}{| l | l | l | X |}
    \hline
    \bf{Slot} & \bf{Format} & \bf{\#bytes} & \bf{Description} \\
    \hline
    % skipped OptionalScope(type arguments, ArrayBuffer())

\end{tabularx}\)
       

\subsubsection{\lst{BlockValue} operation (OpCode 216)}
\label{sec:serialization:operation:BlockValue}

 

\noindent
\(\begin{tabularx}{\textwidth}{| l | l | l | X |}
    \hline
    \bf{Slot} & \bf{Format} & \bf{\#bytes} & \bf{Description} \\
    \hline
         $ numItems $ & \lst{VLQ(UInt)} & [1, *] & number of block items \\
    \hline
          \multicolumn{4}{l}{\lst{for}~$i=1$~\lst{to}~$numItems$} \\
    \hline
             ~~ $ item_i $ & \lst{ValDef} & [1, *] & block's definition in i-th position \\
    \hline
          \multicolumn{4}{l}{\lst{end for}} \\\hline
     $ result $ & \lst{Expr} & [1, *] & result expression of the block \\
    \hline
      
\end{tabularx}\)
       

\subsubsection{\lst{FuncValue} operation (OpCode 217)}
\label{sec:serialization:operation:FuncValue}

 

\noindent
\(\begin{tabularx}{\textwidth}{| l | l | l | X |}
    \hline
    \bf{Slot} & \bf{Format} & \bf{\#bytes} & \bf{Description} \\
    \hline
         $ numArgs $ & \lst{VLQ(UInt)} & [1, *] & number of function arguments \\
    \hline
          \multicolumn{4}{l}{\lst{for}~$i=1$~\lst{to}~$numArgs$} \\
    \hline
             ~~ $ id_i $ & \lst{VLQ(UInt)} & [1, *] & identifier of the i-th argument \\
    \hline
          ~~ $ type_i $ & \lst{Type} & [1, *] & type of the i-th argument \\
    \hline
          \multicolumn{4}{l}{\lst{end for}} \\\hline
     $ body $ & \lst{Expr} & [1, *] & function body, which is parameterized by arguments \\
    \hline
      
\end{tabularx}\)
       

\subsubsection{\lst{Apply} operation (OpCode 218)}
\label{sec:serialization:operation:Apply}

Apply the function to the arguments.  See~\hyperref[sec:appendix:primops:Apply]{\lst{apply}}

\noindent
\(\begin{tabularx}{\textwidth}{| l | l | l | X |}
    \hline
    \bf{Slot} & \bf{Format} & \bf{\#bytes} & \bf{Description} \\
    \hline
         $ func $ & \lst{Expr} & [1, *] & function which is applied \\
    \hline
           $ \#items $ & \lst{VLQ(UInt)} & [1, *] & number of items in the collection \\
    \hline
          \multicolumn{4}{l}{\lst{for}~$i=1$~\lst{to}~$\#items$} \\
    \hline
             ~~ $ args_i $ & \lst{Expr} & [1, *] & i-th item in the list of arguments \\
    \hline
          \multicolumn{4}{l}{\lst{end for}} \\
\end{tabularx}\)
       

\subsubsection{\lst{PropertyCall} operation (OpCode 219)}
\label{sec:serialization:operation:PropertyCall}

 

\noindent
\(\begin{tabularx}{\textwidth}{| l | l | l | X |}
    \hline
    \bf{Slot} & \bf{Format} & \bf{\#bytes} & \bf{Description} \\
    \hline
         $ typeCode $ & \lst{Byte} & 1 & type of the method (see Table~\ref{table:predeftypes}) \\
    \hline
           $ methodCode $ & \lst{Byte} & 1 & a code of the property \\
    \hline
           $ obj $ & \lst{Expr} & [1, *] & receiver object of this property call \\
    \hline
      
\end{tabularx}\)
       

\subsubsection{\lst{MethodCall} operation (OpCode 220)}
\label{sec:serialization:operation:MethodCall}

 

\noindent
\(\begin{tabularx}{\textwidth}{| l | l | l | X |}
    \hline
    \bf{Slot} & \bf{Format} & \bf{\#bytes} & \bf{Description} \\
    \hline
         $ typeCode $ & \lst{Byte} & 1 & type of the method (see Table~\ref{table:predeftypes}) \\
    \hline
           $ methodCode $ & \lst{Byte} & 1 & a code of the method \\
    \hline
           $ obj $ & \lst{Expr} & [1, *] & receiver object of this method call \\
    \hline
           $ \#items $ & \lst{VLQ(UInt)} & [1, *] & number of items in the collection \\
    \hline
          \multicolumn{4}{l}{\lst{for}~$i=1$~\lst{to}~$\#items$} \\
    \hline
             ~~ $ args_i $ & \lst{Expr} & [1, *] & i-th item in the arguments of the method call \\
    \hline
          \multicolumn{4}{l}{\lst{end for}} \\
\end{tabularx}\)
       

\subsubsection{\lst{GetVar} operation (OpCode 227)}
\label{sec:serialization:operation:GetVar}

Get context variable with given \lst{varId} and type. See~\hyperref[sec:type:Context:getVar]{\lst{Context.getVar}}

\noindent
\(\begin{tabularx}{\textwidth}{| l | l | l | X |}
    \hline
    \bf{Slot} & \bf{Format} & \bf{\#bytes} & \bf{Description} \\
    \hline
         $ varId $ & \lst{Byte} & 1 & \lst{Byte} identifier of context variable \\
    \hline
           $ type $ & \lst{Type} & [1, *] & expected type of context variable \\
    \hline
      
\end{tabularx}\)
       

\subsubsection{\lst{OptionGet} operation (OpCode 228)}
\label{sec:serialization:operation:OptionGet}

Returns the option's value. The option must be nonempty. Throws exception if the option is empty. See~\hyperref[sec:type:SOption:get]{\lst{SOption.get}}

\noindent
\(\begin{tabularx}{\textwidth}{| l | l | l | X |}
    \hline
    \bf{Slot} & \bf{Format} & \bf{\#bytes} & \bf{Description} \\
    \hline
         $ this $ & \lst{Expr} & [1, *] & this instance \\
    \hline
      
\end{tabularx}\)
       

\subsubsection{\lst{OptionGetOrElse} operation (OpCode 229)}
\label{sec:serialization:operation:OptionGetOrElse}

Returns the option's value if the option is nonempty, otherwise
returns \lst{default}.
         See~\hyperref[sec:type:SOption:getOrElse]{\lst{SOption.getOrElse}}

\noindent
\(\begin{tabularx}{\textwidth}{| l | l | l | X |}
    \hline
    \bf{Slot} & \bf{Format} & \bf{\#bytes} & \bf{Description} \\
    \hline
         $ this $ & \lst{Expr} & [1, *] & this instance \\
    \hline
           $ default $ & \lst{Expr} & [1, *] & the default value \\
    \hline
      
\end{tabularx}\)
       

\subsubsection{\lst{OptionIsDefined} operation (OpCode 230)}
\label{sec:serialization:operation:OptionIsDefined}

Returns \lst{true} if the option is an instance of \lst{Some}, \lst{false} otherwise. See~\hyperref[sec:type:SOption:isDefined]{\lst{SOption.isDefined}}

\noindent
\(\begin{tabularx}{\textwidth}{| l | l | l | X |}
    \hline
    \bf{Slot} & \bf{Format} & \bf{\#bytes} & \bf{Description} \\
    \hline
         $ this $ & \lst{Expr} & [1, *] & this instance \\
    \hline
      
\end{tabularx}\)
       

\subsubsection{\lst{SigmaAnd} operation (OpCode 234)}
\label{sec:serialization:operation:SigmaAnd}

Returns sigma proposition which is proven when \emph{all} the elements in collection are proven. See~\hyperref[sec:appendix:primops:SigmaAnd]{\lst{allZK}}

\noindent
\(\begin{tabularx}{\textwidth}{| l | l | l | X |}
    \hline
    \bf{Slot} & \bf{Format} & \bf{\#bytes} & \bf{Description} \\
    \hline
         $ \#items $ & \lst{VLQ(UInt)} & [1, *] & number of items in the collection \\
    \hline
          \multicolumn{4}{l}{\lst{for}~$i=1$~\lst{to}~$\#items$} \\
    \hline
             ~~ $ propositions_i $ & \lst{Expr} & [1, *] & i-th item in the a collection of propositions \\
    \hline
          \multicolumn{4}{l}{\lst{end for}} \\
\end{tabularx}\)
       

\subsubsection{\lst{SigmaOr} operation (OpCode 235)}
\label{sec:serialization:operation:SigmaOr}

Returns sigma proposition which is proven when \emph{any} of the elements in collection is proven. See~\hyperref[sec:appendix:primops:SigmaOr]{\lst{anyZK}}

\noindent
\(\begin{tabularx}{\textwidth}{| l | l | l | X |}
    \hline
    \bf{Slot} & \bf{Format} & \bf{\#bytes} & \bf{Description} \\
    \hline
         $ \#items $ & \lst{VLQ(UInt)} & [1, *] & number of items in the collection \\
    \hline
          \multicolumn{4}{l}{\lst{for}~$i=1$~\lst{to}~$\#items$} \\
    \hline
             ~~ $ propositions_i $ & \lst{Expr} & [1, *] & i-th item in the a collection of propositions \\
    \hline
          \multicolumn{4}{l}{\lst{end for}} \\
\end{tabularx}\)
       

\subsubsection{\lst{BinOr} operation (OpCode 236)}
\label{sec:serialization:operation:BinOr}

Logical OR of two operands See~\hyperref[sec:appendix:primops:BinOr]{\lst{||}}

\noindent
\(\begin{tabularx}{\textwidth}{| l | l | l | X |}
    \hline
    \bf{Slot} & \bf{Format} & \bf{\#bytes} & \bf{Description} \\
    \hline
        \multicolumn{4}{l}{\lst{match}~$ (left, right) $} \\
         
    \multicolumn{4}{l}{~~\lst{with}~$(Constant(l, Boolean), Constant(r, Boolean))$ } \\
    \hline
            ~~~~ $ opCode $ & \lst{Byte} & 1 & always contains OpCode 133 \\
    \hline
          ~~~~ $ (l,r) $ & \lst{Bits} & 1 & two higher bits in a byte \\
    \hline
      
    \multicolumn{4}{l}{~~\lst{otherwise} } \\
    \hline
            ~~~~ $ left $ & \lst{Expr} & [1, *] & left operand \\
    \hline
          ~~~~ $ right $ & \lst{Expr} & [1, *] & right operand \\
    \hline
          \multicolumn{4}{l}{\lst{end match}} \\
\end{tabularx}\)
       

\subsubsection{\lst{BinAnd} operation (OpCode 237)}
\label{sec:serialization:operation:BinAnd}

Logical AND of two operands See~\hyperref[sec:appendix:primops:BinAnd]{\lst{&&}}

\noindent
\(\begin{tabularx}{\textwidth}{| l | l | l | X |}
    \hline
    \bf{Slot} & \bf{Format} & \bf{\#bytes} & \bf{Description} \\
    \hline
        \multicolumn{4}{l}{\lst{match}~$ (left, right) $} \\
         
    \multicolumn{4}{l}{~~\lst{otherwise} } \\
    \hline
            ~~~~ $ left $ & \lst{Expr} & [1, *] & left operand \\
    \hline
          ~~~~ $ right $ & \lst{Expr} & [1, *] & right operand \\
    \hline
          \multicolumn{4}{l}{\lst{end match}} \\
\end{tabularx}\)
       

\subsubsection{\lst{DecodePoint} operation (OpCode 238)}
\label{sec:serialization:operation:DecodePoint}

Convert \lst{Coll[Byte]} to \lst{GroupElement} using \lst{GroupElementSerializer} See~\hyperref[sec:appendix:primops:DecodePoint]{\lst{decodePoint}}

\noindent
\(\begin{tabularx}{\textwidth}{| l | l | l | X |}
    \hline
    \bf{Slot} & \bf{Format} & \bf{\#bytes} & \bf{Description} \\
    \hline
         $ input $ & \lst{Expr} & [1, *] & serialized bytes of some \lst{GroupElement} value \\
    \hline
      
\end{tabularx}\)
       

\subsubsection{\lst{LogicalNot} operation (OpCode 239)}
\label{sec:serialization:operation:LogicalNot}

Logical NOT operation. Returns \lst{true} if input is \lst{false} and \lst{false} if input is \lst{true}. See~\hyperref[sec:appendix:primops:LogicalNot]{\lst{unary_!}}

\noindent
\(\begin{tabularx}{\textwidth}{| l | l | l | X |}
    \hline
    \bf{Slot} & \bf{Format} & \bf{\#bytes} & \bf{Description} \\
    \hline
         $ input $ & \lst{Expr} & [1, *] & input \lst{Boolean} value \\
    \hline
      
\end{tabularx}\)
       

\subsubsection{\lst{Negation} operation (OpCode 240)}
\label{sec:serialization:operation:Negation}

Negates numeric value \lst{x} by returning \lst{-x}. See~\hyperref[sec:appendix:primops:Negation]{\lst{unary_-}}

\noindent
\(\begin{tabularx}{\textwidth}{| l | l | l | X |}
    \hline
    \bf{Slot} & \bf{Format} & \bf{\#bytes} & \bf{Description} \\
    \hline
         $ input $ & \lst{Expr} & [1, *] & value of numeric type \\
    \hline
      
\end{tabularx}\)
       

\subsubsection{\lst{BinXor} operation (OpCode 244)}
\label{sec:serialization:operation:BinXor}

Logical XOR of two operands See~\hyperref[sec:appendix:primops:BinXor]{\lst{^}}

\noindent
\(\begin{tabularx}{\textwidth}{| l | l | l | X |}
    \hline
    \bf{Slot} & \bf{Format} & \bf{\#bytes} & \bf{Description} \\
    \hline
        \multicolumn{4}{l}{\lst{match}~$ (left, right) $} \\
         
    \multicolumn{4}{l}{~~\lst{otherwise} } \\
    \hline
            ~~~~ $ left $ & \lst{Expr} & [1, *] & left operand \\
    \hline
          ~~~~ $ right $ & \lst{Expr} & [1, *] & right operand \\
    \hline
          \multicolumn{4}{l}{\lst{end match}} \\
\end{tabularx}\)
       

\subsubsection{\lst{XorOf} operation (OpCode 255)}
\label{sec:serialization:operation:XorOf}

Similar to \lst{allOf}, but performing logical XOR operation between all conditions instead of \lst{&&} See~\hyperref[sec:appendix:primops:XorOf]{\lst{xorOf}}

\noindent
\(\begin{tabularx}{\textwidth}{| l | l | l | X |}
    \hline
    \bf{Slot} & \bf{Format} & \bf{\#bytes} & \bf{Description} \\
    \hline
         $ conditions $ & \lst{Expr} & [1, *] & a collection of conditions \\
    \hline
      
\end{tabularx}\)
       

\subsubsection{\lst{SubstConstants} operation (OpCode 116)}
\label{sec:serialization:operation:SubstConstants}

Transforms serialized bytes of ErgoTree with segregated constants by replacing constants
 at given positions with new values. This operation allow to use serialized scripts as
 pre-defined templates.
 The typical usage is "check that output box have proposition equal to given script bytes,
 where minerPk (constants(0)) is replaced with currentMinerPk".
 Each constant in original scriptBytes have SType serialized before actual data (see ConstantSerializer).
 During substitution each value from newValues is checked to be an instance of the corresponding type.
 This means, the constants during substitution cannot change their types.

 Returns original scriptBytes array where only specified constants are replaced and all other bytes remain exactly the same.
         See~\hyperref[sec:appendix:primops:SubstConstants]{\lst{substConstants}}

\noindent
\(\begin{tabularx}{\textwidth}{| l | l | l | X |}
    \hline
    \bf{Slot} & \bf{Format} & \bf{\#bytes} & \bf{Description} \\
    \hline
         $ scriptBytes $ & \lst{Expr} & [1, *] & serialized ErgoTree with ConstantSegregationFlag set to 1. \\
    \hline
           $ positions $ & \lst{Expr} & [1, *] & 0-based indexes in ErgoTree.constants \\
    \hline
           $ newValues $ & \lst{Expr} & [1, *] & values to be put into the corresponding positions \\
    \hline
      
\end{tabularx}\)
       

\subsubsection{\lst{LongToByteArray} operation (OpCode 122)}
\label{sec:serialization:operation:LongToByteArray}

Converts \lst{Long} value to big-endian bytes representation. See~\hyperref[sec:appendix:primops:LongToByteArray]{\lst{longToByteArray}}

\noindent
\(\begin{tabularx}{\textwidth}{| l | l | l | X |}
    \hline
    \bf{Slot} & \bf{Format} & \bf{\#bytes} & \bf{Description} \\
    \hline
         $ input $ & \lst{Expr} & [1, *] & value to convert \\
    \hline
      
\end{tabularx}\)
       

\subsubsection{\lst{ByteArrayToBigInt} operation (OpCode 123)}
\label{sec:serialization:operation:ByteArrayToBigInt}

Convert big-endian bytes representation (Coll[Byte]) to BigInt value. See~\hyperref[sec:appendix:primops:ByteArrayToBigInt]{\lst{byteArrayToBigInt}}

\noindent
\(\begin{tabularx}{\textwidth}{| l | l | l | X |}
    \hline
    \bf{Slot} & \bf{Format} & \bf{\#bytes} & \bf{Description} \\
    \hline
         $ input $ & \lst{Expr} & [1, *] & collection of bytes in big-endian format \\
    \hline
      
\end{tabularx}\)
       

\subsubsection{\lst{ByteArrayToLong} operation (OpCode 124)}
\label{sec:serialization:operation:ByteArrayToLong}

Convert big-endian bytes representation (Coll[Byte]) to Long value. See~\hyperref[sec:appendix:primops:ByteArrayToLong]{\lst{byteArrayToLong}}

\noindent
\(\begin{tabularx}{\textwidth}{| l | l | l | X |}
    \hline
    \bf{Slot} & \bf{Format} & \bf{\#bytes} & \bf{Description} \\
    \hline
         $ input $ & \lst{Expr} & [1, *] & collection of bytes in big-endian format \\
    \hline
      
\end{tabularx}\)
       

\subsubsection{\lst{Downcast} operation (OpCode 125)}
\label{sec:serialization:operation:Downcast}

Cast this numeric value to a smaller type (e.g. Long to Int). Throws exception if overflow. See~\hyperref[sec:appendix:primops:Downcast]{\lst{downcast}}

\noindent
\(\begin{tabularx}{\textwidth}{| l | l | l | X |}
    \hline
    \bf{Slot} & \bf{Format} & \bf{\#bytes} & \bf{Description} \\
    \hline
         $ input $ & \lst{Expr} & [1, *] & value to cast \\
    \hline
           $ type $ & \lst{Type} & [1, *] & resulting type of the cast operation \\
    \hline
      
\end{tabularx}\)
       

\subsubsection{\lst{Upcast} operation (OpCode 126)}
\label{sec:serialization:operation:Upcast}

Cast this numeric value to a bigger type (e.g. Int to Long) See~\hyperref[sec:appendix:primops:Upcast]{\lst{upcast}}

\noindent
\(\begin{tabularx}{\textwidth}{| l | l | l | X |}
    \hline
    \bf{Slot} & \bf{Format} & \bf{\#bytes} & \bf{Description} \\
    \hline
         $ input $ & \lst{Expr} & [1, *] & value to cast \\
    \hline
           $ type $ & \lst{Type} & [1, *] & resulting type of the cast operation \\
    \hline
      
\end{tabularx}\)
       