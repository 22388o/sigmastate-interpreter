
\subsubsection{\lst{placeholder} method (Code 115)}
\label{sec:appendix:primops:ConstantPlaceholder}
\noindent
\begin{tabularx}{\textwidth}{| l | X |}
   \hline
   \bf{Description} & Create special ErgoTree node which can be replaced by constant with given id. \\
  
  \hline
  \bf{Parameters} &
      \(\begin{array}{l l l}
         \lst{index} & \lst{: Int} & \text{// index of the constant in ErgoTree header} \\
      \end{array}\) \\
       
  \hline
  \bf{Result} & \lst{T} \\
  \hline
  
  \bf{Serialized as} & \hyperref[sec:serialization:operation:ConstantPlaceholder]{\lst{ConstantPlaceholder}} \\
  \hline
       
\end{tabularx}

\subsubsection{\lst{substConstants} method (Code 116)}
\label{sec:appendix:primops:SubstConstants}
\noindent
\begin{tabularx}{\textwidth}{| l | X |}
   \hline
   \bf{Description} & Transforms serialized bytes of ErgoTree with segregated constants by replacing constants
 at given positions with new values. This operation allow to use serialized scripts as
 pre-defined templates.
 The typical usage is "check that output box have proposition equal to given script bytes,
 where minerPk (constants(0)) is replaced with currentMinerPk".
 Each constant in original scriptBytes have SType serialized before actual data (see ConstantSerializer).
 During substitution each value from newValues is checked to be an instance of the corresponding type.
 This means, the constants during substitution cannot change their types.

 Returns original scriptBytes array where only specified constants are replaced and all other bytes remain exactly the same.
         \\
  
  \hline
  \bf{Parameters} &
      \(\begin{array}{l l l}
         \lst{scriptBytes} & \lst{: Coll[Byte]} & \text{// serialized ErgoTree with ConstantSegregationFlag set to 1.} \\
\lst{positions} & \lst{: Coll[Int]} & \text{// zero based indexes in ErgoTree.constants array which should be replaced with new values} \\
\lst{newValues} & \lst{: Coll[T]} & \text{// new values to be injected into the corresponding positions in ErgoTree.constants array} \\
      \end{array}\) \\
       
  \hline
  \bf{Result} & \lst{Coll[Byte]} \\
  \hline
  
  \bf{Serialized as} & \hyperref[sec:serialization:operation:SubstConstants]{\lst{SubstConstants}} \\
  \hline
       
\end{tabularx}

\subsubsection{\lst{longToByteArray} method (Code 122)}
\label{sec:appendix:primops:LongToByteArray}
\noindent
\begin{tabularx}{\textwidth}{| l | X |}
   \hline
   \bf{Description} & Converts \lst{Long} value to big-endian bytes representation. \\
  
  \hline
  \bf{Parameters} &
      \(\begin{array}{l l l}
         \lst{input} & \lst{: Long} & \text{// value to convert} \\
      \end{array}\) \\
       
  \hline
  \bf{Result} & \lst{Coll[Byte]} \\
  \hline
  
  \bf{Serialized as} & \hyperref[sec:serialization:operation:LongToByteArray]{\lst{LongToByteArray}} \\
  \hline
       
\end{tabularx}

\subsubsection{\lst{byteArrayToBigInt} method (Code 123)}
\label{sec:appendix:primops:ByteArrayToBigInt}
\noindent
\begin{tabularx}{\textwidth}{| l | X |}
   \hline
   \bf{Description} & Convert big-endian bytes representation (Coll[Byte]) to BigInt value. \\
  
  \hline
  \bf{Parameters} &
      \(\begin{array}{l l l}
         \lst{input} & \lst{: Coll[Byte]} & \text{// collection of bytes in big-endian format} \\
      \end{array}\) \\
       
  \hline
  \bf{Result} & \lst{BigInt} \\
  \hline
  
  \bf{Serialized as} & \hyperref[sec:serialization:operation:ByteArrayToBigInt]{\lst{ByteArrayToBigInt}} \\
  \hline
       
\end{tabularx}

\subsubsection{\lst{byteArrayToLong} method (Code 124)}
\label{sec:appendix:primops:ByteArrayToLong}
\noindent
\begin{tabularx}{\textwidth}{| l | X |}
   \hline
   \bf{Description} & Convert big-endian bytes representation (Coll[Byte]) to Long value. \\
  
  \hline
  \bf{Parameters} &
      \(\begin{array}{l l l}
         \lst{input} & \lst{: Coll[Byte]} & \text{// collection of bytes in big-endian format} \\
      \end{array}\) \\
       
  \hline
  \bf{Result} & \lst{Long} \\
  \hline
  
  \bf{Serialized as} & \hyperref[sec:serialization:operation:ByteArrayToLong]{\lst{ByteArrayToLong}} \\
  \hline
       
\end{tabularx}

\subsubsection{\lst{downcast} method (Code 125)}
\label{sec:appendix:primops:Downcast}
\noindent
\begin{tabularx}{\textwidth}{| l | X |}
   \hline
   \bf{Description} & Cast this numeric value to a smaller type (e.g. Long to Int). Throws exception if overflow. \\
  
  \hline
  \bf{Parameters} &
      \(\begin{array}{l l l}
         \lst{input} & \lst{: T} & \text{// value to cast} \\
      \end{array}\) \\
       
  \hline
  \bf{Result} & \lst{R} \\
  \hline
  
  \bf{Serialized as} & \hyperref[sec:serialization:operation:Downcast]{\lst{Downcast}} \\
  \hline
       
\end{tabularx}

\subsubsection{\lst{upcast} method (Code 126)}
\label{sec:appendix:primops:Upcast}
\noindent
\begin{tabularx}{\textwidth}{| l | X |}
   \hline
   \bf{Description} & Cast this numeric value to a bigger type (e.g. Int to Long) \\
  
  \hline
  \bf{Parameters} &
      \(\begin{array}{l l l}
         \lst{input} & \lst{: T} & \text{// value to cast} \\
      \end{array}\) \\
       
  \hline
  \bf{Result} & \lst{R} \\
  \hline
  
  \bf{Serialized as} & \hyperref[sec:serialization:operation:Upcast]{\lst{Upcast}} \\
  \hline
       
\end{tabularx}

\subsubsection{\lst{selectField} method (Code 140)}
\label{sec:appendix:primops:SelectField}
\noindent
\begin{tabularx}{\textwidth}{| l | X |}
   \hline
   \bf{Description} & Select tuple field by its 1-based index. E.g. \lst{input._1} is transformed to \lst{SelectField(input, 1)} \\
  
  \hline
  \bf{Parameters} &
      \(\begin{array}{l l l}
         \lst{input} & \lst{: T} & \text{// tuple of items} \\
\lst{fieldIndex} & \lst{: Byte} & \text{// index of an item to select} \\
      \end{array}\) \\
       
  \hline
  \bf{Result} & \lst{R} \\
  \hline
  
  \bf{Serialized as} & \hyperref[sec:serialization:operation:SelectField]{\lst{SelectField}} \\
  \hline
       
\end{tabularx}

\subsubsection{\lst{<} method (Code 143)}
\label{sec:appendix:primops:LT}
\noindent
\begin{tabularx}{\textwidth}{| l | X |}
   \hline
   \bf{Description} & Returns \lst{true} is the left operand is less then the right operand, \lst{false} otherwise. \\
  
  \hline
  \bf{Parameters} &
      \(\begin{array}{l l l}
         \lst{left} & \lst{: T} & \text{// left operand} \\
\lst{right} & \lst{: T} & \text{// right operand} \\
      \end{array}\) \\
       
  \hline
  \bf{Result} & \lst{Boolean} \\
  \hline
  
  \bf{Serialized as} & \hyperref[sec:serialization:operation:LT]{\lst{LT}} \\
  \hline
       
\end{tabularx}

\subsubsection{\lst{<=} method (Code 144)}
\label{sec:appendix:primops:LE}
\noindent
\begin{tabularx}{\textwidth}{| l | X |}
   \hline
   \bf{Description} & Returns \lst{true} is the left operand is less then or equal to the right operand, \lst{false} otherwise. \\
  
  \hline
  \bf{Parameters} &
      \(\begin{array}{l l l}
         \lst{left} & \lst{: T} & \text{// left operand} \\
\lst{right} & \lst{: T} & \text{// right operand} \\
      \end{array}\) \\
       
  \hline
  \bf{Result} & \lst{Boolean} \\
  \hline
  
  \bf{Serialized as} & \hyperref[sec:serialization:operation:LE]{\lst{LE}} \\
  \hline
       
\end{tabularx}

\subsubsection{\lst{>} method (Code 145)}
\label{sec:appendix:primops:GT}
\noindent
\begin{tabularx}{\textwidth}{| l | X |}
   \hline
   \bf{Description} & Returns \lst{true} is the left operand is greater then the right operand, \lst{false} otherwise. \\
  
  \hline
  \bf{Parameters} &
      \(\begin{array}{l l l}
         \lst{left} & \lst{: T} & \text{// left operand} \\
\lst{right} & \lst{: T} & \text{// right operand} \\
      \end{array}\) \\
       
  \hline
  \bf{Result} & \lst{Boolean} \\
  \hline
  
  \bf{Serialized as} & \hyperref[sec:serialization:operation:GT]{\lst{GT}} \\
  \hline
       
\end{tabularx}

\subsubsection{\lst{>=} method (Code 146)}
\label{sec:appendix:primops:GE}
\noindent
\begin{tabularx}{\textwidth}{| l | X |}
   \hline
   \bf{Description} & Returns \lst{true} is the left operand is greater then or equal to the right operand, \lst{false} otherwise. \\
  
  \hline
  \bf{Parameters} &
      \(\begin{array}{l l l}
         \lst{left} & \lst{: T} & \text{// left operand} \\
\lst{right} & \lst{: T} & \text{// right operand} \\
      \end{array}\) \\
       
  \hline
  \bf{Result} & \lst{Boolean} \\
  \hline
  
  \bf{Serialized as} & \hyperref[sec:serialization:operation:GE]{\lst{GE}} \\
  \hline
       
\end{tabularx}

\subsubsection{\lst{==} method (Code 147)}
\label{sec:appendix:primops:EQ}
\noindent
\begin{tabularx}{\textwidth}{| l | X |}
   \hline
   \bf{Description} & Compare equality of \lst{left} and \lst{right} arguments \\
  
  \hline
  \bf{Parameters} &
      \(\begin{array}{l l l}
         \lst{left} & \lst{: T} & \text{// left operand} \\
\lst{right} & \lst{: T} & \text{// right operand} \\
      \end{array}\) \\
       
  \hline
  \bf{Result} & \lst{Boolean} \\
  \hline
  
  \bf{Serialized as} & \hyperref[sec:serialization:operation:EQ]{\lst{EQ}} \\
  \hline
       
\end{tabularx}

\subsubsection{\lst{!=} method (Code 148)}
\label{sec:appendix:primops:NEQ}
\noindent
\begin{tabularx}{\textwidth}{| l | X |}
   \hline
   \bf{Description} & Compare inequality of \lst{left} and \lst{right} arguments \\
  
  \hline
  \bf{Parameters} &
      \(\begin{array}{l l l}
         \lst{left} & \lst{: T} & \text{// left operand} \\
\lst{right} & \lst{: T} & \text{// right operand} \\
      \end{array}\) \\
       
  \hline
  \bf{Result} & \lst{Boolean} \\
  \hline
  
  \bf{Serialized as} & \hyperref[sec:serialization:operation:NEQ]{\lst{NEQ}} \\
  \hline
       
\end{tabularx}

\subsubsection{\lst{if} method (Code 149)}
\label{sec:appendix:primops:If}
\noindent
\begin{tabularx}{\textwidth}{| l | X |}
   \hline
   \bf{Description} & Compute condition, if true then compute trueBranch else compute falseBranch \\
  
  \hline
  \bf{Parameters} &
      \(\begin{array}{l l l}
         \lst{condition} & \lst{: Boolean} & \text{// condition expression} \\
\lst{trueBranch} & \lst{: T} & \text{// expression to execute when \lst{condition == true}} \\
\lst{falseBranch} & \lst{: T} & \text{// expression to execute when \lst{condition == false}} \\
      \end{array}\) \\
       
  \hline
  \bf{Result} & \lst{T} \\
  \hline
  
  \bf{Serialized as} & \hyperref[sec:serialization:operation:If]{\lst{If}} \\
  \hline
       
\end{tabularx}

\subsubsection{\lst{allOf} method (Code 150)}
\label{sec:appendix:primops:AND}
\noindent
\begin{tabularx}{\textwidth}{| l | X |}
   \hline
   \bf{Description} & Returns true if \emph{all} the elements in collection are \lst{true}. \\
  
  \hline
  \bf{Parameters} &
      \(\begin{array}{l l l}
         \lst{conditions} & \lst{: Coll[Boolean]} & \text{// a collection of conditions} \\
      \end{array}\) \\
       
  \hline
  \bf{Result} & \lst{Boolean} \\
  \hline
  
  \bf{Serialized as} & \hyperref[sec:serialization:operation:AND]{\lst{AND}} \\
  \hline
       
\end{tabularx}

\subsubsection{\lst{anyOf} method (Code 151)}
\label{sec:appendix:primops:OR}
\noindent
\begin{tabularx}{\textwidth}{| l | X |}
   \hline
   \bf{Description} & Returns true if \emph{any} the elements in collection are \lst{true}. \\
  
  \hline
  \bf{Parameters} &
      \(\begin{array}{l l l}
         \lst{conditions} & \lst{: Coll[Boolean]} & \text{// a collection of conditions} \\
      \end{array}\) \\
       
  \hline
  \bf{Result} & \lst{Boolean} \\
  \hline
  
  \bf{Serialized as} & \hyperref[sec:serialization:operation:OR]{\lst{OR}} \\
  \hline
       
\end{tabularx}

\subsubsection{\lst{atLeast} method (Code 152)}
\label{sec:appendix:primops:AtLeast}
\noindent
\begin{tabularx}{\textwidth}{| l | X |}
   \hline
   \bf{Description} &  Logical threshold.
 AtLeast has two inputs: integer \lst{bound} and \lst{children} same as in AND/OR.
 The result is true if at least \lst{bound} children are proven.
         \\
  
  \hline
  \bf{Parameters} &
      \(\begin{array}{l l l}
         \lst{bound} & \lst{: Int} & \text{// required minimum of proven children} \\
\lst{children} & \lst{: Coll[SigmaProp]} & \text{// proposition to be proven/validated} \\
      \end{array}\) \\
       
  \hline
  \bf{Result} & \lst{SigmaProp} \\
  \hline
  
  \bf{Serialized as} & \hyperref[sec:serialization:operation:AtLeast]{\lst{AtLeast}} \\
  \hline
       
\end{tabularx}

\subsubsection{\lst{-} method (Code 153)}
\label{sec:appendix:primops:Minus}
\noindent
\begin{tabularx}{\textwidth}{| l | X |}
   \hline
   \bf{Description} & Returns a result of subtracting second numeric operand from the first. \\
  
  \hline
  \bf{Parameters} &
      \(\begin{array}{l l l}
         \lst{left} & \lst{: T} & \text{// left operand} \\
\lst{right} & \lst{: T} & \text{// right operand} \\
      \end{array}\) \\
       
  \hline
  \bf{Result} & \lst{T} \\
  \hline
  
  \bf{Serialized as} & \hyperref[sec:serialization:operation:Minus]{\lst{Minus}} \\
  \hline
       
\end{tabularx}

\subsubsection{\lst{+} method (Code 154)}
\label{sec:appendix:primops:Plus}
\noindent
\begin{tabularx}{\textwidth}{| l | X |}
   \hline
   \bf{Description} & Returns a sum of two numeric operands \\
  
  \hline
  \bf{Parameters} &
      \(\begin{array}{l l l}
         \lst{left} & \lst{: T} & \text{// left operand} \\
\lst{right} & \lst{: T} & \text{// right operand} \\
      \end{array}\) \\
       
  \hline
  \bf{Result} & \lst{T} \\
  \hline
  
  \bf{Serialized as} & \hyperref[sec:serialization:operation:Plus]{\lst{Plus}} \\
  \hline
       
\end{tabularx}

\subsubsection{\lst{binary_|} method (Code 155)}
\label{sec:appendix:primops:Xor}
\noindent
\begin{tabularx}{\textwidth}{| l | X |}
   \hline
   \bf{Description} & Byte-wise XOR of two collections of bytes \\
  
  \hline
  \bf{Parameters} &
      \(\begin{array}{l l l}
         \lst{left} & \lst{: Coll[Byte]} & \text{// left operand} \\
\lst{right} & \lst{: Coll[Byte]} & \text{// right operand} \\
      \end{array}\) \\
       
  \hline
  \bf{Result} & \lst{Coll[Byte]} \\
  \hline
  
  \bf{Serialized as} & \hyperref[sec:serialization:operation:Xor]{\lst{Xor}} \\
  \hline
       
\end{tabularx}

\subsubsection{\lst{*} method (Code 156)}
\label{sec:appendix:primops:Multiply}
\noindent
\begin{tabularx}{\textwidth}{| l | X |}
   \hline
   \bf{Description} & Returns a multiplication of two numeric operands \\
  
  \hline
  \bf{Parameters} &
      \(\begin{array}{l l l}
         \lst{left} & \lst{: T} & \text{// left operand} \\
\lst{right} & \lst{: T} & \text{// right operand} \\
      \end{array}\) \\
       
  \hline
  \bf{Result} & \lst{T} \\
  \hline
  
  \bf{Serialized as} & \hyperref[sec:serialization:operation:Multiply]{\lst{Multiply}} \\
  \hline
       
\end{tabularx}

\subsubsection{\lst{/} method (Code 157)}
\label{sec:appendix:primops:Division}
\noindent
\begin{tabularx}{\textwidth}{| l | X |}
   \hline
   \bf{Description} & Integer division of the first operand by the second operand. \\
  
  \hline
  \bf{Parameters} &
      \(\begin{array}{l l l}
         \lst{left} & \lst{: T} & \text{// left operand} \\
\lst{right} & \lst{: T} & \text{// right operand} \\
      \end{array}\) \\
       
  \hline
  \bf{Result} & \lst{T} \\
  \hline
  
  \bf{Serialized as} & \hyperref[sec:serialization:operation:Division]{\lst{Division}} \\
  \hline
       
\end{tabularx}

\subsubsection{\lst{\%} method (Code 158)}
\label{sec:appendix:primops:Modulo}
\noindent
\begin{tabularx}{\textwidth}{| l | X |}
   \hline
   \bf{Description} & Remainder from division of the first operand by the second operand. \\
  
  \hline
  \bf{Parameters} &
      \(\begin{array}{l l l}
         \lst{left} & \lst{: T} & \text{// left operand} \\
\lst{right} & \lst{: T} & \text{// right operand} \\
      \end{array}\) \\
       
  \hline
  \bf{Result} & \lst{T} \\
  \hline
  
  \bf{Serialized as} & \hyperref[sec:serialization:operation:Modulo]{\lst{Modulo}} \\
  \hline
       
\end{tabularx}

\subsubsection{\lst{min} method (Code 161)}
\label{sec:appendix:primops:Min}
\noindent
\begin{tabularx}{\textwidth}{| l | X |}
   \hline
   \bf{Description} & Minimum value of two operands. \\
  
  \hline
  \bf{Parameters} &
      \(\begin{array}{l l l}
         \lst{left} & \lst{: T} & \text{// left operand} \\
\lst{right} & \lst{: T} & \text{// right operand} \\
      \end{array}\) \\
       
  \hline
  \bf{Result} & \lst{T} \\
  \hline
  
  \bf{Serialized as} & \hyperref[sec:serialization:operation:Min]{\lst{Min}} \\
  \hline
       
\end{tabularx}

\subsubsection{\lst{max} method (Code 162)}
\label{sec:appendix:primops:Max}
\noindent
\begin{tabularx}{\textwidth}{| l | X |}
   \hline
   \bf{Description} & Maximum value of two operands. \\
  
  \hline
  \bf{Parameters} &
      \(\begin{array}{l l l}
         \lst{left} & \lst{: T} & \text{// left operand} \\
\lst{right} & \lst{: T} & \text{// right operand} \\
      \end{array}\) \\
       
  \hline
  \bf{Result} & \lst{T} \\
  \hline
  
  \bf{Serialized as} & \hyperref[sec:serialization:operation:Max]{\lst{Max}} \\
  \hline
       
\end{tabularx}

\subsubsection{\lst{avlTree} method (Code 182)}
\label{sec:appendix:primops:CreateAvlTree}
\noindent
\begin{tabularx}{\textwidth}{| l | X |}
   \hline
   \bf{Description} & Construct a new authenticated dictionary with given parameters and tree root digest. \\
  
  \hline
  \bf{Parameters} &
      \(\begin{array}{l l l}
         \lst{operationFlags} & \lst{: Byte} & \text{// flags of available operations} \\
\lst{digest} & \lst{: Coll[Byte]} & \text{// hash of merkle tree root} \\
\lst{keyLength} & \lst{: Int} & \text{// length of dictionary keys in bytes} \\
\lst{valueLengthOpt} & \lst{: Option[Int]} & \text{// optional width of dictionary values in bytes} \\
      \end{array}\) \\
       
  \hline
  \bf{Result} & \lst{AvlTree} \\
  \hline
  
  \bf{Serialized as} & \hyperref[sec:serialization:operation:CreateAvlTree]{\lst{CreateAvlTree}} \\
  \hline
       
\end{tabularx}

\subsubsection{\lst{treeLookup} method (Code 183)}
\label{sec:appendix:primops:TreeLookup}
\noindent
\begin{tabularx}{\textwidth}{| l | X |}
   \hline
   \bf{Description} &  \\
  
  \hline
  \bf{Parameters} &
      \(\begin{array}{l l l}
         \lst{tree} & \lst{: AvlTree} & \text{// tree to lookup the key} \\
\lst{key} & \lst{: Coll[Byte]} & \text{// a key of an item in the \lst{tree} to lookup} \\
\lst{proof} & \lst{: Coll[Byte]} & \text{// proof to perform verification of the operation} \\
      \end{array}\) \\
       
  \hline
  \bf{Result} & \lst{Option[Coll[Byte]]} \\
  \hline
  
  \bf{Serialized as} & \hyperref[sec:serialization:operation:TreeLookup]{\lst{TreeLookup}} \\
  \hline
       
\end{tabularx}

\subsubsection{\lst{blake2b256} method (Code 203)}
\label{sec:appendix:primops:CalcBlake2b256}
\noindent
\begin{tabularx}{\textwidth}{| l | X |}
   \hline
   \bf{Description} & Calculate Blake2b hash from \lst{input} bytes. \\
  
  \hline
  \bf{Parameters} &
      \(\begin{array}{l l l}
         \lst{input} & \lst{: Coll[Byte]} & \text{// collection of bytes} \\
      \end{array}\) \\
       
  \hline
  \bf{Result} & \lst{Coll[Byte]} \\
  \hline
  
  \bf{Serialized as} & \hyperref[sec:serialization:operation:CalcBlake2b256]{\lst{CalcBlake2b256}} \\
  \hline
       
\end{tabularx}

\subsubsection{\lst{sha256} method (Code 204)}
\label{sec:appendix:primops:CalcSha256}
\noindent
\begin{tabularx}{\textwidth}{| l | X |}
   \hline
   \bf{Description} & Calculate Sha256 hash from \lst{input} bytes. \\
  
  \hline
  \bf{Parameters} &
      \(\begin{array}{l l l}
         \lst{input} & \lst{: Coll[Byte]} & \text{// collection of bytes} \\
      \end{array}\) \\
       
  \hline
  \bf{Result} & \lst{Coll[Byte]} \\
  \hline
  
  \bf{Serialized as} & \hyperref[sec:serialization:operation:CalcSha256]{\lst{CalcSha256}} \\
  \hline
       
\end{tabularx}

\subsubsection{\lst{proveDlog} method (Code 205)}
\label{sec:appendix:primops:CreateProveDlog}
\noindent
\begin{tabularx}{\textwidth}{| l | X |}
   \hline
   \bf{Description} & ErgoTree operation to create a new \lst{SigmaProp} value representing public key
 of discrete logarithm signature protocol.
         \\
  
  \hline
  \bf{Parameters} &
      \(\begin{array}{l l l}
         \lst{value} & \lst{: GroupElement} & \text{// element of elliptic curve group} \\
      \end{array}\) \\
       
  \hline
  \bf{Result} & \lst{SigmaProp} \\
  \hline
  
  \bf{Serialized as} & \hyperref[sec:serialization:operation:CreateProveDlog]{\lst{CreateProveDlog}} \\
  \hline
       
\end{tabularx}

\subsubsection{\lst{proveDHTuple} method (Code 206)}
\label{sec:appendix:primops:CreateProveDHTuple}
\noindent
\begin{tabularx}{\textwidth}{| l | X |}
   \hline
   \bf{Description} &  ErgoTree operation to create a new SigmaProp value representing public key
 of Diffie Hellman signature protocol.
 Common input: (g,h,u,v)
         \\
  
  \hline
  \bf{Parameters} &
      \(\begin{array}{l l l}
         \lst{g} & \lst{: GroupElement} & \text{// } \\
\lst{h} & \lst{: GroupElement} & \text{// } \\
\lst{u} & \lst{: GroupElement} & \text{// } \\
\lst{v} & \lst{: GroupElement} & \text{// } \\
      \end{array}\) \\
       
  \hline
  \bf{Result} & \lst{SigmaProp} \\
  \hline
  
  \bf{Serialized as} & \hyperref[sec:serialization:operation:CreateProveDHTuple]{\lst{CreateProveDHTuple}} \\
  \hline
       
\end{tabularx}

\subsubsection{\lst{sigmaProp} method (Code 209)}
\label{sec:appendix:primops:BoolToSigmaProp}
\noindent
\begin{tabularx}{\textwidth}{| l | X |}
   \hline
   \bf{Description} & Embedding of \lst{Boolean} values to \lst{SigmaProp} values.
 As an example, this operation allows boolean experessions
 to be used as arguments of \lst{atLeast(..., sigmaProp(boolExpr), ...)} operation.
 During execution results to either \lst{TrueProp} or \lst{FalseProp} values of \lst{SigmaProp} type.
         \\
  
  \hline
  \bf{Parameters} &
      \(\begin{array}{l l l}
         \lst{condition} & \lst{: Boolean} & \text{// boolean value to embed in SigmaProp value} \\
      \end{array}\) \\
       
  \hline
  \bf{Result} & \lst{SigmaProp} \\
  \hline
  
  \bf{Serialized as} & \hyperref[sec:serialization:operation:BoolToSigmaProp]{\lst{BoolToSigmaProp}} \\
  \hline
       
\end{tabularx}

\subsubsection{\lst{executeFromVar} method (Code 212)}
\label{sec:appendix:primops:DeserializeContext}
\noindent
\begin{tabularx}{\textwidth}{| l | X |}
   \hline
   \bf{Description} & Extracts context variable as \lst{Coll[Byte]}, deserializes it to script
 and then executes this script in the current context.
 The original \lst{Coll[Byte]} of the script is available as \lst{getVar[Coll[Byte]](id)}.
 Type parameter \lst{V} result type of the deserialized script.
 Throws an exception if the actual script type doesn't conform to T.
 Returns a result of the script execution in the current context
         \\
  
  \hline
  \bf{Parameters} &
      \(\begin{array}{l l l}
         \lst{id} & \lst{: Byte} & \text{// identifier of the context variable} \\
      \end{array}\) \\
       
  \hline
  \bf{Result} & \lst{T} \\
  \hline
  
  \bf{Serialized as} & \hyperref[sec:serialization:operation:DeserializeContext]{\lst{DeserializeContext}} \\
  \hline
       
\end{tabularx}

\subsubsection{\lst{executeFromSelfReg} method (Code 213)}
\label{sec:appendix:primops:DeserializeRegister}
\noindent
\begin{tabularx}{\textwidth}{| l | X |}
   \hline
   \bf{Description} & Extracts SELF register as \lst{Coll[Byte]}, deserializes it to script
 and then executes this script in the current context.
 The original \lst{Coll[Byte]} of the script is available as \lst{SELF.getReg[Coll[Byte]](id)}.
 Type parameter \lst{T} result type of the deserialized script.
 Throws an exception if the actual script type doesn't conform to \lst{T}.
 Returns a result of the script execution in the current context
         \\
  
  \hline
  \bf{Parameters} &
      \(\begin{array}{l l l}
         \lst{id} & \lst{: Byte} & \text{// identifier of the register} \\
\lst{default} & \lst{: Option[T]} & \text{// optional default value, if register is not available} \\
      \end{array}\) \\
       
  \hline
  \bf{Result} & \lst{T} \\
  \hline
  
  \bf{Serialized as} & \hyperref[sec:serialization:operation:DeserializeRegister]{\lst{DeserializeRegister}} \\
  \hline
       
\end{tabularx}

\subsubsection{\lst{apply} method (Code 218)}
\label{sec:appendix:primops:Apply}
\noindent
\begin{tabularx}{\textwidth}{| l | X |}
   \hline
   \bf{Description} & Apply the function to the arguments.  \\
  
  \hline
  \bf{Parameters} &
      \(\begin{array}{l l l}
         \lst{func} & \lst{: (T) => R} & \text{// function which is applied} \\
\lst{args} & \lst{: T} & \text{// list of arguments} \\
      \end{array}\) \\
       
  \hline
  \bf{Result} & \lst{R} \\
  \hline
  
  \bf{Serialized as} & \hyperref[sec:serialization:operation:Apply]{\lst{Apply}} \\
  \hline
       
\end{tabularx}

\subsubsection{\lst{getVar} method (Code 227)}
\label{sec:appendix:primops:GetVar}
\noindent
\begin{tabularx}{\textwidth}{| l | X |}
   \hline
   \bf{Description} & Get context variable with given \lst{varId} and type. \\
  
  \hline
  \bf{Parameters} &
      \(\begin{array}{l l l}
         \lst{varId} & \lst{: Byte} & \text{// \lst{Byte} identifier of context variable} \\
      \end{array}\) \\
       
  \hline
  \bf{Result} & \lst{Option[T]} \\
  \hline
  
  \bf{Serialized as} & \hyperref[sec:serialization:operation:GetVar]{\lst{GetVar}} \\
  \hline
       
\end{tabularx}

\subsubsection{\lst{allZK} method (Code 234)}
\label{sec:appendix:primops:SigmaAnd}
\noindent
\begin{tabularx}{\textwidth}{| l | X |}
   \hline
   \bf{Description} & Returns sigma proposition which is proven when \emph{all} the elements in collection are proven. \\
  
  \hline
  \bf{Parameters} &
      \(\begin{array}{l l l}
         \lst{propositions} & \lst{: Coll[SigmaProp]} & \text{// a collection of propositions} \\
      \end{array}\) \\
       
  \hline
  \bf{Result} & \lst{SigmaProp} \\
  \hline
  
  \bf{Serialized as} & \hyperref[sec:serialization:operation:SigmaAnd]{\lst{SigmaAnd}} \\
  \hline
       
\end{tabularx}

\subsubsection{\lst{anyZK} method (Code 235)}
\label{sec:appendix:primops:SigmaOr}
\noindent
\begin{tabularx}{\textwidth}{| l | X |}
   \hline
   \bf{Description} & Returns sigma proposition which is proven when \emph{any} of the elements in collection is proven. \\
  
  \hline
  \bf{Parameters} &
      \(\begin{array}{l l l}
         \lst{propositions} & \lst{: Coll[SigmaProp]} & \text{// a collection of propositions} \\
      \end{array}\) \\
       
  \hline
  \bf{Result} & \lst{SigmaProp} \\
  \hline
  
  \bf{Serialized as} & \hyperref[sec:serialization:operation:SigmaOr]{\lst{SigmaOr}} \\
  \hline
       
\end{tabularx}

\subsubsection{\lst{||} method (Code 236)}
\label{sec:appendix:primops:BinOr}
\noindent
\begin{tabularx}{\textwidth}{| l | X |}
   \hline
   \bf{Description} & Logical OR of two operands \\
  
  \hline
  \bf{Parameters} &
      \(\begin{array}{l l l}
         \lst{left} & \lst{: Boolean} & \text{// left operand} \\
\lst{right} & \lst{: Boolean} & \text{// right operand} \\
      \end{array}\) \\
       
  \hline
  \bf{Result} & \lst{Boolean} \\
  \hline
  
  \bf{Serialized as} & \hyperref[sec:serialization:operation:BinOr]{\lst{BinOr}} \\
  \hline
       
\end{tabularx}

\subsubsection{\lst{&&} method (Code 237)}
\label{sec:appendix:primops:BinAnd}
\noindent
\begin{tabularx}{\textwidth}{| l | X |}
   \hline
   \bf{Description} & Logical AND of two operands \\
  
  \hline
  \bf{Parameters} &
      \(\begin{array}{l l l}
         \lst{left} & \lst{: Boolean} & \text{// left operand} \\
\lst{right} & \lst{: Boolean} & \text{// right operand} \\
      \end{array}\) \\
       
  \hline
  \bf{Result} & \lst{Boolean} \\
  \hline
  
  \bf{Serialized as} & \hyperref[sec:serialization:operation:BinAnd]{\lst{BinAnd}} \\
  \hline
       
\end{tabularx}

\subsubsection{\lst{decodePoint} method (Code 238)}
\label{sec:appendix:primops:DecodePoint}
\noindent
\begin{tabularx}{\textwidth}{| l | X |}
   \hline
   \bf{Description} & Convert \lst{Coll[Byte]} to \lst{GroupElement} using \lst{GroupElementSerializer} \\
  
  \hline
  \bf{Parameters} &
      \(\begin{array}{l l l}
         \lst{input} & \lst{: Coll[Byte]} & \text{// serialized bytes of some \lst{GroupElement} value} \\
      \end{array}\) \\
       
  \hline
  \bf{Result} & \lst{GroupElement} \\
  \hline
  
  \bf{Serialized as} & \hyperref[sec:serialization:operation:DecodePoint]{\lst{DecodePoint}} \\
  \hline
       
\end{tabularx}

\subsubsection{\lst{unary_!} method (Code 239)}
\label{sec:appendix:primops:LogicalNot}
\noindent
\begin{tabularx}{\textwidth}{| l | X |}
   \hline
   \bf{Description} & Logical NOT operation. Returns \lst{true} if input is \lst{false} and \lst{false} if input is \lst{true}. \\
  
  \hline
  \bf{Parameters} &
      \(\begin{array}{l l l}
         \lst{input} & \lst{: Boolean} & \text{// input \lst{Boolean} value} \\
      \end{array}\) \\
       
  \hline
  \bf{Result} & \lst{Boolean} \\
  \hline
  
  \bf{Serialized as} & \hyperref[sec:serialization:operation:LogicalNot]{\lst{LogicalNot}} \\
  \hline
       
\end{tabularx}

\subsubsection{\lst{unary_-} method (Code 240)}
\label{sec:appendix:primops:Negation}
\noindent
\begin{tabularx}{\textwidth}{| l | X |}
   \hline
   \bf{Description} & Negates numeric value \lst{x} by returning \lst{-x}. \\
  
  \hline
  \bf{Parameters} &
      \(\begin{array}{l l l}
         \lst{input} & \lst{: T} & \text{// value of numeric type} \\
      \end{array}\) \\
       
  \hline
  \bf{Result} & \lst{T} \\
  \hline
  
  \bf{Serialized as} & \hyperref[sec:serialization:operation:Negation]{\lst{Negation}} \\
  \hline
       
\end{tabularx}

\subsubsection{\lst{unary_~} method (Code 241)}
\label{sec:appendix:primops:BitInversion}
\noindent
\begin{tabularx}{\textwidth}{| l | X |}
   \hline
   \bf{Description} & Invert every bit of the numeric value. \\
  
  \hline
  \bf{Parameters} &
      \(\begin{array}{l l l}
         \lst{input} & \lst{: T} & \text{// value of numeric type} \\
      \end{array}\) \\
       
  \hline
  \bf{Result} & \lst{T} \\
  \hline
  
  \bf{Serialized as} & \hyperref[sec:serialization:operation:BitInversion]{\lst{BitInversion}} \\
  \hline
       
\end{tabularx}

\subsubsection{\lst{bit_|} method (Code 242)}
\label{sec:appendix:primops:BitOr}
\noindent
\begin{tabularx}{\textwidth}{| l | X |}
   \hline
   \bf{Description} & Bitwise OR of two numeric operands. \\
  
  \hline
  \bf{Parameters} &
      \(\begin{array}{l l l}
         \lst{left} & \lst{: T} & \text{// left operand} \\
\lst{right} & \lst{: T} & \text{// right operand} \\
      \end{array}\) \\
       
  \hline
  \bf{Result} & \lst{T} \\
  \hline
  
  \bf{Serialized as} & \hyperref[sec:serialization:operation:BitOr]{\lst{BitOr}} \\
  \hline
       
\end{tabularx}

\subsubsection{\lst{bit_&} method (Code 243)}
\label{sec:appendix:primops:BitAnd}
\noindent
\begin{tabularx}{\textwidth}{| l | X |}
   \hline
   \bf{Description} & Bitwise AND of two numeric operands. \\
  
  \hline
  \bf{Parameters} &
      \(\begin{array}{l l l}
         \lst{left} & \lst{: T} & \text{// left operand} \\
\lst{right} & \lst{: T} & \text{// right operand} \\
      \end{array}\) \\
       
  \hline
  \bf{Result} & \lst{T} \\
  \hline
  
  \bf{Serialized as} & \hyperref[sec:serialization:operation:BitAnd]{\lst{BitAnd}} \\
  \hline
       
\end{tabularx}

\subsubsection{\lst{^} method (Code 244)}
\label{sec:appendix:primops:BinXor}
\noindent
\begin{tabularx}{\textwidth}{| l | X |}
   \hline
   \bf{Description} & Logical XOR of two operands \\
  
  \hline
  \bf{Parameters} &
      \(\begin{array}{l l l}
         \lst{left} & \lst{: Boolean} & \text{// left operand} \\
\lst{right} & \lst{: Boolean} & \text{// right operand} \\
      \end{array}\) \\
       
  \hline
  \bf{Result} & \lst{Boolean} \\
  \hline
  
  \bf{Serialized as} & \hyperref[sec:serialization:operation:BinXor]{\lst{BinXor}} \\
  \hline
       
\end{tabularx}

\subsubsection{\lst{bit_^} method (Code 245)}
\label{sec:appendix:primops:BitXor}
\noindent
\begin{tabularx}{\textwidth}{| l | X |}
   \hline
   \bf{Description} & Bitwise XOR of two numeric operands. \\
  
  \hline
  \bf{Parameters} &
      \(\begin{array}{l l l}
         \lst{left} & \lst{: T} & \text{// left operand} \\
\lst{right} & \lst{: T} & \text{// right operand} \\
      \end{array}\) \\
       
  \hline
  \bf{Result} & \lst{T} \\
  \hline
  
  \bf{Serialized as} & \hyperref[sec:serialization:operation:BitXor]{\lst{BitXor}} \\
  \hline
       
\end{tabularx}

\subsubsection{\lst{bit_>>} method (Code 246)}
\label{sec:appendix:primops:BitShiftRight}
\noindent
\begin{tabularx}{\textwidth}{| l | X |}
   \hline
   \bf{Description} & Right shift of bits. \\
  
  \hline
  \bf{Parameters} &
      \(\begin{array}{l l l}
         \lst{left} & \lst{: T} & \text{// left operand} \\
\lst{right} & \lst{: T} & \text{// right operand} \\
      \end{array}\) \\
       
  \hline
  \bf{Result} & \lst{T} \\
  \hline
  
  \bf{Serialized as} & \hyperref[sec:serialization:operation:BitShiftRight]{\lst{BitShiftRight}} \\
  \hline
       
\end{tabularx}

\subsubsection{\lst{bit_<<} method (Code 247)}
\label{sec:appendix:primops:BitShiftLeft}
\noindent
\begin{tabularx}{\textwidth}{| l | X |}
   \hline
   \bf{Description} & Left shift of bits. \\
  
  \hline
  \bf{Parameters} &
      \(\begin{array}{l l l}
         \lst{left} & \lst{: T} & \text{// left operand} \\
\lst{right} & \lst{: T} & \text{// right operand} \\
      \end{array}\) \\
       
  \hline
  \bf{Result} & \lst{T} \\
  \hline
  
  \bf{Serialized as} & \hyperref[sec:serialization:operation:BitShiftLeft]{\lst{BitShiftLeft}} \\
  \hline
       
\end{tabularx}

\subsubsection{\lst{bit_>>>} method (Code 248)}
\label{sec:appendix:primops:BitShiftRightZeroed}
\noindent
\begin{tabularx}{\textwidth}{| l | X |}
   \hline
   \bf{Description} & Right shift of bits. \\
  
  \hline
  \bf{Parameters} &
      \(\begin{array}{l l l}
         \lst{left} & \lst{: T} & \text{// left operand} \\
\lst{right} & \lst{: T} & \text{// right operand} \\
      \end{array}\) \\
       
  \hline
  \bf{Result} & \lst{T} \\
  \hline
  
  \bf{Serialized as} & \hyperref[sec:serialization:operation:BitShiftRightZeroed]{\lst{BitShiftRightZeroed}} \\
  \hline
       
\end{tabularx}

\subsubsection{\lst{xorOf} method (Code 255)}
\label{sec:appendix:primops:XorOf}
\noindent
\begin{tabularx}{\textwidth}{| l | X |}
   \hline
   \bf{Description} & Similar to \lst{allOf}, but performing logical XOR operation between all conditions instead of \lst{&&} \\
  
  \hline
  \bf{Parameters} &
      \(\begin{array}{l l l}
         \lst{conditions} & \lst{: Coll[Boolean]} & \text{// a collection of conditions} \\
      \end{array}\) \\
       
  \hline
  \bf{Result} & \lst{Boolean} \\
  \hline
  
  \bf{Serialized as} & \hyperref[sec:serialization:operation:XorOf]{\lst{XorOf}} \\
  \hline
       
\end{tabularx}
