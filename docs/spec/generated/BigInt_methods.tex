
\subsubsection{\lst{BigInt.toByte} method (Code 106.1)}
\noindent
\begin{tabularx}{\textwidth}{| l | X |}
   \hline
   \bf{Description} & Converts this numeric value to \lst{Byte}, throwing exception if overflow. \\
  
  \hline
  \bf{Result} & \lst{Byte} \\
  \hline
\end{tabularx}



\subsubsection{\lst{BigInt.modQ} method (Code 6.1)}
\noindent
\begin{tabularx}{\textwidth}{| l | X |}
   \hline
   \bf{Description} & Returns this \lst{mod} Q, i.e. remainder of division by Q, where Q is an order of the cryprographic group. \\
  
  \hline
  \bf{Result} & \lst{BigInt} \\
  \hline
\end{tabularx}



\subsubsection{\lst{BigInt.toShort} method (Code 106.2)}
\noindent
\begin{tabularx}{\textwidth}{| l | X |}
   \hline
   \bf{Description} & Converts this numeric value to \lst{Short}, throwing exception if overflow. \\
  
  \hline
  \bf{Result} & \lst{Short} \\
  \hline
\end{tabularx}



\subsubsection{\lst{BigInt.plusModQ} method (Code 6.2)}
\noindent
\begin{tabularx}{\textwidth}{| l | X |}
   \hline
   \bf{Description} & Adds this number with \lst{other} by module Q. \\
  
  \hline
  \bf{Parameters} &
      \(\begin{array}{l l l}
         \lst{other} & \lst{: BigInt} & \text{// Number to add to this.} \\
      \end{array}\) \\
       
  \hline
  \bf{Result} & \lst{BigInt} \\
  \hline
\end{tabularx}



\subsubsection{\lst{BigInt.toInt} method (Code 106.3)}
\noindent
\begin{tabularx}{\textwidth}{| l | X |}
   \hline
   \bf{Description} & Converts this numeric value to \lst{Int}, throwing exception if overflow. \\
  
  \hline
  \bf{Result} & \lst{Int} \\
  \hline
\end{tabularx}



\subsubsection{\lst{BigInt.minusModQ} method (Code 6.3)}
\noindent
\begin{tabularx}{\textwidth}{| l | X |}
   \hline
   \bf{Description} & Subtracts \lst{other} number from this by module Q. \\
  
  \hline
  \bf{Parameters} &
      \(\begin{array}{l l l}
         \lst{other} & \lst{: BigInt} & \text{// Number to subtract from this.} \\
      \end{array}\) \\
       
  \hline
  \bf{Result} & \lst{BigInt} \\
  \hline
\end{tabularx}



\subsubsection{\lst{BigInt.toLong} method (Code 106.4)}
\noindent
\begin{tabularx}{\textwidth}{| l | X |}
   \hline
   \bf{Description} & Converts this numeric value to \lst{Long}, throwing exception if overflow. \\
  
  \hline
  \bf{Result} & \lst{Long} \\
  \hline
\end{tabularx}



\subsubsection{\lst{BigInt.multModQ} method (Code 6.4)}
\noindent
\begin{tabularx}{\textwidth}{| l | X |}
   \hline
   \bf{Description} & Multiply this number with \lst{other} by module Q. \\
  
  \hline
  \bf{Parameters} &
      \(\begin{array}{l l l}
         \lst{other} & \lst{: BigInt} & \text{// Number to multiply with this.} \\
      \end{array}\) \\
       
  \hline
  \bf{Result} & \lst{BigInt} \\
  \hline
\end{tabularx}



\subsubsection{\lst{BigInt.toBigInt} method (Code 106.5)}
\noindent
\begin{tabularx}{\textwidth}{| l | X |}
   \hline
   \bf{Description} & Converts this numeric value to \lst{BigInt} \\
  
  \hline
  \bf{Result} & \lst{BigInt} \\
  \hline
\end{tabularx}



\subsubsection{\lst{BigInt.toBytes} method (Code 106.6)}
\noindent
\begin{tabularx}{\textwidth}{| l | X |}
   \hline
   \bf{Description} & Returns a big-endian representation of this numeric value in a collection of bytes.
 For example, the Int value \lst{0x12131415} would yield the
 byte array  \lst{[0x12, 0x13, 0x14, 0x15]}. \\
  
  \hline
  \bf{Result} & \lst{Coll[Byte]} \\
  \hline
\end{tabularx}



\subsubsection{\lst{BigInt.toBits} method (Code 106.7)}
\noindent
\begin{tabularx}{\textwidth}{| l | X |}
   \hline
   \bf{Description} & Returns a big-endian representation of this numeric in a collection of Booleans.
 Each boolean corresponds to one bit. \\
  
  \hline
  \bf{Result} & \lst{Coll[Boolean]} \\
  \hline
\end{tabularx}
