\section{Predefined types}
\label{sec:appendix:predeftypes}

\begin{table}[h]
    \small
    \begin{tabu}{|l |l |l |l |l |l |l |l|}
     \hline
     \rowfont{\bfseries}
        Name   &   Code   &  IsConstSize & 
        isPrim\footnote{isPrim - primitive type} & 
        isEmbed  & isNum & Set of values \\
        \hline

\lst{Boolean}	&	$1$	&	\lst{true}	& \lst{true}	&	\lst{true} &	\lst{false}	& $\Set{\lst{true}, \lst{false}}$ \\
\hline
\lst{Byte}	&	$2$	&	\lst{true}	& \lst{true}	&	\lst{true} &	\lst{true}	& $\Set{-2^{7} \dots 2^{7}-1}$~\ref{sec:type:Byte} \\
\hline
\lst{Short}	&	$3$	&	\lst{true}	& \lst{true}	&	\lst{true} &	\lst{true}	& $\Set{-2^{15} \dots 2^{15}-1}$~\ref{sec:type:Short} \\
\hline
\lst{Int}	&	$4$	&	\lst{true}	& \lst{true}	&	\lst{true} &	\lst{true}	& $\Set{-2^{31} \dots 2^{31}-1}$~\ref{sec:type:Int} \\
\hline
\lst{Long}	&	$5$	&	\lst{true}	& \lst{true}	&	\lst{true} &	\lst{true}	& $\Set{-2^{63} \dots 2^{63}-1}$~\ref{sec:type:Long} \\
\hline
\lst{BigInt}	&	$6$	&	\lst{true}	& \lst{true}	&	\lst{true} &	\lst{true}	& $\Set{-2^{255} \dots 2^{255}-1}$~\ref{sec:type:BigInt} \\
\hline
\lst{GroupElement}	&	$7$	&	\lst{true}	& \lst{true}	&	\lst{true} &	\lst{false}	& $\Set{p \in \lst{SecP256K1Point}}$ \\
\hline
\lst{SigmaProp}	&	$8$	&	\lst{false}	& \lst{true}	&	\lst{true} &	\lst{false}	& Sec.~\ref{sec:type:SigmaProp} \\
\hline
\lst{Box}	&	$99$	&	\lst{false}	& \lst{false}	&	\lst{false} &	\lst{false}	& Sec.~\ref{sec:type:Box} \\
\hline
\lst{AvlTree}	&	$100$	&	\lst{false}	& \lst{false}	&	\lst{false} &	\lst{false}	& Sec.~\ref{sec:type:AvlTree} \\
\hline
\lst{Context}	&	$101$	&	\lst{false}	& \lst{false}	&	\lst{false} &	\lst{false}	& Sec.~\ref{sec:type:Context} \\
\hline
\lst{Header}	&	$104$	&	\lst{true}	& \lst{false}	&	\lst{false} &	\lst{false}	& Sec.~\ref{sec:type:Header} \\
\hline
\lst{PreHeader}	&	$105$	&	\lst{true}	& \lst{false}	&	\lst{false} &	\lst{false}	& Sec.~\ref{sec:type:PreHeader} \\
\hline
\lst{Global}	&	$106$	&	\lst{true}	& \lst{false}	&	\lst{false} &	\lst{false}	& Sec.~\ref{sec:type:Global} \\

    \hline
    \end{tabu}
    \caption{Predefined types of \langname}
    \label{table:predeftypes}
\end{table}

There is a section for each type with sub-sections for all available methods. Each
method is characterized by the description, signature (i.e. name, parameters and return
type), description of all parameters and reference to the serialization format.

There is universal primitive which can represent any method invocation
(\lst{MethodCall}). However, many method are also mapped to the special primitive
operations to save storage space.

The following sub-sections are auto-generated from type descriptors of \langname reference
implementation.

% \subsection{Boolean type}
% \label{sec:type:Boolean}
% 
\subsubsection{\lst{Boolean.toByte} method (Code 1.1)}
\noindent
\begin{tabularx}{\textwidth}{| l | X |}
   \hline
   \bf{Description} & Convert true to 1 and false to 0 \\
  
  \hline
  \bf{Parameters} &
      \(\begin{array}{l l l}
         
      \end{array}\) \\
       
  \hline
  \bf{Result} & \lst{Byte} \\
  \hline
  
  \bf{Serialized as} & \lst{PropertyCall(opCode=219)} \\
  \hline
       
\end{tabularx}


\subsection{Byte type}
\label{sec:type:Byte}

\noindent
\begin{tabularx}{\textwidth}{| c | c | X |}
  \hline
  \bf{Code} & \bf{Method Signature} & \bf{Description} \\
  \hline
  106.1 & \lst{def toByte()} &  \\
\hline
106.2 & \lst{def toShort()} &  \\
\hline
106.3 & \lst{def toInt()} &  \\
\hline
106.4 & \lst{def toLong()} &  \\
\hline
106.5 & \lst{def toBigInt()} &  \\
\hline
106.6 & \lst{def toBytes()} &  \\
\hline
106.7 & \lst{def toBits()} &  \\
  \hline
\end{tabularx}
     

\subsection{Short type}
\label{sec:type:Short}
 
\subsubsection{\lst{Short.toByte} method (Code 106.1)}
\noindent
\begin{tabularx}{\textwidth}{| l | X |}
   \hline
   \bf{Description} &  \\
  
  \hline
  \bf{Result} & \lst{Byte} \\
  \hline
\end{tabularx}



\subsubsection{\lst{Short.toShort} method (Code 106.2)}
\noindent
\begin{tabularx}{\textwidth}{| l | X |}
   \hline
   \bf{Description} &  \\
  
  \hline
  \bf{Result} & \lst{Short} \\
  \hline
\end{tabularx}



\subsubsection{\lst{Short.toInt} method (Code 106.3)}
\noindent
\begin{tabularx}{\textwidth}{| l | X |}
   \hline
   \bf{Description} &  \\
  
  \hline
  \bf{Result} & \lst{Int} \\
  \hline
\end{tabularx}



\subsubsection{\lst{Short.toLong} method (Code 106.4)}
\noindent
\begin{tabularx}{\textwidth}{| l | X |}
   \hline
   \bf{Description} &  \\
  
  \hline
  \bf{Result} & \lst{Long} \\
  \hline
\end{tabularx}



\subsubsection{\lst{Short.toBigInt} method (Code 106.5)}
\noindent
\begin{tabularx}{\textwidth}{| l | X |}
   \hline
   \bf{Description} &  \\
  
  \hline
  \bf{Result} & \lst{BigInt} \\
  \hline
\end{tabularx}



\subsubsection{\lst{Short.toBytes} method (Code 106.6)}
\noindent
\begin{tabularx}{\textwidth}{| l | X |}
   \hline
   \bf{Description} &  \\
  
  \hline
  \bf{Result} & \lst{Coll[Byte]} \\
  \hline
\end{tabularx}



\subsubsection{\lst{Short.toBits} method (Code 106.7)}
\noindent
\begin{tabularx}{\textwidth}{| l | X |}
   \hline
   \bf{Description} &  \\
  
  \hline
  \bf{Result} & \lst{Coll[Boolean]} \\
  \hline
\end{tabularx}


\subsection{Int type}
\label{sec:type:Int}
 
\subsubsection{\lst{Int.toByte} method (Code 106.1)}
\label{sec:type:Int:toByte}
\noindent
\begin{tabularx}{\textwidth}{| l | X |}
   \hline
   \bf{Description} & Converts this numeric value to \lst{Byte}, throwing exception if overflow. \\
  
  \hline
  \bf{Parameters} &
      \(\begin{array}{l l l}
         
      \end{array}\) \\
       
  \hline
  \bf{Result} & \lst{Byte} \\
  \hline
  
  \bf{Serialized as} & \hyperref[sec:serialization:operation:PropertyCall]{\lst{PropertyCall}} \\
  \hline
       
\end{tabularx}



\subsubsection{\lst{Int.toShort} method (Code 106.2)}
\label{sec:type:Int:toShort}
\noindent
\begin{tabularx}{\textwidth}{| l | X |}
   \hline
   \bf{Description} & Converts this numeric value to \lst{Short}, throwing exception if overflow. \\
  
  \hline
  \bf{Parameters} &
      \(\begin{array}{l l l}
         
      \end{array}\) \\
       
  \hline
  \bf{Result} & \lst{Short} \\
  \hline
  
  \bf{Serialized as} & \hyperref[sec:serialization:operation:PropertyCall]{\lst{PropertyCall}} \\
  \hline
       
\end{tabularx}



\subsubsection{\lst{Int.toInt} method (Code 106.3)}
\label{sec:type:Int:toInt}
\noindent
\begin{tabularx}{\textwidth}{| l | X |}
   \hline
   \bf{Description} & Converts this numeric value to \lst{Int}, throwing exception if overflow. \\
  
  \hline
  \bf{Parameters} &
      \(\begin{array}{l l l}
         
      \end{array}\) \\
       
  \hline
  \bf{Result} & \lst{Int} \\
  \hline
  
  \bf{Serialized as} & \hyperref[sec:serialization:operation:PropertyCall]{\lst{PropertyCall}} \\
  \hline
       
\end{tabularx}



\subsubsection{\lst{Int.toLong} method (Code 106.4)}
\label{sec:type:Int:toLong}
\noindent
\begin{tabularx}{\textwidth}{| l | X |}
   \hline
   \bf{Description} & Converts this numeric value to \lst{Long}, throwing exception if overflow. \\
  
  \hline
  \bf{Parameters} &
      \(\begin{array}{l l l}
         
      \end{array}\) \\
       
  \hline
  \bf{Result} & \lst{Long} \\
  \hline
  
  \bf{Serialized as} & \hyperref[sec:serialization:operation:PropertyCall]{\lst{PropertyCall}} \\
  \hline
       
\end{tabularx}



\subsubsection{\lst{Int.toBigInt} method (Code 106.5)}
\label{sec:type:Int:toBigInt}
\noindent
\begin{tabularx}{\textwidth}{| l | X |}
   \hline
   \bf{Description} & Converts this numeric value to \lst{BigInt} \\
  
  \hline
  \bf{Parameters} &
      \(\begin{array}{l l l}
         
      \end{array}\) \\
       
  \hline
  \bf{Result} & \lst{BigInt} \\
  \hline
  
  \bf{Serialized as} & \hyperref[sec:serialization:operation:PropertyCall]{\lst{PropertyCall}} \\
  \hline
       
\end{tabularx}



\subsubsection{\lst{Int.toBytes} method (Code 106.6)}
\label{sec:type:Int:toBytes}
\noindent
\begin{tabularx}{\textwidth}{| l | X |}
   \hline
   \bf{Description} &  Returns a big-endian representation of this numeric value in a collection of bytes.
 For example, the \lst{Int} value \lst{0x12131415} would yield the
 collection of bytes \lst{[0x12, 0x13, 0x14, 0x15]}.
           \\
  
  \hline
  \bf{Parameters} &
      \(\begin{array}{l l l}
         
      \end{array}\) \\
       
  \hline
  \bf{Result} & \lst{Coll[Byte]} \\
  \hline
  
  \bf{Serialized as} & \hyperref[sec:serialization:operation:PropertyCall]{\lst{PropertyCall}} \\
  \hline
       
\end{tabularx}



\subsubsection{\lst{Int.toBits} method (Code 106.7)}
\label{sec:type:Int:toBits}
\noindent
\begin{tabularx}{\textwidth}{| l | X |}
   \hline
   \bf{Description} &  Returns a big-endian representation of this numeric in a collection of Booleans.
  Each boolean corresponds to one bit.
           \\
  
  \hline
  \bf{Parameters} &
      \(\begin{array}{l l l}
         
      \end{array}\) \\
       
  \hline
  \bf{Result} & \lst{Coll[Boolean]} \\
  \hline
  
  \bf{Serialized as} & \hyperref[sec:serialization:operation:PropertyCall]{\lst{PropertyCall}} \\
  \hline
       
\end{tabularx}


\subsection{Long type}
\label{sec:type:Long}
 
\subsubsection{\lst{Long.toByte} method (Code 106.1)}
\noindent
\begin{tabularx}{\textwidth}{| l | X |}
   \hline
   \bf{Description} & Converts this numeric value to \lst{Byte}, throwing exception if overflow. \\
  
  \hline
  \bf{Result} & \lst{Byte} \\
  \hline
\end{tabularx}



\subsubsection{\lst{Long.toShort} method (Code 106.2)}
\noindent
\begin{tabularx}{\textwidth}{| l | X |}
   \hline
   \bf{Description} & Converts this numeric value to \lst{Short}, throwing exception if overflow. \\
  
  \hline
  \bf{Result} & \lst{Short} \\
  \hline
\end{tabularx}



\subsubsection{\lst{Long.toInt} method (Code 106.3)}
\noindent
\begin{tabularx}{\textwidth}{| l | X |}
   \hline
   \bf{Description} & Converts this numeric value to \lst{Int}, throwing exception if overflow. \\
  
  \hline
  \bf{Result} & \lst{Int} \\
  \hline
\end{tabularx}



\subsubsection{\lst{Long.toLong} method (Code 106.4)}
\noindent
\begin{tabularx}{\textwidth}{| l | X |}
   \hline
   \bf{Description} & Converts this numeric value to \lst{Long}, throwing exception if overflow. \\
  
  \hline
  \bf{Result} & \lst{Long} \\
  \hline
\end{tabularx}



\subsubsection{\lst{Long.toBigInt} method (Code 106.5)}
\noindent
\begin{tabularx}{\textwidth}{| l | X |}
   \hline
   \bf{Description} & Converts this numeric value to \lst{BigInt} \\
  
  \hline
  \bf{Result} & \lst{BigInt} \\
  \hline
\end{tabularx}



\subsubsection{\lst{Long.toBytes} method (Code 106.6)}
\noindent
\begin{tabularx}{\textwidth}{| l | X |}
   \hline
   \bf{Description} & Returns a big-endian representation of this numeric value in a collection of bytes.
 For example, the Int value \lst{0x12131415} would yield the
 byte array  \lst{[0x12, 0x13, 0x14, 0x15]}. \\
  
  \hline
  \bf{Result} & \lst{Coll[Byte]} \\
  \hline
\end{tabularx}



\subsubsection{\lst{Long.toBits} method (Code 106.7)}
\noindent
\begin{tabularx}{\textwidth}{| l | X |}
   \hline
   \bf{Description} & Returns a big-endian representation of this numeric in a collection of Booleans.
 Each boolean corresponds to one bit. \\
  
  \hline
  \bf{Result} & \lst{Coll[Boolean]} \\
  \hline
\end{tabularx}


\subsection{BigInt type}
\label{sec:type:BigInt}
 
\noindent
\begin{tabularx}{\textwidth}{| c | c | X |}
  \hline
  \bf{Code} & \bf{Method Signature} & \bf{Description} \\
  \hline
  106.1 & \lst{def toByte()} &  \\
\hline
6.1 & \lst{def modQ()} &  \\
\hline
106.2 & \lst{def toShort()} &  \\
\hline
6.2 & \lst{def plusModQ()} &  \\
\hline
106.3 & \lst{def toInt()} &  \\
\hline
6.3 & \lst{def minusModQ()} &  \\
\hline
106.4 & \lst{def toLong()} &  \\
\hline
6.4 & \lst{def multModQ()} &  \\
\hline
106.5 & \lst{def toBigInt()} &  \\
\hline
106.6 & \lst{def toBytes()} &  \\
\hline
106.7 & \lst{def toBits()} &  \\
  \hline
\end{tabularx}
     

\subsection{GroupElement type}
\label{sec:type:GroupElement}

\subsubsection{\lst{GroupElement.getEncoded} method (Code 7.2)}
\label{sec:type:GroupElement:getEncoded}
\noindent
\begin{tabularx}{\textwidth}{| l | X |}
   \hline
   \bf{Description} & Get an encoding of the point value. \\
  
  \hline
  \bf{Parameters} &
      \(\begin{array}{l l l}
         
      \end{array}\) \\
       
  \hline
  \bf{Result} & \lst{Coll[Byte]} \\
  \hline
  
  \bf{Serialized as} & \hyperref[sec:serialization:operation:PropertyCall]{\lst{PropertyCall}} \\
  \hline
       
\end{tabularx}



\subsubsection{\lst{GroupElement.exp} method (Code 7.3)}
\label{sec:type:GroupElement:exp}
\noindent
\begin{tabularx}{\textwidth}{| l | X |}
   \hline
   \bf{Description} & Exponentiate this \lst{GroupElement} to the given number. Returns this to the power of k \\
  
  \hline
  \bf{Parameters} &
      \(\begin{array}{l l l}
         \lst{k} & \lst{: BigInt} & \text{// The power} \\
      \end{array}\) \\
       
  \hline
  \bf{Result} & \lst{GroupElement} \\
  \hline
  
  \bf{Serialized as} & \hyperref[sec:serialization:operation:Exponentiate]{\lst{Exponentiate}} \\
  \hline
       
\end{tabularx}



\subsubsection{\lst{GroupElement.multiply} method (Code 7.4)}
\label{sec:type:GroupElement:multiply}
\noindent
\begin{tabularx}{\textwidth}{| l | X |}
   \hline
   \bf{Description} & Group operation. \\
  
  \hline
  \bf{Parameters} &
      \(\begin{array}{l l l}
         \lst{other} & \lst{: GroupElement} & \text{// other element of the group} \\
      \end{array}\) \\
       
  \hline
  \bf{Result} & \lst{GroupElement} \\
  \hline
  
  \bf{Serialized as} & \hyperref[sec:serialization:operation:MultiplyGroup]{\lst{MultiplyGroup}} \\
  \hline
       
\end{tabularx}



\subsubsection{\lst{GroupElement.negate} method (Code 7.5)}
\label{sec:type:GroupElement:negate}
\noindent
\begin{tabularx}{\textwidth}{| l | X |}
   \hline
   \bf{Description} & Inverse element of the group. \\
  
  \hline
  \bf{Parameters} &
      \(\begin{array}{l l l}
         
      \end{array}\) \\
       
  \hline
  \bf{Result} & \lst{GroupElement} \\
  \hline
  
  \bf{Serialized as} & \hyperref[sec:serialization:operation:PropertyCall]{\lst{PropertyCall}} \\
  \hline
       
\end{tabularx}


\subsection{SigmaProp type}
\label{sec:type:SigmaProp}

Values of \lst{SigmaProp} type hold sigma propositions, which can be proved
and verified using Sigma protocols. Each sigma proposition is represented as
an expression where sigma protocol primitives such as \lst{ProveDlog}, and
\lst{ProveDHTuple} are used as constants and special sigma protocol
connectives like \lst{&&},\lst{||} and \lst{THRESHOLD} are used as operations.

The abstract syntax of sigma propositions is shown in
Figure~\ref{fig:sigmaprop:tree}.

\begin{figure}[h] \footnotesize
   \caption{Abstract syntax of sigma propositions}\vspace{-7pt}
   \label{fig:sigmaprop:tree}
   \centering
   \begin{tabular}{@{}l c l l l} 
      \hline
      Set 		&  			& Syntax	   & Mnemonic 	& Description \\
      \hline
      $Tree \ni t$	& := 	& \lst{Trivial(b)} 	& \lst{TrivialProp}	& boolean value \lst{b} as sigma proposition  \\
                     & $\mid$	& \lst{Dlog(ge)} 	& \lst{ProveDLog}	& knowledge of discrete logarithm of \lst{ge} \\
                     & $\mid$ & \lst{DHTuple(g,h,u,v)} 	& \lst{ProveDHTuple}	& knowledge of Diffie-Hellman tuple \\
                     & $\mid$ & \lst{THRESHOLD}$(k,t_1,\dots,t_n)$ 	& \lst{CTHRESHOLD}	& knowledge of $k$ out of $n$ secrets\\
                     & $\mid$ & \lst{OR}$(t_1,\dots,t_n)$ 	& \lst{COR}	& knowledge of any one of $n$ secrets\\
                     & $\mid$ & \lst{AND}$(t_1,\dots,t_n)$ 	& \lst{CAND}	& knowledge of all $n$ secrets\\
      \hline
   \end{tabular} 
\end{figure}

Every well-formed tree of sigma proposition is a value of type
\lst{SigmaProp}, thus following the notation of Section~\ref{sec:evaluation}
we can define a denotation of the \lst{SigmaProp} type (i.e. a set of possible values)

$$\Denot{\lst{SigmaProp}} = \Set{t \in Tree}$$


The following methods can be called on all instances of \lst{SigmaProp} type.


\subsubsection{\lst{SigmaProp.propBytes} method (Code 8.1)}
\noindent
\begin{tabularx}{\textwidth}{| l | X |}
   \hline
   \bf{Description} & Serialized bytes of this sigma proposition taken as ErgoTree. \\
  
  \hline
  \bf{Parameters} &
      \(\begin{array}{l l l}
         
      \end{array}\) \\
       
  \hline
  \bf{Result} & \lst{Coll[Byte]} \\
  \hline
  
  \bf{Serialized as} & \lst{SigmaPropBytes(opCode=208)} \\
  \hline
       
\end{tabularx}



\subsubsection{\lst{SigmaProp.isProven} method (Code 8.2)}
\noindent
\begin{tabularx}{\textwidth}{| l | X |}
   \hline
   \bf{Description} & Verify that sigma proposition is proven. (FRONTEND ONLY) \\
  
  \hline
  \bf{Parameters} &
      \(\begin{array}{l l l}
         
      \end{array}\) \\
       
  \hline
  \bf{Result} & \lst{Boolean} \\
  \hline
  
\end{tabularx}


Additionally, for a list of primitive operations on \lst{SigmaProp} type see
Appendix~\ref{sec:appendix:primops}.

\subsection{Box type}
\label{sec:type:Box}
 
\subsubsection{\lst{Box.value} method (Code 99.1)}
\label{sec:type:Box:value}
\noindent
\begin{tabularx}{\textwidth}{| l | X |}
   \hline
   \bf{Description} & Monetary value in NanoERGs stored in this box. \\
   \hline
   \bf{Signature} & \lst{def value}: \lst{Long} \\
  
  \hline
  
  \bf{Serialized as} & \hyperref[sec:serialization:operation:ExtractAmount]{\lst{ExtractAmount}} \\
  \hline
       
\end{tabularx}



\subsubsection{\lst{Box.propositionBytes} method (Code 99.2)}
\label{sec:type:Box:propositionBytes}
\noindent
\begin{tabularx}{\textwidth}{| l | X |}
   \hline
   \bf{Description} & Serialized bytes of the guarding script which should be evaluated to true in order to
 open this box (spend it in a transaction). \\
   \hline
   \bf{Signature} & \lst{def propositionBytes}: \lst{Coll[Byte]} \\
  
  \hline
  
  \bf{Serialized as} & \hyperref[sec:serialization:operation:ExtractScriptBytes]{\lst{ExtractScriptBytes}} \\
  \hline
       
\end{tabularx}



\subsubsection{\lst{Box.bytes} method (Code 99.3)}
\label{sec:type:Box:bytes}
\noindent
\begin{tabularx}{\textwidth}{| l | X |}
   \hline
   \bf{Description} & Serialized bytes of this box's content, including proposition bytes. \\
   \hline
   \bf{Signature} & \lst{def bytes}: \lst{Coll[Byte]} \\
  
  \hline
  
  \bf{Serialized as} & \hyperref[sec:serialization:operation:ExtractBytes]{\lst{ExtractBytes}} \\
  \hline
       
\end{tabularx}



\subsubsection{\lst{Box.bytesWithoutRef} method (Code 99.4)}
\label{sec:type:Box:bytesWithoutRef}
\noindent
\begin{tabularx}{\textwidth}{| l | X |}
   \hline
   \bf{Description} & Serialized bytes of this box's content, excluding transactionId and index of output. \\
   \hline
   \bf{Signature} & \lst{def bytesWithoutRef}: \lst{Coll[Byte]} \\
  
  \hline
  
  \bf{Serialized as} & \hyperref[sec:serialization:operation:ExtractBytesWithNoRef]{\lst{ExtractBytesWithNoRef}} \\
  \hline
       
\end{tabularx}



\subsubsection{\lst{Box.id} method (Code 99.5)}
\label{sec:type:Box:id}
\noindent
\begin{tabularx}{\textwidth}{| l | X |}
   \hline
   \bf{Description} & Blake2b256 hash of this box's content, basically equals to \lst{blake2b256(bytes)} \\
   \hline
   \bf{Signature} & \lst{def id}: \lst{Coll[Byte]} \\
  
  \hline
  
  \bf{Serialized as} & \hyperref[sec:serialization:operation:ExtractId]{\lst{ExtractId}} \\
  \hline
       
\end{tabularx}



\subsubsection{\lst{Box.creationInfo} method (Code 99.6)}
\label{sec:type:Box:creationInfo}
\noindent
\begin{tabularx}{\textwidth}{| l | X |}
   \hline
   \bf{Description} &  If \lst{tx} is a transaction which generated this box, then \lst{creationInfo._1}
 is a height of the tx's block. The \lst{creationInfo._2} is a serialized transaction
 identifier followed by box index in the transaction outputs.
         \\
   \hline
   \bf{Signature} & \lst{def creationInfo}: \lst{(Int,Coll[Byte])} \\
  
  \hline
  
  \bf{Serialized as} & \hyperref[sec:serialization:operation:ExtractCreationInfo]{\lst{ExtractCreationInfo}} \\
  \hline
       
\end{tabularx}



\subsubsection{\lst{Box.tokens} method (Code 99.8)}
\label{sec:type:Box:tokens}
\noindent
\begin{tabularx}{\textwidth}{| l | X |}
   \hline
   \bf{Description} & Secondary tokens \\
   \hline
   \bf{Signature} & \lst{def tokens}: \lst{Coll[(Coll[Byte],Long)]} \\
  
  \hline
  
  \bf{Serialized as} & \hyperref[sec:serialization:operation:PropertyCall]{\lst{PropertyCall}} \\
  \hline
       
\end{tabularx}



\subsubsection{\lst{Box.R0} method (Code 99.9)}
\label{sec:type:Box:R0}
\noindent
\begin{tabularx}{\textwidth}{| l | X |}
   \hline
   \bf{Description} & Monetary value, in Ergo tokens \\
   \hline
   \bf{Signature} & \lst{def R0}$[$\lst{T}$]$: \lst{Option[T]} \\
  
  \hline
  
  \bf{Serialized as} & \hyperref[sec:serialization:operation:ExtractRegisterAs]{\lst{ExtractRegisterAs}} \\
  \hline
       
\end{tabularx}



\subsubsection{\lst{Box.R1} method (Code 99.10)}
\label{sec:type:Box:R1}
\noindent
\begin{tabularx}{\textwidth}{| l | X |}
   \hline
   \bf{Description} & Guarding script \\
   \hline
   \bf{Signature} & \lst{def R1}$[$\lst{T}$]$: \lst{Option[T]} \\
  
  \hline
  
  \bf{Serialized as} & \hyperref[sec:serialization:operation:ExtractRegisterAs]{\lst{ExtractRegisterAs}} \\
  \hline
       
\end{tabularx}



\subsubsection{\lst{Box.R2} method (Code 99.11)}
\label{sec:type:Box:R2}
\noindent
\begin{tabularx}{\textwidth}{| l | X |}
   \hline
   \bf{Description} & Secondary tokens \\
   \hline
   \bf{Signature} & \lst{def R2}$[$\lst{T}$]$: \lst{Option[T]} \\
  
  \hline
  
  \bf{Serialized as} & \hyperref[sec:serialization:operation:ExtractRegisterAs]{\lst{ExtractRegisterAs}} \\
  \hline
       
\end{tabularx}



\subsubsection{\lst{Box.R3} method (Code 99.12)}
\label{sec:type:Box:R3}
\noindent
\begin{tabularx}{\textwidth}{| l | X |}
   \hline
   \bf{Description} & Reference to transaction and output id where the box was created \\
   \hline
   \bf{Signature} & \lst{def R3}$[$\lst{T}$]$: \lst{Option[T]} \\
  
  \hline
  
  \bf{Serialized as} & \hyperref[sec:serialization:operation:ExtractRegisterAs]{\lst{ExtractRegisterAs}} \\
  \hline
       
\end{tabularx}



\subsubsection{\lst{Box.R4} method (Code 99.13)}
\label{sec:type:Box:R4}
\noindent
\begin{tabularx}{\textwidth}{| l | X |}
   \hline
   \bf{Description} & Non-mandatory register \\
   \hline
   \bf{Signature} & \lst{def R4}$[$\lst{T}$]$: \lst{Option[T]} \\
  
  \hline
  
  \bf{Serialized as} & \hyperref[sec:serialization:operation:ExtractRegisterAs]{\lst{ExtractRegisterAs}} \\
  \hline
       
\end{tabularx}



\subsubsection{\lst{Box.R5} method (Code 99.14)}
\label{sec:type:Box:R5}
\noindent
\begin{tabularx}{\textwidth}{| l | X |}
   \hline
   \bf{Description} & Non-mandatory register \\
   \hline
   \bf{Signature} & \lst{def R5}$[$\lst{T}$]$: \lst{Option[T]} \\
  
  \hline
  
  \bf{Serialized as} & \hyperref[sec:serialization:operation:ExtractRegisterAs]{\lst{ExtractRegisterAs}} \\
  \hline
       
\end{tabularx}



\subsubsection{\lst{Box.R6} method (Code 99.15)}
\label{sec:type:Box:R6}
\noindent
\begin{tabularx}{\textwidth}{| l | X |}
   \hline
   \bf{Description} & Non-mandatory register \\
   \hline
   \bf{Signature} & \lst{def R6}$[$\lst{T}$]$: \lst{Option[T]} \\
  
  \hline
  
  \bf{Serialized as} & \hyperref[sec:serialization:operation:ExtractRegisterAs]{\lst{ExtractRegisterAs}} \\
  \hline
       
\end{tabularx}



\subsubsection{\lst{Box.R7} method (Code 99.16)}
\label{sec:type:Box:R7}
\noindent
\begin{tabularx}{\textwidth}{| l | X |}
   \hline
   \bf{Description} & Non-mandatory register \\
   \hline
   \bf{Signature} & \lst{def R7}$[$\lst{T}$]$: \lst{Option[T]} \\
  
  \hline
  
  \bf{Serialized as} & \hyperref[sec:serialization:operation:ExtractRegisterAs]{\lst{ExtractRegisterAs}} \\
  \hline
       
\end{tabularx}



\subsubsection{\lst{Box.R8} method (Code 99.17)}
\label{sec:type:Box:R8}
\noindent
\begin{tabularx}{\textwidth}{| l | X |}
   \hline
   \bf{Description} & Non-mandatory register \\
   \hline
   \bf{Signature} & \lst{def R8}$[$\lst{T}$]$: \lst{Option[T]} \\
  
  \hline
  
  \bf{Serialized as} & \hyperref[sec:serialization:operation:ExtractRegisterAs]{\lst{ExtractRegisterAs}} \\
  \hline
       
\end{tabularx}



\subsubsection{\lst{Box.R9} method (Code 99.18)}
\label{sec:type:Box:R9}
\noindent
\begin{tabularx}{\textwidth}{| l | X |}
   \hline
   \bf{Description} & Non-mandatory register \\
   \hline
   \bf{Signature} & \lst{def R9}$[$\lst{T}$]$: \lst{Option[T]} \\
  
  \hline
  
  \bf{Serialized as} & \hyperref[sec:serialization:operation:ExtractRegisterAs]{\lst{ExtractRegisterAs}} \\
  \hline
       
\end{tabularx}


\subsection{\lst{AvlTree} type}
\label{sec:type:AvlTree}
 
\subsubsection{\lst{AvlTree.digest} method (Code 100.1)}
\noindent
\begin{tabularx}{\textwidth}{| l | X |}
   \hline
   \bf{Description} &  \\
  
  \hline
  \bf{Result} & \lst{Coll[Byte]} \\
  \hline
  
  \bf{Serialized as} & \lst{PropertyCall(opCode=219)} \\
  \hline
       
\end{tabularx}



\subsubsection{\lst{AvlTree.enabledOperations} method (Code 100.2)}
\noindent
\begin{tabularx}{\textwidth}{| l | X |}
   \hline
   \bf{Description} &  \\
  
  \hline
  \bf{Result} & \lst{Byte} \\
  \hline
  
  \bf{Serialized as} & \lst{PropertyCall(opCode=219)} \\
  \hline
       
\end{tabularx}



\subsubsection{\lst{AvlTree.keyLength} method (Code 100.3)}
\noindent
\begin{tabularx}{\textwidth}{| l | X |}
   \hline
   \bf{Description} &  \\
  
  \hline
  \bf{Result} & \lst{Int} \\
  \hline
  
  \bf{Serialized as} & \lst{PropertyCall(opCode=219)} \\
  \hline
       
\end{tabularx}



\subsubsection{\lst{AvlTree.valueLengthOpt} method (Code 100.4)}
\noindent
\begin{tabularx}{\textwidth}{| l | X |}
   \hline
   \bf{Description} &  \\
  
  \hline
  \bf{Result} & \lst{Option[Int]} \\
  \hline
  
  \bf{Serialized as} & \lst{PropertyCall(opCode=219)} \\
  \hline
       
\end{tabularx}



\subsubsection{\lst{AvlTree.isInsertAllowed} method (Code 100.5)}
\noindent
\begin{tabularx}{\textwidth}{| l | X |}
   \hline
   \bf{Description} &  \\
  
  \hline
  \bf{Result} & \lst{Boolean} \\
  \hline
  
  \bf{Serialized as} & \lst{PropertyCall(opCode=219)} \\
  \hline
       
\end{tabularx}



\subsubsection{\lst{AvlTree.isUpdateAllowed} method (Code 100.6)}
\noindent
\begin{tabularx}{\textwidth}{| l | X |}
   \hline
   \bf{Description} &  \\
  
  \hline
  \bf{Result} & \lst{Boolean} \\
  \hline
  
  \bf{Serialized as} & \lst{PropertyCall(opCode=219)} \\
  \hline
       
\end{tabularx}



\subsubsection{\lst{AvlTree.isRemoveAllowed} method (Code 100.7)}
\noindent
\begin{tabularx}{\textwidth}{| l | X |}
   \hline
   \bf{Description} &  \\
  
  \hline
  \bf{Result} & \lst{Boolean} \\
  \hline
  
  \bf{Serialized as} & \lst{PropertyCall(opCode=219)} \\
  \hline
       
\end{tabularx}



\subsubsection{\lst{AvlTree.updateOperations} method (Code 100.8)}
\noindent
\begin{tabularx}{\textwidth}{| l | X |}
   \hline
   \bf{Description} &  \\
  
  \hline
  \bf{Parameters} &
      \(\begin{array}{l l l}
         \lst{arg0} & \lst{: Byte} & \text{// } \\
      \end{array}\) \\
       
  \hline
  \bf{Result} & \lst{AvlTree} \\
  \hline
  
  \bf{Serialized as} & \lst{MethodCall(opCode=220)} \\
  \hline
       
\end{tabularx}



\subsubsection{\lst{AvlTree.contains} method (Code 100.9)}
\noindent
\begin{tabularx}{\textwidth}{| l | X |}
   \hline
   \bf{Description} &  \\
  
  \hline
  \bf{Parameters} &
      \(\begin{array}{l l l}
         \lst{arg0} & \lst{: Coll[Byte]} & \text{// } \\
\lst{arg1} & \lst{: Coll[Byte]} & \text{// } \\
      \end{array}\) \\
       
  \hline
  \bf{Result} & \lst{Boolean} \\
  \hline
  
  \bf{Serialized as} & \lst{MethodCall(opCode=220)} \\
  \hline
       
\end{tabularx}



\subsubsection{\lst{AvlTree.get} method (Code 100.10)}
\noindent
\begin{tabularx}{\textwidth}{| l | X |}
   \hline
   \bf{Description} &  \\
  
  \hline
  \bf{Parameters} &
      \(\begin{array}{l l l}
         \lst{arg0} & \lst{: Coll[Byte]} & \text{// } \\
\lst{arg1} & \lst{: Coll[Byte]} & \text{// } \\
      \end{array}\) \\
       
  \hline
  \bf{Result} & \lst{Option[Coll[Byte]]} \\
  \hline
  
  \bf{Serialized as} & \lst{MethodCall(opCode=220)} \\
  \hline
       
\end{tabularx}



\subsubsection{\lst{AvlTree.getMany} method (Code 100.11)}
\noindent
\begin{tabularx}{\textwidth}{| l | X |}
   \hline
   \bf{Description} &  \\
  
  \hline
  \bf{Parameters} &
      \(\begin{array}{l l l}
         \lst{arg0} & \lst{: Coll[Coll[Byte]]} & \text{// } \\
\lst{arg1} & \lst{: Coll[Byte]} & \text{// } \\
      \end{array}\) \\
       
  \hline
  \bf{Result} & \lst{Coll[Option[Coll[Byte]]]} \\
  \hline
  
  \bf{Serialized as} & \lst{MethodCall(opCode=220)} \\
  \hline
       
\end{tabularx}



\subsubsection{\lst{AvlTree.insert} method (Code 100.12)}
\noindent
\begin{tabularx}{\textwidth}{| l | X |}
   \hline
   \bf{Description} &  \\
  
  \hline
  \bf{Parameters} &
      \(\begin{array}{l l l}
         \lst{arg0} & \lst{: Coll[(Coll[Byte],Coll[Byte])]} & \text{// } \\
\lst{arg1} & \lst{: Coll[Byte]} & \text{// } \\
      \end{array}\) \\
       
  \hline
  \bf{Result} & \lst{Option[AvlTree]} \\
  \hline
  
  \bf{Serialized as} & \lst{MethodCall(opCode=220)} \\
  \hline
       
\end{tabularx}



\subsubsection{\lst{AvlTree.update} method (Code 100.13)}
\noindent
\begin{tabularx}{\textwidth}{| l | X |}
   \hline
   \bf{Description} &  \\
  
  \hline
  \bf{Parameters} &
      \(\begin{array}{l l l}
         \lst{arg0} & \lst{: Coll[(Coll[Byte],Coll[Byte])]} & \text{// } \\
\lst{arg1} & \lst{: Coll[Byte]} & \text{// } \\
      \end{array}\) \\
       
  \hline
  \bf{Result} & \lst{Option[AvlTree]} \\
  \hline
  
  \bf{Serialized as} & \lst{MethodCall(opCode=220)} \\
  \hline
       
\end{tabularx}



\subsubsection{\lst{AvlTree.remove} method (Code 100.14)}
\noindent
\begin{tabularx}{\textwidth}{| l | X |}
   \hline
   \bf{Description} &  \\
  
  \hline
  \bf{Parameters} &
      \(\begin{array}{l l l}
         \lst{arg0} & \lst{: Coll[Coll[Byte]]} & \text{// } \\
\lst{arg1} & \lst{: Coll[Byte]} & \text{// } \\
      \end{array}\) \\
       
  \hline
  \bf{Result} & \lst{Option[AvlTree]} \\
  \hline
  
  \bf{Serialized as} & \lst{MethodCall(opCode=220)} \\
  \hline
       
\end{tabularx}



\subsubsection{\lst{AvlTree.updateDigest} method (Code 100.15)}
\noindent
\begin{tabularx}{\textwidth}{| l | X |}
   \hline
   \bf{Description} &  \\
  
  \hline
  \bf{Parameters} &
      \(\begin{array}{l l l}
         \lst{arg0} & \lst{: Coll[Byte]} & \text{// } \\
      \end{array}\) \\
       
  \hline
  \bf{Result} & \lst{AvlTree} \\
  \hline
  
  \bf{Serialized as} & \lst{MethodCall(opCode=220)} \\
  \hline
       
\end{tabularx}


\subsection{Header type}
\label{sec:type:Header}
 
\subsubsection{\lst{Header.id} method (Code 104.1)}
\noindent
\begin{tabularx}{\textwidth}{| l | X |}
   \hline
   \bf{Description} &  \\
  
  \hline
  \bf{Result} & \lst{Coll[Byte]} \\
  \hline
  
  \bf{Serialized as} & \lst{PropertyCall(opCode=219)} \\
  \hline
       
\end{tabularx}



\subsubsection{\lst{Header.version} method (Code 104.2)}
\noindent
\begin{tabularx}{\textwidth}{| l | X |}
   \hline
   \bf{Description} &  \\
  
  \hline
  \bf{Result} & \lst{Byte} \\
  \hline
  
  \bf{Serialized as} & \lst{PropertyCall(opCode=219)} \\
  \hline
       
\end{tabularx}



\subsubsection{\lst{Header.parentId} method (Code 104.3)}
\noindent
\begin{tabularx}{\textwidth}{| l | X |}
   \hline
   \bf{Description} &  \\
  
  \hline
  \bf{Result} & \lst{Coll[Byte]} \\
  \hline
  
  \bf{Serialized as} & \lst{PropertyCall(opCode=219)} \\
  \hline
       
\end{tabularx}



\subsubsection{\lst{Header.ADProofsRoot} method (Code 104.4)}
\noindent
\begin{tabularx}{\textwidth}{| l | X |}
   \hline
   \bf{Description} &  \\
  
  \hline
  \bf{Result} & \lst{Coll[Byte]} \\
  \hline
  
  \bf{Serialized as} & \lst{PropertyCall(opCode=219)} \\
  \hline
       
\end{tabularx}



\subsubsection{\lst{Header.stateRoot} method (Code 104.5)}
\noindent
\begin{tabularx}{\textwidth}{| l | X |}
   \hline
   \bf{Description} &  \\
  
  \hline
  \bf{Result} & \lst{AvlTree} \\
  \hline
  
  \bf{Serialized as} & \lst{PropertyCall(opCode=219)} \\
  \hline
       
\end{tabularx}



\subsubsection{\lst{Header.transactionsRoot} method (Code 104.6)}
\noindent
\begin{tabularx}{\textwidth}{| l | X |}
   \hline
   \bf{Description} &  \\
  
  \hline
  \bf{Result} & \lst{Coll[Byte]} \\
  \hline
  
  \bf{Serialized as} & \lst{PropertyCall(opCode=219)} \\
  \hline
       
\end{tabularx}



\subsubsection{\lst{Header.timestamp} method (Code 104.7)}
\noindent
\begin{tabularx}{\textwidth}{| l | X |}
   \hline
   \bf{Description} &  \\
  
  \hline
  \bf{Result} & \lst{Long} \\
  \hline
  
  \bf{Serialized as} & \lst{PropertyCall(opCode=219)} \\
  \hline
       
\end{tabularx}



\subsubsection{\lst{Header.nBits} method (Code 104.8)}
\noindent
\begin{tabularx}{\textwidth}{| l | X |}
   \hline
   \bf{Description} &  \\
  
  \hline
  \bf{Result} & \lst{Long} \\
  \hline
  
  \bf{Serialized as} & \lst{PropertyCall(opCode=219)} \\
  \hline
       
\end{tabularx}



\subsubsection{\lst{Header.height} method (Code 104.9)}
\noindent
\begin{tabularx}{\textwidth}{| l | X |}
   \hline
   \bf{Description} &  \\
  
  \hline
  \bf{Result} & \lst{Int} \\
  \hline
  
  \bf{Serialized as} & \lst{PropertyCall(opCode=219)} \\
  \hline
       
\end{tabularx}



\subsubsection{\lst{Header.extensionRoot} method (Code 104.10)}
\noindent
\begin{tabularx}{\textwidth}{| l | X |}
   \hline
   \bf{Description} &  \\
  
  \hline
  \bf{Result} & \lst{Coll[Byte]} \\
  \hline
  
  \bf{Serialized as} & \lst{PropertyCall(opCode=219)} \\
  \hline
       
\end{tabularx}



\subsubsection{\lst{Header.minerPk} method (Code 104.11)}
\noindent
\begin{tabularx}{\textwidth}{| l | X |}
   \hline
   \bf{Description} &  \\
  
  \hline
  \bf{Result} & \lst{GroupElement} \\
  \hline
  
  \bf{Serialized as} & \lst{PropertyCall(opCode=219)} \\
  \hline
       
\end{tabularx}



\subsubsection{\lst{Header.powOnetimePk} method (Code 104.12)}
\noindent
\begin{tabularx}{\textwidth}{| l | X |}
   \hline
   \bf{Description} &  \\
  
  \hline
  \bf{Result} & \lst{GroupElement} \\
  \hline
  
  \bf{Serialized as} & \lst{PropertyCall(opCode=219)} \\
  \hline
       
\end{tabularx}



\subsubsection{\lst{Header.powNonce} method (Code 104.13)}
\noindent
\begin{tabularx}{\textwidth}{| l | X |}
   \hline
   \bf{Description} &  \\
  
  \hline
  \bf{Result} & \lst{Coll[Byte]} \\
  \hline
  
  \bf{Serialized as} & \lst{PropertyCall(opCode=219)} \\
  \hline
       
\end{tabularx}



\subsubsection{\lst{Header.powDistance} method (Code 104.14)}
\noindent
\begin{tabularx}{\textwidth}{| l | X |}
   \hline
   \bf{Description} &  \\
  
  \hline
  \bf{Result} & \lst{BigInt} \\
  \hline
  
  \bf{Serialized as} & \lst{PropertyCall(opCode=219)} \\
  \hline
       
\end{tabularx}



\subsubsection{\lst{Header.votes} method (Code 104.15)}
\noindent
\begin{tabularx}{\textwidth}{| l | X |}
   \hline
   \bf{Description} &  \\
  
  \hline
  \bf{Result} & \lst{Coll[Byte]} \\
  \hline
  
  \bf{Serialized as} & \lst{PropertyCall(opCode=219)} \\
  \hline
       
\end{tabularx}


\subsection{PreHeader type}
\label{sec:type:PreHeader}
 
\subsubsection{\lst{PreHeader.version} method (Code 105.1)}
\noindent
\begin{tabularx}{\textwidth}{| l | X |}
   \hline
   \bf{Description} &  \\
  
  \hline
  \bf{Result} & \lst{Byte} \\
  \hline
  
  \bf{Serialized as} & \lst{PropertyCall(opCode=219)} \\
  \hline
       
\end{tabularx}



\subsubsection{\lst{PreHeader.parentId} method (Code 105.2)}
\noindent
\begin{tabularx}{\textwidth}{| l | X |}
   \hline
   \bf{Description} &  \\
  
  \hline
  \bf{Result} & \lst{Coll[Byte]} \\
  \hline
  
  \bf{Serialized as} & \lst{PropertyCall(opCode=219)} \\
  \hline
       
\end{tabularx}



\subsubsection{\lst{PreHeader.timestamp} method (Code 105.3)}
\noindent
\begin{tabularx}{\textwidth}{| l | X |}
   \hline
   \bf{Description} &  \\
  
  \hline
  \bf{Result} & \lst{Long} \\
  \hline
  
  \bf{Serialized as} & \lst{PropertyCall(opCode=219)} \\
  \hline
       
\end{tabularx}



\subsubsection{\lst{PreHeader.nBits} method (Code 105.4)}
\noindent
\begin{tabularx}{\textwidth}{| l | X |}
   \hline
   \bf{Description} &  \\
  
  \hline
  \bf{Result} & \lst{Long} \\
  \hline
  
  \bf{Serialized as} & \lst{PropertyCall(opCode=219)} \\
  \hline
       
\end{tabularx}



\subsubsection{\lst{PreHeader.height} method (Code 105.5)}
\noindent
\begin{tabularx}{\textwidth}{| l | X |}
   \hline
   \bf{Description} &  \\
  
  \hline
  \bf{Result} & \lst{Int} \\
  \hline
  
  \bf{Serialized as} & \lst{PropertyCall(opCode=219)} \\
  \hline
       
\end{tabularx}



\subsubsection{\lst{PreHeader.minerPk} method (Code 105.6)}
\noindent
\begin{tabularx}{\textwidth}{| l | X |}
   \hline
   \bf{Description} &  \\
  
  \hline
  \bf{Result} & \lst{GroupElement} \\
  \hline
  
  \bf{Serialized as} & \lst{PropertyCall(opCode=219)} \\
  \hline
       
\end{tabularx}



\subsubsection{\lst{PreHeader.votes} method (Code 105.7)}
\noindent
\begin{tabularx}{\textwidth}{| l | X |}
   \hline
   \bf{Description} &  \\
  
  \hline
  \bf{Result} & \lst{Coll[Byte]} \\
  \hline
  
  \bf{Serialized as} & \lst{PropertyCall(opCode=219)} \\
  \hline
       
\end{tabularx}


\subsection{Context type}
\label{sec:type:Context}

\noindent
\begin{tabularx}{\textwidth}{| c | c | X |}
  \hline
  \bf{Code} & \bf{Method Signature} & \bf{Description} \\
  \hline
  101.1 & \lst{def dataInputs()} &  \\
\hline
101.2 & \lst{def headers()} &  \\
\hline
101.3 & \lst{def preHeader()} &  \\
\hline
101.4 & \lst{def INPUTS()} &  \\
\hline
101.5 & \lst{def OUTPUTS()} &  \\
\hline
101.6 & \lst{def HEIGHT()} &  \\
\hline
101.7 & \lst{def SELF()} &  \\
\hline
101.8 & \lst{def selfBoxIndex()} &  \\
\hline
101.9 & \lst{def LastBlockUtxoRootHash()} &  \\
\hline
101.10 & \lst{def minerPubKey()} &  \\
\hline
101.11 & \lst{def getVar()} &  \\
  \hline
\end{tabularx}
     

\subsection{Global type}
\label{sec:type:Global}

\subsubsection{\lst{SigmaDslBuilder.groupGenerator} method (Code 106.1)}
\label{sec:type:SigmaDslBuilder:groupGenerator}
\noindent
\begin{tabularx}{\textwidth}{| l | X |}
   \hline
   \bf{Description} &  \\
  
  \hline
  \bf{Parameters} &
      \(\begin{array}{l l l}
         
      \end{array}\) \\
       
  \hline
  \bf{Result} & \lst{GroupElement} \\
  \hline
  
  \bf{Serialized as} & \hyperref[sec:serialization:operation:GroupGenerator]{\lst{GroupGenerator}} \\
  \hline
       
\end{tabularx}



\subsubsection{\lst{SigmaDslBuilder.xor} method (Code 106.2)}
\label{sec:type:SigmaDslBuilder:xor}
\noindent
\begin{tabularx}{\textwidth}{| l | X |}
   \hline
   \bf{Description} & Byte-wise XOR of two collections of bytes \\
  
  \hline
  \bf{Parameters} &
      \(\begin{array}{l l l}
         \lst{left} & \lst{: Coll[Byte]} & \text{// left operand} \\
\lst{right} & \lst{: Coll[Byte]} & \text{// right operand} \\
      \end{array}\) \\
       
  \hline
  \bf{Result} & \lst{Coll[Byte]} \\
  \hline
  
  \bf{Serialized as} & \hyperref[sec:serialization:operation:Xor]{\lst{Xor}} \\
  \hline
       
\end{tabularx}


\subsection{Coll type}
\label{sec:type:Coll}
 
\subsubsection{\lst{SCollection.size} method (Code 12.1)}
\label{sec:type:SCollection:size}
\noindent
\begin{tabularx}{\textwidth}{| l | X |}
   \hline
   \bf{Description} & The size of the collection in elements. \\
  
  \hline
  \bf{Parameters} &
      \(\begin{array}{l l l}
         
      \end{array}\) \\
       
  \hline
  \bf{Result} & \lst{Int} \\
  \hline
  
  \bf{Serialized as} & \hyperref[sec:serialization:operation:SizeOf]{\lst{SizeOf}} \\
  \hline
       
\end{tabularx}



\subsubsection{\lst{SCollection.getOrElse} method (Code 12.2)}
\label{sec:type:SCollection:getOrElse}
\noindent
\begin{tabularx}{\textwidth}{| l | X |}
   \hline
   \bf{Description} & Return the element of collection if \lst{index} is in range \lst{0 .. size-1} \\
  
  \hline
  \bf{Parameters} &
      \(\begin{array}{l l l}
         \lst{index} & \lst{: Int} & \text{// index of the element of this collection} \\
\lst{default} & \lst{: IV} & \text{// value to return when \lst{index} is out of range} \\
      \end{array}\) \\
       
  \hline
  \bf{Result} & \lst{IV} \\
  \hline
  
  \bf{Serialized as} & \hyperref[sec:serialization:operation:ByIndex]{\lst{ByIndex}} \\
  \hline
       
\end{tabularx}



\subsubsection{\lst{SCollection.map} method (Code 12.3)}
\label{sec:type:SCollection:map}
\noindent
\begin{tabularx}{\textwidth}{| l | X |}
   \hline
   \bf{Description} &  Builds a new collection by applying a function to all elements of this collection.
 Returns a new collection of type \lst{Coll[B]} resulting from applying the given function
 \lst{f} to each element of this collection and collecting the results.
         \\
  
  \hline
  \bf{Parameters} &
      \(\begin{array}{l l l}
         \lst{f} & \lst{: (IV) => OV} & \text{// the function to apply to each element} \\
      \end{array}\) \\
       
  \hline
  \bf{Result} & \lst{Coll[OV]} \\
  \hline
  
  \bf{Serialized as} & \hyperref[sec:serialization:operation:MapCollection]{\lst{MapCollection}} \\
  \hline
       
\end{tabularx}



\subsubsection{\lst{SCollection.exists} method (Code 12.4)}
\label{sec:type:SCollection:exists}
\noindent
\begin{tabularx}{\textwidth}{| l | X |}
   \hline
   \bf{Description} & Tests whether a predicate holds for at least one element of this collection.
Returns \lst{true} if the given predicate \lst{p} is satisfied by at least one element of this collection, otherwise \lst{false}
         \\
  
  \hline
  \bf{Parameters} &
      \(\begin{array}{l l l}
         \lst{p} & \lst{: (IV) => Boolean} & \text{// the predicate used to test elements} \\
      \end{array}\) \\
       
  \hline
  \bf{Result} & \lst{Boolean} \\
  \hline
  
  \bf{Serialized as} & \hyperref[sec:serialization:operation:Exists]{\lst{Exists}} \\
  \hline
       
\end{tabularx}



\subsubsection{\lst{SCollection.fold} method (Code 12.5)}
\label{sec:type:SCollection:fold}
\noindent
\begin{tabularx}{\textwidth}{| l | X |}
   \hline
   \bf{Description} & Applies a binary operator to a start value and all elements of this collection, going left to right. \\
  
  \hline
  \bf{Parameters} &
      \(\begin{array}{l l l}
         \lst{zero} & \lst{: OV} & \text{// a starting value} \\
\lst{op} & \lst{: (OV,IV) => OV} & \text{// the binary operator} \\
      \end{array}\) \\
       
  \hline
  \bf{Result} & \lst{OV} \\
  \hline
  
  \bf{Serialized as} & \hyperref[sec:serialization:operation:Fold]{\lst{Fold}} \\
  \hline
       
\end{tabularx}



\subsubsection{\lst{SCollection.forall} method (Code 12.6)}
\label{sec:type:SCollection:forall}
\noindent
\begin{tabularx}{\textwidth}{| l | X |}
   \hline
   \bf{Description} & Tests whether a predicate holds for all elements of this collection.
Returns \lst{true} if this collection is empty or the given predicate \lst{p}
holds for all elements of this collection, otherwise \lst{false}.
         \\
  
  \hline
  \bf{Parameters} &
      \(\begin{array}{l l l}
         \lst{p} & \lst{: (IV) => Boolean} & \text{// the predicate used to test elements} \\
      \end{array}\) \\
       
  \hline
  \bf{Result} & \lst{Boolean} \\
  \hline
  
  \bf{Serialized as} & \hyperref[sec:serialization:operation:ForAll]{\lst{ForAll}} \\
  \hline
       
\end{tabularx}



\subsubsection{\lst{SCollection.slice} method (Code 12.7)}
\label{sec:type:SCollection:slice}
\noindent
\begin{tabularx}{\textwidth}{| l | X |}
   \hline
   \bf{Description} & Selects an interval of elements.  The returned collection is made up
  of all elements \lst{x} which satisfy the invariant:
  \lst{
     from <= indexOf(x) < until
  }
         \\
  
  \hline
  \bf{Parameters} &
      \(\begin{array}{l l l}
         \lst{from} & \lst{: Int} & \text{// the lowest index to include from this collection} \\
\lst{until} & \lst{: Int} & \text{// the lowest index to EXCLUDE from this collection} \\
      \end{array}\) \\
       
  \hline
  \bf{Result} & \lst{Coll[IV]} \\
  \hline
  
  \bf{Serialized as} & \hyperref[sec:serialization:operation:Slice]{\lst{Slice}} \\
  \hline
       
\end{tabularx}



\subsubsection{\lst{SCollection.filter} method (Code 12.8)}
\label{sec:type:SCollection:filter}
\noindent
\begin{tabularx}{\textwidth}{| l | X |}
   \hline
   \bf{Description} & Selects all elements of this collection which satisfy a predicate.
 Returns  a new collection consisting of all elements of this collection that satisfy the given
 predicate \lst{p}. The order of the elements is preserved.
         \\
  
  \hline
  \bf{Parameters} &
      \(\begin{array}{l l l}
         \lst{p} & \lst{: (IV) => Boolean} & \text{// the predicate used to test elements.} \\
      \end{array}\) \\
       
  \hline
  \bf{Result} & \lst{Coll[IV]} \\
  \hline
  
  \bf{Serialized as} & \hyperref[sec:serialization:operation:Filter]{\lst{Filter}} \\
  \hline
       
\end{tabularx}



\subsubsection{\lst{SCollection.append} method (Code 12.9)}
\label{sec:type:SCollection:append}
\noindent
\begin{tabularx}{\textwidth}{| l | X |}
   \hline
   \bf{Description} & Puts the elements of other collection after the elements of this collection (concatenation of 2 collections) \\
  
  \hline
  \bf{Parameters} &
      \(\begin{array}{l l l}
         \lst{other} & \lst{: Coll[IV]} & \text{// the collection to append at the end of this} \\
      \end{array}\) \\
       
  \hline
  \bf{Result} & \lst{Coll[IV]} \\
  \hline
  
  \bf{Serialized as} & \hyperref[sec:serialization:operation:Append]{\lst{Append}} \\
  \hline
       
\end{tabularx}



\subsubsection{\lst{SCollection.apply} method (Code 12.10)}
\label{sec:type:SCollection:apply}
\noindent
\begin{tabularx}{\textwidth}{| l | X |}
   \hline
   \bf{Description} & The element at given index.
 Indices start at \lst{0}; \lst{xs.apply(0)} is the first element of collection \lst{xs}.
 Note the indexing syntax \lst{xs(i)} is a shorthand for \lst{xs.apply(i)}.
 Returns the element at the given index.
 Throws an exception if \lst{i < 0} or \lst{length <= i}
         \\
  
  \hline
  \bf{Parameters} &
      \(\begin{array}{l l l}
         \lst{i} & \lst{: Int} & \text{// the index} \\
      \end{array}\) \\
       
  \hline
  \bf{Result} & \lst{IV} \\
  \hline
  
  \bf{Serialized as} & \hyperref[sec:serialization:operation:ByIndex]{\lst{ByIndex}} \\
  \hline
       
\end{tabularx}



\subsubsection{\lst{SCollection.<<} method (Code 12.11)}
\label{sec:type:SCollection:<<}
\noindent
\begin{tabularx}{\textwidth}{| l | X |}
   \hline
   \bf{Description} &  \\
  
  \hline
  \bf{Parameters} &
      \(\begin{array}{l l l}
         \lst{arg0} & \lst{: Coll[IV]} & \text{// } \\
\lst{arg1} & \lst{: Int} & \text{// } \\
      \end{array}\) \\
       
  \hline
  \bf{Result} & \lst{Coll[IV]} \\
  \hline
  
\end{tabularx}



\subsubsection{\lst{SCollection.>>} method (Code 12.12)}
\label{sec:type:SCollection:>>}
\noindent
\begin{tabularx}{\textwidth}{| l | X |}
   \hline
   \bf{Description} &  \\
  
  \hline
  \bf{Parameters} &
      \(\begin{array}{l l l}
         \lst{arg0} & \lst{: Coll[IV]} & \text{// } \\
\lst{arg1} & \lst{: Int} & \text{// } \\
      \end{array}\) \\
       
  \hline
  \bf{Result} & \lst{Coll[IV]} \\
  \hline
  
\end{tabularx}



\subsubsection{\lst{SCollection.>>>} method (Code 12.13)}
\label{sec:type:SCollection:>>>}
\noindent
\begin{tabularx}{\textwidth}{| l | X |}
   \hline
   \bf{Description} &  \\
  
  \hline
  \bf{Parameters} &
      \(\begin{array}{l l l}
         \lst{arg0} & \lst{: Coll[Boolean]} & \text{// } \\
\lst{arg1} & \lst{: Int} & \text{// } \\
      \end{array}\) \\
       
  \hline
  \bf{Result} & \lst{Coll[Boolean]} \\
  \hline
  
\end{tabularx}



\subsubsection{\lst{SCollection.indices} method (Code 12.14)}
\label{sec:type:SCollection:indices}
\noindent
\begin{tabularx}{\textwidth}{| l | X |}
   \hline
   \bf{Description} & Produces the range of all indices of this collection as a new collection
 containing [0 .. length-1] values.
         \\
  
  \hline
  \bf{Parameters} &
      \(\begin{array}{l l l}
         
      \end{array}\) \\
       
  \hline
  \bf{Result} & \lst{Coll[Int]} \\
  \hline
  
  \bf{Serialized as} & \hyperref[sec:serialization:operation:PropertyCall]{\lst{PropertyCall}} \\
  \hline
       
\end{tabularx}



\subsubsection{\lst{SCollection.flatMap} method (Code 12.15)}
\label{sec:type:SCollection:flatMap}
\noindent
\begin{tabularx}{\textwidth}{| l | X |}
   \hline
   \bf{Description} &  Builds a new collection by applying a function to all elements of this collection
 and using the elements of the resulting collections.
 Function \lst{f} is constrained to be of the form \lst{x => x.someProperty}, otherwise
 it is illegal.
 Returns a new collection of type \lst{Coll[B]} resulting from applying the given collection-valued function
 \lst{f} to each element of this collection and concatenating the results.
         \\
  
  \hline
  \bf{Parameters} &
      \(\begin{array}{l l l}
         \lst{f} & \lst{: (IV) => Coll[OV]} & \text{// the function to apply to each element.} \\
      \end{array}\) \\
       
  \hline
  \bf{Result} & \lst{Coll[OV]} \\
  \hline
  
  \bf{Serialized as} & \hyperref[sec:serialization:operation:MethodCall]{\lst{MethodCall}} \\
  \hline
       
\end{tabularx}



\subsubsection{\lst{SCollection.segmentLength} method (Code 12.16)}
\label{sec:type:SCollection:segmentLength}
\noindent
\begin{tabularx}{\textwidth}{| l | X |}
   \hline
   \bf{Description} & Computes length of longest segment whose elements all satisfy some predicate.
 Returns the length of the longest segment of this collection starting from index \lst{from}
 such that every element of the segment satisfies the predicate \lst{p}.
         \\
  
  \hline
  \bf{Parameters} &
      \(\begin{array}{l l l}
         \lst{p} & \lst{: (IV) => Boolean} & \text{// the predicate used to test elements.} \\
\lst{from} & \lst{: Int} & \text{// the index where the search starts.} \\
      \end{array}\) \\
       
  \hline
  \bf{Result} & \lst{Int} \\
  \hline
  
  \bf{Serialized as} & \hyperref[sec:serialization:operation:MethodCall]{\lst{MethodCall}} \\
  \hline
       
\end{tabularx}



\subsubsection{\lst{SCollection.indexWhere} method (Code 12.17)}
\label{sec:type:SCollection:indexWhere}
\noindent
\begin{tabularx}{\textwidth}{| l | X |}
   \hline
   \bf{Description} & Finds index of the first element satisfying some predicate after or at some start index.
 Returns the index \lst{>= from} of the first element of this collection that satisfies the predicate \lst{p},
 or \lst{-1}, if none exists.
         \\
  
  \hline
  \bf{Parameters} &
      \(\begin{array}{l l l}
         \lst{p} & \lst{: (IV) => Boolean} & \text{// the predicate used to test elements.} \\
\lst{from} & \lst{: Int} & \text{// the start index} \\
      \end{array}\) \\
       
  \hline
  \bf{Result} & \lst{Int} \\
  \hline
  
  \bf{Serialized as} & \hyperref[sec:serialization:operation:MethodCall]{\lst{MethodCall}} \\
  \hline
       
\end{tabularx}



\subsubsection{\lst{SCollection.lastIndexWhere} method (Code 12.18)}
\label{sec:type:SCollection:lastIndexWhere}
\noindent
\begin{tabularx}{\textwidth}{| l | X |}
   \hline
   \bf{Description} & Finds index of last element satisfying some predicate before or at given end index.
 Return the index \lst{<= end} of the last element of this collection that satisfies the predicate \lst{p},
 or \lst{-1}, if none exists.
         \\
  
  \hline
  \bf{Parameters} &
      \(\begin{array}{l l l}
         \lst{p} & \lst{: (IV) => Boolean} & \text{// the predicate used to test elements.} \\
      \end{array}\) \\
       
  \hline
  \bf{Result} & \lst{Int} \\
  \hline
  
  \bf{Serialized as} & \hyperref[sec:serialization:operation:MethodCall]{\lst{MethodCall}} \\
  \hline
       
\end{tabularx}



\subsubsection{\lst{SCollection.patch} method (Code 12.19)}
\label{sec:type:SCollection:patch}
\noindent
\begin{tabularx}{\textwidth}{| l | X |}
   \hline
   \bf{Description} &  \\
  
  \hline
  \bf{Parameters} &
      \(\begin{array}{l l l}
         
      \end{array}\) \\
       
  \hline
  \bf{Result} & \lst{Coll[IV]} \\
  \hline
  
  \bf{Serialized as} & \hyperref[sec:serialization:operation:MethodCall]{\lst{MethodCall}} \\
  \hline
       
\end{tabularx}



\subsubsection{\lst{SCollection.updated} method (Code 12.20)}
\label{sec:type:SCollection:updated}
\noindent
\begin{tabularx}{\textwidth}{| l | X |}
   \hline
   \bf{Description} &  \\
  
  \hline
  \bf{Parameters} &
      \(\begin{array}{l l l}
         
      \end{array}\) \\
       
  \hline
  \bf{Result} & \lst{Coll[IV]} \\
  \hline
  
  \bf{Serialized as} & \hyperref[sec:serialization:operation:MethodCall]{\lst{MethodCall}} \\
  \hline
       
\end{tabularx}



\subsubsection{\lst{SCollection.updateMany} method (Code 12.21)}
\label{sec:type:SCollection:updateMany}
\noindent
\begin{tabularx}{\textwidth}{| l | X |}
   \hline
   \bf{Description} &  \\
  
  \hline
  \bf{Parameters} &
      \(\begin{array}{l l l}
         
      \end{array}\) \\
       
  \hline
  \bf{Result} & \lst{Coll[IV]} \\
  \hline
  
  \bf{Serialized as} & \hyperref[sec:serialization:operation:MethodCall]{\lst{MethodCall}} \\
  \hline
       
\end{tabularx}



\subsubsection{\lst{SCollection.unionSets} method (Code 12.22)}
\label{sec:type:SCollection:unionSets}
\noindent
\begin{tabularx}{\textwidth}{| l | X |}
   \hline
   \bf{Description} &  \\
  
  \hline
  \bf{Parameters} &
      \(\begin{array}{l l l}
         
      \end{array}\) \\
       
  \hline
  \bf{Result} & \lst{Coll[IV]} \\
  \hline
  
  \bf{Serialized as} & \hyperref[sec:serialization:operation:MethodCall]{\lst{MethodCall}} \\
  \hline
       
\end{tabularx}



\subsubsection{\lst{SCollection.diff} method (Code 12.23)}
\label{sec:type:SCollection:diff}
\noindent
\begin{tabularx}{\textwidth}{| l | X |}
   \hline
   \bf{Description} &  \\
  
  \hline
  \bf{Parameters} &
      \(\begin{array}{l l l}
         
      \end{array}\) \\
       
  \hline
  \bf{Result} & \lst{Coll[IV]} \\
  \hline
  
  \bf{Serialized as} & \hyperref[sec:serialization:operation:MethodCall]{\lst{MethodCall}} \\
  \hline
       
\end{tabularx}



\subsubsection{\lst{SCollection.intersect} method (Code 12.24)}
\label{sec:type:SCollection:intersect}
\noindent
\begin{tabularx}{\textwidth}{| l | X |}
   \hline
   \bf{Description} &  \\
  
  \hline
  \bf{Parameters} &
      \(\begin{array}{l l l}
         
      \end{array}\) \\
       
  \hline
  \bf{Result} & \lst{Coll[IV]} \\
  \hline
  
  \bf{Serialized as} & \hyperref[sec:serialization:operation:MethodCall]{\lst{MethodCall}} \\
  \hline
       
\end{tabularx}



\subsubsection{\lst{SCollection.prefixLength} method (Code 12.25)}
\label{sec:type:SCollection:prefixLength}
\noindent
\begin{tabularx}{\textwidth}{| l | X |}
   \hline
   \bf{Description} &  \\
  
  \hline
  \bf{Parameters} &
      \(\begin{array}{l l l}
         
      \end{array}\) \\
       
  \hline
  \bf{Result} & \lst{Int} \\
  \hline
  
  \bf{Serialized as} & \hyperref[sec:serialization:operation:MethodCall]{\lst{MethodCall}} \\
  \hline
       
\end{tabularx}



\subsubsection{\lst{SCollection.indexOf} method (Code 12.26)}
\label{sec:type:SCollection:indexOf}
\noindent
\begin{tabularx}{\textwidth}{| l | X |}
   \hline
   \bf{Description} &  \\
  
  \hline
  \bf{Parameters} &
      \(\begin{array}{l l l}
         
      \end{array}\) \\
       
  \hline
  \bf{Result} & \lst{Int} \\
  \hline
  
  \bf{Serialized as} & \hyperref[sec:serialization:operation:MethodCall]{\lst{MethodCall}} \\
  \hline
       
\end{tabularx}



\subsubsection{\lst{SCollection.lastIndexOf} method (Code 12.27)}
\label{sec:type:SCollection:lastIndexOf}
\noindent
\begin{tabularx}{\textwidth}{| l | X |}
   \hline
   \bf{Description} &  \\
  
  \hline
  \bf{Parameters} &
      \(\begin{array}{l l l}
         
      \end{array}\) \\
       
  \hline
  \bf{Result} & \lst{Int} \\
  \hline
  
  \bf{Serialized as} & \hyperref[sec:serialization:operation:MethodCall]{\lst{MethodCall}} \\
  \hline
       
\end{tabularx}



\subsubsection{\lst{SCollection.find} method (Code 12.28)}
\label{sec:type:SCollection:find}
\noindent
\begin{tabularx}{\textwidth}{| l | X |}
   \hline
   \bf{Description} &  \\
  
  \hline
  \bf{Parameters} &
      \(\begin{array}{l l l}
         
      \end{array}\) \\
       
  \hline
  \bf{Result} & \lst{Option[IV]} \\
  \hline
  
  \bf{Serialized as} & \hyperref[sec:serialization:operation:MethodCall]{\lst{MethodCall}} \\
  \hline
       
\end{tabularx}



\subsubsection{\lst{SCollection.zip} method (Code 12.29)}
\label{sec:type:SCollection:zip}
\noindent
\begin{tabularx}{\textwidth}{| l | X |}
   \hline
   \bf{Description} &  \\
  
  \hline
  \bf{Parameters} &
      \(\begin{array}{l l l}
         
      \end{array}\) \\
       
  \hline
  \bf{Result} & \lst{Coll[(IV,OV)]} \\
  \hline
  
  \bf{Serialized as} & \hyperref[sec:serialization:operation:MethodCall]{\lst{MethodCall}} \\
  \hline
       
\end{tabularx}



\subsubsection{\lst{SCollection.distinct} method (Code 12.30)}
\label{sec:type:SCollection:distinct}
\noindent
\begin{tabularx}{\textwidth}{| l | X |}
   \hline
   \bf{Description} &  \\
  
  \hline
  \bf{Parameters} &
      \(\begin{array}{l l l}
         
      \end{array}\) \\
       
  \hline
  \bf{Result} & \lst{Coll[IV]} \\
  \hline
  
  \bf{Serialized as} & \hyperref[sec:serialization:operation:PropertyCall]{\lst{PropertyCall}} \\
  \hline
       
\end{tabularx}



\subsubsection{\lst{SCollection.startsWith} method (Code 12.31)}
\label{sec:type:SCollection:startsWith}
\noindent
\begin{tabularx}{\textwidth}{| l | X |}
   \hline
   \bf{Description} &  \\
  
  \hline
  \bf{Parameters} &
      \(\begin{array}{l l l}
         
      \end{array}\) \\
       
  \hline
  \bf{Result} & \lst{Boolean} \\
  \hline
  
  \bf{Serialized as} & \hyperref[sec:serialization:operation:MethodCall]{\lst{MethodCall}} \\
  \hline
       
\end{tabularx}



\subsubsection{\lst{SCollection.endsWith} method (Code 12.32)}
\label{sec:type:SCollection:endsWith}
\noindent
\begin{tabularx}{\textwidth}{| l | X |}
   \hline
   \bf{Description} &  \\
  
  \hline
  \bf{Parameters} &
      \(\begin{array}{l l l}
         
      \end{array}\) \\
       
  \hline
  \bf{Result} & \lst{Boolean} \\
  \hline
  
  \bf{Serialized as} & \hyperref[sec:serialization:operation:MethodCall]{\lst{MethodCall}} \\
  \hline
       
\end{tabularx}



\subsubsection{\lst{SCollection.partition} method (Code 12.33)}
\label{sec:type:SCollection:partition}
\noindent
\begin{tabularx}{\textwidth}{| l | X |}
   \hline
   \bf{Description} &  \\
  
  \hline
  \bf{Parameters} &
      \(\begin{array}{l l l}
         
      \end{array}\) \\
       
  \hline
  \bf{Result} & \lst{(Coll[IV],Coll[IV])} \\
  \hline
  
  \bf{Serialized as} & \hyperref[sec:serialization:operation:MethodCall]{\lst{MethodCall}} \\
  \hline
       
\end{tabularx}



\subsubsection{\lst{SCollection.mapReduce} method (Code 12.34)}
\label{sec:type:SCollection:mapReduce}
\noindent
\begin{tabularx}{\textwidth}{| l | X |}
   \hline
   \bf{Description} &  \\
  
  \hline
  \bf{Parameters} &
      \(\begin{array}{l l l}
         
      \end{array}\) \\
       
  \hline
  \bf{Result} & \lst{Coll[(K,V)]} \\
  \hline
  
  \bf{Serialized as} & \hyperref[sec:serialization:operation:MethodCall]{\lst{MethodCall}} \\
  \hline
       
\end{tabularx}


\subsection{Option type}
\label{sec:type:Option}

\subsubsection{\lst{SOption.isEmpty} method (Code 36.1)}
\label{sec:type:SOption:isEmpty}
\noindent
\begin{tabularx}{\textwidth}{| l | X |}
   \hline
   \bf{Description} &  \\
  
  \hline
  \bf{Parameters} &
      \(\begin{array}{l l l}
         \lst{arg0} & \lst{: Option[T]} & \text{// } \\
      \end{array}\) \\
       
  \hline
  \bf{Result} & \lst{Boolean} \\
  \hline
  
\end{tabularx}



\subsubsection{\lst{SOption.isDefined} method (Code 36.2)}
\label{sec:type:SOption:isDefined}
\noindent
\begin{tabularx}{\textwidth}{| l | X |}
   \hline
   \bf{Description} & Returns \lst{true} if the option is an instance of \lst{Some}, \lst{false} otherwise. \\
  
  \hline
  \bf{Parameters} &
      \(\begin{array}{l l l}
         
      \end{array}\) \\
       
  \hline
  \bf{Result} & \lst{Boolean} \\
  \hline
  
  \bf{Serialized as} & \hyperref[sec:serialization:operation:OptionIsDefined]{\lst{OptionIsDefined}} \\
  \hline
       
\end{tabularx}



\subsubsection{\lst{SOption.get} method (Code 36.3)}
\label{sec:type:SOption:get}
\noindent
\begin{tabularx}{\textwidth}{| l | X |}
   \hline
   \bf{Description} & Returns the option's value. The option must be nonempty. Throws exception if the option is empty. \\
  
  \hline
  \bf{Parameters} &
      \(\begin{array}{l l l}
         
      \end{array}\) \\
       
  \hline
  \bf{Result} & \lst{T} \\
  \hline
  
  \bf{Serialized as} & \hyperref[sec:serialization:operation:OptionGet]{\lst{OptionGet}} \\
  \hline
       
\end{tabularx}



\subsubsection{\lst{SOption.getOrElse} method (Code 36.4)}
\label{sec:type:SOption:getOrElse}
\noindent
\begin{tabularx}{\textwidth}{| l | X |}
   \hline
   \bf{Description} & Returns the option's value if the option is nonempty, otherwise
return the result of evaluating \lst{default}.
         \\
  
  \hline
  \bf{Parameters} &
      \(\begin{array}{l l l}
         \lst{default} & \lst{: T} & \text{// the default value} \\
      \end{array}\) \\
       
  \hline
  \bf{Result} & \lst{T} \\
  \hline
  
  \bf{Serialized as} & \hyperref[sec:serialization:operation:OptionGetOrElse]{\lst{OptionGetOrElse}} \\
  \hline
       
\end{tabularx}



\subsubsection{\lst{SOption.fold} method (Code 36.5)}
\label{sec:type:SOption:fold}
\noindent
\begin{tabularx}{\textwidth}{| l | X |}
   \hline
   \bf{Description} & Returns the result of applying \lst{f} to this option's
  value if the option is nonempty.  Otherwise, evaluates
  expression \lst{ifEmpty}.
  This is equivalent to \lst{option map f getOrElse ifEmpty}.
         \\
  
  \hline
  \bf{Parameters} &
      \(\begin{array}{l l l}
         \lst{ifEmpty} & \lst{: R} & \text{// the expression to evaluate if empty} \\
\lst{f} & \lst{: (T) => R} & \text{// the function to apply if nonempty} \\
      \end{array}\) \\
       
  \hline
  \bf{Result} & \lst{R} \\
  \hline
  
  \bf{Serialized as} & \hyperref[sec:serialization:operation:MethodCall]{\lst{MethodCall}} \\
  \hline
       
\end{tabularx}



\subsubsection{\lst{SOption.toColl} method (Code 36.6)}
\label{sec:type:SOption:toColl}
\noindent
\begin{tabularx}{\textwidth}{| l | X |}
   \hline
   \bf{Description} & Convert this Option to a collection with zero or one element. \\
  
  \hline
  \bf{Parameters} &
      \(\begin{array}{l l l}
         
      \end{array}\) \\
       
  \hline
  \bf{Result} & \lst{Coll[T]} \\
  \hline
  
  \bf{Serialized as} & \hyperref[sec:serialization:operation:PropertyCall]{\lst{PropertyCall}} \\
  \hline
       
\end{tabularx}



\subsubsection{\lst{SOption.map} method (Code 36.7)}
\label{sec:type:SOption:map}
\noindent
\begin{tabularx}{\textwidth}{| l | X |}
   \hline
   \bf{Description} & Returns a \lst{Some} containing the result of applying \lst{f} to this option's
   value if this option is nonempty.
   Otherwise return \lst{None}.
         \\
  
  \hline
  \bf{Parameters} &
      \(\begin{array}{l l l}
         \lst{f} & \lst{: (T) => R} & \text{// the function to apply} \\
      \end{array}\) \\
       
  \hline
  \bf{Result} & \lst{Option[R]} \\
  \hline
  
  \bf{Serialized as} & \hyperref[sec:serialization:operation:MethodCall]{\lst{MethodCall}} \\
  \hline
       
\end{tabularx}



\subsubsection{\lst{SOption.filter} method (Code 36.8)}
\label{sec:type:SOption:filter}
\noindent
\begin{tabularx}{\textwidth}{| l | X |}
   \hline
   \bf{Description} & Returns this option if it is nonempty and applying the predicate \lst{p} to
  this option's value returns true. Otherwise, return \lst{None}.
         \\
  
  \hline
  \bf{Parameters} &
      \(\begin{array}{l l l}
         \lst{p} & \lst{: (T) => Boolean} & \text{// the predicate used for testing} \\
      \end{array}\) \\
       
  \hline
  \bf{Result} & \lst{Option[T]} \\
  \hline
  
  \bf{Serialized as} & \hyperref[sec:serialization:operation:MethodCall]{\lst{MethodCall}} \\
  \hline
       
\end{tabularx}



\subsubsection{\lst{SOption.flatMap} method (Code 36.9)}
\label{sec:type:SOption:flatMap}
\noindent
\begin{tabularx}{\textwidth}{| l | X |}
   \hline
   \bf{Description} & Returns the result of applying \lst{f} to this option's value if
   this option is nonempty.
   Returns \lst{None} if this option is empty.
   Slightly different from \lst{map} in that \lst{f} is expected to
   return an option (which could be \lst{one}).
         \\
  
  \hline
  \bf{Parameters} &
      \(\begin{array}{l l l}
         \lst{f} & \lst{: (T) => Option[R]} & \text{// the function to apply} \\
      \end{array}\) \\
       
  \hline
  \bf{Result} & \lst{Option[R]} \\
  \hline
  
  \bf{Serialized as} & \hyperref[sec:serialization:operation:MethodCall]{\lst{MethodCall}} \\
  \hline
       
\end{tabularx}

