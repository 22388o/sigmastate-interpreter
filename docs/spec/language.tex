\section{Language}
\label{sec:language}

Here we define abstract syntax for \langname language. It is a typed
functional language with tuples, collections, optional types and \lst{val}
binding expressions. The semantics of \langname is specified by first
translating it to a core calculus (\corelang) and then by giving its
evaluation semantics. Typing rules is given in Section~\ref{sec:typing} and
evaluation semantics is given in Section~\ref{sec:evaluation}.

\langname is defined here using abstract syntax notation as shown in
Figure~\ref{fig:language}. This corresponds to \lst{ErgoTree} data structure,
which can be serialized to an array of bytes. The mnemonics shown in the
figure correspond to classes of \lst{ErgoTree} reference implementation.

\begin{figure}[h]
    \footnotesize
    \[\begin{tabular}{@{}l c l l l} 
\hline
Set Name				&  			& Syntax	& Mnemonic 		& Description \\
\hline
$\mathcal{T} \ni T$	& ::= 		& \lst{P} 	& \lst{SPredefType}	& predefined types (see Appendix~\ref{sec:appendix:predeftypes}) \\

			&	$\mid$	& $\tau$	& \lst{STypeVar} & type variable \\
			&	$\mid$	& $(T_1, \dots ,T_n) $	& \lst{STuple} & tuple of $n$ elements (see \lst{Tuple} type)\\

			&   $\mid$  & $(T_1,\dots,T_n) \to T $	& \lst{SFunc} & function of $n$ arguments (see \lst{Func} type) \\
			&   $\mid$  & $\text{\lst{Coll}}[T]$			& \lst{SCollection} & collection of elements of type $T$   \\
			&   $\mid$  & $\text{\lst{Option}}[T]$		& \lst{SOption} & optional value of type $T$  \\
			& 	     	&									& 				&		\\

$Term\ni e$	& ::= 		&   $C(v, T)$				& \lst{Constant} & typed constants  \\
			& 	$\mid$ 	& 	$x$ 					& \lst{ValUse} & variables  \\
			& 	$\mid$ 	& 	$\TyLam{x_i}{T_i}{e}$ 		& \lst{FuncExpr} & lambda expression \\
			& 	$\mid$ 	& 	$\Apply{e_f}{\Ov{e_i}}$ 	& \lst{Apply} & application of functional expression \\
			& 	$\mid$ 	& 	$\Apply{e.m}{\Ov{e_i}}$		& \lst{MethodCall} & method invocation  \\
			& 	$\mid$ 	&   $\Tup{e_1, \dots ,e_n}$ 	& \lst{Tuple} & constructor of tuple with $n$ items \\
			& 	$\mid$ 	& 	$\Apply{\delta}{\Ov{e_i}}$ 	& & primitive application (see Appendix~\ref{sec:appendix:primops}) \\
			& 	$\mid$ 	& 	\lst{if} $(e_{cond})$ $e_1$ \lst{else} $e_2$ & \lst{If} & if-then-else expression \\
			& 	$\mid$ 	&   $\{ \Ov{\text{\lst{val}}~x_i = e_i;}~e\}$  & \lst{BlockExpr} & block expression \\
			& 	     	&										& &				\\
$cd$   		& ::= 		& 	$\Trait{I}{\Ov{ms_i}}$				& \lst{STypeCompanion} & interface declaration    \\
			& 	     	&										& &				\\
$ms$	   	& ::= 	& $\MSig{m[\Ov{\tau_i}]}{\overline{x_i : T_i}}{T}$ 	& \lst{SMethod} & method signature declaration   \\
\end{tabular}\] 


    \caption{Abstract syntax of ErgoScript language}
    \label{fig:language}
\end{figure}
    
We assign types to the terms in a standard way following typing rules shown
in Figure~\ref{fig:typing}.

Constants keep both the type and the data value of that type. To be
well-formed the type of the constant should correspond to its value.

Variables are always typed and identified by unique $id$, which refers to
either lambda bound variable of \lst{val} bound variable. The encoding of
variables and their resolution is described in Section~\ref{sec:blocks}.

Lambda expressions can take a list of lambda-bound variables which can be
used in the body expression, which can be \emph{block expression}. 

Function application takes an expression of functional type (e.g. $T_1 \to
T_n$) and a list of arguments. The reason we do not write it $e_f(\Ov{e})$
is that this notation suggests that $(\Ov{e})$ is a subterm, which it is not.

Method invocation allows to apply functions defined as methods of
\emph{interface types}. If expression $e$ has interface type $I$ and and
method $m$ is declared in the interface $I$ then method invocation
$e.m(args)$ is defined for the appropriate $args$.

Conditional expressions of \langname are strict in condition and lazy in both
of the branches. Each branch is an expression which is executed depending on
the result of condition. This laziness of branches specified by lowering to
\corelang (see Figure~\ref{fig:lowering}).

Block expression contains a list of \lst{val} definitions of variables. To be
wellformed each subsequent definition can only refer to the previously defined
variables. Result of block execution is the result of the resulting
expression $e$, which can use any variable of the block.

Each type may be associated with a list of method declarations, in which case
we say that \emph{the type has methods}. The semantics of the methods is the
same as in Java. Having an instance of some type with methods it is possible
to call methods on the instance with some additional arguments.
Each method can be parameterized by type variables, which
can be used in method signature. Because \langname supports only monomorphic
values each method call is monomorphic and all type variables are assigned to
concrete types (see \lst{MethodCall} typing rule in Figure~\ref{fig:typing}).

The semantics of \langname is specified by translating all its terms to a
somewhat lower and simplified language, which we call \corelang. This
\emph{lowering} translation is shown in Figure~\ref{fig:lowering}. 

\begin{figure}[h]
\begin{center}
\begin{tabular}{ l c l }
	\hline
$Term_{\langname}$ &  & $Term_{Core}$  \\	
	\hline

$\Low{ \TyLam{x_i}{T_i}{e} 		}$ & \To & 
		$\Lam{x:(T_0,\dots,T_n)}{ \Low{ \{ \Ov{\lst{val}~x_i: T_i = x.\_i;}~e\} } }$ \\	

$\Low{ \Apply{e_f}{\Ov{e_i}} 	}$ & \To & $\Apply{ \Low{e_f} }{ \Low{(\Ov{e_i})} }$ \\	
$\Low{ \Apply{e.m}{\Ov{e_i}}	}$ & \To & $\Apply{ \Low{e}.m}{\Ov{ \Low{e_i} }}$ \\	
$\Low{ \Tup{e_1, \dots ,e_n}	}$ & \To & $\Tup{\Low{e_1}, \dots ,\Low{e_n}}$ \\	

$\Low{ e_1~\text{\lst{||}}~e_2	    }$ & \To & $\Low{ \IfThenElse{ e_1 }{ \True }{ e_2 } }$ \\	
$\Low{ e_1~\text{\lst{\&\&}}~e_2	}$ & \To & $\Low{ \IfThenElse{ e_1 }{ e_2 }{ \False } }$ \\	

$\Low{ \IfThenElse{e_{cond}}{e_1}{e_2} }$ & \To & 
		$\Apply{(\lst{if}(\Low{e_{cond}} ,~\Lam{(\_:Unit)}{\Low{e_1}} ,~\Lam{(\_:Unit)}{\Low{e_2}} ))}{}$ \\ 

$\Low{ \{ \Ov{\text{\lst{val}}~x_i: T_i = e_i;}~e\} }$ & \To &  
		$\Apply{ (\Lam{(x_1:T_1)}{( \dots \Apply{(\Lam{(x_n:T_n)}{\Low{e}})}{\Low{e_n}} \dots )}) }{\Low{e_1}}$\\

$\Low{ \Apply{\delta}{\Ov{e_i}}	}$ & \To & $\Apply{\delta}{\Ov{ \Low{e_i} }}$ \\	
$\Low{ e }$ 	& \To &  $e$ \\	
\end{tabular}
\end{center}
\caption{Lowering to \corelang}
\label{fig:lowering}
\end{figure}

All $n$-ary lambdas when $n>1$ are transformed to single arguments lambdas
using tupled arguments.
Note that $\IfThenElse{e_{cond}}{e_1}{e_2}$ term of \langname has lazy
evaluation of its branches whereas right-hand-side \lst{if} is a primitive
operation and have strict evaluation of the arguments. The laziness
is achieved by using lambda expressions of \lst{Unit} $\to$ \lst{Boolean}
type.

We translate logical operations (\lst{||}, \lst{&&}) of \langname, which are
lazy on second argument to \lst{if} term of \langname, which is recursively
translated to the corresponding \corelang term.

Syntactic blocks of \langname are completely eliminated and translated to
nested lambda expressions, which unambiguously specify evaluation semantics
of blocks. The \corelang is specified in Section~\ref{sec:evaluation}.

