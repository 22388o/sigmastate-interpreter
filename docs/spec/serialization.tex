\section{Serialization}
\label{sec:serialization}

Terms of the language described in Section~\ref{sec:language} can be
serialized to array of bytes to be stored in Ergo blockchain (e.g. as
Box.propositionBytes).

When the guarding script of an input box of a transaction is validated the
\lst{propositionBytes} array is deserialized to an \langname term, which can
be evaluated as it is specified in Section~\ref{sec:evaluation}.

Here we specify the serialization procedure in general. The serialization
format of \langname terms and types is specified in
Appendix~\ref{sec:appendix:ergotree_serialization} and
~\ref{sec:appendix:type_serialization} correspondingly. 

\begin{table}[h]
    \footnotesize
\(\begin{tabularx}{\textwidth}{| l | p{2cm} | X |}
    \hline
    \bf{Name}   & \bf{Value} & \bf{Description} \\
    \hline
    $\MaxVlqSize$  & $10$ & Maximum size of VLQ encoded byte sequence (See VLQ formats)  \\
    \hline
    $\MaxTypeSize$ & $100$ & Maximum size of serialized type term (see Type format) \\
    \hline
    $\MaxDataSize$ & $10Kb$ & Maximum size of serialized data instance (see Data format) \\
    \hline
    $\MaxConstSize$ & $=\MaxTypeSize + \MaxDataSize$  & Maximum size of serialized data instance (see Const format) \\
    \hline
    $\MaxExprSize$ & $1Kb$ & Maximum size of serialized \langname term (see Expr format) \\
    \hline
    $\MaxErgoTreeSize$ & $24Kb$ & Maximum size of serialized \langname contract (see ErgoTree format) \\
    \hline
\end{tabularx}\)
\caption{Serialization limits}
\label{table:ser:formats}
\end{table}

All serialization formats which are uses and defined thoughout this
section are listed in Table~\ref{table:ser:formats}.

\begin{table}[h]
    \footnotesize
\(\begin{tabularx}{\textwidth}{| l | l | X |}
    \hline
    \bf{Format} & \bf{\#bytes} & \bf{Description} \\
    \hline
    \lst{Byte} & $1$ & 8-bit signed two's-complement integer \\
    \hline
    \lst{Short} & $2$ & 16-bit signed two's-complement integer (big-endian) \\
    \hline    
    \lst{Int} & $4$ & 32-bit signed two's-complement integer (big-endian) \\
    \hline
    \lst{Long} & $8$ & 64-bit signed two's-complement integer (big-endian) \\
    \hline
    \lst{UByte} & $1$ & 8-bit unsigned integer \\
    \hline
    \lst{UShort} & $2$ & 16-bit unsigned integer (big-endian) \\
    \hline    
    \lst{UInt} & $4$ & 32-bit unsigned integer (big-endian) \\
    \hline
    \lst{ULong} & $8$ & 64-bit unsigned integer (big-endian) \\

    \hline
    \lst{VLQ(UShort)} & $[1..3]$ & Encoded unsigned \lst{Short} value using VLQ. See~\cite{VLQWikipedia,VLQRosetta} and~\ref{sec:vlq-encoding} \\
    \hline    
    \lst{VLQ(UInt)} & $[1..5]$ & Encoded unsigned 32-bit integer using VLQ. \\
    \hline
    \lst{VLQ(ULong)} & $[1..\MaxVlqSize]$ & Encoded unsigned 64-bit integer using VLQ. \\

    \hline
    \lst{Bits} & $[1..\MaxBits]$ & A collection of bits packed in a sequence of bytes. \\
    \hline
    \lst{Bytes} & $[1..\MaxBytes]$ & A sequence (block) of bytes. 
    The size of the block should either stored elsewhere or wellknown. \\

    \hline
    \lst{Type} & $[1..\MaxTypeSize]$ & Serialized type terms of \langname. See~\ref{sec:ser:type} \\
    \hline
    \lst{Data} & $[1..\MaxDataSize]$ & Serialized \langname values. See~\ref{sec:ser:data} \\
    \hline
    \lst{GroupElement} & $33$ & Serialized elements of eliptic curve group. See~\ref{sec:ser:data:groupelement} \\
    \hline
    \lst{SigmaProp} & $[1..\MaxSigmaProp]$ & Serialized sigma propositions. See~\ref{sec:ser:data:sigmaprop} \\
    \hline
    \lst{Box} & $[1..\MaxBox]$ & Serialized box data. See~\ref{sec:ser:data:box} \\
    \hline
    \lst{AvlTree} & $44$ & Serialized dynamic dictionary digest. See~\ref{sec:ser:data:avltree} \\
    \hline
    \lst{Const} & $[1..\MaxConstSize]$ & Serialized \langname constants (values with types). See~\ref{sec:ser:const} \\
    \hline
    \lst{Expr} & $[1..\MaxExprSize]$ & Serialized expression terms of \langname. See~\ref{sec:ser:expr} \\
    \hline
    \lst{ErgoTree} & $[1..\MaxErgoTreeSize]$ & Serialized instances of \langname contracts. See~\ref{sec:ser:ergotree} \\
    \hline
\end{tabularx}\)
\caption{Serialization formats}
\label{table:ser:formats}
\end{table}

Table~\ref{table:ser:formats} introduce a name for each format and also shows
the number of bytes each format may occupy in the byte stream. We use $[1..n]$
notation when serialization may produce from 1 to n bytes depending of actual
data instance. 

\subsection{Type Serialization}
\label{sec:ser:type}

\mnote{Migrate TypeSerialization.md here}


\subsection{Data Serialization}
\label{sec:ser:data}

In \langname all runtime data values have an associated type also available
at runtime (this is called \emph{type reification}\cite{Reification}).
However serialization format separates data values from its type descriptors. 
This allows to save space when for example a collection of items is serialized.

\begin{figure}[h]
\footnotesize
\(\begin{tabularx}{\textwidth}{| l | l | l | X |}
    \hline
    \bf{Slot} & \bf{Format} & \bf{\#bytes} & \bf{Description} \\
    \hline
    \hline
    \multicolumn{4}{l}{\lst{def serializeData(}$t, v$\lst{)}} \\
    \multicolumn{4}{l}{~~\lst{match} $(t, v)$ } \\

    \multicolumn{4}{l}{~~~~\lst{with} $(Unit, v \in \Denot{Unit})$~~~// nothing serialized } \\
    \multicolumn{4}{l}{~~~~\lst{with} $(Boolean, v \in \Denot{Boolean})$} \\
    \hline
    $~~~~~~v$ & \lst{Byte} & 1 & 0 or 1 in a single byte \\

    \hline
    \multicolumn{4}{l}{~~~~\lst{with} $(Byte, v \in \Denot{Byte})$} \\
    \hline
    $~~~~~~v$  & \lst{Byte} & 1 &  in a single byte \\

    \hline
    \multicolumn{4}{l}{~~~~\lst{with} $(N, v \in \Denot{Short}), N \in {Short, Int, Long}$} \\
    \hline
    $~~~~~~v$  & \lst{VLQ(ZigZag($$N$$))} & [1..3] & 
      16,32,64-bit signed integer encoded using \hyperref[sec:zigzag-encoding]{ZigZag} 
      and then using \hyperref[sec:vlq-encoding]{VLQ} \\

    \hline
    \multicolumn{4}{l}{~~~~\lst{with} $(BigInt, v \in \Denot{BigInt})$} \\
    \multicolumn{4}{l}{~~~~~~$bytes = v$\lst{.toByteArray} } \\
    \hline
    $~~~~~~numBytes$  & \lst{VLQ(UInt)} &  & number of bytes in $bytes$ array \\
    \hline
    $~~~~~~bytes$  & \lst{Bytes} &  & serialized $bytes$ array \\

    \hline
    \multicolumn{4}{l}{~~~~\lst{with} $(GroupElement, v \in \Denot{GroupElement})$} \\
    \hline
    ~~~~~~$v$  & \lst{GroupElement} &  & serialization of GroupElement data. See~\ref{sec:ser:data:groupelement} \\

    \hline
    \multicolumn{4}{l}{~~~~\lst{with} $(SigmaProp, v \in \Denot{SigmaProp})$} \\
    \hline
    ~~~~~~$v$  & \lst{SigmaProp} &  & serialization of SigmaProp data. See~\ref{sec:ser:data:sigmaprop} \\

    \hline
    \multicolumn{4}{l}{~~~~\lst{with} $(Box, v \in \Denot{Box})$} \\
    \hline
    ~~~~~~$v$  & \lst{Box} &  & serialization of Box data. See~\ref{sec:ser:data:box} \\

    \hline
    \multicolumn{4}{l}{~~~~\lst{with} $(AvlTree, v \in \Denot{AvlTree})$} \\
    \hline
    ~~~~~~$v$  & \lst{AvlTree} &  & serialization of AvlTree data. See~\ref{sec:ser:data:avltree} \\

    \hline
    \multicolumn{4}{l}{~~~~\lst{with} $(Coll[T], v \in \Denot{Coll[T]})$} \\
    \hline
    $~~~~~~len$  & \lst{VLQ(UShort)} & [1..3] & length of the collection \\
    \hline
    \multicolumn{4}{l}{~~~~~~\lst{match} $(T, v)$ } \\

    \multicolumn{4}{l}{~~~~~~~~\lst{with} $(Boolean, v \in \Denot{Coll[Boolean]})$} \\
    \hline
    $~~~~~~~~~~items$  & \lst{Bits} & [1..1024] & boolean values packed in bits \\
    \hline

    \multicolumn{4}{l}{~~~~~~~~\lst{with} $(Byte, v \in \Denot{Coll[Byte]})$} \\
    \hline
    $~~~~~~~~~~items$  & \lst{Bytes} & $[1..len]$ & items of the collection  \\
    \hline
    \multicolumn{4}{l}{~~~~~~~~\lst{otherwise} } \\
    \multicolumn{4}{l}{~~~~~~~~~~\lst{for}~$i=1$~\lst{to}~$len$} \\
    \multicolumn{4}{l}{~~~~~~~~~~~~\lst{serializeData(}$T, v_i$\lst{)}} \\
    \multicolumn{4}{l}{~~~~~~~~~~\lst{end for}} \\
    \multicolumn{4}{l}{~~~~~~\lst{end match}} \\

    \multicolumn{4}{l}{~~\lst{end match}} \\
    \multicolumn{4}{l}{\lst{end serializeData}} \\
    \hline
    \hline
\end{tabularx}\)
\caption{Data serialization format}
\label{fig:ser:data}
\end{figure}

\subsubsection{GroupElement serialization}
\label{sec:ser:data:groupelement}

\subsubsection{SigmaProp serialization}
\label{sec:ser:data:sigmaprop}

\subsubsection{Box serialization}
\label{sec:ser:data:box}

\subsubsection{AvlTree serialization}
\label{sec:ser:data:avltree}

\subsection{Constant Serialization}
\label{sec:ser:const}

\lst{Constant} format is simple and self sufficient to represent any data value in
\langname. Every data block of \lst{Constant} format contains both type and
data, such it can be stored or wire transfered and then later unambiguously
interpreted. The format is shown in Figure~\ref{fig:ser:const}

\begin{figure}[h]
\footnotesize
\(\begin{tabularx}{\textwidth}{| l | l | l | X |}
    \hline
    \bf{Slot} & \bf{Format} & \bf{\#bytes} & \bf{Description} \\
    \hline
    $type$  & \lst{Type} & $[1..\MaxTypeSize]$ & type of the data instance (see~\ref{sec:ser:type}) \\
    \hline
    $value$  & \lst{Data} & $[1..\MaxDataSize]$ & serialized data instance (see~\ref{sec:ser:data}) \\
    \hline
\end{tabularx}\)
\caption{Constant serialization format}
\label{fig:ser:const}
\end{figure}

\subsection{Expression Serialization}
\label{sec:ser:expr}

Expressions of \langname are serialized as tree data structure using
recursive procedure described here. 

\begin{figure}[h]
\footnotesize
\(\begin{tabularx}{\textwidth}{| l | l | l | X |}
    \hline
    \bf{Slot} & \bf{Format} & \bf{\#bytes} & \bf{Description} \\
    \hline
    \multicolumn{4}{l}{\lst{def serializeExpr(}$e$\lst{)}} \\
    \hline
    ~~$e.opCode$  & \lst{Byte} & $1$ & opcode of ErgoTree node, 
    used for selection of an appropriate node serializer from Appendix~\ref{sec:appendix:ergotree_serialization} \\
    \hline
    \multicolumn{4}{l}{~~\lst{if} $opCode <= LastConstantCode$ \lst{then}} \\
    \hline
    ~~~~$c$  & \lst{Const} & $[1..\MaxConstSize]$ & Constant serializaton slot \\ 
    \hline
    \multicolumn{4}{l}{~~\lst{else}} \\
    \hline
    ~~~~$body$  & Op & $[1..\MaxExprSize]$ & serialization of operation arguments 
    depending on $e.opCode$ as defined in Appendix~\ref{sec:appendix:ergotree_serialization} \\ 
    \hline
    \multicolumn{4}{l}{~~\lst{end if}} \\
    \multicolumn{4}{l}{\lst{end serializeExpr}} \\
    \hline
\end{tabularx}\)
\caption{Expression serialization format}
\label{fig:ser:expr}
\end{figure}


\subsection{\ASDag~serialization}
\label{sec:ser:ergotree}

The root of a serializable \langname term is a data structure called \ASDag
which serialization format shown in Figure~\ref{fig:ergotree}

\begin{figure}[h]
\footnotesize
\(\begin{tabularx}{\textwidth}{| l | l | l | X |}
  \hline
  \bf{Slot} & \bf{Format} & \bf{\#bytes} & \bf{Description} \\
  \hline
  $ header $ & \lst{VLQ(UInt)} & [1, *] & the first bytes of serialized byte array which
  determines interpretation of the rest of the array \\
  \hline
  $numConstants$ & \lst{VLQ(UInt)} & [1, *] & size of $constants$ array \\
  \hline
  \multicolumn{4}{l}{\lst{for}~$i=1$~\lst{to}~$numConstants$} \\
  \hline
      ~~ $ const_i $ & \lst{Const} & [1, *] & constant in i-th position \\
  \hline
  \multicolumn{4}{l}{\lst{end for}} \\
  \hline
  $ root $ & \lst{Expr} & [1, *] & If constantSegregationFlag is true, the  contains ConstantPlaceholder instead of some Constant nodes.
                       Otherwise may not contain placeholders.
                       It is possible to have both constants and placeholders in the tree, but for every placeholder
                       there should be a constant in $constants$ array. \\
  \hline
\end{tabularx}\)
\caption{\ASDag serialization format}
\label{fig:ser:ergotree}
\end{figure}


Serialized instances of \ASDag are self sufficient and can be stored and passed around.
\ASDag format defines top-level serialization format of \langname scripts.
The interpretation of the byte array depend on the first $header$ bytes, which uses VLQ encoding up to 30 bits.
Currently we define meaning for only first byte, which may be extended in future versions.

\begin{figure}[h]
    \footnotesize
\(\begin{tabularx}{\textwidth}{| l | l | X |}
    \hline
    \bf{Bits} & \bf{Default Value} & \bf{Description} \\
    \hline
    Bits 0-2 & 0 & language version (current version == 0) \\
    \hline
    Bit 3 & 0 & reserved (should be 0) \\
    \hline
    Bit 4 & 0 & == 1 if constant segregation is used for this ErgoTree (see~ Section~\ref{sec:ser:constant_segregation}\\
    \hline
    Bit 5 & 0 & == 1 - reserved for context dependent costing (should be = 0) \\
    \hline
    Bit 6 & 0 & reserved for GZIP compression (should be 0) \\
    \hline
    Bit 7 & 0 & == 1 if the header contains more than 1 byte (should be 0) \\
    \hline
\end{tabularx}\)
\caption{\ASDag $header$ bits}
\label{fig:ergotree:header}
\end{figure}

Currently we don't specify interpretation for the second and other bytes of
the header. We reserve the possibility to extend header by using Bit 7 == 1
and chain additional bytes as in VLQ. Once the new bytes are required, a new
version of the language should be created and implemented via
soft-forkability. That new language will give an interpretation for the new
bytes.

The default behavior of ErgoTreeSerializer is to preserve original structure
of \ASDag and check consistency. In case of any inconsistency the
serializer throws exception.

If constant segregation bit is set to 1 then $constants$ collection contains
the constants for which there may be \lst{ConstantPlaceholder} nodes in the
tree. If is however constant segregation bit is 0, then \lst{constants}
collection should be empty and any placeholder in the tree will lead to
exception.

\subsection{Constant Segregation}
\label{sec:ser:constant_segregation}

