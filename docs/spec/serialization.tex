\section{Serialization}
\label{sec:serialization}

This section defines a binary format, which is used to store \langname
contracts in persistent stores, to transfer them over wire and to enable
cross-platform interoperation.

Terms of the language described in Section~\ref{sec:language} can be
serialized to array of bytes to be stored in Ergo blockchain (e.g. as
Box.propositionBytes).

When the guarding script of an input box of a transaction is validated the
\lst{propositionBytes} array is deserialized to an \langname IR (called \ASDag), which can
be evaluated as it is specified in Section~\ref{sec:evaluation}.

Here we specify the serialization procedure in general. The serialization
format of \langname terms and types is specified in
Appendix~\ref{sec:appendix:ergotree_serialization} and
~\ref{sec:appendix:type_serialization} correspondingly.

Table~\ref{table:ser:formats} shows size limits which are checked during
contract deserialization.

\begin{table}[h]
    \footnotesize
\(\begin{tabularx}{\textwidth}{| l | p{2cm} | X |}
    \hline
    \bf{Name}   & \bf{Value} & \bf{Description} \\
    \hline
    $\MaxVlqSize$  & $10$ & Maximum size of VLQ encoded byte sequence (See VLQ formats)  \\
    \hline
    $\MaxTypeSize$ & $100$ & Maximum size of serialized type term (see Type format) \\
    \hline
    $\MaxDataSize$ & $10Kb$ & Maximum size of serialized data instance (see Data format) \\
    \hline
    $\MaxConstSize$ & $=\MaxTypeSize + \MaxDataSize$  & Maximum size of serialized data instance (see Const format) \\
    \hline
    $\MaxExprSize$ & $1Kb$ & Maximum size of serialized \langname term (see Expr format) \\
    \hline
    $\MaxErgoTreeSize$ & $24Kb$ & Maximum size of serialized \langname contract (see ErgoTree format) \\
    \hline
\end{tabularx}\)
\caption{Serialization limits}
\label{table:ser:formats}
\end{table}

All serialization formats which are uses and defined thoughout this
section are listed in Table~\ref{table:ser:formats}.

\begin{table}[h]
    \footnotesize
\(\begin{tabularx}{\textwidth}{| l | l | X |}
    \hline
    \bf{Format} & \bf{\#bytes} & \bf{Description} \\
    \hline
    \lst{Byte} & $1$ & 8-bit signed two's-complement integer \\
    \hline
    \lst{Short} & $2$ & 16-bit signed two's-complement integer (big-endian) \\
    \hline    
    \lst{Int} & $4$ & 32-bit signed two's-complement integer (big-endian) \\
    \hline
    \lst{Long} & $8$ & 64-bit signed two's-complement integer (big-endian) \\
    \hline
    \lst{UByte} & $1$ & 8-bit unsigned integer \\
    \hline
    \lst{UShort} & $2$ & 16-bit unsigned integer (big-endian) \\
    \hline    
    \lst{UInt} & $4$ & 32-bit unsigned integer (big-endian) \\
    \hline
    \lst{ULong} & $8$ & 64-bit unsigned integer (big-endian) \\

    \hline
    \lst{VLQ(UShort)} & $[1..3]$ & Encoded unsigned \lst{Short} value using VLQ. See~\cite{VLQWikipedia,VLQRosetta} and~\ref{sec:vlq-encoding} \\
    \hline    
    \lst{VLQ(UInt)} & $[1..5]$ & Encoded unsigned 32-bit integer using VLQ. \\
    \hline
    \lst{VLQ(ULong)} & $[1..\MaxVlqSize]$ & Encoded unsigned 64-bit integer using VLQ. \\

    \hline
    \lst{Bits} & $[1..\MaxBits]$ & A collection of bits packed in a sequence of bytes. \\
    \hline
    \lst{Bytes} & $[1..\MaxBytes]$ & A sequence (block) of bytes. 
    The size of the block should either stored elsewhere or wellknown. \\

    \hline
    \lst{Type} & $[1..\MaxTypeSize]$ & Serialized type terms of \langname. See~\ref{sec:ser:type} \\
    \hline
    \lst{Data} & $[1..\MaxDataSize]$ & Serialized \langname values. See~\ref{sec:ser:data} \\
    \hline
    \lst{GroupElement} & $33$ & Serialized elements of eliptic curve group. See~\ref{sec:ser:data:groupelement} \\
    \hline
    \lst{SigmaProp} & $[1..\MaxSigmaProp]$ & Serialized sigma propositions. See~\ref{sec:ser:data:sigmaprop} \\
    \hline
    \lst{Box} & $[1..\MaxBox]$ & Serialized box data. See~\ref{sec:ser:data:box} \\
    \hline
    \lst{AvlTree} & $44$ & Serialized dynamic dictionary digest. See~\ref{sec:ser:data:avltree} \\
    \hline
    \lst{Const} & $[1..\MaxConstSize]$ & Serialized \langname constants (values with types). See~\ref{sec:ser:const} \\
    \hline
    \lst{Expr} & $[1..\MaxExprSize]$ & Serialized expression terms of \langname. See~\ref{sec:ser:expr} \\
    \hline
    \lst{ErgoTree} & $[1..\MaxErgoTreeSize]$ & Serialized instances of \langname contracts. See~\ref{sec:ser:ergotree} \\
    \hline
\end{tabularx}\)
\caption{Serialization formats}
\label{table:ser:formats}
\end{table}

Table~\ref{table:ser:formats} introduce a name for each format and also shows
the number of bytes each format may occupy in the byte stream. We use $[1..n]$
notation when serialization may produce from 1 to n bytes depending of actual
data instance. 

Serialization format of \ASDag is optimized for compact storage. In many
cases serialization procedure is data dependent and thus have branching
logic. To express this complex serialization logic we use
\emph{pseudo-language operators} like \lst{for, match, if, optional} which
allow to specify a \emph{structure} on \emph{simple serialization slots}.
Each \emph{slot} specifies a fragment of serialized stream of bytes, whereas
\emph{operators} specifiy how the slots are combined together to form the
stream of bytes.

\subsection{Type Serialization}
\label{sec:ser:type}

For motivation behind this type encoding please see
Appendix~\ref{sec:appendix:motivation:type}.

\subsubsection{Distribution of type codes}
\label{sec:ser:type:codedist}

The whole space of 256 one byte codes is divided as shown in
Figure~\ref{fig:ser:type:codedist}.

\begin{figure}[h] \footnotesize
\(\begin{tabularx}{\textwidth}{| l | X |}
    \hline
    \bf{Value/Interval} & \bf{Distribution} \\
    \hline
    \lst{0x00} & special value to represent undefined type (\lst{NoType} in \ASDag) \\
    \hline
    \lst{0x01 - 0x6F(111)} & \emph{data types} including primitive types, arrays, options
    aka nullable types, classes (in future), 111 = 255 - 144 different codes \\
    \hline
    \lst{0x70(112) - 0xFF(255)} & \emph{function types} \lst{T1 => T2}, 144 = 12 x 12
    different codes~\footnote{Note that the function types are never serialized in version 1 of the Ergo
    protocol, this encoding is reserved for future development of the protocol.} \\
    \hline 
\end{tabularx}\)
\caption{Distribution of type codes between Data and Function types}
\label{fig:ser:type:codedist}
\end{figure}

\subsubsection{Encoding of Data Types}

There are eight different values for \emph{embeddable} types and 3 more are reserved
for the future extensions. Each embeddable type has a type code in the range {1,...,11}
as shown in Figure~\ref{fig:ser:type:embeddable}.

\begin{figure}[h] \footnotesize
    \(\begin{tabularx}{\textwidth}{| l | X |}
        \hline
        \bf{Code} & \bf{Type} \\ \hline
1     &   Boolean \\  \hline
2     &   Byte\\  \hline
3     &   Short (16 bit)\\  \hline
4     &   Int (32 bit)\\  \hline
5     &   Long (64 bit)\\  \hline
6     &   BigInt (represented by java.math.BigInteger)\\  \hline
7     &   GroupElement (represented by org.bouncycastle.math.ec.ECPoint)\\  \hline
8     &   SigmaProp \\  \hline
9     &   reserved for Char \\  \hline
10    &   reserved \\  \hline
11    &   reserved \\ \hline 
\end{tabularx}\)
\caption{Embeddable Types}
\label{fig:ser:type:embeddable}
\end{figure}

For each type constructor like \lst{Coll} or \lst{Option} we use the encoding schema
defined below. Type constructor has an associated \emph{base code} which is multiple
of~$12$ (e.g.~$12$ for \lst{Coll[_]}, $24$ for \lst{Coll[Coll[_]]} etc.).
The base code can be added to the embeddable type code to produce the code of the constructed
type, for example $12 + 1 = 13$ is a code of \lst{Coll[Byte]}. 
The code of type
constructor (e.g. $12$ in this example) is used when type parameter is non-embeddable
type (e.g. \lst{Coll[(Byte, Int)]}). In this case the code of type
constructor is read first, and then recursive descent is performed to read
bytes of the parameter type (in this case \lst{(Byte, Int)}). This encoding
allows very simple and fast decoding by using \lst{div} and \lst{mod} operations.

Following the above encoding schema the interval of codes for data types is divided as
shown in Figure~\ref{fig:ser:type:datatypes}.

\begin{figure}[h] \footnotesize
\(\begin{tabularx}{\textwidth}{| l | l | X |}
\hline
\bf{Interval}       & \bf{Type constructor} & \bf{Description} \\ \hline
0x01 - 0x0B(11)     &                       & embeddable types (including 3 reserved) \\ \hline
0x0C(12)            & \lst{Coll[_]}         & Collection of non-embeddable types (\lst{Coll[(Int,Boolean)]}) \\ \hline
0x0D(13) - 0x17(23) & \lst{Coll[_]}         & Collection of embeddable types (\lst{Coll[Byte]}, \lst{Coll[Int]}, etc.) \\ \hline
0x18(24)            & \lst{Coll[Coll[_]]}   & Nested collection of non-embeddable types (\lst{Coll[Coll[(Int,Boolean)]]}) \\ \hline
0x19(25) - 0x23(35) & \lst{Coll[Coll[_]]}   & Nested collection of embeddable types (\lst{Coll[Coll[Byte]]}, \lst{Coll[Coll[Int]]}) \\ \hline
0x24(36)            & \lst{Option[_]}       & Option of non-embeddable type (\lst{Option[(Int, Byte)]}) \\ \hline
0x25(37) - 0x2F(47) & \lst{Option[_]}       & Option of embeddable type (\lst{Option[Int]}) \\ \hline
0x30(48)            & \lst{Option[Coll[_]]} & Option of Coll of non-embeddable type (\lst{Option[Coll[(Int, Boolean)]]}) \\ \hline
0x31(49) - 0x3B(59) & \lst{Option[Coll[_]]} & Option of Coll of embeddable type (\lst{Option[Coll[Int]]}) \\ \hline
0x3C(60)            & \lst{(_,_)}           & Pair of non-embeddable types (\lst{((Int, Byte), (Boolean,Box))}, etc.) \\ \hline
0x3D(61) - 0x47(71) & \lst{(_, Int)}        & Pair of types where first is embeddable (\lst{(_, Int)}) \\ \hline
0x48(72)            & \lst{(_,_,_)}         & Triple of types  \\ \hline
0x49(73) - 0x53(83) & \lst{(Int, _)}        & Pair of types where second is embeddable (\lst{(Int, _)}) \\ \hline
0x54(84)            & \lst{(_,_,_,_)}       & Quadruple of types  \\ \hline
0x55(85) - 0x5F(95) & \lst{(_, _)}          & Symmetric pair of embeddable types (\lst{(Int, Int)}, \lst{(Byte,Byte)}, etc.) \\ \hline
0x60(96)            & \lst{(_,...,_)}       & \lst{Tuple} type with more than 4 items \lst{(Int, Byte, Box, Boolean, Int)} \\ \hline
0x61(97)            & \lst{Any}             & Any type  \\ \hline
0x62(98)            & \lst{Unit}            & Unit type \\ \hline
0x63(99)            & \lst{Box}             & Box type  \\ \hline
0x64(100)           & \lst{AvlTree}         & AvlTree type  \\ \hline
0x65(101)           & \lst{Context}         & Context type  \\ \hline
0x66(102)           &                       & reserved for String  \\ \hline
0x67(103)           &                       & reserved for TypeVar  \\ \hline
0x68(104)           & \lst{Header}          & Header type  \\ \hline
0x69(105)           & \lst{PreHeader}       & PreHeader type  \\ \hline
0x6A(106)           & \lst{Global}          & Global type  \\ \hline
0x6B(107)- 0x6E(110)&                       & reserved for future use  \\ \hline
0x6F(111)           &                       & Reserved for future \lst{Class} type (e.g. user-defined types)  \\ \hline
\end{tabularx}\)
\caption{Code Ranges of Data Types}
\label{fig:ser:type:datatypes}
\end{figure}

\subsubsection{Encoding of Function Types}

We use $12$ different values for both domain and range types of functions. This
gives us $12 * 12 = 144$ function types in total and allows to represent $11 *
11 = 121$ functions over primitive types using just single byte.

Each code $F$ in a range of the function types (i.e $F \in \{112, \dots, 255\}$) can be
represented as $F~=~D * 12 + R + 112$, where $D, R \in \{0,\dots,11\}$ - indices of
domain and range types correspondingly, $112$ - is the first code in an interval of
function types.

If $D = 0$ then the domain type is not embeddable and the recursive descent is
necessary to write/read the domain type.

If $R = 0$ then the range type is not embeddable and the recursive descent is necessary
to write/read the range type.

\subsubsection{Recursive Descent}
\label{sec:ser:type:recursive}

When an argument of a type constructor is not a primitive type we fallback to the
simple encoding schema in which case we emit the separate code for the type constructor
according to the table above and descend recursively to every child node of the type
tree.

We do this descend only for those children whose code cannot be embedded in
the parent code. For example, serialization of \lst{Coll[(Int,Boolean)]}
proceeds as the following:
\begin{enumerate}
\item Emit \lst{0x0C} because the elements type of the collection is not embeddable 
\item Recursively serialize \lst{(Int, Boolean)}
\item Emit \lst{0x41(=0x3D+4)} because the first type of the pair is embeddable and its code is~$4$
\item Recursivley serialize \lst{Boolean}
\item Emit \lst{0x02} - the code for embeddable type \lst{Boolean}
\end{enumerate}

More examples of type serialization are shown in Figure~\ref{fig:ser:type:examples}
\begin{figure}[h] \footnotesize
\(\begin{tabularx}{\textwidth}{| l | c | c | l | c | X |}
\hline
\bf{Type}                &\bf{D} & \bf{R} & \bf{Bytes} & \bf{\#Bytes} &  \bf{Comments} \\ \hline
\lst{Byte}               &     &     &  1                   &  1     &    \\ \hline
\lst{Coll[Byte]}         &     &     &  12 + 1 = 13         &  1     &    \\ \hline
\lst{Coll[Coll[Byte]]}   &     &     &  24 + 1 = 25         &  1     &     \\ \hline
\lst{Option[Byte]}       &     &     &  36 + 1 = 37         &  1     & register    \\ \hline
\lst{Option[Coll[Byte]]} &     &     &  48 + 1 = 49         &  1     & register    \\ \hline
\lst{(Int,Int)}          &     &     &  84 + 3 = 87         &  1     & fold    \\ \hline
\lst{Box=>Boolean}       & 7   & 2   &  198 = 7*12+2+112    &  1     & exist, forall    \\ \hline
\lst{(Int,Int)=>Int}     & 0   & 3   &  115=0*12+3+112, 87  &  2     &  fold    \\ \hline
\lst{(Int,Boolean)}      &     &     &  60 + 3, 2           &  2     &      \\ \hline
\lst{(Int,Box)=>Boolean} & 0   & 2   &  0*12+2+112, 60+3, 7 &  3     &     \\ \hline
\end{tabularx}\)
\caption{Examples of type serialization}
\label{fig:ser:type:examples}
\end{figure}


\subsection{Data Serialization}
\label{sec:ser:data}

In \langname all runtime data values have an associated type also available
at runtime (this is called \emph{type reification}\cite{Reification}).
However serialization format separates data values from its type descriptors. 
This allows to save space when for example a collection of items is serialized.

The contents of a typed data structure can be fully described by a type tree.
For example having a typed data object \lst{d: (Int, Coll[Byte], Boolean)} we can
tell that \lst{d} has 3 items, the first item contain 32-bit integer, the second
- collection of bytes, and the third - logical true/false value.

To serialize/deserialize typed data we need to know its type descriptor (type
tree). Serialization procedure is recursive over type tree and the
corresponding subcomponents of an object. For primitive types (the leaves of
the type tree) the format is fixed. The data values of \langname types are
serialized using predefined function shown in Figure~\ref{fig:ser:data}.

\begin{figure}[h]
\footnotesize
\(\begin{tabularx}{\textwidth}{| l | l | l | X |}
    \hline
    \bf{Slot} & \bf{Format} & \bf{\#bytes} & \bf{Description} \\
    \hline
    \hline
    \multicolumn{4}{l}{\lst{def serializeData(}$t, v$\lst{)}} \\
    \multicolumn{4}{l}{~~\lst{match} $(t, v)$ } \\

    \multicolumn{4}{l}{~~~~\lst{with} $(Unit, v \in \Denot{Unit})$~~~// nothing serialized } \\
    \multicolumn{4}{l}{~~~~\lst{with} $(Boolean, v \in \Denot{Boolean})$} \\
    \hline
    $~~~~~~v$ & \lst{Byte} & 1 & 0 or 1 in a single byte \\

    \hline
    \multicolumn{4}{l}{~~~~\lst{with} $(Byte, v \in \Denot{Byte})$} \\
    \hline
    $~~~~~~v$  & \lst{Byte} & 1 &  in a single byte \\

    \hline
    \multicolumn{4}{l}{~~~~\lst{with} $(N, v \in \Denot{Short}), N \in {Short, Int, Long}$} \\
    \hline
    $~~~~~~v$  & \lst{VLQ(ZigZag($$N$$))} & [1..3] & 
      16,32,64-bit signed integer encoded using \hyperref[sec:zigzag-encoding]{ZigZag} 
      and then using \hyperref[sec:vlq-encoding]{VLQ} \\

    \hline
    \multicolumn{4}{l}{~~~~\lst{with} $(BigInt, v \in \Denot{BigInt})$} \\
    \multicolumn{4}{l}{~~~~~~$bytes = v$\lst{.toByteArray} } \\
    \hline
    $~~~~~~numBytes$  & \lst{VLQ(UInt)} &  & number of bytes in $bytes$ array \\
    \hline
    $~~~~~~bytes$  & \lst{Bytes} &  & serialized $bytes$ array \\

    \hline
    \multicolumn{4}{l}{~~~~\lst{with} $(GroupElement, v \in \Denot{GroupElement})$} \\
    \hline
    ~~~~~~$v$  & \lst{GroupElement} &  & serialization of GroupElement data. See~\ref{sec:ser:data:groupelement} \\

    \hline
    \multicolumn{4}{l}{~~~~\lst{with} $(SigmaProp, v \in \Denot{SigmaProp})$} \\
    \hline
    ~~~~~~$v$  & \lst{SigmaProp} &  & serialization of SigmaProp data. See~\ref{sec:ser:data:sigmaprop} \\

    \hline
    \multicolumn{4}{l}{~~~~\lst{with} $(Box, v \in \Denot{Box})$} \\
    \hline
    ~~~~~~$v$  & \lst{Box} &  & serialization of Box data. See~\ref{sec:ser:data:box} \\

    \hline
    \multicolumn{4}{l}{~~~~\lst{with} $(AvlTree, v \in \Denot{AvlTree})$} \\
    \hline
    ~~~~~~$v$  & \lst{AvlTree} &  & serialization of AvlTree data. See~\ref{sec:ser:data:avltree} \\

    \hline
    \multicolumn{4}{l}{~~~~\lst{with} $(Coll[T], v \in \Denot{Coll[T]})$} \\
    \hline
    $~~~~~~len$  & \lst{VLQ(UShort)} & [1..3] & length of the collection \\
    \hline
    \multicolumn{4}{l}{~~~~~~\lst{match} $(T, v)$ } \\

    \multicolumn{4}{l}{~~~~~~~~\lst{with} $(Boolean, v \in \Denot{Coll[Boolean]})$} \\
    \hline
    $~~~~~~~~~~items$  & \lst{Bits} & [1..1024] & boolean values packed in bits \\
    \hline

    \multicolumn{4}{l}{~~~~~~~~\lst{with} $(Byte, v \in \Denot{Coll[Byte]})$} \\
    \hline
    $~~~~~~~~~~items$  & \lst{Bytes} & $[1..len]$ & items of the collection  \\
    \hline
    \multicolumn{4}{l}{~~~~~~~~\lst{otherwise} } \\
    \multicolumn{4}{l}{~~~~~~~~~~\lst{for}~$i=1$~\lst{to}~$len$} \\
    \multicolumn{4}{l}{~~~~~~~~~~~~\lst{serializeData(}$T, v_i$\lst{)}} \\
    \multicolumn{4}{l}{~~~~~~~~~~\lst{end for}} \\
    \multicolumn{4}{l}{~~~~~~\lst{end match}} \\

    \multicolumn{4}{l}{~~\lst{end match}} \\
    \multicolumn{4}{l}{\lst{end serializeData}} \\
    \hline
    \hline
\end{tabularx}\)
\caption{Data serialization format}
\label{fig:ser:data}
\end{figure}

\subsubsection{GroupElement serialization}
\label{sec:ser:data:groupelement}

\subsubsection{SigmaProp serialization}
\label{sec:ser:data:sigmaprop}

\subsubsection{Box serialization}
\label{sec:ser:data:box}

\subsubsection{AvlTree serialization}
\label{sec:ser:data:avltree}

\subsection{Constant Serialization}
\label{sec:ser:const}

\lst{Constant} format is simple and self sufficient to represent any data value in
\langname. Every data block of \lst{Constant} format contains both type and
data, such it can be stored or wire transfered and then later unambiguously
interpreted. The format is shown in Figure~\ref{fig:ser:const}

\begin{figure}[h]
\footnotesize
\(\begin{tabularx}{\textwidth}{| l | l | l | X |}
    \hline
    \bf{Slot} & \bf{Format} & \bf{\#bytes} & \bf{Description} \\
    \hline
    $type$  & \lst{Type} & $[1..\MaxTypeSize]$ & type of the data instance (see~\ref{sec:ser:type}) \\
    \hline
    $value$  & \lst{Data} & $[1..\MaxDataSize]$ & serialized data instance (see~\ref{sec:ser:data}) \\
    \hline
\end{tabularx}\)
\caption{Constant serialization format}
\label{fig:ser:const}
\end{figure}

\subsection{Expression Serialization}
\label{sec:ser:expr}

Expressions of \langname are serialized as tree data structure using
recursive procedure described here. 

\begin{figure}[h]
\footnotesize
\(\begin{tabularx}{\textwidth}{| l | l | l | X |}
    \hline
    \bf{Slot} & \bf{Format} & \bf{\#bytes} & \bf{Description} \\
    \hline
    \multicolumn{4}{l}{\lst{def serializeExpr(}$e$\lst{)}} \\
    \hline
    ~~$e.opCode$  & \lst{Byte} & $1$ & opcode of ErgoTree node, 
    used for selection of an appropriate node serializer from Appendix~\ref{sec:appendix:ergotree_serialization} \\
    \hline
    \multicolumn{4}{l}{~~\lst{if} $opCode <= LastConstantCode$ \lst{then}} \\
    \hline
    ~~~~$c$  & \lst{Const} & $[1..\MaxConstSize]$ & Constant serializaton slot \\ 
    \hline
    \multicolumn{4}{l}{~~\lst{else}} \\
    \hline
    ~~~~$body$  & Op & $[1..\MaxExprSize]$ & serialization of operation arguments 
    depending on $e.opCode$ as defined in Appendix~\ref{sec:appendix:ergotree_serialization} \\ 
    \hline
    \multicolumn{4}{l}{~~\lst{end if}} \\
    \multicolumn{4}{l}{\lst{end serializeExpr}} \\
    \hline
\end{tabularx}\)
\caption{Expression serialization format}
\label{fig:ser:expr}
\end{figure}


\subsection{\ASDag~serialization}
\label{sec:ser:ergotree}

The root of a serializable \langname term is a data structure called \ASDag
which serialization format shown in Figure~\ref{fig:ergotree}

\begin{figure}[h]
\footnotesize
\(\begin{tabularx}{\textwidth}{| l | l | l | X |}
  \hline
  \bf{Slot} & \bf{Format} & \bf{\#bytes} & \bf{Description} \\
  \hline
  $ header $ & \lst{VLQ(UInt)} & [1, *] & the first bytes of serialized byte array which
  determines interpretation of the rest of the array \\
  \hline
  $numConstants$ & \lst{VLQ(UInt)} & [1, *] & size of $constants$ array \\
  \hline
  \multicolumn{4}{l}{\lst{for}~$i=1$~\lst{to}~$numConstants$} \\
  \hline
      ~~ $ const_i $ & \lst{Const} & [1, *] & constant in i-th position \\
  \hline
  \multicolumn{4}{l}{\lst{end for}} \\
  \hline
  $ root $ & \lst{Expr} & [1, *] & If constantSegregationFlag is true, the  contains ConstantPlaceholder instead of some Constant nodes.
                       Otherwise may not contain placeholders.
                       It is possible to have both constants and placeholders in the tree, but for every placeholder
                       there should be a constant in $constants$ array. \\
  \hline
\end{tabularx}\)
\caption{\ASDag serialization format}
\label{fig:ser:ergotree}
\end{figure}


Serialized instances of \ASDag are self sufficient and can be stored and passed around.
\ASDag format defines top-level serialization format of \langname scripts.
The interpretation of the byte array depend on the first $header$ bytes, which uses VLQ encoding up to 30 bits.
Currently we define meaning for only first byte, which may be extended in future versions.

\begin{figure}[h]
    \footnotesize
\(\begin{tabularx}{\textwidth}{| l | l | X |}
    \hline
    \bf{Bits} & \bf{Default Value} & \bf{Description} \\
    \hline
    Bits 0-2 & 0 & language version (current version == 0) \\
    \hline
    Bit 3 & 0 & reserved (should be 0) \\
    \hline
    Bit 4 & 0 & == 1 if constant segregation is used for this ErgoTree (see~ Section~\ref{sec:ser:constant_segregation}\\
    \hline
    Bit 5 & 0 & == 1 - reserved for context dependent costing (should be = 0) \\
    \hline
    Bit 6 & 0 & reserved for GZIP compression (should be 0) \\
    \hline
    Bit 7 & 0 & == 1 if the header contains more than 1 byte (should be 0) \\
    \hline
\end{tabularx}\)
\caption{\ASDag $header$ bits}
\label{fig:ergotree:header}
\end{figure}

Currently we don't specify interpretation for the second and other bytes of
the header. We reserve the possibility to extend header by using Bit 7 == 1
and chain additional bytes as in VLQ. Once the new bytes are required, a new
version of the language should be created and implemented via
soft-forkability. That new language will give an interpretation for the new
bytes.

The default behavior of ErgoTreeSerializer is to preserve original structure
of \ASDag and check consistency. In case of any inconsistency the
serializer throws exception.

If constant segregation bit is set to 1 then $constants$ collection contains
the constants for which there may be \lst{ConstantPlaceholder} nodes in the
tree. If is however constant segregation bit is 0, then \lst{constants}
collection should be empty and any placeholder in the tree will lead to
exception.

\subsection{Constant Segregation}
\label{sec:ser:constant_segregation}

