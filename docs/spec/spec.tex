\documentclass[11pt]{article}

\usepackage{fullpage}

\usepackage{mathtools,color,xcolor,hyperref,graphicx,wrapfig,listings,array,xspace,tabu,stmaryrd,tabularx,verbatim,longtable}

% "define" Scala
\lstdefinelanguage{scala}{
  morekeywords={abstract,case,catch,class,def,%
    do,else,extends,false,final,finally,%
    for,if,implicit,import,match,mixin,%
    new,null,object,override,package,%
    private,protected,requires,return,sealed,%
    super,this,throw,trait,true,try,%
    type,val,var,while,with,yield},
  otherkeywords={=>,<-,<\%,<:,>:,\#,@},
  sensitive=true,
  morecomment=[l]{//},
  morecomment=[n]{/*}{*/},
  morestring=[b]",
  morestring=[b]',
  morestring=[b]"""
}

\newcommand{\authnote}[2]{\textsf{#1 \textcolor{blue}{: #2}}}

\newcommand{\knote}[1]{{\authnote{\textcolor{green}{kushti}}{#1}}}
\newcommand{\mnote}[1]{{\authnote{\textcolor{red}{Morphic}}{#1}}}
\newcommand{\dnote}[1]{{\authnote{\textcolor{brown}{Dima}}{#1}}}


\newcommand{\ret}{\mathsf{ret}}
\newcommand{\new}{\mathsf{new}}
\newcommand{\hnew}{h_\mathsf{new}}
\newcommand{\old}{\mathsf{old}}
\newcommand{\op}{\mathsf{op}}
\newcommand{\verifier}{\mathcal{V}}
\newcommand{\prover}{\mathcal{P}}
\newcommand{\key}{\mathsf{key}}
\newcommand{\nextkey}{\mathsf{nextKey}}
\newcommand{\node}{\mathsf{t}}
\newcommand{\parent}{\mathsf{p}}
\newcommand{\leaf}{\mathsf{f}}
\newcommand{\vl}{\mathsf{value}}
\newcommand{\balance}{\mathsf{balance}}
\newcommand{\lft}{\mathsf{left}}
\newcommand{\rgt}{\mathsf{right}}
\newcommand{\lbl}{\mathsf{label}}
\newcommand{\direction}{\mathsf{d}}
\newcommand{\oppositedirection}{\bar{\mathsf{d}}}
\newcommand{\found}{\mathsf{found}}
\newcommand{\mypar}[1]{\smallskip\noindent\textbf{#1.}\ \ \ }
\newcommand{\ignore}[1]{}

\newcommand{\langname}{ErgoTree\xspace}
\newcommand{\corelang}{$\lst{Core-}\lambda$\xspace}
\newcommand{\lst}[1]{\text{\lstinline[basicstyle={\ttfamily}]$#1$}}

\newcommand{\andnode}{\ensuremath{\mathsf{AND}}}
\newcommand{\ornode}{\ensuremath{\mathsf{OR}}}
\newcommand{\tnode}{\ensuremath{\mathsf{THRESHOLD}}}
\newcommand{\GF}{\ensuremath{\mathrm{GF}}}

\newcommand{\ASDag}{ErgoTree\xspace}

\newcommand{\I}[1]{\mathit{#1}}
\newcommand{\B}[1]{\mathbf{#1}}
\newcommand{\PA}[1]{\I{PA}\langle\I{#1}\rangle}
\newcommand{\NA}[1]{\I{NA}\langle\I{#1}\rangle}
\newcommand{\nlindent}[1][0.2cm]{\newline\hangindent=#1}
\newcommand{\MU}[1]{\mu\B{\alpha}.\B{#1}}
\newcommand{\Monoid}[1]{\I{Monoid}\TY{#1}}
%\newcommand{\indentline}{\hangindent=0.7cm}
\newcommand{\tick}{\checkmark}
\newcommand{\Left}[3]{\text{\lst{l}}[#1,#2]\cdot #3}
\newcommand{\Right}[3]{\text{\lst{r}}[#1,#2]\cdot #3}
\newcommand{\SelectField}[2]{\text{\lst{Field}}(#1, #2)}
\newcommand{\Fst}[1]{\text{\lst{fst}}~#1}
\newcommand{\Snd}[1]{\text{\lst{snd}}~#1} 
% \newcommand{\Fst}[1]{$#1$\lst{.fst}}
% \newcommand{\Snd}[1]{$#1$\lst{.snd}}
\newcommand{\Ctx}{\mathcal{E}}
\newcommand{\Apply}[2]{#1\langle#2\rangle}
\newcommand{\RCtx}{\mathcal{R}}
\newcommand{\RMatch}[1]{(#1 :: \mathcal{R})}
\newcommand{\RCtxEmpty}{\epsilon}
\newcommand{\Frame}{\mathcal{F}}
\newcommand{\Prim}{\delta}
\newcommand{\Sp}{\mathcal{S}}
\newcommand{\Spec}[1]{\mathcal{S}|[#1|]}
\newcommand{\Build}[1]{\mathcal{B}|[#1|]}
\newcommand{\Hole}{\diamondsuit{}}
\newcommand{\Trait}[2]{\text{\lst{trait}}~#1~\{ #2 \}}
\newcommand{\Class}[3]{\text{\lst{class}}~#1~\text{\lst{extends}}~#2 \{ #3 \}}
\newcommand{\MSig}[3]{\text{\lst{def}}~#1(#2): #3}
\newcommand{\CaseOfxxx}[3]{\lst{case} $#1$ \lst{of} \{ $#2 \to #3$ \}}
\newcommand{\LetXXX}[3]{\lst{let} $#1$ \lst{=} $#2$ \lst{in} $#3$}
\newcommand{\LetrecXXX}[3]{\lst{letrec} $#1$ \lst{=} $#2$ \lst{in} $#3$}
\newcommand{\CaseOfXX}[2]{\text{\lst{case}}~#1~\text{\lst{of}}~\{ #2 \}}
\newcommand{\CaseOf}[3]{\text{\lst{case}}~#1~\text{\lst{of}}~\{ #2 \to #3 \}}
\newcommand{\True}{\text{\lst{true}}}
\newcommand{\False}{\text{\lst{false}}}
\newcommand{\IfThenElse}[3]{\text{\lst{if}}~(#1)~#2~\text{\lst{else}}~#3}
\newcommand{\Let}[3]{\text{\lst{let}}~#1~\text{\lst{=}}~#2~\text{\lst{in}}~#3}
\newcommand{\Field}[2]{#1.\text{\lst{#2}}}
\newcommand{\FDecl}[2]{\text{\lst{val}}~#1 : #2}
\newcommand{\New}[1]{\text{\lst{new}}~#1}
\newcommand{\Meth}[2]{\text{\lst{#1.#2}}}

\newcommand{\KSet}{\mathcal{K}}  
\newcommand{\VSet}{\mathcal{V}}  
\newcommand{\LSet}{\mathcal{L}}  
\newcommand{\Low}[1]{\mathcal{L}\llbracket#1\rrbracket}  
\newcommand{\Denot}[1]{\llbracket#1\rrbracket}  
\newcommand{\PSet}{\mathcal{P}}  
\newcommand{\DSet}{\mathcal{D}}  
\newcommand{\CSet}{\mathcal{CLS}}  
\newcommand{\ISet}{\mathcal{ABS}}  

\newcommand{\Ov}[1]{\overline{#1}}
\newcommand{\Un}[1]{\underline{#1}}
\newcommand{\Tup}[1]{(#1)}
\newcommand{\Coll}[1]{\text{\lst{Coll}}(#1)}
\newcommand{\Some}[1]{\text{\lst{Some}}(#1)}
\newcommand{\None}[1]{\text{\lst{None}}[#1]}
\newcommand{\Def}[1]{\llparenthesis#1\rrparenthesis}
\newcommand{\ByDef}{\overset{def}{=}}
\newcommand{\Dag}{\Delta}
\newcommand{\Dom}[1]{\mathcal{D}om~#1}
\newcommand{\TAddr}{Addr}
\newcommand{\TDef}{Def}
\newcommand{\TNode}{Node}
\newcommand{\TDag}{Dag}
\newcommand{\TPair}[2]{#1\times#2}
\newcommand{\TList}[1]{List~#1}
\newcommand{\TMDag}{\TDag * \TAddr}
\newcommand{\Focus}[1]{\langle#1\rangle}
\newcommand{\MDag}[1]{\Delta\Focus{#1}}
\newcommand{\MDagPr}[1]{\Delta'\Focus{#1}}
\newcommand{\Map}[2]{#1  \mapsto #2}
\newcommand{\AddMap}[3]{#1 \cup \{#2  \mapsto #3\}}
\newcommand{\To}{$\mapsto$}
\newcommand{\TP}[2]{#1 \to #2}
\newcommand{\Set}[1]{\{ #1 \}}
\newcommand{\DHole}[2]{d~#1\Hole#2}
\newcommand{\PrimPat}{\Prim~\overline{\beta}}
\newcommand{\DefPat}{d~(\Ov{\beta})}
\newcommand{\Lam}[2]{\lambda#1.#2}
\newcommand{\TyLam}[3]{\lambda(\Ov{#1:#2}).#3}
\newcommand{\LamPat}[2]{\lambda#1.#2}
\newcommand{\DagPat}[2]{\{ \TP{#1}{#2} \}}
\newcommand{\MDagPat}[4]{\{ \TP{#1}{#2} \}^#3\Focus{#4}}
\newcommand{\Inj}[3]{#1\xleftarrow{}#3}
\newcommand{\SE}[3]{SE'|[#1|]~#2~#3}
\newcommand{\SEtop}[2]{SE|[#1|]~#2}
\newcommand{\TEnv}{\Gamma}
\newcommand{\Der}[2]{#1~\vdash~#2}
\newcommand{\DerV}[2]{#1~\vdash^{\text{\lst{v}}}~#2}
\newcommand{\DerC}[2]{#1~\vdash^{\text{\lst{c}}}~#2}
\newcommand{\DerEnv}[1]{\Der{\TEnv}{#1}}
\newcommand{\DerEnvV}[1]{\DerV{\TEnv}{#1}}
\newcommand{\DerEnvC}[1]{\DerC{\TEnv}{#1}}
\newcommand{\Dif}[1]{\partial#1}
\newcommand{\View}[2]{#1\sphericalangle #2}
\newcommand{\This}{\text{\lst{this}}}
\newcommand{\Section}[1]{Section~\ref{section:#1}}
\newcommand{\MaxVlqSize}{VLQ_{max}}
\newcommand{\MaxBits}{Bits_{max}}
\newcommand{\MaxBytes}{Bytes_{max}}
\newcommand{\MaxTypeSize}{Type_{max}}
\newcommand{\MaxDataSize}{Data_{max}}
\newcommand{\MaxBox}{Box_{max}}
\newcommand{\MaxSigmaProp}{SigmaProp_{max}}
\newcommand{\MaxAvlTree}{AvlTree_{max}}
\newcommand{\MaxConstSize}{Const_{max}}
\newcommand{\MaxExprSize}{Expr_{max}}
\newcommand{\MaxErgoTreeSize}{ErgoTree_{max}}

\newtheorem{definition}{Definition}

\setcounter{tocdepth}{2}

\begin{document}

\title{ErgoTree Specification}

\author{authors}


\maketitle

\begin{abstract}
In this document we consider typed abstract syntax of the language
called \ASDag which defines semantics of a condition which protects a closed
box in the Ergo Platform blockchain. Serialized graph is written into a box.
Most of Ergo users are unaware of the graph since they are developing contracts in higher-level languages, such as
ErgoScript. However, for developers of alternative higher-level languages, client libraries and clients knowledge of
internals would be highly useful. This document is providing the internals, namely, the following data structures and
algorithms:
\begin{itemize}
\item{} Serialization to a binary format and graph deserialization from the binary form.
\item{} When a graph is considered to be well-formed and when not.
\item{} Type system and typing rules.
\item{} How graph is transformed into an execution trace.
\item{} How execution trace is costed.
\item{} How execution trace is reduced into a Sigma-expression.
\item{} How Sigma-expression is proven and verified.
\end{itemize}
\end{abstract}

\knote{Please note that the document is intended for general high-skilled tech audience, so avoid describing Scala
classes etc.}

\tableofcontents

\section{Introduction}
\label{sec:intro}

The design space of programming languages is very broad ranging from
general-purpose languages like C,Java,Python up to specialized languages
like SQL, HTML, CSS, etc.

Since Ergo's goal is to provide a platform for contractual money, the choice
of the language for writing contracts is very important.

First of all the language and contract execution environment should be
\emph{deterministic}. Once created and stored in Ergo blockchain, smart
contract should always behave predictably and deterministically, it should only depend on well-defined data context and nothing else.
As long as data context doesn't change, any execution of the contract
should return the same value any time it is executed, on any execution
platform, and even on any \emph{compliant} language implementation.
No general purpose programming language is deterministic because
all of them provide non-deterministic operations. ErgoScript doesn't have
non-deterministic operations.

Second, the language should be \emph{spam-resistant}, meaning it should
facilitate in defending against attacks when malicious contracts can overload
network nodes and bring the blockchain down. To fullfill this goal ErgoScript
support \emph{ahead-of-time cost estimation}, the fast check performed before
contract execution to ensure the evaluation cost is within acceptable
bounds. In general, such cost prediction is not possible, however if the
language is simple enough (which is the case of ErgoScript) and if operations
are carefully selected, then costing is possible and doesn't require
usage of Gas~\mnote{cite etherium} and allow to avoid related problems~\mnote{cite Gas related problems}.

Third, being simple, the contracts language should be \emph{expressive
enough}. It should be possible to implement most of the practical scenarios,
which is the case of ErgoScript. In our experience expressivity of contracts
language comes hand in hand with design and capabilities of Ergo blockchain
platform itself, making the whole system \emph{turing-complete} as we
demonstrated in \mnote{cite TuringPaper}.

Forth, simplicity and expressivity are often characteristics of
domain-specific languages~\mnote{cite DSL}. From this perspective ErgoScript
is a DSL for writing smart contracts. The language directly captures the
Ubiquites Language~\cite{UbiqLang} of smart contracts
domain directly manipulating with first-class Boxes, Tokens, Zero-Knowledge
Sigma-Propostions etc., these are the novel features Ergo aims to provide as a platform/service
for custom user applicatons. Domain-specific nature nature of ErgoScript also fasilitates spam-resistance,
because operations of ErgoScript are all carefully selected to be \emph{costing
friendly}.

And last, but not the least, we wanted our new language to be, nevertheless,
\emph{familiar to the most} since we aim to address as large audience of
programmers as possible with minimum surprise and WTF ratio
\cite{WTFLang}.
The syntax of ErgoScript is inspired by Scala/Kotlin, but in fact it shares a
common subset with Java and C\#, thus if you are proficient in any of these
languages you will be right at home with ErgoScript as well.

Guided by this requirements we designed ErgoScript as a new yet familiar
looking language which directly support all novel features of Ergo blockchain.
We also implemented reference implementation of the specification described in this document.

\section{\langname As A Language}
\label{sec:language}

In this section we define an abstract syntax for the \langname language. It is a typed
call-by-value, higher-order functional language without recursion. It supports
single-assignment blocks, tuples, optional values, indexed collections with
higher-order operations, short-cutting logicals, ternary 'if' with lazy branches. All
operations are deterministic, without side effects and all values are immutable.

The semantics of \langname is specified by first translating it to a core calculus
(\corelang) and then by giving its denotational evaluation semantics. Typing rules are
given in section~\ref{sec:typing} and the evaluation semantics is given in
section~\ref{sec:evaluation}. Guidance on compliant interpreter implementation is
provided in section~\ref{sec:implementation}.

\langname is defined below using abstract syntax notation as shown in
Figure~\ref{fig:language}. This corresponds to \lst{Value} class of the reference
implementation, which can be serialized to an array of bytes using
\lst{ValueSerializer}. The mnemonic names shown in the figure correspond to classes of
the reference implementation.

\begin{figure}[h]
    \footnotesize
    \[\begin{tabular}{@{}l c l l l} 
\hline
Set Name				&  			& Syntax	& Mnemonic 		& Description \\
\hline
$\mathcal{T} \ni T$	& ::= 		& \lst{P} 	& \lst{SPredefType}	& predefined types (see Appendix~\ref{sec:appendix:predeftypes}) \\

			&	$\mid$	& $\tau$	& \lst{STypeVar} & type variable \\
			&	$\mid$	& $(T_1, \dots ,T_n) $	& \lst{STuple} & tuple of $n$ elements (see \lst{Tuple} type)\\

			&   $\mid$  & $(T_1,\dots,T_n) \to T $	& \lst{SFunc} & function of $n$ arguments (see \lst{Func} type) \\
			&   $\mid$  & $\text{\lst{Coll}}[T]$			& \lst{SCollection} & collection of elements of type $T$   \\
			&   $\mid$  & $\text{\lst{Option}}[T]$		& \lst{SOption} & optional value of type $T$  \\
			& 	     	&									& 				&		\\

$Term\ni e$	& ::= 		&   $C(v, T)$				& \lst{Constant} & typed constants  \\
			& 	$\mid$ 	& 	$x$ 					& \lst{ValUse} & variables  \\
			& 	$\mid$ 	& 	$\TyLam{x_i}{T_i}{e}$ 		& \lst{FuncExpr} & lambda expression \\
			& 	$\mid$ 	& 	$\Apply{e_f}{\Ov{e_i}}$ 	& \lst{Apply} & application of functional expression \\
			& 	$\mid$ 	& 	$\Apply{e.m}{\Ov{e_i}}$		& \lst{MethodCall} & method invocation  \\
			& 	$\mid$ 	&   $\Tup{e_1, \dots ,e_n}$ 	& \lst{Tuple} & constructor of tuple with $n$ items \\
			& 	$\mid$ 	& 	$\Apply{\delta}{\Ov{e_i}}$ 	& & primitive application (see Appendix~\ref{sec:appendix:primops}) \\
			& 	$\mid$ 	& 	\lst{if} $(e_{cond})$ $e_1$ \lst{else} $e_2$ & \lst{If} & if-then-else expression \\
			& 	$\mid$ 	&   $\{ \Ov{\text{\lst{val}}~x_i = e_i;}~e\}$  & \lst{BlockExpr} & block expression \\
			& 	     	&										& &				\\
$cd$   		& ::= 		& 	$\Trait{I}{\Ov{ms_i}}$				& \lst{STypeCompanion} & interface declaration    \\
			& 	     	&										& &				\\
$ms$	   	& ::= 	& $\MSig{m[\Ov{\tau_i}]}{\overline{x_i : T_i}}{T}$ 	& \lst{SMethod} & method signature declaration   \\
\end{tabular}\] 


    \caption{Abstract syntax of ErgoScript language}
    \label{fig:language}
\end{figure}
    
We assign types to the terms in a standard way following typing rules shown
in Figure~\ref{fig:typing}.

Constants keep both the type and the data value of that type. To be
well-formed the type of the constant should correspond to its value.

Variables are always typed and identified by unique $id$, which refers to
either lambda bound variable of \lst{val} bound variable. The encoding of
variables and their resolution is described in Section~\ref{sec:blocks}.

Lambda expressions can take a list of lambda-bound variables which can be
used in the body expression, which can be \emph{block expression}. 

Function application takes an expression of functional type (e.g. $T_1 \to
T_n$) and a list of arguments. The reason we do not write it $e_f(\Ov{e})$
is that this notation suggests that $(\Ov{e})$ is a subterm, which it is not.

Method invocation allows to apply functions defined as methods of
\emph{interface types}. If expression $e$ has interface type $I$ and and
method $m$ is declared in the interface $I$ then method invocation
$e.m(args)$ is defined for the appropriate $args$.

Conditional expressions of \langname are strict in condition and lazy in both
of the branches. Each branch is an expression which is executed depending on
the result of condition. This laziness of branches specified by lowering to
\corelang (see Figure~\ref{fig:lowering}).

Block expression contains a list of \lst{val} definitions of variables. To be
wellformed each subsequent definition can only refer to the previously defined
variables. Result of block execution is the result of the resulting
expression $e$, which can use any variable of the block.

Each type may be associated with a list of method declarations, in which case
we say that \emph{the type has methods}. The semantics of the methods is the
same as in Java. Having an instance of some type with methods it is possible
to call methods on the instance with some additional arguments.
Each method can be parameterized by type variables, which
can be used in method signature. Because \langname supports only monomorphic
values each method call is monomorphic and all type variables are assigned to
concrete types (see \lst{MethodCall} typing rule in Figure~\ref{fig:typing}).

The semantics of \langname is specified by translating all its terms to a
somewhat lower and simplified language, which we call \corelang. This
\emph{lowering} translation is shown in Figure~\ref{fig:lowering}. 

\begin{figure}[h]
\begin{center}
\begin{tabular}{ l c l }
	\hline
$Term_{\langname}$ &  & $Term_{Core}$  \\	
	\hline

$\Low{ \TyLam{x_i}{T_i}{e} 		}$ & \To & 
		$\Lam{x:(T_0,\dots,T_n)}{ \Low{ \{ \Ov{\lst{val}~x_i: T_i = x.\_i;}~e\} } }$ \\	

$\Low{ \Apply{e_f}{\Ov{e_i}} 	}$ & \To & $\Apply{ \Low{e_f} }{ \Low{(\Ov{e_i})} }$ \\	
$\Low{ \Apply{e.m}{\Ov{e_i}}	}$ & \To & $\Apply{ \Low{e}.m}{\Ov{ \Low{e_i} }}$ \\	
$\Low{ \Tup{e_1, \dots ,e_n}	}$ & \To & $\Tup{\Low{e_1}, \dots ,\Low{e_n}}$ \\	

$\Low{ e_1~\text{\lst{||}}~e_2	    }$ & \To & $\Low{ \IfThenElse{ e_1 }{ \True }{ e_2 } }$ \\	
$\Low{ e_1~\text{\lst{\&\&}}~e_2	}$ & \To & $\Low{ \IfThenElse{ e_1 }{ e_2 }{ \False } }$ \\	

$\Low{ \IfThenElse{e_{cond}}{e_1}{e_2} }$ & \To & 
		$\Apply{(\lst{if}(\Low{e_{cond}} ,~\Lam{(\_:Unit)}{\Low{e_1}} ,~\Lam{(\_:Unit)}{\Low{e_2}} ))}{}$ \\ 

$\Low{ \{ \Ov{\text{\lst{val}}~x_i: T_i = e_i;}~e\} }$ & \To &  
		$\Apply{ (\Lam{(x_1:T_1)}{( \dots \Apply{(\Lam{(x_n:T_n)}{\Low{e}})}{\Low{e_n}} \dots )}) }{\Low{e_1}}$\\

$\Low{ \Apply{\delta}{\Ov{e_i}}	}$ & \To & $\Apply{\delta}{\Ov{ \Low{e_i} }}$ \\	
$\Low{ e }$ 	& \To &  $e$ \\	
\end{tabular}
\end{center}
\caption{Lowering to \corelang}
\label{fig:lowering}
\end{figure}

All $n$-ary lambdas when $n>1$ are transformed to single arguments lambdas
using tupled arguments.
Note that $\IfThenElse{e_{cond}}{e_1}{e_2}$ term of \langname has lazy
evaluation of its branches whereas right-hand-side \lst{if} is a primitive
operation and have strict evaluation of the arguments. The laziness
is achieved by using lambda expressions of \lst{Unit} $\to$ \lst{Boolean}
type.

We translate logical operations (\lst{||}, \lst{&&}) of \langname, which are
lazy on second argument to \lst{if} term of \langname, which is recursively
translated to the corresponding \corelang term.

Syntactic blocks of \langname are completely eliminated and translated to
nested lambda expressions, which unambiguously specify evaluation semantics
of blocks. The \corelang is specified in Section~\ref{sec:evaluation}.



\section{Typing}
\label{sec:typing}

\langname is a strictly typed language, in which every term should have a
type in order to be wellformed and evaluated. Typing judgement of the form
$\Der{\Gamma}{e : T}$ say that $e$ is a term of type $T$ in the typing
context $\Gamma$.

\begin{figure}[h]

\begin{center}
% var and consts
\(\begin{array}{c c c}
\frac{}{\Der{\Gamma,x : \tau}{x : \tau}}
 	 & 
\frac{}{\Der{\Gamma}{l : \text{\lst{Int}}}}
     &
\frac{}{\Der{\Gamma}{() : \text{\lst{Unit}}}}
	 \\
	 & & \\
\end{array}\) 

% primitive
\(
\frac{\oplus : (\tau_1\times\tau_2) \to \tau_3~~~\DerEnv{e_1 : \tau_1}~~~\DerEnv{e_2 : \tau_2}}{\DerEnv{e_1 \oplus e_2 : \tau_3}}
\)

% pairs
\(\begin{array}{c c c}
 & \\
\frac{\DerEnv{e : \TPair{\tau_1}{\tau_2}}}{\DerEnv{\Fst{e} : \tau_1}}
     &
\frac{\DerEnv{e : \TPair{\tau_1}{\tau_2}}}{\DerEnv{\Snd{e} : \tau_2}}
     &
\frac{\DerEnv{e_1 : \tau_1}~~~\DerEnv{e_2 : \tau_2}}{\DerEnv{\Tup{e_1,e_2} : \TPair{\tau_1}{\tau_2}}}
	 \\
\end{array}\) 

% sums
\(\begin{array}{c c}
 & \\
\frac{\DerEnv{e : \tau_1}}{\DerEnv{\Left{\tau_1}{\tau_2}{e} : \tau_1 + \tau_2}}
     &
\frac{\DerEnv{e : \tau_2}}{\DerEnv{\Right{\tau_1}{\tau_2}{e} : \tau_1 + \tau_2}}
	 \\
	& \\ 
\end{array}\) 

% case
\(\begin{array}{c}
\frac{\DerEnv{e : \tau_1 + \tau_2}~~~\Der{\Gamma,x_1 : \tau_1}{e_1 : \tau}~~~\Der{\Gamma,x_2 : \tau_2}{e_2 : \tau}}
     {\DerEnv{\CaseOfXX{e}{\Left{\tau_1}{\tau_2}{x_1} \to e_1;~~\Right{\tau_1}{\tau_2}{x_2} \to e_2} : \tau}} \\
  \\
\end{array}\) 

% case
\(\begin{array}{c}
\frac{\DerEnv{e : \text{\lst{Int}}}~~~\DerEnv{e_i : \tau}}
     {\DerEnv{\CaseOf{e}{l_i}{e_i} : \tau}} \\
  \\
\end{array}\)

% let
% \(
% \frac{\Der{\TEnv,x : \tau_1}{e_2 : \tau_2}}{\Der{\Gamma}{\Let{x}{e_1}{e_2} : \tau_2}}
% \)

% functions
\(\begin{array}{c c}
\frac{\Der{\TEnv,x:\tau_1}{e:\tau_2}}{\Der{\Gamma}{\TyLam{x}{\tau_1}{e} : \tau_1 \to \tau_2}}
 	 & 
\frac{\Der{\TEnv}{e_1 : \tau_2 \to \tau}~~~\Der{\TEnv}{e_2 : \tau_2}}{\Der{\Gamma}{e_1~e_2 : \tau}}
	 \\
\end{array}\) 

\end{center}



\caption{Typing rules of \langname}
\label{fig:typing}
\end{figure}

Note that each well-typed term has exactly one type hence we assume there
exists a funcion $termType: Term \to \mathcal{T}$ which relates each well-typed
term with the corresponding type.

\section{Evaluation Semantics}
\label{sec:evaluation}

  Evaluation of \langname is specified by its translation to \corelang, whose
terms form a subset of \langname terms. Thus, typing rules of \corelang form
a subset of typing rules of \langname.

Here we specifiy evaluation semantics of \corelang, which is based on
call-by-value (CBV) lambda calculus. Evaluation of \corelang is specified
using denotational semantics. To do that we first specify denotations of
types, then typed terms and then equations of denotational semantics.

\begin{definition}
  (values, producers)
  \begin{itemize}
    \item The following CBV terms are called values:
    $$ V :== x \mid C(d, T) \mid \Lam{x}{M}$$
    \item All CBV terms are called producers. (This is because, when evaluated, they produce a value.)
  \end{itemize}
\end{definition}

We now describe and explain a denotational semantics for the \corelang
language. The key principle is that each type $A$ denotes a set $\Denot{A}$
whose elements are the denotations of values of type $A$.

Thus the type \lst{Boolean} denotes the 2-element set
$\{\lst{true},\lst{false}\}$, because there are two values of type
\lst{Boolean}. Likewise the type $(T_1,\dots,T_n)$ denotes
$(\Denot{T_1},\dots,\Denot{T_n})$ because a value of type $(T_1,\dots,T_n)$
must be of the form $(V_1,\dots,V_n)$, where each $V_i$ is value of type
$T_i$.

Given a value $V$ of type $A$, we write $\Denot{V}$ for the element of $A$
that it denotes. Given a close term $M$ of type $A$, we recall that it
produces a value $V$ of type $A$. So $M$ will denote an element $\Denot{M}$
of $\Denot{A}$.

A value of type $A \to B$ is of the form $\Lam{x}{M}$. This, when
applied to a value of type $A$ gives a value of type $B$. So $A \to B$
denotes $\Denot{A} \to \Denot{B}$. It is true that the syntax appears to
allow us to apply $\Lam{x}{M}$ to any term $N$ of type $A$. But $N$ will be
evaluated before it interracts with $\Lam{x}{M}$, so $\Lam{x}{M}$ is really only applied to the value that $N$ produces.

\begin{definition}
 A \emph{context} $\Gamma$ is a finite sequence of identifiers with value
 types $x_1:A_1, \dots ,x_n:A_n$. Sometimes we omit the identifiers and
 write $\Gamma$ as a list of value types.
\end{definition}

Given a context $\Gamma = x_1:A_1,\dots,x_n:A_n$, an environment (list of
bindings for identifiers) associates to each $x_i$ as value of type $A_i$. So
the environment denotes an element of $(\Denot{A_1},\dots,\Denot{A_n})$, and
we write $\Denot{\Gamma}$ for this set.

Given a \corelang term $\DerEnv{M: B}$, we see that $M$, together with
environment, gives a closed term of type $B$. So $M$ denotes a function
$\Denot{M}$ from $\Denot{\Gamma}$ to $\Denot{B}$.

In summary, the denotational semantics is organized as follows.

\begin{itemize}
  \item A type $A$ denotes a set $\Denot{A}$
  \item A context $x_1:A_1,\dots,x_n:A_n$ denotes the set $(\Denot{A_1},\dots,\Denot{A_n})$
  \item A term $\DerEnv{M: B}$ denotes a function $\Denot{M}:
  \Denot{\Gamma} \to \Denot{B}$
\end{itemize}

The denotations of types and terms is given in Figure~\ref{fig:denotations}.

\begin{figure}[h]

The denotations of \corelang types

\begin{center}
  \(\begin{array}{ l c l }
  \Denot{\lst{Boolean}} & = & \{ \lst{true}, \lst{false} \}  \\	
  \Denot{\lst{P}} & = & \text{see Appendix~\ref{sec:appendix:predeftypes}} \\	
  \Denot{(T_1,\dots,T_n)} & = & (\Denot{T_1},\dots,\Denot{T_n})  \\	
  \Denot{A \to B} & = & \Denot{A} \to \Denot{B}  \\	
  \end{array}\)
\end{center}

The denotations of \corelang terms

\begin{center}
  \(\begin{array}{ l c l }
  \Apply{ \Denot{\lst{x}}			}{(\rho,\lst{x}\mapsto x, \rho')} & = & x \\	
  \Apply{ \Denot{C(d, T)} 			}{\rho} & = & d \\	
  \Apply{ \Denot{(\Ov{M_i})} 		}{\rho} & = & (\Ov{ \Apply{\Denot{M_i}}{\rho} }) \\	

  \Apply{ \Denot{\Apply{\delta}{N}} }{\rho} & = 
		& \Apply{ (\Apply{\Denot{\delta}}{\rho}) }{ v }~where~v = \Apply{\Denot{N}}{\rho} \\	

  \Apply{ \Denot{\Lam{\lst{x}}{M}}	}{\rho} & = 
		& \Lam{x}{ \Apply{\Denot{M}}{(\rho, \lst{x}\mapsto x)} } \\	

  \Apply{ \Denot{\Apply{M_f}{N}}	}{\rho} & = 
		& \Apply{ (\Apply{\Denot{M_f}}{\rho}) }{ v }~where~v = \Apply{\Denot{N}}{\rho} \\	

  \Apply{ \Denot{ \Apply{M_I.\lst{m}}{\Ov{N_i}} }	}{\rho} & = 
		& \Apply{ (\Apply{\Denot{M_I}}{\rho}).m }{ \Ov{v_i} }~where~\Ov{v_i = \Apply{\Denot{N_i}}{\rho}} \\	
  \end{array}\)
\end{center}

\caption{Denotational semantics of \corelang}
\label{fig:denotations}
\end{figure}

\section{Serialization}
\label{sec:serialization}

This section defines a binary format, which is used to store \langname
contracts in persistent stores, to transfer them over the wire and to enable
cross-platform interoperation.

Terms of the language described in Section~\ref{sec:language} can be
serialized to array of bytes to be stored in Ergo blockchain (e.g. as
\lst{Box.propositionBytes}).

When the guarding script of an input box of a transaction is validated the
\lst{propositionBytes} array is deserialized to an \langname IR (represented by the
\lst{ErgoTree} class), which can be evaluated as it is specified in
Section~\ref{sec:evaluation}.

Here we specify the serialization procedure in general. The serialization format of
\langname types (\lst{SType} class) and nodes (\lst{Value} class) is specified in
section~\ref{sec:ser:type} and Appendix~\ref{sec:appendix:ergotree_serialization}
correspondingly.

Table~\ref{table:ser:limits} shows size limits which are checked during
contract deserialization, which is important to resist malicious script attacks.

\begin{table}[h]
    \caption{Serialization limits}\vspace{-7pt}
    \label{table:ser:limits}
    \footnotesize
\(\begin{tabularx}{\textwidth}{| l | p{2.5cm} | X |}
    \hline
    \bf{Name}   & \bf{Value} & \bf{Description} \\
    \hline
    $\MaxVlqSize$  & $10$ & Maximum size of VLQ encoded byte sequence (See VLQ formats~\ref{sec:vlq-encoding})  \\
    \hline
    $\MaxTypeSize$ & $100$ & Maximum size of serialized type term (see Type format~\ref{sec:ser:type}) \\
    \hline
    $\MaxDataSize$ & $10Kb$ & Maximum size of serialized data instance (see Data format~\ref{sec:ser:data}) \\
    \hline
    $\MaxConstSize$ & $=\MaxTypeSize+\MaxDataSize$  & Maximum size of serialized data instance (see Const format~\ref{sec:ser:const}) \\
    \hline
    $\MaxExprSize$ & $4Kb$ & Maximum size of serialized \langname term (see Expr format~\ref{sec:ser:expr}) \\
    \hline
    $\MaxErgoTreeSize$ & $4Kb$ & Maximum size of serialized \langname contract (see ErgoTree format~\ref{sec:ser:ergotree}) \\
    \hline
\end{tabularx}\)

\end{table}

All the serialization formats which are uses and defined thoughout this section are
listed in Table~\ref{table:ser:formats} which introduces a name for each format and also
shows the number of bytes each format may occupy in the byte stream.

\begin{table}[H] \scriptsize
\caption{Serialization formats}\vspace{-7pt}
\label{table:ser:formats}
\(\begin{tabularx}{\textwidth}{| l | l | X |}
    \hline
    \bf{Format} & \bf{\#bytes} & \bf{Description} \\
    \hline
    \lst{Byte} & $1$ & 8-bit signed two's-complement integer \\
    \hline
    \lst{Short} & $2$ & 16-bit signed two's-complement integer (big-endian) \\
    \hline    
    \lst{Int} & $4$ & 32-bit signed two's-complement integer (big-endian) \\
    \hline
    \lst{Long} & $8$ & 64-bit signed two's-complement integer (big-endian) \\
    \hline
    \lst{UByte} & $1$ & 8-bit unsigned integer \\
    \hline
    \lst{UShort} & $2$ & 16-bit unsigned integer (big-endian) \\
    \hline    
    \lst{UInt} & $4$ & 32-bit unsigned integer (big-endian) \\
    \hline
    \lst{ULong} & $8$ & 64-bit unsigned integer (big-endian) \\

    \hline
    \lst{VLQ(UShort)} & $[1..3]$ & Encoded unsigned \lst{Short} value using VLQ. See~\cite{VLQWikipedia,VLQRosetta} and~\ref{sec:vlq-encoding} \\
    \hline    
    \lst{VLQ(UInt)} & $[1..5]$ & Encoded unsigned 32-bit integer using VLQ. \\
    \hline
    \lst{VLQ(ULong)} & $[1..\MaxVlqSize]$ & Encoded unsigned 64-bit integer using VLQ. \\

    \hline
    \lst{Bits} & $[1..\MaxBits]$ & A collection of bits packed in a sequence of bytes. \\
    \hline
    \lst{Bytes} & $[1..\MaxBytes]$ & A sequence of bytes, which size is stored elsewhere or wellknown. \\

    \hline
    \lst{Type} & $[1..\MaxTypeSize]$ & Serialized type terms of \langname. See~\ref{sec:ser:type} \\
    \hline
    \lst{Data} & $[1..\MaxDataSize]$ & Serialized \langname values. See~\ref{sec:ser:data} \\
    \hline
    \lst{GroupElement} & $33$ & Serialized elements of eliptic curve group. See~\ref{sec:ser:data:groupelement} \\
    \hline
    \lst{SigmaProp} & $[1..\MaxSigmaProp]$ & Serialized sigma propositions. See~\ref{sec:ser:data:sigmaprop} \\
    \hline
    \lst{Box} & $[1..\MaxBox]$ & Serialized box data. See~\ref{sec:ser:data:box} \\
    \hline
    \lst{AvlTree} & $44$ & Serialized dynamic dictionary digest. See~\ref{sec:ser:data:avltree} \\
    \hline
    \lst{Const} & $[1..\MaxConstSize]$ & Serialized \langname constants (values with types). See~\ref{sec:ser:const} \\
    \hline
    \lst{Expr} & $[1..\MaxExprSize]$ & Serialized expression terms of \langname. See~\ref{sec:ser:expr} \\
    \hline
    \lst{ErgoTree} & $[1..\MaxErgoTreeSize]$ & Serialized instances of \langname contracts. See~\ref{sec:ser:ergotree} \\
    \hline
\end{tabularx}\)
\end{table}

We use $[1..n]$ notation when serialization may produce from 1 to n bytes depending of
actual data instance.

Serialization format of \ASDag is optimized for compact storage and very fast
deserialization. In many cases serialization procedure is data dependent and thus have
branching logic. To express this complex serialization logic in the specification we
use a \emph{pseudo-language} with operators like \lst{for, match, if, optional}. The
language allows to specify a \emph{structure} out of \emph{simple serialization slots}.
Each \emph{slot} specifies a fragment of serialized stream of bytes, whereas
\emph{operators} specifiy how the slots are combined together to form the resulting
stream of bytes. The notation is summarized in Table~\ref{table:ser:notation}.

\begin{table}[h] \footnotesize
\caption{Serialization Notation}
\label{table:ser:notation}
\(\begin{tabularx}{\textwidth}{| l | X |}
    \hline
    \bf{Notation} & \bf{Description} \\
    \hline
    $\Denot{T}$ where $T$ - type & Denotes a set of values of type $T$  \\
    \hline
    $v \in \Denot{T}$ & The value $v$ belongs to the set $\Denot{T}$ \\
    \hline   
    $v : T$ & Same as $v \in \Denot{T}$ \\
    \hline    
    \lst{match} $(t, v)$ & Pattern match on pair $(t, v)$ where $t, v$ - values \\
    \hline    
    \lst{with} $(Unit, v \in \Denot{Unit})$ & Pattern case \\
    \hline    

    \lst{for}~$i=1$~\lst{to}~$len$ & \multirow{3}{=}{Call 
        the given \lst{serialize} function repeatedly. 
        The outputs bytes of all invocations are concatenated and become 
        the output of the \lst{for} statement. } \\
    ~~\lst{serialize(}$v_i$\lst{)} &  \\
    \lst{end for} & \\
    \hline

    \lst{if}~$condition$~\lst{then} & \multirow[t]{5}{=}{Serialize 
        one of the branches depending of the $condition$.  
        The output bytes of the executed branch becomes the output of the \lst{if} statement. } \\
    ~~\lst{serialize1(}$v_1$\lst{)} &  \\
    \lst{else} & \\
    ~~\lst{serialize2(}$v_2$\lst{)} &  \\
    \lst{end if} & \\
    \hline
\end{tabularx}\)
\end{table}

In the next section we describe how types (like \lst{Int}, \lst{Coll[Byte]},
etc.) are serialized, then we define serialization of typed data. This will
give us a basis to describe serialization of Constant nodes of \ASDag. From
that we will proceed to serialization of arbitrary \ASDag trees.

\subsection{Type Serialization}
\label{sec:ser:type}

For motivation behind this type encoding please see
Appendix~\ref{sec:appendix:motivation:type}.

\subsubsection{Distribution of type codes}
\label{sec:ser:type:codedist}

The whole space of 256 one byte codes is divided as shown in
Figure~\ref{fig:ser:type:codedist}.

\begin{figure}[h] \footnotesize
\(\begin{tabularx}{\textwidth}{| l | X |}
    \hline
    \bf{Value/Interval} & \bf{Distribution} \\
    \hline
    \lst{0x00} & special value to represent undefined type (\lst{NoType} in \ASDag) \\
    \hline
    \lst{0x01 - 0x6F(111)} & \emph{data types} including primitive types, arrays, options
    aka nullable types, classes (in future), 111 = 255 - 144 different codes \\
    \hline
    \lst{0x70(112) - 0xFF(255)} & \emph{function types} \lst{T1 => T2}, 144 = 12 x 12
    different codes~\footnote{Note that the function types are never serialized in version 1 of the Ergo
    protocol, this encoding is reserved for future development of the protocol.} \\
    \hline 
\end{tabularx}\)
\caption{Distribution of type codes between Data and Function types}
\label{fig:ser:type:codedist}
\end{figure}

\subsubsection{Encoding of Data Types}

There are eight different values for \emph{embeddable} types and 3 more are reserved
for the future extensions. Each embeddable type has a type code in the range {1,...,11}
as shown in Figure~\ref{fig:ser:type:embeddable}.

\begin{figure}[h] \footnotesize
    \(\begin{tabularx}{\textwidth}{| l | X |}
        \hline
        \bf{Code} & \bf{Type} \\ \hline
1     &   Boolean \\  \hline
2     &   Byte\\  \hline
3     &   Short (16 bit)\\  \hline
4     &   Int (32 bit)\\  \hline
5     &   Long (64 bit)\\  \hline
6     &   BigInt (represented by java.math.BigInteger)\\  \hline
7     &   GroupElement (represented by org.bouncycastle.math.ec.ECPoint)\\  \hline
8     &   SigmaProp \\  \hline
9     &   reserved for Char \\  \hline
10    &   reserved \\  \hline
11    &   reserved \\ \hline 
\end{tabularx}\)
\caption{Embeddable Types}
\label{fig:ser:type:embeddable}
\end{figure}

For each type constructor like \lst{Coll} or \lst{Option} we use the encoding schema
defined below. Type constructor has an associated \emph{base code} which is multiple
of~$12$ (e.g.~$12$ for \lst{Coll[_]}, $24$ for \lst{Coll[Coll[_]]} etc.).
The base code can be added to the embeddable type code to produce the code of the constructed
type, for example $12 + 1 = 13$ is a code of \lst{Coll[Byte]}. 
The code of type
constructor (e.g. $12$ in this example) is used when type parameter is non-embeddable
type (e.g. \lst{Coll[(Byte, Int)]}). In this case the code of type
constructor is read first, and then recursive descent is performed to read
bytes of the parameter type (in this case \lst{(Byte, Int)}). This encoding
allows very simple and fast decoding by using \lst{div} and \lst{mod} operations.

Following the above encoding schema the interval of codes for data types is divided as
shown in Figure~\ref{fig:ser:type:datatypes}.

\begin{figure}[h] \footnotesize
\(\begin{tabularx}{\textwidth}{| l | l | X |}
\hline
\bf{Interval}       & \bf{Type constructor} & \bf{Description} \\ \hline
0x01 - 0x0B(11)     &                       & embeddable types (including 3 reserved) \\ \hline
0x0C(12)            & \lst{Coll[_]}         & Collection of non-embeddable types (\lst{Coll[(Int,Boolean)]}) \\ \hline
0x0D(13) - 0x17(23) & \lst{Coll[_]}         & Collection of embeddable types (\lst{Coll[Byte]}, \lst{Coll[Int]}, etc.) \\ \hline
0x18(24)            & \lst{Coll[Coll[_]]}   & Nested collection of non-embeddable types (\lst{Coll[Coll[(Int,Boolean)]]}) \\ \hline
0x19(25) - 0x23(35) & \lst{Coll[Coll[_]]}   & Nested collection of embeddable types (\lst{Coll[Coll[Byte]]}, \lst{Coll[Coll[Int]]}) \\ \hline
0x24(36)            & \lst{Option[_]}       & Option of non-embeddable type (\lst{Option[(Int, Byte)]}) \\ \hline
0x25(37) - 0x2F(47) & \lst{Option[_]}       & Option of embeddable type (\lst{Option[Int]}) \\ \hline
0x30(48)            & \lst{Option[Coll[_]]} & Option of Coll of non-embeddable type (\lst{Option[Coll[(Int, Boolean)]]}) \\ \hline
0x31(49) - 0x3B(59) & \lst{Option[Coll[_]]} & Option of Coll of embeddable type (\lst{Option[Coll[Int]]}) \\ \hline
0x3C(60)            & \lst{(_,_)}           & Pair of non-embeddable types (\lst{((Int, Byte), (Boolean,Box))}, etc.) \\ \hline
0x3D(61) - 0x47(71) & \lst{(_, Int)}        & Pair of types where first is embeddable (\lst{(_, Int)}) \\ \hline
0x48(72)            & \lst{(_,_,_)}         & Triple of types  \\ \hline
0x49(73) - 0x53(83) & \lst{(Int, _)}        & Pair of types where second is embeddable (\lst{(Int, _)}) \\ \hline
0x54(84)            & \lst{(_,_,_,_)}       & Quadruple of types  \\ \hline
0x55(85) - 0x5F(95) & \lst{(_, _)}          & Symmetric pair of embeddable types (\lst{(Int, Int)}, \lst{(Byte,Byte)}, etc.) \\ \hline
0x60(96)            & \lst{(_,...,_)}       & \lst{Tuple} type with more than 4 items \lst{(Int, Byte, Box, Boolean, Int)} \\ \hline
0x61(97)            & \lst{Any}             & Any type  \\ \hline
0x62(98)            & \lst{Unit}            & Unit type \\ \hline
0x63(99)            & \lst{Box}             & Box type  \\ \hline
0x64(100)           & \lst{AvlTree}         & AvlTree type  \\ \hline
0x65(101)           & \lst{Context}         & Context type  \\ \hline
0x66(102)           &                       & reserved for String  \\ \hline
0x67(103)           &                       & reserved for TypeVar  \\ \hline
0x68(104)           & \lst{Header}          & Header type  \\ \hline
0x69(105)           & \lst{PreHeader}       & PreHeader type  \\ \hline
0x6A(106)           & \lst{Global}          & Global type  \\ \hline
0x6B(107)- 0x6E(110)&                       & reserved for future use  \\ \hline
0x6F(111)           &                       & Reserved for future \lst{Class} type (e.g. user-defined types)  \\ \hline
\end{tabularx}\)
\caption{Code Ranges of Data Types}
\label{fig:ser:type:datatypes}
\end{figure}

\subsubsection{Encoding of Function Types}

We use $12$ different values for both domain and range types of functions. This
gives us $12 * 12 = 144$ function types in total and allows to represent $11 *
11 = 121$ functions over primitive types using just single byte.

Each code $F$ in a range of the function types (i.e $F \in \{112, \dots, 255\}$) can be
represented as $F~=~D * 12 + R + 112$, where $D, R \in \{0,\dots,11\}$ - indices of
domain and range types correspondingly, $112$ - is the first code in an interval of
function types.

If $D = 0$ then the domain type is not embeddable and the recursive descent is
necessary to write/read the domain type.

If $R = 0$ then the range type is not embeddable and the recursive descent is necessary
to write/read the range type.

\subsubsection{Recursive Descent}
\label{sec:ser:type:recursive}

When an argument of a type constructor is not a primitive type we fallback to the
simple encoding schema in which case we emit the separate code for the type constructor
according to the table above and descend recursively to every child node of the type
tree.

We do this descend only for those children whose code cannot be embedded in
the parent code. For example, serialization of \lst{Coll[(Int,Boolean)]}
proceeds as the following:
\begin{enumerate}
\item Emit \lst{0x0C} because the elements type of the collection is not embeddable 
\item Recursively serialize \lst{(Int, Boolean)}
\item Emit \lst{0x41(=0x3D+4)} because the first type of the pair is embeddable and its code is~$4$
\item Recursivley serialize \lst{Boolean}
\item Emit \lst{0x02} - the code for embeddable type \lst{Boolean}
\end{enumerate}

More examples of type serialization are shown in Figure~\ref{fig:ser:type:examples}
\begin{figure}[h] \footnotesize
\(\begin{tabularx}{\textwidth}{| l | c | c | l | c | X |}
\hline
\bf{Type}                &\bf{D} & \bf{R} & \bf{Bytes} & \bf{\#Bytes} &  \bf{Comments} \\ \hline
\lst{Byte}               &     &     &  1                   &  1     &    \\ \hline
\lst{Coll[Byte]}         &     &     &  12 + 1 = 13         &  1     &    \\ \hline
\lst{Coll[Coll[Byte]]}   &     &     &  24 + 1 = 25         &  1     &     \\ \hline
\lst{Option[Byte]}       &     &     &  36 + 1 = 37         &  1     & register    \\ \hline
\lst{Option[Coll[Byte]]} &     &     &  48 + 1 = 49         &  1     & register    \\ \hline
\lst{(Int,Int)}          &     &     &  84 + 3 = 87         &  1     & fold    \\ \hline
\lst{Box=>Boolean}       & 7   & 2   &  198 = 7*12+2+112    &  1     & exist, forall    \\ \hline
\lst{(Int,Int)=>Int}     & 0   & 3   &  115=0*12+3+112, 87  &  2     &  fold    \\ \hline
\lst{(Int,Boolean)}      &     &     &  60 + 3, 2           &  2     &      \\ \hline
\lst{(Int,Box)=>Boolean} & 0   & 2   &  0*12+2+112, 60+3, 7 &  3     &     \\ \hline
\end{tabularx}\)
\caption{Examples of type serialization}
\label{fig:ser:type:examples}
\end{figure}


\subsection{Data Serialization}
\label{sec:ser:data}

In \langname all runtime data values have an associated type also available
at runtime (this is called \emph{type reification}\cite{Reification}).
However serialization format separates data values from its type descriptors. 
This allows to save space when for example a collection of items is serialized.

It is done is such a way that the contents of a typed data structure can be fully
described by a type tree. For example having a typed data object \lst{d: (Int,
Coll[Byte], Boolean)} we can tell, by examining the structure of the type, that \lst{d}
is a tuple with 3 items, the first item contain 32-bit integer, the second - collection
of bytes, and the third - logical true/false value.

To serialize/deserialize typed data we need to know its \emph{type descriptor} (type
tree). The data serialization procedure is recursive over a type tree and the
corresponding sub-components of the data object. For primitive types (the leaves of the
type tree) the format is fixed. The data values of \langname types are serialized
according to the predefined recursive function shown in Figure~\ref{fig:ser:data} which
uses the notation from Table~\ref{table:ser:notation}.

\begin{figure}[H] \footnotesize
\caption{Data serialization format}\vspace{-7pt}
\label{fig:ser:data}
\(\begin{tabularx}{\textwidth}{| l | l | l | X |}
    \hline
    \bf{Slot} & \bf{Format} & \bf{\#bytes} & \bf{Description} \\
    \hline
    \hline
    \multicolumn{4}{l}{\lst{def serializeData(}$t, v$\lst{)}} \\
    \multicolumn{4}{l}{~~\lst{match} $(t, v)$ } \\

    \multicolumn{4}{l}{~~~~\lst{with} $(Unit, v \in \Denot{Unit})$~~~// nothing serialized } \\
    \multicolumn{4}{l}{~~~~\lst{with} $(Boolean, v \in \Denot{Boolean})$} \\
    \hline
    $~~~~~~v$ & \lst{Byte} & 1 & 0 if $v = false$ or 1 otherwise \\

    \hline
    \multicolumn{4}{l}{~~~~\lst{with} $(Byte, v \in \Denot{Byte})$} \\
    \hline
    $~~~~~~v$  & \lst{Byte} & 1 &  in a single byte \\

    \hline
    \multicolumn{4}{l}{~~~~\lst{with} $(N, v \in \Denot{N}), N \in {Short, Int, Long}$} \\
    \hline
    $~~~~~~v$  & \lst{VLQ(ZigZag($$N$$))} & [1..3] & 
      16,32,64-bit signed integer encoded using \hyperref[sec:zigzag-encoding]{ZigZag} 
      and then \hyperref[sec:vlq-encoding]{VLQ} \\

    \hline
    \multicolumn{4}{l}{~~~~\lst{with} $(BigInt, v \in \Denot{BigInt})$} \\
    \multicolumn{4}{l}{~~~~~~$bytes = v$\lst{.toByteArray} } \\
    \hline
    $~~~~~~numBytes$  & \lst{VLQ(UInt)} &  & number of bytes in $bytes$ array \\
    \hline
    $~~~~~~bytes$  & \lst{Bytes} &  & serialized $bytes$ array \\

    \hline
    \multicolumn{4}{l}{~~~~\lst{with} $(GroupElement, v \in \Denot{GroupElement})$} \\
    \hline
    ~~~~~~$v$  & \lst{GroupElement} &  & serialization of GroupElement data. See~\ref{sec:ser:data:groupelement} \\

    \hline
    \multicolumn{4}{l}{~~~~\lst{with} $(SigmaProp, v \in \Denot{SigmaProp})$} \\
    \hline
    ~~~~~~$v$  & \lst{SigmaProp} &  & serialization of SigmaProp data. See~\ref{sec:ser:data:sigmaprop} \\

    \hline
    \multicolumn{4}{l}{~~~~\lst{with} $(Box, v \in \Denot{Box})$} \\
    \hline
    ~~~~~~$v$  & \lst{Box} &  & serialization of Box data. See~\ref{sec:ser:data:box} \\

    \hline
    \multicolumn{4}{l}{~~~~\lst{with} $(AvlTree, v \in \Denot{AvlTree})$} \\
    \hline
    ~~~~~~$v$  & \lst{AvlTree} &  & serialization of AvlTree data. See~\ref{sec:ser:data:avltree} \\

    \hline
    \multicolumn{4}{l}{~~~~\lst{with} $(Coll[T], v \in \Denot{Coll[T]})$} \\
    \hline
    $~~~~~~len$  & \lst{VLQ(UShort)} & [1..3] & length of the collection \\
    \hline
    \multicolumn{4}{l}{~~~~~~\lst{match} $(T, v)$ } \\

    \multicolumn{4}{l}{~~~~~~~~\lst{with} $(Boolean, v \in \Denot{Coll[Boolean]})$} \\
    \hline
    $~~~~~~~~~~v$  & \lst{Bits} & [1..1024] & boolean values packed in bits \\
    \hline

    \multicolumn{4}{l}{~~~~~~~~\lst{with} $(Byte, v \in \Denot{Coll[Byte]})$} \\
    \hline
    $~~~~~~~~~~v$  & \lst{Bytes} & $[1..len]$ & items of the collection  \\
    \hline
    \multicolumn{4}{l}{~~~~~~~~\lst{otherwise} } \\
    \multicolumn{4}{l}{~~~~~~~~~~\lst{for}~$i=1$~\lst{to}~$len$~\lst{do}~\lst{serializeData(}$T, v_i$\lst{) end for}} \\
    \multicolumn{4}{l}{~~~~~~\lst{end match}} \\
    \multicolumn{4}{l}{~~\lst{end match}} \\
    \multicolumn{4}{l}{\lst{end serializeData}} \\
    \hline
    \hline
\end{tabularx}\)
\end{figure}

\subsubsection{GroupElement serialization}
\label{sec:ser:data:groupelement}

A value of the \lst{GroupElement} type is represented in reference implementation using
\lst{SecP256K1Point} class of the \lst{org.bouncycastle.math.ec.custom.sec} package and
serialized into ASN.1 encoding. During deserialization the different encodings are
taken account of, including point compression for $F_p$ (see X9.62 sec. 4.2.1 pg. 17).

\begin{figure}[H] \footnotesize
\caption{GroupElement serialization format}\vspace{-7pt}
\label{fig:ser:data:groupelement}
\(\begin{tabularx}{\textwidth}{| l | l | l | X |}
    \hline
    \bf{Slot} & \bf{Format} & \bf{\#bytes} & \bf{Description} \\
    \hline
    \multicolumn{4}{l}{\lst{def serialize(}$ge$\lst{)}} \\
    \multicolumn{4}{l}{~~\lst{if} $ge.isInfinity$ \lst{then}} \\
    \hline
    ~~~~$bytes$  & \lst{rep(}$0, 33$\lst{)} & $ 33 $ & all bytes = 0 \\ 
    \hline
    \multicolumn{4}{l}{~~\lst{else}} \\
    \hline
    ~~~~$bytes$  & Bytes & $33$ & see \lst{sigmastate.serialization.GroupElementSerializer} \\ 
    \hline
    \multicolumn{4}{l}{~~\lst{end if}} \\
    \multicolumn{4}{l}{\lst{end def}} \\
    \hline
    \hline
\end{tabularx}\)
\end{figure}

\subsubsection{SigmaProp serialization}
\label{sec:ser:data:sigmaprop}

\begin{figure}[H] \footnotesize
\caption{SigmaProp serialization format}\vspace{-7pt}
\label{fig:ser:data:sigmaprop}
\(\begin{tabularx}{\textwidth}{| l | l | l | X |}
    \hline
    \bf{Slot} & \bf{Format} & \bf{\#bytes} & \bf{Description} \\
    \hline
    \hline
\end{tabularx}\)
\end{figure}

\subsubsection{Box serialization}
\label{sec:ser:data:box}

\begin{figure}[H] \footnotesize
\caption{Box serialization format}\vspace{-7pt}
\label{fig:ser:data:box}
\(\begin{tabularx}{\textwidth}{| l | l | l | X |}
    \hline
    \bf{Slot} & \bf{Format} & \bf{\#bytes} & \bf{Description} \\
    \hline
    \hline
\end{tabularx}\)
\end{figure}

\subsubsection{AvlTree serialization}
\label{sec:ser:data:avltree}

\begin{figure}[H] \footnotesize
\caption{AvlTree serialization format}\vspace{-7pt}
\label{fig:ser:data:avltree}
\(\begin{tabularx}{\textwidth}{| l | l | l | X |}
    \hline
    \bf{Slot} & \bf{Format} & \bf{\#bytes} & \bf{Description} \\
    \hline
    \hline
\end{tabularx}\)
\end{figure}


\subsection{Constant Serialization}
\label{sec:ser:const}

\lst{Constant} format is simple and self sufficient to represent any data value in
\langname. Every data block of \lst{Constant} format contains both type and
data, such it can be stored or wire transfered and then later unambiguously
interpreted. The format is shown in Figure~\ref{fig:ser:const}

\begin{figure}[h]
\footnotesize
\(\begin{tabularx}{\textwidth}{| l | l | l | X |}
    \hline
    \bf{Slot} & \bf{Format} & \bf{\#bytes} & \bf{Description} \\
    \hline
    $type$  & \lst{Type} & $[1..\MaxTypeSize]$ & type of the data instance (see~\ref{sec:ser:type}) \\
    \hline
    $value$  & \lst{Data} & $[1..\MaxDataSize]$ & serialized data instance (see~\ref{sec:ser:data}) \\
    \hline
\end{tabularx}\)
\caption{Constant serialization format}
\label{fig:ser:const}
\end{figure}


\subsection{Expression Serialization}
\label{sec:ser:expr}

Expressions of \langname are serialized as tree data structure using
recursive procedure described here. 

\begin{figure}[h]
\footnotesize
\(\begin{tabularx}{\textwidth}{| l | l | l | X |}
    \hline
    \bf{Slot} & \bf{Format} & \bf{\#bytes} & \bf{Description} \\
    \hline
    \multicolumn{4}{l}{\lst{def serializeExpr(}$e$\lst{)}} \\
    \hline
    ~~$e.opCode$  & \lst{Byte} & $1$ & opcode of ErgoTree node, 
    used for selection of an appropriate node serializer from Appendix~\ref{sec:appendix:ergotree_serialization} \\
    \hline
    \multicolumn{4}{l}{~~\lst{if} $opCode <= LastConstantCode$ \lst{then}} \\
    \hline
    ~~~~$c$  & \lst{Const} & $[1..\MaxConstSize]$ & Constant serializaton slot \\ 
    \hline
    \multicolumn{4}{l}{~~\lst{else}} \\
    \hline
    ~~~~$body$  & Op & $[1..\MaxExprSize]$ & serialization of operation arguments 
    depending on $e.opCode$ as defined in Appendix~\ref{sec:appendix:ergotree_serialization} \\ 
    \hline
    \multicolumn{4}{l}{~~\lst{end if}} \\
    \multicolumn{4}{l}{\lst{end serializeExpr}} \\
    \hline
\end{tabularx}\)
\caption{Expression serialization format}
\label{fig:ser:expr}
\end{figure}


\subsection{\ASDag~serialization}
\label{sec:ser:ergotree}

The \langname propostions stored in UTXO boxes are represented in the reference
implementation using \lst{ErgoTree} class. The class is serialized using the \lst{ErgoTree}
serialization format  shown in Figure~\ref{fig:ser:ergotree}. 

\begin{figure}[h] \footnotesize
\caption{\ASDag serialization format}\vspace{-7pt}
\label{fig:ser:ergotree}
\(\begin{tabularx}{\textwidth}{| l | l | l | X |}
  \hline
  \bf{Slot} & \bf{Format} & \bf{\#bytes} & \bf{Description} \\
  \hline
  $ header $ & \lst{VLQ(UInt)} & [1, *] & the first bytes of serialized byte array which
  determines interpretation of the rest of the array \\
  \hline
  \multicolumn{4}{l}{\lst{if} $header[3] = 1$ \lst{then}} \\
  \hline
  ~~$size$ & \lst{VLQ(UInt)} & [1, *] & size of the constants and root, i.e. the number of bytes after $header$ and $size$ \\
  \hline
  \multicolumn{4}{l}{\lst{end for}} \\
  \hline
  $numConstants$ & \lst{VLQ(UInt)} & [1, *] & size of $constants$ array \\
  \hline
  \multicolumn{4}{l}{\lst{for}~$i=0$~\lst{to}~$numConstants-1$} \\
  \hline
      ~~ $ const_i $ & \lst{Const} & [1, *] & constant in i-th position \\
  \hline
  \multicolumn{4}{l}{\lst{end for}} \\
  \hline
  $ root $ & \lst{Expr} & [1, *] & 
    If $header[4] = 1$, the $root$ tree may contain ConstantPlaceholder
    nodes instead of Constant nodes (and may by only some of them, not all).
    Otherwise (i.e. if $header[4] = 0$) the root cannot contain placeholders (exception
    should be thrown). It is possible to have both constants and placeholders in the
    tree, but for every placeholder there should be a constant in $constants$ array of
    ErgoTree instance. \\
  \hline
\end{tabularx}\)
\end{figure}

Serialized instances of \lst{ErgoTree} class are self sufficient and can be stored and
passed around. \lst{ErgoTree} format defines top-level serialization format of
\langname scripts. The interpretation of the byte array depend on the first $header$
bytes, which uses VLQ encoding up to 30 bits. Currently we define meaning for only
first byte, which may be extended in future versions. The meaning of the bits is shown
in Figure~\ref{fig:ergotree:header}.

\begin{figure}[h] \footnotesize
\caption{\ASDag $header$ bits}\vspace{-7pt}
\label{fig:ergotree:header}
\(\begin{tabularx}{\textwidth}{| l | l | X |}
    \hline
    \bf{Bits} & \bf{Default} & \bf{Description} \\
    \hline
    Bits 0-2 & 0 & language version (current version == 0) \\
    \hline
    Bit 3 & 0 & $= 1$ if size of the whole tree is serialized after the header byte \\
    \hline
    Bit 4 & 0 & $= 1$ if constant segregation is used for this ErgoTree \\
    \hline
    Bit 5 & 0 & $= 1$ - reserved for context dependent costing (should be = 0) \\
    \hline
    Bit 6 & 0 & reserved for GZIP compression (should be 0) \\
    \hline
    Bit 7 & 0 & $= 1$ if the header contains more than 1 byte (should be 0) \\
    \hline
\end{tabularx}\)
\end{figure}

Currently we don't specify interpretation for the second and other bytes of
the header. We reserve the possibility to extend header by using Bit 7 == 1
and chain additional bytes as in VLQ. Once the new bytes are required, a new
version of the language should be created and implemented via
soft-forkability. That new language will give an interpretation for the new
bytes.

The default behavior of ErgoTreeSerializer is to preserve original structure
of \ASDag and check consistency. In case of any inconsistency the
serializer throws exception.

If constant segregation Bit4 is set to 1 then $constants$ collection contains
the constants for which there may be \lst{ConstantPlaceholder} nodes in the
tree. Nowever, if the constant segregation bit is 0, then $constants$
collection should be empty and any placeholder in the tree will lead to
exception.


\subsection{Constant Segregation}
\label{sec:ser:constant_segregation}



\section{The Graph}
\label{sec:graph}

\section{Costing}
\label{sec:costing}

can check the graphs generated and saved in file.
This is how the file name is specified
\begin{lstlisting}  

    val env: ScriptEnv = Map(
    ScriptNameProp -> s"filename_verify",
  ...
\end{lstlisting}  

The file should be in \lst{test-out} directory. The graph should have
explicit nodes like \lst{CostOf(...)}, which represent access to CostTable
entries. The actual cost is counted in the nodes like this \lst{s1340: Int =
OpCost(2, List(s1361, s1360), s983)}. Each such node is handled like
\lst{costAccumulator.add(s1340, OpCost(2, List(s1361, s1360), s983), dataEnv)}
See \lst{CostAccumulator}

How much cost is represented by OpCost node?
\begin{enumerate}
    \item Symbols s1361, s1360 are dependencies. They represent cost that
    should be accumulated before s983. 
    \item If upon handling of OpCost, the dependencies are not yet
    accumulated, then they are accumulated first, and then s983 is
    accumulated.
    \item the values of s1340 is the value of s983.
    \item Thus execution of OpCost, consists of 2 parts: a) data flow b) side
    effect on CostAccumulator
    \item OpCost is special node, interpreted in a special way. See method
    evaluate in Evaluation.
\end{enumerate}



\bibliographystyle{alpha}
\bibliography{spec.bib}

\appendix

\section{Predefined types}
\label{sec:appendix:predeftypes}

\begin{table}[h]
    \small
    \begin{tabu}{|l |l |l |l |l |l |l |l|}
     \hline
     \rowfont{\bfseries}
        Name   &   Code   &  IsConstSize & 
        isPrim\footnote{isPrim - primitive type} & 
        isEmbed  & isNum & Set of values \\
        \hline

\lst{Boolean}	&	$1$	&	\lst{true}	& \lst{true}	&	\lst{true} &	\lst{false}	& $\Set{\lst{true}, \lst{false}}$ \\
\hline
\lst{Byte}	&	$2$	&	\lst{true}	& \lst{true}	&	\lst{true} &	\lst{true}	& $\Set{-2^{7} \dots 2^{7}-1}$ \\
\hline
\lst{Short}	&	$3$	&	\lst{true}	& \lst{true}	&	\lst{true} &	\lst{true}	& $\Set{-2^{15} \dots 2^{15}-1}$ \\
\hline
\lst{Int}	&	$4$	&	\lst{true}	& \lst{true}	&	\lst{true} &	\lst{true}	& $\Set{-2^{31} \dots 2^{31}-1}$ \\
\hline
\lst{Long}	&	$5$	&	\lst{true}	& \lst{true}	&	\lst{true} &	\lst{true}	& $\Set{-2^{63} \dots 2^{63}-1}$ \\
\hline
\lst{BigInt}	&	$6$	&	\lst{true}	& \lst{true}	&	\lst{true} &	\lst{true}	& $\Set{-2^{255} \dots 2^{255}-1}$ \\
\hline
\lst{GroupElement}	&	$7$	&	\lst{true}	& \lst{true}	&	\lst{true} &	\lst{false}	& $\Set{p \in \lst{SecP256K1Point}}$ \\
\hline
\lst{SigmaProp}	&	$8$	&	\lst{false}	& \lst{true}	&	\lst{true} &	\lst{false}	& Sec.~\ref{sec:type:SigmaProp} \\
\hline
\lst{Box}	&	$99$	&	\lst{false}	& \lst{false}	&	\lst{false} &	\lst{false}	& Sec.~\ref{sec:type:Box} \\
\hline
\lst{AvlTree}	&	$100$	&	\lst{false}	& \lst{false}	&	\lst{false} &	\lst{false}	& Sec.~\ref{sec:type:AvlTree} \\
\hline
\lst{Context}	&	$101$	&	\lst{false}	& \lst{false}	&	\lst{false} &	\lst{false}	& Sec.~\ref{sec:type:Context} \\
\hline
\lst{Header}	&	$104$	&	\lst{true}	& \lst{false}	&	\lst{false} &	\lst{false}	& Sec.~\ref{sec:type:Header} \\
\hline
\lst{PreHeader}	&	$105$	&	\lst{true}	& \lst{false}	&	\lst{false} &	\lst{false}	& Sec.~\ref{sec:type:PreHeader} \\
\hline
\lst{Global}	&	$106$	&	\lst{true}	& \lst{false}	&	\lst{false} &	\lst{false}	& Sec.~\ref{sec:type:Global} \\

    \hline
    \end{tabu}
    \caption{Predefined types of \langname}
    \label{table:predeftypes}
\end{table}

\mnote{This table should be autogenerated from sigma SType descriptors}

\subsection{Boolean type}
\label{sec:type:Boolean}

\subsubsection{\lst{Boolean.toByte} method (Code 1.1)}
\noindent
\begin{tabularx}{\textwidth}{| l | X |}
   \hline
   \bf{Description} & Convert true to 1 and false to 0 \\
  
  \hline
  \bf{Parameters} &
      \(\begin{array}{l l l}
         
      \end{array}\) \\
       
  \hline
  \bf{Result} & \lst{Byte} \\
  \hline
  
  \bf{Serialized as} & \lst{PropertyCall(opCode=219)} \\
  \hline
       
\end{tabularx}


\subsection{Byte type}
\label{sec:type:Byte}

\noindent
\begin{tabularx}{\textwidth}{| c | c | X |}
  \hline
  \bf{Code} & \bf{Method Signature} & \bf{Description} \\
  \hline
  106.1 & \lst{def toByte()} &  \\
\hline
106.2 & \lst{def toShort()} &  \\
\hline
106.3 & \lst{def toInt()} &  \\
\hline
106.4 & \lst{def toLong()} &  \\
\hline
106.5 & \lst{def toBigInt()} &  \\
\hline
106.6 & \lst{def toBytes()} &  \\
\hline
106.7 & \lst{def toBits()} &  \\
  \hline
\end{tabularx}
     

\subsection{Short type}
\label{sec:type:Short}

\subsubsection{\lst{Short.toByte} method (Code 106.1)}
\noindent
\begin{tabularx}{\textwidth}{| l | X |}
   \hline
   \bf{Description} &  \\
  
  \hline
  \bf{Result} & \lst{Byte} \\
  \hline
\end{tabularx}



\subsubsection{\lst{Short.toShort} method (Code 106.2)}
\noindent
\begin{tabularx}{\textwidth}{| l | X |}
   \hline
   \bf{Description} &  \\
  
  \hline
  \bf{Result} & \lst{Short} \\
  \hline
\end{tabularx}



\subsubsection{\lst{Short.toInt} method (Code 106.3)}
\noindent
\begin{tabularx}{\textwidth}{| l | X |}
   \hline
   \bf{Description} &  \\
  
  \hline
  \bf{Result} & \lst{Int} \\
  \hline
\end{tabularx}



\subsubsection{\lst{Short.toLong} method (Code 106.4)}
\noindent
\begin{tabularx}{\textwidth}{| l | X |}
   \hline
   \bf{Description} &  \\
  
  \hline
  \bf{Result} & \lst{Long} \\
  \hline
\end{tabularx}



\subsubsection{\lst{Short.toBigInt} method (Code 106.5)}
\noindent
\begin{tabularx}{\textwidth}{| l | X |}
   \hline
   \bf{Description} &  \\
  
  \hline
  \bf{Result} & \lst{BigInt} \\
  \hline
\end{tabularx}



\subsubsection{\lst{Short.toBytes} method (Code 106.6)}
\noindent
\begin{tabularx}{\textwidth}{| l | X |}
   \hline
   \bf{Description} &  \\
  
  \hline
  \bf{Result} & \lst{Coll[Byte]} \\
  \hline
\end{tabularx}



\subsubsection{\lst{Short.toBits} method (Code 106.7)}
\noindent
\begin{tabularx}{\textwidth}{| l | X |}
   \hline
   \bf{Description} &  \\
  
  \hline
  \bf{Result} & \lst{Coll[Boolean]} \\
  \hline
\end{tabularx}


\subsection{Int type}
\label{sec:type:Int}

\subsubsection{\lst{Int.toByte} method (Code 106.1)}
\label{sec:type:Int:toByte}
\noindent
\begin{tabularx}{\textwidth}{| l | X |}
   \hline
   \bf{Description} & Converts this numeric value to \lst{Byte}, throwing exception if overflow. \\
  
  \hline
  \bf{Parameters} &
      \(\begin{array}{l l l}
         
      \end{array}\) \\
       
  \hline
  \bf{Result} & \lst{Byte} \\
  \hline
  
  \bf{Serialized as} & \hyperref[sec:serialization:operation:PropertyCall]{\lst{PropertyCall}} \\
  \hline
       
\end{tabularx}



\subsubsection{\lst{Int.toShort} method (Code 106.2)}
\label{sec:type:Int:toShort}
\noindent
\begin{tabularx}{\textwidth}{| l | X |}
   \hline
   \bf{Description} & Converts this numeric value to \lst{Short}, throwing exception if overflow. \\
  
  \hline
  \bf{Parameters} &
      \(\begin{array}{l l l}
         
      \end{array}\) \\
       
  \hline
  \bf{Result} & \lst{Short} \\
  \hline
  
  \bf{Serialized as} & \hyperref[sec:serialization:operation:PropertyCall]{\lst{PropertyCall}} \\
  \hline
       
\end{tabularx}



\subsubsection{\lst{Int.toInt} method (Code 106.3)}
\label{sec:type:Int:toInt}
\noindent
\begin{tabularx}{\textwidth}{| l | X |}
   \hline
   \bf{Description} & Converts this numeric value to \lst{Int}, throwing exception if overflow. \\
  
  \hline
  \bf{Parameters} &
      \(\begin{array}{l l l}
         
      \end{array}\) \\
       
  \hline
  \bf{Result} & \lst{Int} \\
  \hline
  
  \bf{Serialized as} & \hyperref[sec:serialization:operation:PropertyCall]{\lst{PropertyCall}} \\
  \hline
       
\end{tabularx}



\subsubsection{\lst{Int.toLong} method (Code 106.4)}
\label{sec:type:Int:toLong}
\noindent
\begin{tabularx}{\textwidth}{| l | X |}
   \hline
   \bf{Description} & Converts this numeric value to \lst{Long}, throwing exception if overflow. \\
  
  \hline
  \bf{Parameters} &
      \(\begin{array}{l l l}
         
      \end{array}\) \\
       
  \hline
  \bf{Result} & \lst{Long} \\
  \hline
  
  \bf{Serialized as} & \hyperref[sec:serialization:operation:PropertyCall]{\lst{PropertyCall}} \\
  \hline
       
\end{tabularx}



\subsubsection{\lst{Int.toBigInt} method (Code 106.5)}
\label{sec:type:Int:toBigInt}
\noindent
\begin{tabularx}{\textwidth}{| l | X |}
   \hline
   \bf{Description} & Converts this numeric value to \lst{BigInt} \\
  
  \hline
  \bf{Parameters} &
      \(\begin{array}{l l l}
         
      \end{array}\) \\
       
  \hline
  \bf{Result} & \lst{BigInt} \\
  \hline
  
  \bf{Serialized as} & \hyperref[sec:serialization:operation:PropertyCall]{\lst{PropertyCall}} \\
  \hline
       
\end{tabularx}



\subsubsection{\lst{Int.toBytes} method (Code 106.6)}
\label{sec:type:Int:toBytes}
\noindent
\begin{tabularx}{\textwidth}{| l | X |}
   \hline
   \bf{Description} &  Returns a big-endian representation of this numeric value in a collection of bytes.
 For example, the \lst{Int} value \lst{0x12131415} would yield the
 collection of bytes \lst{[0x12, 0x13, 0x14, 0x15]}.
           \\
  
  \hline
  \bf{Parameters} &
      \(\begin{array}{l l l}
         
      \end{array}\) \\
       
  \hline
  \bf{Result} & \lst{Coll[Byte]} \\
  \hline
  
  \bf{Serialized as} & \hyperref[sec:serialization:operation:PropertyCall]{\lst{PropertyCall}} \\
  \hline
       
\end{tabularx}



\subsubsection{\lst{Int.toBits} method (Code 106.7)}
\label{sec:type:Int:toBits}
\noindent
\begin{tabularx}{\textwidth}{| l | X |}
   \hline
   \bf{Description} &  Returns a big-endian representation of this numeric in a collection of Booleans.
  Each boolean corresponds to one bit.
           \\
  
  \hline
  \bf{Parameters} &
      \(\begin{array}{l l l}
         
      \end{array}\) \\
       
  \hline
  \bf{Result} & \lst{Coll[Boolean]} \\
  \hline
  
  \bf{Serialized as} & \hyperref[sec:serialization:operation:PropertyCall]{\lst{PropertyCall}} \\
  \hline
       
\end{tabularx}


\subsection{Long type}
\label{sec:type:Long}

\subsubsection{\lst{Long.toByte} method (Code 106.1)}
\noindent
\begin{tabularx}{\textwidth}{| l | X |}
   \hline
   \bf{Description} & Converts this numeric value to \lst{Byte}, throwing exception if overflow. \\
  
  \hline
  \bf{Result} & \lst{Byte} \\
  \hline
\end{tabularx}



\subsubsection{\lst{Long.toShort} method (Code 106.2)}
\noindent
\begin{tabularx}{\textwidth}{| l | X |}
   \hline
   \bf{Description} & Converts this numeric value to \lst{Short}, throwing exception if overflow. \\
  
  \hline
  \bf{Result} & \lst{Short} \\
  \hline
\end{tabularx}



\subsubsection{\lst{Long.toInt} method (Code 106.3)}
\noindent
\begin{tabularx}{\textwidth}{| l | X |}
   \hline
   \bf{Description} & Converts this numeric value to \lst{Int}, throwing exception if overflow. \\
  
  \hline
  \bf{Result} & \lst{Int} \\
  \hline
\end{tabularx}



\subsubsection{\lst{Long.toLong} method (Code 106.4)}
\noindent
\begin{tabularx}{\textwidth}{| l | X |}
   \hline
   \bf{Description} & Converts this numeric value to \lst{Long}, throwing exception if overflow. \\
  
  \hline
  \bf{Result} & \lst{Long} \\
  \hline
\end{tabularx}



\subsubsection{\lst{Long.toBigInt} method (Code 106.5)}
\noindent
\begin{tabularx}{\textwidth}{| l | X |}
   \hline
   \bf{Description} & Converts this numeric value to \lst{BigInt} \\
  
  \hline
  \bf{Result} & \lst{BigInt} \\
  \hline
\end{tabularx}



\subsubsection{\lst{Long.toBytes} method (Code 106.6)}
\noindent
\begin{tabularx}{\textwidth}{| l | X |}
   \hline
   \bf{Description} & Returns a big-endian representation of this numeric value in a collection of bytes.
 For example, the Int value \lst{0x12131415} would yield the
 byte array  \lst{[0x12, 0x13, 0x14, 0x15]}. \\
  
  \hline
  \bf{Result} & \lst{Coll[Byte]} \\
  \hline
\end{tabularx}



\subsubsection{\lst{Long.toBits} method (Code 106.7)}
\noindent
\begin{tabularx}{\textwidth}{| l | X |}
   \hline
   \bf{Description} & Returns a big-endian representation of this numeric in a collection of Booleans.
 Each boolean corresponds to one bit. \\
  
  \hline
  \bf{Result} & \lst{Coll[Boolean]} \\
  \hline
\end{tabularx}


\subsection{BigInt type}
\label{sec:type:BigInt}

\noindent
\begin{tabularx}{\textwidth}{| c | c | X |}
  \hline
  \bf{Code} & \bf{Method Signature} & \bf{Description} \\
  \hline
  106.1 & \lst{def toByte()} &  \\
\hline
6.1 & \lst{def modQ()} &  \\
\hline
106.2 & \lst{def toShort()} &  \\
\hline
6.2 & \lst{def plusModQ()} &  \\
\hline
106.3 & \lst{def toInt()} &  \\
\hline
6.3 & \lst{def minusModQ()} &  \\
\hline
106.4 & \lst{def toLong()} &  \\
\hline
6.4 & \lst{def multModQ()} &  \\
\hline
106.5 & \lst{def toBigInt()} &  \\
\hline
106.6 & \lst{def toBytes()} &  \\
\hline
106.7 & \lst{def toBits()} &  \\
  \hline
\end{tabularx}
     

\subsection{GroupElement type}
\label{sec:type:GroupElement}

\subsubsection{\lst{GroupElement.getEncoded} method (Code 7.2)}
\label{sec:type:GroupElement:getEncoded}
\noindent
\begin{tabularx}{\textwidth}{| l | X |}
   \hline
   \bf{Description} & Get an encoding of the point value. \\
  
  \hline
  \bf{Parameters} &
      \(\begin{array}{l l l}
         
      \end{array}\) \\
       
  \hline
  \bf{Result} & \lst{Coll[Byte]} \\
  \hline
  
  \bf{Serialized as} & \hyperref[sec:serialization:operation:PropertyCall]{\lst{PropertyCall}} \\
  \hline
       
\end{tabularx}



\subsubsection{\lst{GroupElement.exp} method (Code 7.3)}
\label{sec:type:GroupElement:exp}
\noindent
\begin{tabularx}{\textwidth}{| l | X |}
   \hline
   \bf{Description} & Exponentiate this \lst{GroupElement} to the given number. Returns this to the power of k \\
  
  \hline
  \bf{Parameters} &
      \(\begin{array}{l l l}
         \lst{k} & \lst{: BigInt} & \text{// The power} \\
      \end{array}\) \\
       
  \hline
  \bf{Result} & \lst{GroupElement} \\
  \hline
  
  \bf{Serialized as} & \hyperref[sec:serialization:operation:Exponentiate]{\lst{Exponentiate}} \\
  \hline
       
\end{tabularx}



\subsubsection{\lst{GroupElement.multiply} method (Code 7.4)}
\label{sec:type:GroupElement:multiply}
\noindent
\begin{tabularx}{\textwidth}{| l | X |}
   \hline
   \bf{Description} & Group operation. \\
  
  \hline
  \bf{Parameters} &
      \(\begin{array}{l l l}
         \lst{other} & \lst{: GroupElement} & \text{// other element of the group} \\
      \end{array}\) \\
       
  \hline
  \bf{Result} & \lst{GroupElement} \\
  \hline
  
  \bf{Serialized as} & \hyperref[sec:serialization:operation:MultiplyGroup]{\lst{MultiplyGroup}} \\
  \hline
       
\end{tabularx}



\subsubsection{\lst{GroupElement.negate} method (Code 7.5)}
\label{sec:type:GroupElement:negate}
\noindent
\begin{tabularx}{\textwidth}{| l | X |}
   \hline
   \bf{Description} & Inverse element of the group. \\
  
  \hline
  \bf{Parameters} &
      \(\begin{array}{l l l}
         
      \end{array}\) \\
       
  \hline
  \bf{Result} & \lst{GroupElement} \\
  \hline
  
  \bf{Serialized as} & \hyperref[sec:serialization:operation:PropertyCall]{\lst{PropertyCall}} \\
  \hline
       
\end{tabularx}



\subsection{SigmaProp type}
\label{sec:type:SigmaProp}

Values of \lst{SigmaProp} type hold sigma propositions, which can be proved
and verified using Sigma protocols. Each sigma proposition is represented as
an expression where sigma protocol primitives such as \lst{ProveDlog}, and
\lst{ProveDHTuple} are used as constants and special sigma protocol
connectives like \lst{&&},\lst{||} and \lst{THRESHOLD} are used as operations.

The abstract syntax of sigma propositions is shown in
Figure~\ref{fig:sigmaprop:tree}.

\begin{figure}[h]
   \centering
   \begin{tabular}{@{}l c l l l} 
      \hline
      Set 		&  			& Syntax	   & Mnemonic 	& Description \\
      \hline
      $Tree \ni t$	& := 	& \lst{Trivial(b)} 	& \lst{TrivialProp}	& boolean value \lst{b} as sigma proposition  \\
                     & $\mid$	& \lst{Dlog(ge)} 	& \lst{ProveDLog}	& knowledge of discrete logarithm of \lst{ge} \\
                     & $\mid$ & \lst{DHTuple(g,h,u,v)} 	& \lst{ProveDHTuple}	& knowledge of Diffie-Hellman tuple \\
                     & $\mid$ & \lst{THRESHOLD}$(k,t_1,\dots,t_n)$ 	& \lst{THRESHOLD}	& knowledge of $k$ out of $n$ secrets\\
                     & $\mid$ & \lst{OR}$(t_1,\dots,t_n)$ 	& \lst{OR}	& knowledge of any one of $n$ secrets\\
                     & $\mid$ & \lst{AND}$(t_1,\dots,t_n)$ 	& \lst{AND}	& knowledge of all $n$ secrets\\
      \end{tabular} 
   \caption{Abstract syntax of sigma propositions}
   \label{fig:sigmaprop:tree}
\end{figure}

Every well-formed tree of sigma proposition is a value of type
\lst{SigmaProp}, thus following the notation of Section~\ref{sec:evaluation}
we can define denotation of \lst{SigmaProp}
$$\Denot{\lst{SigmaProp}} = \Set{t \in Tree}$$


The following methods can be called on all instances of \lst{SigmaProp} type.

\noindent
\begin{tabularx}{\textwidth}{| l | X |}
  \hline
  \bf{Method Signature} & \bf{Description} \\
  \hline
  \lst{def propBytes: Coll[Byte]} & 
    Serialized bytes of this sigma proposition taken as ErgoTree and then serialized. \\
  \hline
\end{tabularx}


\subsubsection{\lst{SigmaProp.propBytes} method (Code 8.1)}
\noindent
\begin{tabularx}{\textwidth}{| l | X |}
   \hline
   \bf{Description} & Serialized bytes of this sigma proposition taken as ErgoTree. \\
  
  \hline
  \bf{Parameters} &
      \(\begin{array}{l l l}
         
      \end{array}\) \\
       
  \hline
  \bf{Result} & \lst{Coll[Byte]} \\
  \hline
  
  \bf{Serialized as} & \lst{SigmaPropBytes(opCode=208)} \\
  \hline
       
\end{tabularx}



\subsubsection{\lst{SigmaProp.isProven} method (Code 8.2)}
\noindent
\begin{tabularx}{\textwidth}{| l | X |}
   \hline
   \bf{Description} & Verify that sigma proposition is proven. (FRONTEND ONLY) \\
  
  \hline
  \bf{Parameters} &
      \(\begin{array}{l l l}
         
      \end{array}\) \\
       
  \hline
  \bf{Result} & \lst{Boolean} \\
  \hline
  
\end{tabularx}


For a list of primitive operations on \lst{SigmaProp} type see Appendix~\ref{sec:appendix:primops}.

\subsection{Box type}
\label{sec:type:Box}

\subsubsection{\lst{Box.value} method (Code 99.1)}
\label{sec:type:Box:value}
\noindent
\begin{tabularx}{\textwidth}{| l | X |}
   \hline
   \bf{Description} & Monetary value in NanoERGs stored in this box. \\
   \hline
   \bf{Signature} & \lst{def value}: \lst{Long} \\
  
  \hline
  
  \bf{Serialized as} & \hyperref[sec:serialization:operation:ExtractAmount]{\lst{ExtractAmount}} \\
  \hline
       
\end{tabularx}



\subsubsection{\lst{Box.propositionBytes} method (Code 99.2)}
\label{sec:type:Box:propositionBytes}
\noindent
\begin{tabularx}{\textwidth}{| l | X |}
   \hline
   \bf{Description} & Serialized bytes of the guarding script which should be evaluated to true in order to
 open this box (spend it in a transaction). \\
   \hline
   \bf{Signature} & \lst{def propositionBytes}: \lst{Coll[Byte]} \\
  
  \hline
  
  \bf{Serialized as} & \hyperref[sec:serialization:operation:ExtractScriptBytes]{\lst{ExtractScriptBytes}} \\
  \hline
       
\end{tabularx}



\subsubsection{\lst{Box.bytes} method (Code 99.3)}
\label{sec:type:Box:bytes}
\noindent
\begin{tabularx}{\textwidth}{| l | X |}
   \hline
   \bf{Description} & Serialized bytes of this box's content, including proposition bytes. \\
   \hline
   \bf{Signature} & \lst{def bytes}: \lst{Coll[Byte]} \\
  
  \hline
  
  \bf{Serialized as} & \hyperref[sec:serialization:operation:ExtractBytes]{\lst{ExtractBytes}} \\
  \hline
       
\end{tabularx}



\subsubsection{\lst{Box.bytesWithoutRef} method (Code 99.4)}
\label{sec:type:Box:bytesWithoutRef}
\noindent
\begin{tabularx}{\textwidth}{| l | X |}
   \hline
   \bf{Description} & Serialized bytes of this box's content, excluding transactionId and index of output. \\
   \hline
   \bf{Signature} & \lst{def bytesWithoutRef}: \lst{Coll[Byte]} \\
  
  \hline
  
  \bf{Serialized as} & \hyperref[sec:serialization:operation:ExtractBytesWithNoRef]{\lst{ExtractBytesWithNoRef}} \\
  \hline
       
\end{tabularx}



\subsubsection{\lst{Box.id} method (Code 99.5)}
\label{sec:type:Box:id}
\noindent
\begin{tabularx}{\textwidth}{| l | X |}
   \hline
   \bf{Description} & Blake2b256 hash of this box's content, basically equals to \lst{blake2b256(bytes)} \\
   \hline
   \bf{Signature} & \lst{def id}: \lst{Coll[Byte]} \\
  
  \hline
  
  \bf{Serialized as} & \hyperref[sec:serialization:operation:ExtractId]{\lst{ExtractId}} \\
  \hline
       
\end{tabularx}



\subsubsection{\lst{Box.creationInfo} method (Code 99.6)}
\label{sec:type:Box:creationInfo}
\noindent
\begin{tabularx}{\textwidth}{| l | X |}
   \hline
   \bf{Description} &  If \lst{tx} is a transaction which generated this box, then \lst{creationInfo._1}
 is a height of the tx's block. The \lst{creationInfo._2} is a serialized transaction
 identifier followed by box index in the transaction outputs.
         \\
   \hline
   \bf{Signature} & \lst{def creationInfo}: \lst{(Int,Coll[Byte])} \\
  
  \hline
  
  \bf{Serialized as} & \hyperref[sec:serialization:operation:ExtractCreationInfo]{\lst{ExtractCreationInfo}} \\
  \hline
       
\end{tabularx}



\subsubsection{\lst{Box.tokens} method (Code 99.8)}
\label{sec:type:Box:tokens}
\noindent
\begin{tabularx}{\textwidth}{| l | X |}
   \hline
   \bf{Description} & Secondary tokens \\
   \hline
   \bf{Signature} & \lst{def tokens}: \lst{Coll[(Coll[Byte],Long)]} \\
  
  \hline
  
  \bf{Serialized as} & \hyperref[sec:serialization:operation:PropertyCall]{\lst{PropertyCall}} \\
  \hline
       
\end{tabularx}



\subsubsection{\lst{Box.R0} method (Code 99.9)}
\label{sec:type:Box:R0}
\noindent
\begin{tabularx}{\textwidth}{| l | X |}
   \hline
   \bf{Description} & Monetary value, in Ergo tokens \\
   \hline
   \bf{Signature} & \lst{def R0}$[$\lst{T}$]$: \lst{Option[T]} \\
  
  \hline
  
  \bf{Serialized as} & \hyperref[sec:serialization:operation:ExtractRegisterAs]{\lst{ExtractRegisterAs}} \\
  \hline
       
\end{tabularx}



\subsubsection{\lst{Box.R1} method (Code 99.10)}
\label{sec:type:Box:R1}
\noindent
\begin{tabularx}{\textwidth}{| l | X |}
   \hline
   \bf{Description} & Guarding script \\
   \hline
   \bf{Signature} & \lst{def R1}$[$\lst{T}$]$: \lst{Option[T]} \\
  
  \hline
  
  \bf{Serialized as} & \hyperref[sec:serialization:operation:ExtractRegisterAs]{\lst{ExtractRegisterAs}} \\
  \hline
       
\end{tabularx}



\subsubsection{\lst{Box.R2} method (Code 99.11)}
\label{sec:type:Box:R2}
\noindent
\begin{tabularx}{\textwidth}{| l | X |}
   \hline
   \bf{Description} & Secondary tokens \\
   \hline
   \bf{Signature} & \lst{def R2}$[$\lst{T}$]$: \lst{Option[T]} \\
  
  \hline
  
  \bf{Serialized as} & \hyperref[sec:serialization:operation:ExtractRegisterAs]{\lst{ExtractRegisterAs}} \\
  \hline
       
\end{tabularx}



\subsubsection{\lst{Box.R3} method (Code 99.12)}
\label{sec:type:Box:R3}
\noindent
\begin{tabularx}{\textwidth}{| l | X |}
   \hline
   \bf{Description} & Reference to transaction and output id where the box was created \\
   \hline
   \bf{Signature} & \lst{def R3}$[$\lst{T}$]$: \lst{Option[T]} \\
  
  \hline
  
  \bf{Serialized as} & \hyperref[sec:serialization:operation:ExtractRegisterAs]{\lst{ExtractRegisterAs}} \\
  \hline
       
\end{tabularx}



\subsubsection{\lst{Box.R4} method (Code 99.13)}
\label{sec:type:Box:R4}
\noindent
\begin{tabularx}{\textwidth}{| l | X |}
   \hline
   \bf{Description} & Non-mandatory register \\
   \hline
   \bf{Signature} & \lst{def R4}$[$\lst{T}$]$: \lst{Option[T]} \\
  
  \hline
  
  \bf{Serialized as} & \hyperref[sec:serialization:operation:ExtractRegisterAs]{\lst{ExtractRegisterAs}} \\
  \hline
       
\end{tabularx}



\subsubsection{\lst{Box.R5} method (Code 99.14)}
\label{sec:type:Box:R5}
\noindent
\begin{tabularx}{\textwidth}{| l | X |}
   \hline
   \bf{Description} & Non-mandatory register \\
   \hline
   \bf{Signature} & \lst{def R5}$[$\lst{T}$]$: \lst{Option[T]} \\
  
  \hline
  
  \bf{Serialized as} & \hyperref[sec:serialization:operation:ExtractRegisterAs]{\lst{ExtractRegisterAs}} \\
  \hline
       
\end{tabularx}



\subsubsection{\lst{Box.R6} method (Code 99.15)}
\label{sec:type:Box:R6}
\noindent
\begin{tabularx}{\textwidth}{| l | X |}
   \hline
   \bf{Description} & Non-mandatory register \\
   \hline
   \bf{Signature} & \lst{def R6}$[$\lst{T}$]$: \lst{Option[T]} \\
  
  \hline
  
  \bf{Serialized as} & \hyperref[sec:serialization:operation:ExtractRegisterAs]{\lst{ExtractRegisterAs}} \\
  \hline
       
\end{tabularx}



\subsubsection{\lst{Box.R7} method (Code 99.16)}
\label{sec:type:Box:R7}
\noindent
\begin{tabularx}{\textwidth}{| l | X |}
   \hline
   \bf{Description} & Non-mandatory register \\
   \hline
   \bf{Signature} & \lst{def R7}$[$\lst{T}$]$: \lst{Option[T]} \\
  
  \hline
  
  \bf{Serialized as} & \hyperref[sec:serialization:operation:ExtractRegisterAs]{\lst{ExtractRegisterAs}} \\
  \hline
       
\end{tabularx}



\subsubsection{\lst{Box.R8} method (Code 99.17)}
\label{sec:type:Box:R8}
\noindent
\begin{tabularx}{\textwidth}{| l | X |}
   \hline
   \bf{Description} & Non-mandatory register \\
   \hline
   \bf{Signature} & \lst{def R8}$[$\lst{T}$]$: \lst{Option[T]} \\
  
  \hline
  
  \bf{Serialized as} & \hyperref[sec:serialization:operation:ExtractRegisterAs]{\lst{ExtractRegisterAs}} \\
  \hline
       
\end{tabularx}



\subsubsection{\lst{Box.R9} method (Code 99.18)}
\label{sec:type:Box:R9}
\noindent
\begin{tabularx}{\textwidth}{| l | X |}
   \hline
   \bf{Description} & Non-mandatory register \\
   \hline
   \bf{Signature} & \lst{def R9}$[$\lst{T}$]$: \lst{Option[T]} \\
  
  \hline
  
  \bf{Serialized as} & \hyperref[sec:serialization:operation:ExtractRegisterAs]{\lst{ExtractRegisterAs}} \\
  \hline
       
\end{tabularx}


\subsection{AvlTree type}
\label{sec:type:AvlTree}

\subsubsection{\lst{AvlTree.digest} method (Code 100.1)}
\noindent
\begin{tabularx}{\textwidth}{| l | X |}
   \hline
   \bf{Description} &  \\
  
  \hline
  \bf{Result} & \lst{Coll[Byte]} \\
  \hline
  
  \bf{Serialized as} & \lst{PropertyCall(opCode=219)} \\
  \hline
       
\end{tabularx}



\subsubsection{\lst{AvlTree.enabledOperations} method (Code 100.2)}
\noindent
\begin{tabularx}{\textwidth}{| l | X |}
   \hline
   \bf{Description} &  \\
  
  \hline
  \bf{Result} & \lst{Byte} \\
  \hline
  
  \bf{Serialized as} & \lst{PropertyCall(opCode=219)} \\
  \hline
       
\end{tabularx}



\subsubsection{\lst{AvlTree.keyLength} method (Code 100.3)}
\noindent
\begin{tabularx}{\textwidth}{| l | X |}
   \hline
   \bf{Description} &  \\
  
  \hline
  \bf{Result} & \lst{Int} \\
  \hline
  
  \bf{Serialized as} & \lst{PropertyCall(opCode=219)} \\
  \hline
       
\end{tabularx}



\subsubsection{\lst{AvlTree.valueLengthOpt} method (Code 100.4)}
\noindent
\begin{tabularx}{\textwidth}{| l | X |}
   \hline
   \bf{Description} &  \\
  
  \hline
  \bf{Result} & \lst{Option[Int]} \\
  \hline
  
  \bf{Serialized as} & \lst{PropertyCall(opCode=219)} \\
  \hline
       
\end{tabularx}



\subsubsection{\lst{AvlTree.isInsertAllowed} method (Code 100.5)}
\noindent
\begin{tabularx}{\textwidth}{| l | X |}
   \hline
   \bf{Description} &  \\
  
  \hline
  \bf{Result} & \lst{Boolean} \\
  \hline
  
  \bf{Serialized as} & \lst{PropertyCall(opCode=219)} \\
  \hline
       
\end{tabularx}



\subsubsection{\lst{AvlTree.isUpdateAllowed} method (Code 100.6)}
\noindent
\begin{tabularx}{\textwidth}{| l | X |}
   \hline
   \bf{Description} &  \\
  
  \hline
  \bf{Result} & \lst{Boolean} \\
  \hline
  
  \bf{Serialized as} & \lst{PropertyCall(opCode=219)} \\
  \hline
       
\end{tabularx}



\subsubsection{\lst{AvlTree.isRemoveAllowed} method (Code 100.7)}
\noindent
\begin{tabularx}{\textwidth}{| l | X |}
   \hline
   \bf{Description} &  \\
  
  \hline
  \bf{Result} & \lst{Boolean} \\
  \hline
  
  \bf{Serialized as} & \lst{PropertyCall(opCode=219)} \\
  \hline
       
\end{tabularx}



\subsubsection{\lst{AvlTree.updateOperations} method (Code 100.8)}
\noindent
\begin{tabularx}{\textwidth}{| l | X |}
   \hline
   \bf{Description} &  \\
  
  \hline
  \bf{Parameters} &
      \(\begin{array}{l l l}
         \lst{arg0} & \lst{: Byte} & \text{// } \\
      \end{array}\) \\
       
  \hline
  \bf{Result} & \lst{AvlTree} \\
  \hline
  
  \bf{Serialized as} & \lst{MethodCall(opCode=220)} \\
  \hline
       
\end{tabularx}



\subsubsection{\lst{AvlTree.contains} method (Code 100.9)}
\noindent
\begin{tabularx}{\textwidth}{| l | X |}
   \hline
   \bf{Description} &  \\
  
  \hline
  \bf{Parameters} &
      \(\begin{array}{l l l}
         \lst{arg0} & \lst{: Coll[Byte]} & \text{// } \\
\lst{arg1} & \lst{: Coll[Byte]} & \text{// } \\
      \end{array}\) \\
       
  \hline
  \bf{Result} & \lst{Boolean} \\
  \hline
  
  \bf{Serialized as} & \lst{MethodCall(opCode=220)} \\
  \hline
       
\end{tabularx}



\subsubsection{\lst{AvlTree.get} method (Code 100.10)}
\noindent
\begin{tabularx}{\textwidth}{| l | X |}
   \hline
   \bf{Description} &  \\
  
  \hline
  \bf{Parameters} &
      \(\begin{array}{l l l}
         \lst{arg0} & \lst{: Coll[Byte]} & \text{// } \\
\lst{arg1} & \lst{: Coll[Byte]} & \text{// } \\
      \end{array}\) \\
       
  \hline
  \bf{Result} & \lst{Option[Coll[Byte]]} \\
  \hline
  
  \bf{Serialized as} & \lst{MethodCall(opCode=220)} \\
  \hline
       
\end{tabularx}



\subsubsection{\lst{AvlTree.getMany} method (Code 100.11)}
\noindent
\begin{tabularx}{\textwidth}{| l | X |}
   \hline
   \bf{Description} &  \\
  
  \hline
  \bf{Parameters} &
      \(\begin{array}{l l l}
         \lst{arg0} & \lst{: Coll[Coll[Byte]]} & \text{// } \\
\lst{arg1} & \lst{: Coll[Byte]} & \text{// } \\
      \end{array}\) \\
       
  \hline
  \bf{Result} & \lst{Coll[Option[Coll[Byte]]]} \\
  \hline
  
  \bf{Serialized as} & \lst{MethodCall(opCode=220)} \\
  \hline
       
\end{tabularx}



\subsubsection{\lst{AvlTree.insert} method (Code 100.12)}
\noindent
\begin{tabularx}{\textwidth}{| l | X |}
   \hline
   \bf{Description} &  \\
  
  \hline
  \bf{Parameters} &
      \(\begin{array}{l l l}
         \lst{arg0} & \lst{: Coll[(Coll[Byte],Coll[Byte])]} & \text{// } \\
\lst{arg1} & \lst{: Coll[Byte]} & \text{// } \\
      \end{array}\) \\
       
  \hline
  \bf{Result} & \lst{Option[AvlTree]} \\
  \hline
  
  \bf{Serialized as} & \lst{MethodCall(opCode=220)} \\
  \hline
       
\end{tabularx}



\subsubsection{\lst{AvlTree.update} method (Code 100.13)}
\noindent
\begin{tabularx}{\textwidth}{| l | X |}
   \hline
   \bf{Description} &  \\
  
  \hline
  \bf{Parameters} &
      \(\begin{array}{l l l}
         \lst{arg0} & \lst{: Coll[(Coll[Byte],Coll[Byte])]} & \text{// } \\
\lst{arg1} & \lst{: Coll[Byte]} & \text{// } \\
      \end{array}\) \\
       
  \hline
  \bf{Result} & \lst{Option[AvlTree]} \\
  \hline
  
  \bf{Serialized as} & \lst{MethodCall(opCode=220)} \\
  \hline
       
\end{tabularx}



\subsubsection{\lst{AvlTree.remove} method (Code 100.14)}
\noindent
\begin{tabularx}{\textwidth}{| l | X |}
   \hline
   \bf{Description} &  \\
  
  \hline
  \bf{Parameters} &
      \(\begin{array}{l l l}
         \lst{arg0} & \lst{: Coll[Coll[Byte]]} & \text{// } \\
\lst{arg1} & \lst{: Coll[Byte]} & \text{// } \\
      \end{array}\) \\
       
  \hline
  \bf{Result} & \lst{Option[AvlTree]} \\
  \hline
  
  \bf{Serialized as} & \lst{MethodCall(opCode=220)} \\
  \hline
       
\end{tabularx}



\subsubsection{\lst{AvlTree.updateDigest} method (Code 100.15)}
\noindent
\begin{tabularx}{\textwidth}{| l | X |}
   \hline
   \bf{Description} &  \\
  
  \hline
  \bf{Parameters} &
      \(\begin{array}{l l l}
         \lst{arg0} & \lst{: Coll[Byte]} & \text{// } \\
      \end{array}\) \\
       
  \hline
  \bf{Result} & \lst{AvlTree} \\
  \hline
  
  \bf{Serialized as} & \lst{MethodCall(opCode=220)} \\
  \hline
       
\end{tabularx}


\subsection{Header type}
\label{sec:type:Header}

\subsubsection{\lst{Header.id} method (Code 104.1)}
\noindent
\begin{tabularx}{\textwidth}{| l | X |}
   \hline
   \bf{Description} &  \\
  
  \hline
  \bf{Result} & \lst{Coll[Byte]} \\
  \hline
  
  \bf{Serialized as} & \lst{PropertyCall(opCode=219)} \\
  \hline
       
\end{tabularx}



\subsubsection{\lst{Header.version} method (Code 104.2)}
\noindent
\begin{tabularx}{\textwidth}{| l | X |}
   \hline
   \bf{Description} &  \\
  
  \hline
  \bf{Result} & \lst{Byte} \\
  \hline
  
  \bf{Serialized as} & \lst{PropertyCall(opCode=219)} \\
  \hline
       
\end{tabularx}



\subsubsection{\lst{Header.parentId} method (Code 104.3)}
\noindent
\begin{tabularx}{\textwidth}{| l | X |}
   \hline
   \bf{Description} &  \\
  
  \hline
  \bf{Result} & \lst{Coll[Byte]} \\
  \hline
  
  \bf{Serialized as} & \lst{PropertyCall(opCode=219)} \\
  \hline
       
\end{tabularx}



\subsubsection{\lst{Header.ADProofsRoot} method (Code 104.4)}
\noindent
\begin{tabularx}{\textwidth}{| l | X |}
   \hline
   \bf{Description} &  \\
  
  \hline
  \bf{Result} & \lst{Coll[Byte]} \\
  \hline
  
  \bf{Serialized as} & \lst{PropertyCall(opCode=219)} \\
  \hline
       
\end{tabularx}



\subsubsection{\lst{Header.stateRoot} method (Code 104.5)}
\noindent
\begin{tabularx}{\textwidth}{| l | X |}
   \hline
   \bf{Description} &  \\
  
  \hline
  \bf{Result} & \lst{AvlTree} \\
  \hline
  
  \bf{Serialized as} & \lst{PropertyCall(opCode=219)} \\
  \hline
       
\end{tabularx}



\subsubsection{\lst{Header.transactionsRoot} method (Code 104.6)}
\noindent
\begin{tabularx}{\textwidth}{| l | X |}
   \hline
   \bf{Description} &  \\
  
  \hline
  \bf{Result} & \lst{Coll[Byte]} \\
  \hline
  
  \bf{Serialized as} & \lst{PropertyCall(opCode=219)} \\
  \hline
       
\end{tabularx}



\subsubsection{\lst{Header.timestamp} method (Code 104.7)}
\noindent
\begin{tabularx}{\textwidth}{| l | X |}
   \hline
   \bf{Description} &  \\
  
  \hline
  \bf{Result} & \lst{Long} \\
  \hline
  
  \bf{Serialized as} & \lst{PropertyCall(opCode=219)} \\
  \hline
       
\end{tabularx}



\subsubsection{\lst{Header.nBits} method (Code 104.8)}
\noindent
\begin{tabularx}{\textwidth}{| l | X |}
   \hline
   \bf{Description} &  \\
  
  \hline
  \bf{Result} & \lst{Long} \\
  \hline
  
  \bf{Serialized as} & \lst{PropertyCall(opCode=219)} \\
  \hline
       
\end{tabularx}



\subsubsection{\lst{Header.height} method (Code 104.9)}
\noindent
\begin{tabularx}{\textwidth}{| l | X |}
   \hline
   \bf{Description} &  \\
  
  \hline
  \bf{Result} & \lst{Int} \\
  \hline
  
  \bf{Serialized as} & \lst{PropertyCall(opCode=219)} \\
  \hline
       
\end{tabularx}



\subsubsection{\lst{Header.extensionRoot} method (Code 104.10)}
\noindent
\begin{tabularx}{\textwidth}{| l | X |}
   \hline
   \bf{Description} &  \\
  
  \hline
  \bf{Result} & \lst{Coll[Byte]} \\
  \hline
  
  \bf{Serialized as} & \lst{PropertyCall(opCode=219)} \\
  \hline
       
\end{tabularx}



\subsubsection{\lst{Header.minerPk} method (Code 104.11)}
\noindent
\begin{tabularx}{\textwidth}{| l | X |}
   \hline
   \bf{Description} &  \\
  
  \hline
  \bf{Result} & \lst{GroupElement} \\
  \hline
  
  \bf{Serialized as} & \lst{PropertyCall(opCode=219)} \\
  \hline
       
\end{tabularx}



\subsubsection{\lst{Header.powOnetimePk} method (Code 104.12)}
\noindent
\begin{tabularx}{\textwidth}{| l | X |}
   \hline
   \bf{Description} &  \\
  
  \hline
  \bf{Result} & \lst{GroupElement} \\
  \hline
  
  \bf{Serialized as} & \lst{PropertyCall(opCode=219)} \\
  \hline
       
\end{tabularx}



\subsubsection{\lst{Header.powNonce} method (Code 104.13)}
\noindent
\begin{tabularx}{\textwidth}{| l | X |}
   \hline
   \bf{Description} &  \\
  
  \hline
  \bf{Result} & \lst{Coll[Byte]} \\
  \hline
  
  \bf{Serialized as} & \lst{PropertyCall(opCode=219)} \\
  \hline
       
\end{tabularx}



\subsubsection{\lst{Header.powDistance} method (Code 104.14)}
\noindent
\begin{tabularx}{\textwidth}{| l | X |}
   \hline
   \bf{Description} &  \\
  
  \hline
  \bf{Result} & \lst{BigInt} \\
  \hline
  
  \bf{Serialized as} & \lst{PropertyCall(opCode=219)} \\
  \hline
       
\end{tabularx}



\subsubsection{\lst{Header.votes} method (Code 104.15)}
\noindent
\begin{tabularx}{\textwidth}{| l | X |}
   \hline
   \bf{Description} &  \\
  
  \hline
  \bf{Result} & \lst{Coll[Byte]} \\
  \hline
  
  \bf{Serialized as} & \lst{PropertyCall(opCode=219)} \\
  \hline
       
\end{tabularx}


\subsection{PreHeader type}
\label{sec:type:PreHeader}

\subsubsection{\lst{PreHeader.version} method (Code 105.1)}
\noindent
\begin{tabularx}{\textwidth}{| l | X |}
   \hline
   \bf{Description} &  \\
  
  \hline
  \bf{Result} & \lst{Byte} \\
  \hline
  
  \bf{Serialized as} & \lst{PropertyCall(opCode=219)} \\
  \hline
       
\end{tabularx}



\subsubsection{\lst{PreHeader.parentId} method (Code 105.2)}
\noindent
\begin{tabularx}{\textwidth}{| l | X |}
   \hline
   \bf{Description} &  \\
  
  \hline
  \bf{Result} & \lst{Coll[Byte]} \\
  \hline
  
  \bf{Serialized as} & \lst{PropertyCall(opCode=219)} \\
  \hline
       
\end{tabularx}



\subsubsection{\lst{PreHeader.timestamp} method (Code 105.3)}
\noindent
\begin{tabularx}{\textwidth}{| l | X |}
   \hline
   \bf{Description} &  \\
  
  \hline
  \bf{Result} & \lst{Long} \\
  \hline
  
  \bf{Serialized as} & \lst{PropertyCall(opCode=219)} \\
  \hline
       
\end{tabularx}



\subsubsection{\lst{PreHeader.nBits} method (Code 105.4)}
\noindent
\begin{tabularx}{\textwidth}{| l | X |}
   \hline
   \bf{Description} &  \\
  
  \hline
  \bf{Result} & \lst{Long} \\
  \hline
  
  \bf{Serialized as} & \lst{PropertyCall(opCode=219)} \\
  \hline
       
\end{tabularx}



\subsubsection{\lst{PreHeader.height} method (Code 105.5)}
\noindent
\begin{tabularx}{\textwidth}{| l | X |}
   \hline
   \bf{Description} &  \\
  
  \hline
  \bf{Result} & \lst{Int} \\
  \hline
  
  \bf{Serialized as} & \lst{PropertyCall(opCode=219)} \\
  \hline
       
\end{tabularx}



\subsubsection{\lst{PreHeader.minerPk} method (Code 105.6)}
\noindent
\begin{tabularx}{\textwidth}{| l | X |}
   \hline
   \bf{Description} &  \\
  
  \hline
  \bf{Result} & \lst{GroupElement} \\
  \hline
  
  \bf{Serialized as} & \lst{PropertyCall(opCode=219)} \\
  \hline
       
\end{tabularx}



\subsubsection{\lst{PreHeader.votes} method (Code 105.7)}
\noindent
\begin{tabularx}{\textwidth}{| l | X |}
   \hline
   \bf{Description} &  \\
  
  \hline
  \bf{Result} & \lst{Coll[Byte]} \\
  \hline
  
  \bf{Serialized as} & \lst{PropertyCall(opCode=219)} \\
  \hline
       
\end{tabularx}


\subsection{Context type}
\label{sec:type:Context}

\noindent
\begin{tabularx}{\textwidth}{| c | c | X |}
  \hline
  \bf{Code} & \bf{Method Signature} & \bf{Description} \\
  \hline
  101.1 & \lst{def dataInputs()} &  \\
\hline
101.2 & \lst{def headers()} &  \\
\hline
101.3 & \lst{def preHeader()} &  \\
\hline
101.4 & \lst{def INPUTS()} &  \\
\hline
101.5 & \lst{def OUTPUTS()} &  \\
\hline
101.6 & \lst{def HEIGHT()} &  \\
\hline
101.7 & \lst{def SELF()} &  \\
\hline
101.8 & \lst{def selfBoxIndex()} &  \\
\hline
101.9 & \lst{def LastBlockUtxoRootHash()} &  \\
\hline
101.10 & \lst{def minerPubKey()} &  \\
\hline
101.11 & \lst{def getVar()} &  \\
  \hline
\end{tabularx}
     

\subsection{Global type}
\label{sec:type:Global}

\subsubsection{\lst{SigmaDslBuilder.groupGenerator} method (Code 106.1)}
\label{sec:type:SigmaDslBuilder:groupGenerator}
\noindent
\begin{tabularx}{\textwidth}{| l | X |}
   \hline
   \bf{Description} &  \\
  
  \hline
  \bf{Parameters} &
      \(\begin{array}{l l l}
         
      \end{array}\) \\
       
  \hline
  \bf{Result} & \lst{GroupElement} \\
  \hline
  
  \bf{Serialized as} & \hyperref[sec:serialization:operation:GroupGenerator]{\lst{GroupGenerator}} \\
  \hline
       
\end{tabularx}



\subsubsection{\lst{SigmaDslBuilder.xor} method (Code 106.2)}
\label{sec:type:SigmaDslBuilder:xor}
\noindent
\begin{tabularx}{\textwidth}{| l | X |}
   \hline
   \bf{Description} & Byte-wise XOR of two collections of bytes \\
  
  \hline
  \bf{Parameters} &
      \(\begin{array}{l l l}
         \lst{left} & \lst{: Coll[Byte]} & \text{// left operand} \\
\lst{right} & \lst{: Coll[Byte]} & \text{// right operand} \\
      \end{array}\) \\
       
  \hline
  \bf{Result} & \lst{Coll[Byte]} \\
  \hline
  
  \bf{Serialized as} & \hyperref[sec:serialization:operation:Xor]{\lst{Xor}} \\
  \hline
       
\end{tabularx}


\subsection{Coll type}
\label{sec:type:Coll}

\subsubsection{\lst{SCollection.size} method (Code 12.1)}
\label{sec:type:SCollection:size}
\noindent
\begin{tabularx}{\textwidth}{| l | X |}
   \hline
   \bf{Description} & The size of the collection in elements. \\
  
  \hline
  \bf{Parameters} &
      \(\begin{array}{l l l}
         
      \end{array}\) \\
       
  \hline
  \bf{Result} & \lst{Int} \\
  \hline
  
  \bf{Serialized as} & \hyperref[sec:serialization:operation:SizeOf]{\lst{SizeOf}} \\
  \hline
       
\end{tabularx}



\subsubsection{\lst{SCollection.getOrElse} method (Code 12.2)}
\label{sec:type:SCollection:getOrElse}
\noindent
\begin{tabularx}{\textwidth}{| l | X |}
   \hline
   \bf{Description} & Return the element of collection if \lst{index} is in range \lst{0 .. size-1} \\
  
  \hline
  \bf{Parameters} &
      \(\begin{array}{l l l}
         \lst{index} & \lst{: Int} & \text{// index of the element of this collection} \\
\lst{default} & \lst{: IV} & \text{// value to return when \lst{index} is out of range} \\
      \end{array}\) \\
       
  \hline
  \bf{Result} & \lst{IV} \\
  \hline
  
  \bf{Serialized as} & \hyperref[sec:serialization:operation:ByIndex]{\lst{ByIndex}} \\
  \hline
       
\end{tabularx}



\subsubsection{\lst{SCollection.map} method (Code 12.3)}
\label{sec:type:SCollection:map}
\noindent
\begin{tabularx}{\textwidth}{| l | X |}
   \hline
   \bf{Description} &  Builds a new collection by applying a function to all elements of this collection.
 Returns a new collection of type \lst{Coll[B]} resulting from applying the given function
 \lst{f} to each element of this collection and collecting the results.
         \\
  
  \hline
  \bf{Parameters} &
      \(\begin{array}{l l l}
         \lst{f} & \lst{: (IV) => OV} & \text{// the function to apply to each element} \\
      \end{array}\) \\
       
  \hline
  \bf{Result} & \lst{Coll[OV]} \\
  \hline
  
  \bf{Serialized as} & \hyperref[sec:serialization:operation:MapCollection]{\lst{MapCollection}} \\
  \hline
       
\end{tabularx}



\subsubsection{\lst{SCollection.exists} method (Code 12.4)}
\label{sec:type:SCollection:exists}
\noindent
\begin{tabularx}{\textwidth}{| l | X |}
   \hline
   \bf{Description} & Tests whether a predicate holds for at least one element of this collection.
Returns \lst{true} if the given predicate \lst{p} is satisfied by at least one element of this collection, otherwise \lst{false}
         \\
  
  \hline
  \bf{Parameters} &
      \(\begin{array}{l l l}
         \lst{p} & \lst{: (IV) => Boolean} & \text{// the predicate used to test elements} \\
      \end{array}\) \\
       
  \hline
  \bf{Result} & \lst{Boolean} \\
  \hline
  
  \bf{Serialized as} & \hyperref[sec:serialization:operation:Exists]{\lst{Exists}} \\
  \hline
       
\end{tabularx}



\subsubsection{\lst{SCollection.fold} method (Code 12.5)}
\label{sec:type:SCollection:fold}
\noindent
\begin{tabularx}{\textwidth}{| l | X |}
   \hline
   \bf{Description} & Applies a binary operator to a start value and all elements of this collection, going left to right. \\
  
  \hline
  \bf{Parameters} &
      \(\begin{array}{l l l}
         \lst{zero} & \lst{: OV} & \text{// a starting value} \\
\lst{op} & \lst{: (OV,IV) => OV} & \text{// the binary operator} \\
      \end{array}\) \\
       
  \hline
  \bf{Result} & \lst{OV} \\
  \hline
  
  \bf{Serialized as} & \hyperref[sec:serialization:operation:Fold]{\lst{Fold}} \\
  \hline
       
\end{tabularx}



\subsubsection{\lst{SCollection.forall} method (Code 12.6)}
\label{sec:type:SCollection:forall}
\noindent
\begin{tabularx}{\textwidth}{| l | X |}
   \hline
   \bf{Description} & Tests whether a predicate holds for all elements of this collection.
Returns \lst{true} if this collection is empty or the given predicate \lst{p}
holds for all elements of this collection, otherwise \lst{false}.
         \\
  
  \hline
  \bf{Parameters} &
      \(\begin{array}{l l l}
         \lst{p} & \lst{: (IV) => Boolean} & \text{// the predicate used to test elements} \\
      \end{array}\) \\
       
  \hline
  \bf{Result} & \lst{Boolean} \\
  \hline
  
  \bf{Serialized as} & \hyperref[sec:serialization:operation:ForAll]{\lst{ForAll}} \\
  \hline
       
\end{tabularx}



\subsubsection{\lst{SCollection.slice} method (Code 12.7)}
\label{sec:type:SCollection:slice}
\noindent
\begin{tabularx}{\textwidth}{| l | X |}
   \hline
   \bf{Description} & Selects an interval of elements.  The returned collection is made up
  of all elements \lst{x} which satisfy the invariant:
  \lst{
     from <= indexOf(x) < until
  }
         \\
  
  \hline
  \bf{Parameters} &
      \(\begin{array}{l l l}
         \lst{from} & \lst{: Int} & \text{// the lowest index to include from this collection} \\
\lst{until} & \lst{: Int} & \text{// the lowest index to EXCLUDE from this collection} \\
      \end{array}\) \\
       
  \hline
  \bf{Result} & \lst{Coll[IV]} \\
  \hline
  
  \bf{Serialized as} & \hyperref[sec:serialization:operation:Slice]{\lst{Slice}} \\
  \hline
       
\end{tabularx}



\subsubsection{\lst{SCollection.filter} method (Code 12.8)}
\label{sec:type:SCollection:filter}
\noindent
\begin{tabularx}{\textwidth}{| l | X |}
   \hline
   \bf{Description} & Selects all elements of this collection which satisfy a predicate.
 Returns  a new collection consisting of all elements of this collection that satisfy the given
 predicate \lst{p}. The order of the elements is preserved.
         \\
  
  \hline
  \bf{Parameters} &
      \(\begin{array}{l l l}
         \lst{p} & \lst{: (IV) => Boolean} & \text{// the predicate used to test elements.} \\
      \end{array}\) \\
       
  \hline
  \bf{Result} & \lst{Coll[IV]} \\
  \hline
  
  \bf{Serialized as} & \hyperref[sec:serialization:operation:Filter]{\lst{Filter}} \\
  \hline
       
\end{tabularx}



\subsubsection{\lst{SCollection.append} method (Code 12.9)}
\label{sec:type:SCollection:append}
\noindent
\begin{tabularx}{\textwidth}{| l | X |}
   \hline
   \bf{Description} & Puts the elements of other collection after the elements of this collection (concatenation of 2 collections) \\
  
  \hline
  \bf{Parameters} &
      \(\begin{array}{l l l}
         \lst{other} & \lst{: Coll[IV]} & \text{// the collection to append at the end of this} \\
      \end{array}\) \\
       
  \hline
  \bf{Result} & \lst{Coll[IV]} \\
  \hline
  
  \bf{Serialized as} & \hyperref[sec:serialization:operation:Append]{\lst{Append}} \\
  \hline
       
\end{tabularx}



\subsubsection{\lst{SCollection.apply} method (Code 12.10)}
\label{sec:type:SCollection:apply}
\noindent
\begin{tabularx}{\textwidth}{| l | X |}
   \hline
   \bf{Description} & The element at given index.
 Indices start at \lst{0}; \lst{xs.apply(0)} is the first element of collection \lst{xs}.
 Note the indexing syntax \lst{xs(i)} is a shorthand for \lst{xs.apply(i)}.
 Returns the element at the given index.
 Throws an exception if \lst{i < 0} or \lst{length <= i}
         \\
  
  \hline
  \bf{Parameters} &
      \(\begin{array}{l l l}
         \lst{i} & \lst{: Int} & \text{// the index} \\
      \end{array}\) \\
       
  \hline
  \bf{Result} & \lst{IV} \\
  \hline
  
  \bf{Serialized as} & \hyperref[sec:serialization:operation:ByIndex]{\lst{ByIndex}} \\
  \hline
       
\end{tabularx}



\subsubsection{\lst{SCollection.<<} method (Code 12.11)}
\label{sec:type:SCollection:<<}
\noindent
\begin{tabularx}{\textwidth}{| l | X |}
   \hline
   \bf{Description} &  \\
  
  \hline
  \bf{Parameters} &
      \(\begin{array}{l l l}
         \lst{arg0} & \lst{: Coll[IV]} & \text{// } \\
\lst{arg1} & \lst{: Int} & \text{// } \\
      \end{array}\) \\
       
  \hline
  \bf{Result} & \lst{Coll[IV]} \\
  \hline
  
\end{tabularx}



\subsubsection{\lst{SCollection.>>} method (Code 12.12)}
\label{sec:type:SCollection:>>}
\noindent
\begin{tabularx}{\textwidth}{| l | X |}
   \hline
   \bf{Description} &  \\
  
  \hline
  \bf{Parameters} &
      \(\begin{array}{l l l}
         \lst{arg0} & \lst{: Coll[IV]} & \text{// } \\
\lst{arg1} & \lst{: Int} & \text{// } \\
      \end{array}\) \\
       
  \hline
  \bf{Result} & \lst{Coll[IV]} \\
  \hline
  
\end{tabularx}



\subsubsection{\lst{SCollection.>>>} method (Code 12.13)}
\label{sec:type:SCollection:>>>}
\noindent
\begin{tabularx}{\textwidth}{| l | X |}
   \hline
   \bf{Description} &  \\
  
  \hline
  \bf{Parameters} &
      \(\begin{array}{l l l}
         \lst{arg0} & \lst{: Coll[Boolean]} & \text{// } \\
\lst{arg1} & \lst{: Int} & \text{// } \\
      \end{array}\) \\
       
  \hline
  \bf{Result} & \lst{Coll[Boolean]} \\
  \hline
  
\end{tabularx}



\subsubsection{\lst{SCollection.indices} method (Code 12.14)}
\label{sec:type:SCollection:indices}
\noindent
\begin{tabularx}{\textwidth}{| l | X |}
   \hline
   \bf{Description} & Produces the range of all indices of this collection as a new collection
 containing [0 .. length-1] values.
         \\
  
  \hline
  \bf{Parameters} &
      \(\begin{array}{l l l}
         
      \end{array}\) \\
       
  \hline
  \bf{Result} & \lst{Coll[Int]} \\
  \hline
  
  \bf{Serialized as} & \hyperref[sec:serialization:operation:PropertyCall]{\lst{PropertyCall}} \\
  \hline
       
\end{tabularx}



\subsubsection{\lst{SCollection.flatMap} method (Code 12.15)}
\label{sec:type:SCollection:flatMap}
\noindent
\begin{tabularx}{\textwidth}{| l | X |}
   \hline
   \bf{Description} &  Builds a new collection by applying a function to all elements of this collection
 and using the elements of the resulting collections.
 Function \lst{f} is constrained to be of the form \lst{x => x.someProperty}, otherwise
 it is illegal.
 Returns a new collection of type \lst{Coll[B]} resulting from applying the given collection-valued function
 \lst{f} to each element of this collection and concatenating the results.
         \\
  
  \hline
  \bf{Parameters} &
      \(\begin{array}{l l l}
         \lst{f} & \lst{: (IV) => Coll[OV]} & \text{// the function to apply to each element.} \\
      \end{array}\) \\
       
  \hline
  \bf{Result} & \lst{Coll[OV]} \\
  \hline
  
  \bf{Serialized as} & \hyperref[sec:serialization:operation:MethodCall]{\lst{MethodCall}} \\
  \hline
       
\end{tabularx}



\subsubsection{\lst{SCollection.segmentLength} method (Code 12.16)}
\label{sec:type:SCollection:segmentLength}
\noindent
\begin{tabularx}{\textwidth}{| l | X |}
   \hline
   \bf{Description} & Computes length of longest segment whose elements all satisfy some predicate.
 Returns the length of the longest segment of this collection starting from index \lst{from}
 such that every element of the segment satisfies the predicate \lst{p}.
         \\
  
  \hline
  \bf{Parameters} &
      \(\begin{array}{l l l}
         \lst{p} & \lst{: (IV) => Boolean} & \text{// the predicate used to test elements.} \\
\lst{from} & \lst{: Int} & \text{// the index where the search starts.} \\
      \end{array}\) \\
       
  \hline
  \bf{Result} & \lst{Int} \\
  \hline
  
  \bf{Serialized as} & \hyperref[sec:serialization:operation:MethodCall]{\lst{MethodCall}} \\
  \hline
       
\end{tabularx}



\subsubsection{\lst{SCollection.indexWhere} method (Code 12.17)}
\label{sec:type:SCollection:indexWhere}
\noindent
\begin{tabularx}{\textwidth}{| l | X |}
   \hline
   \bf{Description} & Finds index of the first element satisfying some predicate after or at some start index.
 Returns the index \lst{>= from} of the first element of this collection that satisfies the predicate \lst{p},
 or \lst{-1}, if none exists.
         \\
  
  \hline
  \bf{Parameters} &
      \(\begin{array}{l l l}
         \lst{p} & \lst{: (IV) => Boolean} & \text{// the predicate used to test elements.} \\
\lst{from} & \lst{: Int} & \text{// the start index} \\
      \end{array}\) \\
       
  \hline
  \bf{Result} & \lst{Int} \\
  \hline
  
  \bf{Serialized as} & \hyperref[sec:serialization:operation:MethodCall]{\lst{MethodCall}} \\
  \hline
       
\end{tabularx}



\subsubsection{\lst{SCollection.lastIndexWhere} method (Code 12.18)}
\label{sec:type:SCollection:lastIndexWhere}
\noindent
\begin{tabularx}{\textwidth}{| l | X |}
   \hline
   \bf{Description} & Finds index of last element satisfying some predicate before or at given end index.
 Return the index \lst{<= end} of the last element of this collection that satisfies the predicate \lst{p},
 or \lst{-1}, if none exists.
         \\
  
  \hline
  \bf{Parameters} &
      \(\begin{array}{l l l}
         \lst{p} & \lst{: (IV) => Boolean} & \text{// the predicate used to test elements.} \\
      \end{array}\) \\
       
  \hline
  \bf{Result} & \lst{Int} \\
  \hline
  
  \bf{Serialized as} & \hyperref[sec:serialization:operation:MethodCall]{\lst{MethodCall}} \\
  \hline
       
\end{tabularx}



\subsubsection{\lst{SCollection.patch} method (Code 12.19)}
\label{sec:type:SCollection:patch}
\noindent
\begin{tabularx}{\textwidth}{| l | X |}
   \hline
   \bf{Description} &  \\
  
  \hline
  \bf{Parameters} &
      \(\begin{array}{l l l}
         
      \end{array}\) \\
       
  \hline
  \bf{Result} & \lst{Coll[IV]} \\
  \hline
  
  \bf{Serialized as} & \hyperref[sec:serialization:operation:MethodCall]{\lst{MethodCall}} \\
  \hline
       
\end{tabularx}



\subsubsection{\lst{SCollection.updated} method (Code 12.20)}
\label{sec:type:SCollection:updated}
\noindent
\begin{tabularx}{\textwidth}{| l | X |}
   \hline
   \bf{Description} &  \\
  
  \hline
  \bf{Parameters} &
      \(\begin{array}{l l l}
         
      \end{array}\) \\
       
  \hline
  \bf{Result} & \lst{Coll[IV]} \\
  \hline
  
  \bf{Serialized as} & \hyperref[sec:serialization:operation:MethodCall]{\lst{MethodCall}} \\
  \hline
       
\end{tabularx}



\subsubsection{\lst{SCollection.updateMany} method (Code 12.21)}
\label{sec:type:SCollection:updateMany}
\noindent
\begin{tabularx}{\textwidth}{| l | X |}
   \hline
   \bf{Description} &  \\
  
  \hline
  \bf{Parameters} &
      \(\begin{array}{l l l}
         
      \end{array}\) \\
       
  \hline
  \bf{Result} & \lst{Coll[IV]} \\
  \hline
  
  \bf{Serialized as} & \hyperref[sec:serialization:operation:MethodCall]{\lst{MethodCall}} \\
  \hline
       
\end{tabularx}



\subsubsection{\lst{SCollection.unionSets} method (Code 12.22)}
\label{sec:type:SCollection:unionSets}
\noindent
\begin{tabularx}{\textwidth}{| l | X |}
   \hline
   \bf{Description} &  \\
  
  \hline
  \bf{Parameters} &
      \(\begin{array}{l l l}
         
      \end{array}\) \\
       
  \hline
  \bf{Result} & \lst{Coll[IV]} \\
  \hline
  
  \bf{Serialized as} & \hyperref[sec:serialization:operation:MethodCall]{\lst{MethodCall}} \\
  \hline
       
\end{tabularx}



\subsubsection{\lst{SCollection.diff} method (Code 12.23)}
\label{sec:type:SCollection:diff}
\noindent
\begin{tabularx}{\textwidth}{| l | X |}
   \hline
   \bf{Description} &  \\
  
  \hline
  \bf{Parameters} &
      \(\begin{array}{l l l}
         
      \end{array}\) \\
       
  \hline
  \bf{Result} & \lst{Coll[IV]} \\
  \hline
  
  \bf{Serialized as} & \hyperref[sec:serialization:operation:MethodCall]{\lst{MethodCall}} \\
  \hline
       
\end{tabularx}



\subsubsection{\lst{SCollection.intersect} method (Code 12.24)}
\label{sec:type:SCollection:intersect}
\noindent
\begin{tabularx}{\textwidth}{| l | X |}
   \hline
   \bf{Description} &  \\
  
  \hline
  \bf{Parameters} &
      \(\begin{array}{l l l}
         
      \end{array}\) \\
       
  \hline
  \bf{Result} & \lst{Coll[IV]} \\
  \hline
  
  \bf{Serialized as} & \hyperref[sec:serialization:operation:MethodCall]{\lst{MethodCall}} \\
  \hline
       
\end{tabularx}



\subsubsection{\lst{SCollection.prefixLength} method (Code 12.25)}
\label{sec:type:SCollection:prefixLength}
\noindent
\begin{tabularx}{\textwidth}{| l | X |}
   \hline
   \bf{Description} &  \\
  
  \hline
  \bf{Parameters} &
      \(\begin{array}{l l l}
         
      \end{array}\) \\
       
  \hline
  \bf{Result} & \lst{Int} \\
  \hline
  
  \bf{Serialized as} & \hyperref[sec:serialization:operation:MethodCall]{\lst{MethodCall}} \\
  \hline
       
\end{tabularx}



\subsubsection{\lst{SCollection.indexOf} method (Code 12.26)}
\label{sec:type:SCollection:indexOf}
\noindent
\begin{tabularx}{\textwidth}{| l | X |}
   \hline
   \bf{Description} &  \\
  
  \hline
  \bf{Parameters} &
      \(\begin{array}{l l l}
         
      \end{array}\) \\
       
  \hline
  \bf{Result} & \lst{Int} \\
  \hline
  
  \bf{Serialized as} & \hyperref[sec:serialization:operation:MethodCall]{\lst{MethodCall}} \\
  \hline
       
\end{tabularx}



\subsubsection{\lst{SCollection.lastIndexOf} method (Code 12.27)}
\label{sec:type:SCollection:lastIndexOf}
\noindent
\begin{tabularx}{\textwidth}{| l | X |}
   \hline
   \bf{Description} &  \\
  
  \hline
  \bf{Parameters} &
      \(\begin{array}{l l l}
         
      \end{array}\) \\
       
  \hline
  \bf{Result} & \lst{Int} \\
  \hline
  
  \bf{Serialized as} & \hyperref[sec:serialization:operation:MethodCall]{\lst{MethodCall}} \\
  \hline
       
\end{tabularx}



\subsubsection{\lst{SCollection.find} method (Code 12.28)}
\label{sec:type:SCollection:find}
\noindent
\begin{tabularx}{\textwidth}{| l | X |}
   \hline
   \bf{Description} &  \\
  
  \hline
  \bf{Parameters} &
      \(\begin{array}{l l l}
         
      \end{array}\) \\
       
  \hline
  \bf{Result} & \lst{Option[IV]} \\
  \hline
  
  \bf{Serialized as} & \hyperref[sec:serialization:operation:MethodCall]{\lst{MethodCall}} \\
  \hline
       
\end{tabularx}



\subsubsection{\lst{SCollection.zip} method (Code 12.29)}
\label{sec:type:SCollection:zip}
\noindent
\begin{tabularx}{\textwidth}{| l | X |}
   \hline
   \bf{Description} &  \\
  
  \hline
  \bf{Parameters} &
      \(\begin{array}{l l l}
         
      \end{array}\) \\
       
  \hline
  \bf{Result} & \lst{Coll[(IV,OV)]} \\
  \hline
  
  \bf{Serialized as} & \hyperref[sec:serialization:operation:MethodCall]{\lst{MethodCall}} \\
  \hline
       
\end{tabularx}



\subsubsection{\lst{SCollection.distinct} method (Code 12.30)}
\label{sec:type:SCollection:distinct}
\noindent
\begin{tabularx}{\textwidth}{| l | X |}
   \hline
   \bf{Description} &  \\
  
  \hline
  \bf{Parameters} &
      \(\begin{array}{l l l}
         
      \end{array}\) \\
       
  \hline
  \bf{Result} & \lst{Coll[IV]} \\
  \hline
  
  \bf{Serialized as} & \hyperref[sec:serialization:operation:PropertyCall]{\lst{PropertyCall}} \\
  \hline
       
\end{tabularx}



\subsubsection{\lst{SCollection.startsWith} method (Code 12.31)}
\label{sec:type:SCollection:startsWith}
\noindent
\begin{tabularx}{\textwidth}{| l | X |}
   \hline
   \bf{Description} &  \\
  
  \hline
  \bf{Parameters} &
      \(\begin{array}{l l l}
         
      \end{array}\) \\
       
  \hline
  \bf{Result} & \lst{Boolean} \\
  \hline
  
  \bf{Serialized as} & \hyperref[sec:serialization:operation:MethodCall]{\lst{MethodCall}} \\
  \hline
       
\end{tabularx}



\subsubsection{\lst{SCollection.endsWith} method (Code 12.32)}
\label{sec:type:SCollection:endsWith}
\noindent
\begin{tabularx}{\textwidth}{| l | X |}
   \hline
   \bf{Description} &  \\
  
  \hline
  \bf{Parameters} &
      \(\begin{array}{l l l}
         
      \end{array}\) \\
       
  \hline
  \bf{Result} & \lst{Boolean} \\
  \hline
  
  \bf{Serialized as} & \hyperref[sec:serialization:operation:MethodCall]{\lst{MethodCall}} \\
  \hline
       
\end{tabularx}



\subsubsection{\lst{SCollection.partition} method (Code 12.33)}
\label{sec:type:SCollection:partition}
\noindent
\begin{tabularx}{\textwidth}{| l | X |}
   \hline
   \bf{Description} &  \\
  
  \hline
  \bf{Parameters} &
      \(\begin{array}{l l l}
         
      \end{array}\) \\
       
  \hline
  \bf{Result} & \lst{(Coll[IV],Coll[IV])} \\
  \hline
  
  \bf{Serialized as} & \hyperref[sec:serialization:operation:MethodCall]{\lst{MethodCall}} \\
  \hline
       
\end{tabularx}



\subsubsection{\lst{SCollection.mapReduce} method (Code 12.34)}
\label{sec:type:SCollection:mapReduce}
\noindent
\begin{tabularx}{\textwidth}{| l | X |}
   \hline
   \bf{Description} &  \\
  
  \hline
  \bf{Parameters} &
      \(\begin{array}{l l l}
         
      \end{array}\) \\
       
  \hline
  \bf{Result} & \lst{Coll[(K,V)]} \\
  \hline
  
  \bf{Serialized as} & \hyperref[sec:serialization:operation:MethodCall]{\lst{MethodCall}} \\
  \hline
       
\end{tabularx}


\subsection{Option type}
\label{sec:type:Option}

\subsubsection{\lst{SOption.isEmpty} method (Code 36.1)}
\label{sec:type:SOption:isEmpty}
\noindent
\begin{tabularx}{\textwidth}{| l | X |}
   \hline
   \bf{Description} &  \\
  
  \hline
  \bf{Parameters} &
      \(\begin{array}{l l l}
         \lst{arg0} & \lst{: Option[T]} & \text{// } \\
      \end{array}\) \\
       
  \hline
  \bf{Result} & \lst{Boolean} \\
  \hline
  
\end{tabularx}



\subsubsection{\lst{SOption.isDefined} method (Code 36.2)}
\label{sec:type:SOption:isDefined}
\noindent
\begin{tabularx}{\textwidth}{| l | X |}
   \hline
   \bf{Description} & Returns \lst{true} if the option is an instance of \lst{Some}, \lst{false} otherwise. \\
  
  \hline
  \bf{Parameters} &
      \(\begin{array}{l l l}
         
      \end{array}\) \\
       
  \hline
  \bf{Result} & \lst{Boolean} \\
  \hline
  
  \bf{Serialized as} & \hyperref[sec:serialization:operation:OptionIsDefined]{\lst{OptionIsDefined}} \\
  \hline
       
\end{tabularx}



\subsubsection{\lst{SOption.get} method (Code 36.3)}
\label{sec:type:SOption:get}
\noindent
\begin{tabularx}{\textwidth}{| l | X |}
   \hline
   \bf{Description} & Returns the option's value. The option must be nonempty. Throws exception if the option is empty. \\
  
  \hline
  \bf{Parameters} &
      \(\begin{array}{l l l}
         
      \end{array}\) \\
       
  \hline
  \bf{Result} & \lst{T} \\
  \hline
  
  \bf{Serialized as} & \hyperref[sec:serialization:operation:OptionGet]{\lst{OptionGet}} \\
  \hline
       
\end{tabularx}



\subsubsection{\lst{SOption.getOrElse} method (Code 36.4)}
\label{sec:type:SOption:getOrElse}
\noindent
\begin{tabularx}{\textwidth}{| l | X |}
   \hline
   \bf{Description} & Returns the option's value if the option is nonempty, otherwise
return the result of evaluating \lst{default}.
         \\
  
  \hline
  \bf{Parameters} &
      \(\begin{array}{l l l}
         \lst{default} & \lst{: T} & \text{// the default value} \\
      \end{array}\) \\
       
  \hline
  \bf{Result} & \lst{T} \\
  \hline
  
  \bf{Serialized as} & \hyperref[sec:serialization:operation:OptionGetOrElse]{\lst{OptionGetOrElse}} \\
  \hline
       
\end{tabularx}



\subsubsection{\lst{SOption.fold} method (Code 36.5)}
\label{sec:type:SOption:fold}
\noindent
\begin{tabularx}{\textwidth}{| l | X |}
   \hline
   \bf{Description} & Returns the result of applying \lst{f} to this option's
  value if the option is nonempty.  Otherwise, evaluates
  expression \lst{ifEmpty}.
  This is equivalent to \lst{option map f getOrElse ifEmpty}.
         \\
  
  \hline
  \bf{Parameters} &
      \(\begin{array}{l l l}
         \lst{ifEmpty} & \lst{: R} & \text{// the expression to evaluate if empty} \\
\lst{f} & \lst{: (T) => R} & \text{// the function to apply if nonempty} \\
      \end{array}\) \\
       
  \hline
  \bf{Result} & \lst{R} \\
  \hline
  
  \bf{Serialized as} & \hyperref[sec:serialization:operation:MethodCall]{\lst{MethodCall}} \\
  \hline
       
\end{tabularx}



\subsubsection{\lst{SOption.toColl} method (Code 36.6)}
\label{sec:type:SOption:toColl}
\noindent
\begin{tabularx}{\textwidth}{| l | X |}
   \hline
   \bf{Description} & Convert this Option to a collection with zero or one element. \\
  
  \hline
  \bf{Parameters} &
      \(\begin{array}{l l l}
         
      \end{array}\) \\
       
  \hline
  \bf{Result} & \lst{Coll[T]} \\
  \hline
  
  \bf{Serialized as} & \hyperref[sec:serialization:operation:PropertyCall]{\lst{PropertyCall}} \\
  \hline
       
\end{tabularx}



\subsubsection{\lst{SOption.map} method (Code 36.7)}
\label{sec:type:SOption:map}
\noindent
\begin{tabularx}{\textwidth}{| l | X |}
   \hline
   \bf{Description} & Returns a \lst{Some} containing the result of applying \lst{f} to this option's
   value if this option is nonempty.
   Otherwise return \lst{None}.
         \\
  
  \hline
  \bf{Parameters} &
      \(\begin{array}{l l l}
         \lst{f} & \lst{: (T) => R} & \text{// the function to apply} \\
      \end{array}\) \\
       
  \hline
  \bf{Result} & \lst{Option[R]} \\
  \hline
  
  \bf{Serialized as} & \hyperref[sec:serialization:operation:MethodCall]{\lst{MethodCall}} \\
  \hline
       
\end{tabularx}



\subsubsection{\lst{SOption.filter} method (Code 36.8)}
\label{sec:type:SOption:filter}
\noindent
\begin{tabularx}{\textwidth}{| l | X |}
   \hline
   \bf{Description} & Returns this option if it is nonempty and applying the predicate \lst{p} to
  this option's value returns true. Otherwise, return \lst{None}.
         \\
  
  \hline
  \bf{Parameters} &
      \(\begin{array}{l l l}
         \lst{p} & \lst{: (T) => Boolean} & \text{// the predicate used for testing} \\
      \end{array}\) \\
       
  \hline
  \bf{Result} & \lst{Option[T]} \\
  \hline
  
  \bf{Serialized as} & \hyperref[sec:serialization:operation:MethodCall]{\lst{MethodCall}} \\
  \hline
       
\end{tabularx}



\subsubsection{\lst{SOption.flatMap} method (Code 36.9)}
\label{sec:type:SOption:flatMap}
\noindent
\begin{tabularx}{\textwidth}{| l | X |}
   \hline
   \bf{Description} & Returns the result of applying \lst{f} to this option's value if
   this option is nonempty.
   Returns \lst{None} if this option is empty.
   Slightly different from \lst{map} in that \lst{f} is expected to
   return an option (which could be \lst{one}).
         \\
  
  \hline
  \bf{Parameters} &
      \(\begin{array}{l l l}
         \lst{f} & \lst{: (T) => Option[R]} & \text{// the function to apply} \\
      \end{array}\) \\
       
  \hline
  \bf{Result} & \lst{Option[R]} \\
  \hline
  
  \bf{Serialized as} & \hyperref[sec:serialization:operation:MethodCall]{\lst{MethodCall}} \\
  \hline
       
\end{tabularx}


\section{Predefined global functions}
\label{sec:appendix:primops}

% \begin{table}[h]
    % \caption{Predefined primitive operations of \langname}
    % \label{table:primops}
    % \footnotesize
    % \tiny
    \tiny
    \begin{longtable}[h]{|l |l | p{.25\linewidth} | p{.5\linewidth} |}
	\hline
	% \rowfont{\bfseries}
Code &   Mnemonic   &  Signature & Description \\
    \hline
     115 & \hyperref[sec:serialization:operation:ConstantPlaceholder]{\lst{ConstantPlaceholder}} & \parbox{4cm}{\lst{placeholder:} \\ \lst{(Int)} \\ \lst{  => T}} & Create special ErgoTree node which can be replaced by constant with given id. \\
 \hline
          116 & \hyperref[sec:serialization:operation:SubstConstants]{\lst{SubstConstants}} & \parbox{4cm}{\lst{substConstants:} \\ \lst{(Coll[Byte], Coll[Int], Coll[T])} \\ \lst{  => Coll[Byte]}} & ... \\
 \hline
          122 & \hyperref[sec:serialization:operation:LongToByteArray]{\lst{LongToByteArray}} & \parbox{4cm}{\lst{longToByteArray:} \\ \lst{(Long)} \\ \lst{  => Coll[Byte]}} & Converts \lst{Long} value to big-endian bytes representation. \\
 \hline
          123 & \hyperref[sec:serialization:operation:ByteArrayToBigInt]{\lst{ByteArrayToBigInt}} & \parbox{4cm}{\lst{byteArrayToBigInt:} \\ \lst{(Coll[Byte])} \\ \lst{  => BigInt}} & Convert big-endian bytes representation (Coll[Byte]) to BigInt value. \\
 \hline
          124 & \hyperref[sec:serialization:operation:ByteArrayToLong]{\lst{ByteArrayToLong}} & \parbox{4cm}{\lst{byteArrayToLong:} \\ \lst{(Coll[Byte])} \\ \lst{  => Long}} & Convert big-endian bytes representation (Coll[Byte]) to Long value. \\
 \hline
          125 & \hyperref[sec:serialization:operation:Downcast]{\lst{Downcast}} & \parbox{4cm}{\lst{downcast:} \\ \lst{(T)} \\ \lst{  => R}} & Cast this numeric value to a smaller type (e.g. Long to Int). Throws exception if overflow. \\
 \hline
          126 & \hyperref[sec:serialization:operation:Upcast]{\lst{Upcast}} & \parbox{4cm}{\lst{upcast:} \\ \lst{(T)} \\ \lst{  => R}} & Cast this numeric value to a bigger type (e.g. Int to Long) \\
 \hline
          140 & \hyperref[sec:serialization:operation:SelectField]{\lst{SelectField}} & \parbox{4cm}{\lst{selectField:} \\ \lst{(T, Byte)} \\ \lst{  => R}} & Select tuple field by its 1-based index. E.g. \lst{input._1} is transformed to \lst{SelectField(input, 1)} \\
 \hline
          143 & \hyperref[sec:serialization:operation:LT]{\lst{LT}} & \parbox{4cm}{\lst{<:} \\ \lst{(T, T)} \\ \lst{  => Boolean}} & Returns \lst{true} is the left operand is less then the right operand, \lst{false} otherwise. \\
 \hline
          144 & \hyperref[sec:serialization:operation:LE]{\lst{LE}} & \parbox{4cm}{\lst{<=:} \\ \lst{(T, T)} \\ \lst{  => Boolean}} & Returns \lst{true} is the left operand is less then or equal to the right operand, \lst{false} otherwise. \\
 \hline
          145 & \hyperref[sec:serialization:operation:GT]{\lst{GT}} & \parbox{4cm}{\lst{>:} \\ \lst{(T, T)} \\ \lst{  => Boolean}} & Returns \lst{true} is the left operand is greater then the right operand, \lst{false} otherwise. \\
 \hline
          146 & \hyperref[sec:serialization:operation:GE]{\lst{GE}} & \parbox{4cm}{\lst{>=:} \\ \lst{(T, T)} \\ \lst{  => Boolean}} & Returns \lst{true} is the left operand is greater then or equal to the right operand, \lst{false} otherwise. \\
 \hline
          147 & \hyperref[sec:serialization:operation:EQ]{\lst{EQ}} & \parbox{4cm}{\lst{==:} \\ \lst{(T, T)} \\ \lst{  => Boolean}} & Compare equality of \lst{left} and \lst{right} arguments \\
 \hline
          148 & \hyperref[sec:serialization:operation:NEQ]{\lst{NEQ}} & \parbox{4cm}{\lst{!=:} \\ \lst{(T, T)} \\ \lst{  => Boolean}} & Compare inequality of \lst{left} and \lst{right} arguments \\
 \hline
          149 & \hyperref[sec:serialization:operation:If]{\lst{If}} & \parbox{4cm}{\lst{if:} \\ \lst{(Boolean, T, T)} \\ \lst{  => T}} & Compute condition, if true then compute trueBranch else compute falseBranch \\
 \hline
          150 & \hyperref[sec:serialization:operation:AND]{\lst{AND}} & \parbox{4cm}{\lst{allOf:} \\ \lst{(Coll[Boolean])} \\ \lst{  => Boolean}} & Returns true if \emph{all} the elements in collection are \lst{true}. \\
 \hline
          151 & \hyperref[sec:serialization:operation:OR]{\lst{OR}} & \parbox{4cm}{\lst{anyOf:} \\ \lst{(Coll[Boolean])} \\ \lst{  => Boolean}} & Returns true if \emph{any} the elements in collection are \lst{true}. \\
 \hline
          152 & \hyperref[sec:serialization:operation:AtLeast]{\lst{AtLeast}} & \parbox{4cm}{\lst{atLeast:} \\ \lst{(Int, Coll[SigmaProp])} \\ \lst{  => SigmaProp}} & ... \\
 \hline
          153 & \hyperref[sec:serialization:operation:Minus]{\lst{Minus}} & \parbox{4cm}{\lst{-:} \\ \lst{(T, T)} \\ \lst{  => T}} & Returns a result of subtracting second numeric operand from the first. \\
 \hline
          154 & \hyperref[sec:serialization:operation:Plus]{\lst{Plus}} & \parbox{4cm}{\lst{+:} \\ \lst{(T, T)} \\ \lst{  => T}} & Returns a sum of two numeric operands \\
 \hline
          155 & \hyperref[sec:serialization:operation:Xor]{\lst{Xor}} & \parbox{4cm}{\lst{binary_|:} \\ \lst{(Coll[Byte], Coll[Byte])} \\ \lst{  => Coll[Byte]}} & Byte-wise XOR of two collections of bytes \\
 \hline
          156 & \hyperref[sec:serialization:operation:Multiply]{\lst{Multiply}} & \parbox{4cm}{\lst{*:} \\ \lst{(T, T)} \\ \lst{  => T}} & Returns a multiplication of two numeric operands \\
 \hline
          157 & \hyperref[sec:serialization:operation:Division]{\lst{Division}} & \parbox{4cm}{\lst{/:} \\ \lst{(T, T)} \\ \lst{  => T}} & Integer division of the first operand by the second operand. \\
 \hline
          158 & \hyperref[sec:serialization:operation:Modulo]{\lst{Modulo}} & \parbox{4cm}{\lst{\%:} \\ \lst{(T, T)} \\ \lst{  => T}} & Reminder from division of the first operand by the second operand. \\
 \hline
          161 & \hyperref[sec:serialization:operation:Min]{\lst{Min}} & \parbox{4cm}{\lst{min:} \\ \lst{(T, T)} \\ \lst{  => T}} & Minimum value of two operands. \\
 \hline
          162 & \hyperref[sec:serialization:operation:Max]{\lst{Max}} & \parbox{4cm}{\lst{max:} \\ \lst{(T, T)} \\ \lst{  => T}} & Maximum value of two operands. \\
 \hline
          182 & \hyperref[sec:serialization:operation:CreateAvlTree]{\lst{CreateAvlTree}} & \parbox{4cm}{\lst{avlTree:} \\ \lst{(Byte, Coll[Byte], Int, Option[Int])} \\ \lst{  => AvlTree}} & Construct a new authenticated dictionary with given parameters and tree root digest. \\
 \hline
          183 & \hyperref[sec:serialization:operation:TreeLookup]{\lst{TreeLookup}} & \parbox{4cm}{\lst{treeLookup:} \\ \lst{(AvlTree, Coll[Byte], Coll[Byte])} \\ \lst{  => Option[Coll[Byte]]}} &  \\
 \hline
          203 & \hyperref[sec:serialization:operation:CalcBlake2b256]{\lst{CalcBlake2b256}} & \parbox{4cm}{\lst{blake2b256:} \\ \lst{(Coll[Byte])} \\ \lst{  => Coll[Byte]}} & Calculate Blake2b hash from \lst{input} bytes. \\
 \hline
          204 & \hyperref[sec:serialization:operation:CalcSha256]{\lst{CalcSha256}} & \parbox{4cm}{\lst{sha256:} \\ \lst{(Coll[Byte])} \\ \lst{  => Coll[Byte]}} & Calculate Sha256 hash from \lst{input} bytes. \\
 \hline
          205 & \hyperref[sec:serialization:operation:CreateProveDlog]{\lst{CreateProveDlog}} & \parbox{4cm}{\lst{proveDlog:} \\ \lst{(GroupElement)} \\ \lst{  => SigmaProp}} & ErgoTree operation to create a new \lst{SigmaProp} value representing public key
 of discrete logarithm signature protocol.
         \\
 \hline
          206 & \hyperref[sec:serialization:operation:CreateProveDHTuple]{\lst{CreateProveDHTuple}} & \parbox{4cm}{\lst{proveDHTuple:} \\ \lst{(GroupElement, GroupElement, GroupElement, GroupElement)} \\ \lst{  => SigmaProp}} &  ErgoTree operation to create a new SigmaProp value representing public key
 of Diffie Hellman signature protocol.
 Common input: (g,h,u,v)
         \\
 \hline
          209 & \hyperref[sec:serialization:operation:BoolToSigmaProp]{\lst{BoolToSigmaProp}} & \parbox{4cm}{\lst{sigmaProp:} \\ \lst{(Boolean)} \\ \lst{  => SigmaProp}} & ... \\
 \hline
          212 & \hyperref[sec:serialization:operation:DeserializeContext]{\lst{DeserializeContext}} & \parbox{4cm}{\lst{executeFromVar:} \\ \lst{(Byte)} \\ \lst{  => T}} & ... \\
 \hline
          213 & \hyperref[sec:serialization:operation:DeserializeRegister]{\lst{DeserializeRegister}} & \parbox{4cm}{\lst{executeFromSelfReg:} \\ \lst{(Byte, Option[T])} \\ \lst{  => T}} & ... \\
 \hline
          218 & \hyperref[sec:serialization:operation:Apply]{\lst{Apply}} & \parbox{4cm}{\lst{apply:} \\ \lst{((T) => R, T)} \\ \lst{  => R}} & Apply the function to the arguments.  \\
 \hline
          227 & \hyperref[sec:serialization:operation:GetVar]{\lst{GetVar}} & \parbox{4cm}{\lst{getVar:} \\ \lst{(Byte)} \\ \lst{  => Option[T]}} & Get context variable with given \lst{varId} and type. \\
 \hline
          234 & \hyperref[sec:serialization:operation:SigmaAnd]{\lst{SigmaAnd}} & \parbox{4cm}{\lst{allZK:} \\ \lst{(Coll[SigmaProp])} \\ \lst{  => SigmaProp}} & Returns sigma proposition which is proven when \emph{all} the elements in collection are proven. \\
 \hline
          235 & \hyperref[sec:serialization:operation:SigmaOr]{\lst{SigmaOr}} & \parbox{4cm}{\lst{anyZK:} \\ \lst{(Coll[SigmaProp])} \\ \lst{  => SigmaProp}} & Returns sigma proposition which is proven when \emph{any} of the elements in collection is proven. \\
 \hline
          236 & \hyperref[sec:serialization:operation:BinOr]{\lst{BinOr}} & \parbox{4cm}{\lst{||:} \\ \lst{(Boolean, Boolean)} \\ \lst{  => Boolean}} & Logical OR of two operands \\
 \hline
          237 & \hyperref[sec:serialization:operation:BinAnd]{\lst{BinAnd}} & \parbox{4cm}{\lst{&&:} \\ \lst{(Boolean, Boolean)} \\ \lst{  => Boolean}} & Logical AND of two operands \\
 \hline
          238 & \hyperref[sec:serialization:operation:DecodePoint]{\lst{DecodePoint}} & \parbox{4cm}{\lst{decodePoint:} \\ \lst{(Coll[Byte])} \\ \lst{  => GroupElement}} & Convert \lst{Coll[Byte]} to \lst{GroupElement} using \lst{GroupElementSerializer} \\
 \hline
          239 & \hyperref[sec:serialization:operation:LogicalNot]{\lst{LogicalNot}} & \parbox{4cm}{\lst{unary_!:} \\ \lst{(Boolean)} \\ \lst{  => Boolean}} & Logical NOT operation. Returns \lst{true} if input is \lst{false} and \lst{false} if input is \lst{true}. \\
 \hline
          240 & \hyperref[sec:serialization:operation:Negation]{\lst{Negation}} & \parbox{4cm}{\lst{unary_-:} \\ \lst{(T)} \\ \lst{  => T}} & Negates numeric value \lst{x} by returning \lst{-x}. \\
 \hline
          241 & \hyperref[sec:serialization:operation:BitInversion]{\lst{BitInversion}} & \parbox{4cm}{\lst{unary_~:} \\ \lst{(T)} \\ \lst{  => T}} & Invert every bit of the numeric value. \\
 \hline
          242 & \hyperref[sec:serialization:operation:BitOr]{\lst{BitOr}} & \parbox{4cm}{\lst{bit_|:} \\ \lst{(T, T)} \\ \lst{  => T}} & Bitwise OR of two numeric operands. \\
 \hline
          243 & \hyperref[sec:serialization:operation:BitAnd]{\lst{BitAnd}} & \parbox{4cm}{\lst{bit_&:} \\ \lst{(T, T)} \\ \lst{  => T}} & Bitwise AND of two numeric operands. \\
 \hline
          244 & \hyperref[sec:serialization:operation:BinXor]{\lst{BinXor}} & \parbox{4cm}{\lst{^:} \\ \lst{(Boolean, Boolean)} \\ \lst{  => Boolean}} & Logical XOR of two operands \\
 \hline
          245 & \hyperref[sec:serialization:operation:BitXor]{\lst{BitXor}} & \parbox{4cm}{\lst{bit_^:} \\ \lst{(T, T)} \\ \lst{  => T}} & Bitwise XOR of two numeric operands. \\
 \hline
          246 & \hyperref[sec:serialization:operation:BitShiftRight]{\lst{BitShiftRight}} & \parbox{4cm}{\lst{bit_>>:} \\ \lst{(T, T)} \\ \lst{  => T}} & Right shift of bits. \\
 \hline
          247 & \hyperref[sec:serialization:operation:BitShiftLeft]{\lst{BitShiftLeft}} & \parbox{4cm}{\lst{bit_<<:} \\ \lst{(T, T)} \\ \lst{  => T}} & Left shift of bits. \\
 \hline
          248 & \hyperref[sec:serialization:operation:BitShiftRightZeroed]{\lst{BitShiftRightZeroed}} & \parbox{4cm}{\lst{bit_>>>:} \\ \lst{(T, T)} \\ \lst{  => T}} & Right shift of bits. \\
 \hline
          255 & \hyperref[sec:serialization:operation:XorOf]{\lst{XorOf}} & \parbox{4cm}{\lst{xorOf:} \\ \lst{(Coll[Boolean])} \\ \lst{  => Boolean}} & Similar to \lst{allOf}, but performing logical XOR operation between all conditions instead of \lst{&&} \\
 \hline
         

% SelectField		&	& $((\tau_1,\dots,\tau_n), i: Byte) \to \tau_i$	&	$\SelectField{(e_1,\dots,e_n)}{i}$	&  \\
% 	\hline
% SomeValue		&	& $[T](x: T) \to Option[T]$	& $\Some{e}$ 	& injects value into non-empty optional value \\
% 	\hline
% NoneValue		&	& $[T]()\to Option[T]$		& $\None{\tau}$	& constructs empty optional value of type $\tau$ \\
% 	\hline
% Collection	&	& $[T](T, \dots, T)\to Coll[T]$	& $\Coll{e_1,\dots,e_n}$ & constructor of collection with $n$ items \\
%     \hline
    \end{longtable}
    \normalsize

% \end{table}

Note, the following table is autogenerated from sigma operation descriptors. See 
\lst{SigmaPredef.scala}


\subsubsection{\lst{substConstants} method (Code 116)}
\label{sec:appendix:primops:SubstConstants}
\noindent
\begin{tabularx}{\textwidth}{| l | X |}
   \hline
   \bf{Description} & Transforms serialized bytes of ErgoTree with segregated constants by replacing constants
 at given positions with new values. This operation allow to use serialized scripts as
 pre-defined templates.
 The typical usage is "check that output box have proposition equal to given script bytes,
 where minerPk (constants(0)) is replaced with currentMinerPk".
 Each constant in original scriptBytes have SType serialized before actual data (see ConstantSerializer).
 During substitution each value from newValues is checked to be an instance of the corresponding type.
 This means, the constants during substitution cannot change their types.

 Returns original scriptBytes array where only specified constants are replaced and all other bytes remain exactly the same.
         \\
   \hline
   \bf{Signature} & \footnotesize \lst{def substConstants}$[$\lst{T}$]$(\lst{scriptBytes}$:$~\lst{Coll[Byte]}, \lst{positions}$:$~\lst{Coll[Int]}, \lst{newValues}$:$~\lst{Coll[T]}): \lst{Coll[Byte]} \\
  
  \hline
  \bf{Parameters} &
      \(\begin{array}{l l}
         \lst{scriptBytes} & \text{serialized ErgoTree with ConstantSegregationFlag set to 1.} \\
\lst{positions} & \text{0-based indexes in ErgoTree.constants} \\
\lst{newValues} & \text{values to be put into the corresponding positions} \\
      \end{array}\) \\
       
  \hline
  
  \bf{Serialized as} & \hyperref[sec:serialization:operation:SubstConstants]{\lst{SubstConstants}} \\
  \hline
       
\end{tabularx}

\subsubsection{\lst{longToByteArray} method (Code 122)}
\label{sec:appendix:primops:LongToByteArray}
\noindent
\begin{tabularx}{\textwidth}{| l | X |}
   \hline
   \bf{Description} & Converts \lst{Long} value to big-endian bytes representation. \\
   \hline
   \bf{Signature} & \lst{def longToByteArray}(\lst{input}$:$~\lst{Long}): \lst{Coll[Byte]} \\
  
  \hline
  \bf{Parameters} &
      \(\begin{array}{l l}
         \lst{input} & \text{value to convert} \\
      \end{array}\) \\
       
  \hline
  
  \bf{Serialized as} & \hyperref[sec:serialization:operation:LongToByteArray]{\lst{LongToByteArray}} \\
  \hline
       
\end{tabularx}

\subsubsection{\lst{byteArrayToBigInt} method (Code 123)}
\label{sec:appendix:primops:ByteArrayToBigInt}
\noindent
\begin{tabularx}{\textwidth}{| l | X |}
   \hline
   \bf{Description} & Convert big-endian bytes representation (Coll[Byte]) to BigInt value. \\
   \hline
   \bf{Signature} & \lst{def byteArrayToBigInt}(\lst{input}$:$~\lst{Coll[Byte]}): \lst{BigInt} \\
  
  \hline
  \bf{Parameters} &
      \(\begin{array}{l l}
         \lst{input} & \text{collection of bytes in big-endian format} \\
      \end{array}\) \\
       
  \hline
  
  \bf{Serialized as} & \hyperref[sec:serialization:operation:ByteArrayToBigInt]{\lst{ByteArrayToBigInt}} \\
  \hline
       
\end{tabularx}

\subsubsection{\lst{byteArrayToLong} method (Code 124)}
\label{sec:appendix:primops:ByteArrayToLong}
\noindent
\begin{tabularx}{\textwidth}{| l | X |}
   \hline
   \bf{Description} & Convert big-endian bytes representation (Coll[Byte]) to Long value. \\
   \hline
   \bf{Signature} & \lst{def byteArrayToLong}(\lst{input}$:$~\lst{Coll[Byte]}): \lst{Long} \\
  
  \hline
  \bf{Parameters} &
      \(\begin{array}{l l}
         \lst{input} & \text{collection of bytes in big-endian format} \\
      \end{array}\) \\
       
  \hline
  
  \bf{Serialized as} & \hyperref[sec:serialization:operation:ByteArrayToLong]{\lst{ByteArrayToLong}} \\
  \hline
       
\end{tabularx}

\subsubsection{\lst{downcast} method (Code 125)}
\label{sec:appendix:primops:Downcast}
\noindent
\begin{tabularx}{\textwidth}{| l | X |}
   \hline
   \bf{Description} & Cast this numeric value to a smaller type (e.g. Long to Int). Throws exception if overflow. \\
   \hline
   \bf{Signature} & \lst{def downcast}$[$\lst{T}, \lst{R}$]$(\lst{input}$:$~\lst{T}): \lst{R} \\
  
  \hline
  \bf{Parameters} &
      \(\begin{array}{l l}
         \lst{input} & \text{value to cast} \\
      \end{array}\) \\
       
  \hline
  
  \bf{Serialized as} & \hyperref[sec:serialization:operation:Downcast]{\lst{Downcast}} \\
  \hline
       
\end{tabularx}

\subsubsection{\lst{upcast} method (Code 126)}
\label{sec:appendix:primops:Upcast}
\noindent
\begin{tabularx}{\textwidth}{| l | X |}
   \hline
   \bf{Description} & Cast this numeric value to a bigger type (e.g. Int to Long) \\
   \hline
   \bf{Signature} & \lst{def upcast}$[$\lst{T}, \lst{R}$]$(\lst{input}$:$~\lst{T}): \lst{R} \\
  
  \hline
  \bf{Parameters} &
      \(\begin{array}{l l}
         \lst{input} & \text{value to cast} \\
      \end{array}\) \\
       
  \hline
  
  \bf{Serialized as} & \hyperref[sec:serialization:operation:Upcast]{\lst{Upcast}} \\
  \hline
       
\end{tabularx}

\subsubsection{\lst{selectField} method (Code 140)}
\label{sec:appendix:primops:SelectField}
\noindent
\begin{tabularx}{\textwidth}{| l | X |}
   \hline
   \bf{Description} & Select tuple field by its 1-based index. E.g. \lst{input._1} is transformed to \lst{SelectField(input, 1)} \\
   \hline
   \bf{Signature} & \footnotesize \lst{def selectField}$[$\lst{T}, \lst{R}$]$(\lst{input}$:$~\lst{T}, \lst{fieldIndex}$:$~\lst{Byte}): \lst{R} \\
  
  \hline
  \bf{Parameters} &
      \(\begin{array}{l l}
         \lst{input} & \text{tuple of items} \\
\lst{fieldIndex} & \text{index of an item to select} \\
      \end{array}\) \\
       
  \hline
  
  \bf{Serialized as} & \hyperref[sec:serialization:operation:SelectField]{\lst{SelectField}} \\
  \hline
       
\end{tabularx}

\subsubsection{\lst{<} method (Code 143)}
\label{sec:appendix:primops:LT}
\noindent
\begin{tabularx}{\textwidth}{| l | X |}
   \hline
   \bf{Description} & Returns \lst{true} is the left operand is less then the right operand, \lst{false} otherwise. \\
   \hline
   \bf{Signature} & \lst{def <}$[$\lst{T}$]$(\lst{left}$:$~\lst{T}, \lst{right}$:$~\lst{T}): \lst{Boolean} \\
  
  \hline
  \bf{Parameters} &
      \(\begin{array}{l l}
         \lst{left} & \text{left operand} \\
\lst{right} & \text{right operand} \\
      \end{array}\) \\
       
  \hline
  
  \bf{Serialized as} & \hyperref[sec:serialization:operation:LT]{\lst{LT}} \\
  \hline
       
\end{tabularx}

\subsubsection{\lst{<=} method (Code 144)}
\label{sec:appendix:primops:LE}
\noindent
\begin{tabularx}{\textwidth}{| l | X |}
   \hline
   \bf{Description} & Returns \lst{true} is the left operand is less then or equal to the right operand, \lst{false} otherwise. \\
   \hline
   \bf{Signature} & \lst{def <=}$[$\lst{T}$]$(\lst{left}$:$~\lst{T}, \lst{right}$:$~\lst{T}): \lst{Boolean} \\
  
  \hline
  \bf{Parameters} &
      \(\begin{array}{l l}
         \lst{left} & \text{left operand} \\
\lst{right} & \text{right operand} \\
      \end{array}\) \\
       
  \hline
  
  \bf{Serialized as} & \hyperref[sec:serialization:operation:LE]{\lst{LE}} \\
  \hline
       
\end{tabularx}

\subsubsection{\lst{>} method (Code 145)}
\label{sec:appendix:primops:GT}
\noindent
\begin{tabularx}{\textwidth}{| l | X |}
   \hline
   \bf{Description} & Returns \lst{true} is the left operand is greater then the right operand, \lst{false} otherwise. \\
   \hline
   \bf{Signature} & \lst{def >}$[$\lst{T}$]$(\lst{left}$:$~\lst{T}, \lst{right}$:$~\lst{T}): \lst{Boolean} \\
  
  \hline
  \bf{Parameters} &
      \(\begin{array}{l l}
         \lst{left} & \text{left operand} \\
\lst{right} & \text{right operand} \\
      \end{array}\) \\
       
  \hline
  
  \bf{Serialized as} & \hyperref[sec:serialization:operation:GT]{\lst{GT}} \\
  \hline
       
\end{tabularx}

\subsubsection{\lst{>=} method (Code 146)}
\label{sec:appendix:primops:GE}
\noindent
\begin{tabularx}{\textwidth}{| l | X |}
   \hline
   \bf{Description} & Returns \lst{true} is the left operand is greater then or equal to the right operand, \lst{false} otherwise. \\
   \hline
   \bf{Signature} & \lst{def >=}$[$\lst{T}$]$(\lst{left}$:$~\lst{T}, \lst{right}$:$~\lst{T}): \lst{Boolean} \\
  
  \hline
  \bf{Parameters} &
      \(\begin{array}{l l}
         \lst{left} & \text{left operand} \\
\lst{right} & \text{right operand} \\
      \end{array}\) \\
       
  \hline
  
  \bf{Serialized as} & \hyperref[sec:serialization:operation:GE]{\lst{GE}} \\
  \hline
       
\end{tabularx}

\subsubsection{\lst{==} method (Code 147)}
\label{sec:appendix:primops:EQ}
\noindent
\begin{tabularx}{\textwidth}{| l | X |}
   \hline
   \bf{Description} & Compare equality of \lst{left} and \lst{right} arguments \\
   \hline
   \bf{Signature} & \lst{def ==}$[$\lst{T}$]$(\lst{left}$:$~\lst{T}, \lst{right}$:$~\lst{T}): \lst{Boolean} \\
  
  \hline
  \bf{Parameters} &
      \(\begin{array}{l l}
         \lst{left} & \text{left operand} \\
\lst{right} & \text{right operand} \\
      \end{array}\) \\
       
  \hline
  
  \bf{Serialized as} & \hyperref[sec:serialization:operation:EQ]{\lst{EQ}} \\
  \hline
       
\end{tabularx}

\subsubsection{\lst{!=} method (Code 148)}
\label{sec:appendix:primops:NEQ}
\noindent
\begin{tabularx}{\textwidth}{| l | X |}
   \hline
   \bf{Description} & Compare inequality of \lst{left} and \lst{right} arguments \\
   \hline
   \bf{Signature} & \lst{def !=}$[$\lst{T}$]$(\lst{left}$:$~\lst{T}, \lst{right}$:$~\lst{T}): \lst{Boolean} \\
  
  \hline
  \bf{Parameters} &
      \(\begin{array}{l l}
         \lst{left} & \text{left operand} \\
\lst{right} & \text{right operand} \\
      \end{array}\) \\
       
  \hline
  
  \bf{Serialized as} & \hyperref[sec:serialization:operation:NEQ]{\lst{NEQ}} \\
  \hline
       
\end{tabularx}

\subsubsection{\lst{if} method (Code 149)}
\label{sec:appendix:primops:If}
\noindent
\begin{tabularx}{\textwidth}{| l | X |}
   \hline
   \bf{Description} & Compute condition, if true then compute trueBranch else compute falseBranch \\
   \hline
   \bf{Signature} & \footnotesize \lst{def if}$[$\lst{T}$]$(\lst{condition}$:$~\lst{Boolean}, \lst{trueBranch}$:$~\lst{T}, \lst{falseBranch}$:$~\lst{T}): \lst{T} \\
  
  \hline
  \bf{Parameters} &
      \(\begin{array}{l l}
         \lst{condition} & \text{condition expression} \\
\lst{trueBranch} & \text{expression to execute when \lst{condition == true}} \\
\lst{falseBranch} & \text{expression to execute when \lst{condition == false}} \\
      \end{array}\) \\
       
  \hline
  
  \bf{Serialized as} & \hyperref[sec:serialization:operation:If]{\lst{If}} \\
  \hline
       
\end{tabularx}

\subsubsection{\lst{allOf} method (Code 150)}
\label{sec:appendix:primops:AND}
\noindent
\begin{tabularx}{\textwidth}{| l | X |}
   \hline
   \bf{Description} & Returns true if \emph{all} the elements in collection are \lst{true}. \\
   \hline
   \bf{Signature} & \lst{def allOf}(\lst{conditions}$:$~\lst{Coll[Boolean]}): \lst{Boolean} \\
  
  \hline
  \bf{Parameters} &
      \(\begin{array}{l l}
         \lst{conditions} & \text{a collection of conditions} \\
      \end{array}\) \\
       
  \hline
  
  \bf{Serialized as} & \hyperref[sec:serialization:operation:AND]{\lst{AND}} \\
  \hline
       
\end{tabularx}

\subsubsection{\lst{anyOf} method (Code 151)}
\label{sec:appendix:primops:OR}
\noindent
\begin{tabularx}{\textwidth}{| l | X |}
   \hline
   \bf{Description} & Returns true if \emph{any} the elements in collection are \lst{true}. \\
   \hline
   \bf{Signature} & \lst{def anyOf}(\lst{conditions}$:$~\lst{Coll[Boolean]}): \lst{Boolean} \\
  
  \hline
  \bf{Parameters} &
      \(\begin{array}{l l}
         \lst{conditions} & \text{a collection of conditions} \\
      \end{array}\) \\
       
  \hline
  
  \bf{Serialized as} & \hyperref[sec:serialization:operation:OR]{\lst{OR}} \\
  \hline
       
\end{tabularx}

\subsubsection{\lst{atLeast} method (Code 152)}
\label{sec:appendix:primops:AtLeast}
\noindent
\begin{tabularx}{\textwidth}{| l | X |}
   \hline
   \bf{Description} &  Logical threshold.
 AtLeast has two inputs: integer \lst{bound} and \lst{children} same as in AND/OR.
 The result is true if at least \lst{bound} children are proven.
         \\
   \hline
   \bf{Signature} & \footnotesize \lst{def atLeast}(\lst{bound}$:$~\lst{Int}, \lst{children}$:$~\lst{Coll[SigmaProp]}): \lst{SigmaProp} \\
  
  \hline
  \bf{Parameters} &
      \(\begin{array}{l l}
         \lst{bound} & \text{required minimum of proven children} \\
\lst{children} & \text{proposition to be proven/validated} \\
      \end{array}\) \\
       
  \hline
  
  \bf{Serialized as} & \hyperref[sec:serialization:operation:AtLeast]{\lst{AtLeast}} \\
  \hline
       
\end{tabularx}

\subsubsection{\lst{-} method (Code 153)}
\label{sec:appendix:primops:Minus}
\noindent
\begin{tabularx}{\textwidth}{| l | X |}
   \hline
   \bf{Description} & Returns a result of subtracting second numeric operand from the first. \\
   \hline
   \bf{Signature} & \lst{def -}$[$\lst{T}$]$(\lst{left}$:$~\lst{T}, \lst{right}$:$~\lst{T}): \lst{T} \\
  
  \hline
  \bf{Parameters} &
      \(\begin{array}{l l}
         \lst{left} & \text{left operand} \\
\lst{right} & \text{right operand} \\
      \end{array}\) \\
       
  \hline
  
  \bf{Serialized as} & \hyperref[sec:serialization:operation:Minus]{\lst{Minus}} \\
  \hline
       
\end{tabularx}

\subsubsection{\lst{+} method (Code 154)}
\label{sec:appendix:primops:Plus}
\noindent
\begin{tabularx}{\textwidth}{| l | X |}
   \hline
   \bf{Description} & Returns a sum of two numeric operands \\
   \hline
   \bf{Signature} & \lst{def +}$[$\lst{T}$]$(\lst{left}$:$~\lst{T}, \lst{right}$:$~\lst{T}): \lst{T} \\
  
  \hline
  \bf{Parameters} &
      \(\begin{array}{l l}
         \lst{left} & \text{left operand} \\
\lst{right} & \text{right operand} \\
      \end{array}\) \\
       
  \hline
  
  \bf{Serialized as} & \hyperref[sec:serialization:operation:Plus]{\lst{Plus}} \\
  \hline
       
\end{tabularx}

\subsubsection{\lst{binary_|} method (Code 155)}
\label{sec:appendix:primops:Xor}
\noindent
\begin{tabularx}{\textwidth}{| l | X |}
   \hline
   \bf{Description} & Byte-wise XOR of two collections of bytes \\
   \hline
   \bf{Signature} & \footnotesize \lst{def binary_|}(\lst{left}$:$~\lst{Coll[Byte]}, \lst{right}$:$~\lst{Coll[Byte]}): \lst{Coll[Byte]} \\
  
  \hline
  \bf{Parameters} &
      \(\begin{array}{l l}
         \lst{left} & \text{left operand} \\
\lst{right} & \text{right operand} \\
      \end{array}\) \\
       
  \hline
  
  \bf{Serialized as} & \hyperref[sec:serialization:operation:Xor]{\lst{Xor}} \\
  \hline
       
\end{tabularx}

\subsubsection{\lst{*} method (Code 156)}
\label{sec:appendix:primops:Multiply}
\noindent
\begin{tabularx}{\textwidth}{| l | X |}
   \hline
   \bf{Description} & Returns a multiplication of two numeric operands \\
   \hline
   \bf{Signature} & \lst{def *}$[$\lst{T}$]$(\lst{left}$:$~\lst{T}, \lst{right}$:$~\lst{T}): \lst{T} \\
  
  \hline
  \bf{Parameters} &
      \(\begin{array}{l l}
         \lst{left} & \text{left operand} \\
\lst{right} & \text{right operand} \\
      \end{array}\) \\
       
  \hline
  
  \bf{Serialized as} & \hyperref[sec:serialization:operation:Multiply]{\lst{Multiply}} \\
  \hline
       
\end{tabularx}

\subsubsection{\lst{/} method (Code 157)}
\label{sec:appendix:primops:Division}
\noindent
\begin{tabularx}{\textwidth}{| l | X |}
   \hline
   \bf{Description} & Integer division of the first operand by the second operand. \\
   \hline
   \bf{Signature} & \lst{def /}$[$\lst{T}$]$(\lst{left}$:$~\lst{T}, \lst{right}$:$~\lst{T}): \lst{T} \\
  
  \hline
  \bf{Parameters} &
      \(\begin{array}{l l}
         \lst{left} & \text{left operand} \\
\lst{right} & \text{right operand} \\
      \end{array}\) \\
       
  \hline
  
  \bf{Serialized as} & \hyperref[sec:serialization:operation:Division]{\lst{Division}} \\
  \hline
       
\end{tabularx}

\subsubsection{\lst{\%} method (Code 158)}
\label{sec:appendix:primops:Modulo}
\noindent
\begin{tabularx}{\textwidth}{| l | X |}
   \hline
   \bf{Description} & Reminder from division of the first operand by the second operand. \\
   \hline
   \bf{Signature} & \lst{def \%}$[$\lst{T}$]$(\lst{left}$:$~\lst{T}, \lst{right}$:$~\lst{T}): \lst{T} \\
  
  \hline
  \bf{Parameters} &
      \(\begin{array}{l l}
         \lst{left} & \text{left operand} \\
\lst{right} & \text{right operand} \\
      \end{array}\) \\
       
  \hline
  
  \bf{Serialized as} & \hyperref[sec:serialization:operation:Modulo]{\lst{Modulo}} \\
  \hline
       
\end{tabularx}

\subsubsection{\lst{min} method (Code 161)}
\label{sec:appendix:primops:Min}
\noindent
\begin{tabularx}{\textwidth}{| l | X |}
   \hline
   \bf{Description} & Minimum value of two operands. \\
   \hline
   \bf{Signature} & \lst{def min}$[$\lst{T}$]$(\lst{left}$:$~\lst{T}, \lst{right}$:$~\lst{T}): \lst{T} \\
  
  \hline
  \bf{Parameters} &
      \(\begin{array}{l l}
         \lst{left} & \text{left operand} \\
\lst{right} & \text{right operand} \\
      \end{array}\) \\
       
  \hline
  
  \bf{Serialized as} & \hyperref[sec:serialization:operation:Min]{\lst{Min}} \\
  \hline
       
\end{tabularx}

\subsubsection{\lst{max} method (Code 162)}
\label{sec:appendix:primops:Max}
\noindent
\begin{tabularx}{\textwidth}{| l | X |}
   \hline
   \bf{Description} & Maximum value of two operands. \\
   \hline
   \bf{Signature} & \lst{def max}$[$\lst{T}$]$(\lst{left}$:$~\lst{T}, \lst{right}$:$~\lst{T}): \lst{T} \\
  
  \hline
  \bf{Parameters} &
      \(\begin{array}{l l}
         \lst{left} & \text{left operand} \\
\lst{right} & \text{right operand} \\
      \end{array}\) \\
       
  \hline
  
  \bf{Serialized as} & \hyperref[sec:serialization:operation:Max]{\lst{Max}} \\
  \hline
       
\end{tabularx}

\subsubsection{\lst{avlTree} method (Code 182)}
\label{sec:appendix:primops:CreateAvlTree}
\noindent
\begin{tabularx}{\textwidth}{| l | X |}
   \hline
   \bf{Description} & Construct a new authenticated dictionary with given parameters and tree root digest. \\
   \hline
   \bf{Signature} & \footnotesize \lst{def avlTree}(\lst{operationFlags}$:$~\lst{Byte}, \lst{digest}$:$~\lst{Coll[Byte]}, \lst{keyLength}$:$~\lst{Int}, \lst{valueLengthOpt}$:$~\lst{Option[Int]}): \lst{AvlTree} \\
  
  \hline
  \bf{Parameters} &
      \(\begin{array}{l l}
         \lst{operationFlags} & \text{flags of available operations} \\
\lst{digest} & \text{hash of merkle tree root} \\
\lst{keyLength} & \text{length of dictionary keys in bytes} \\
\lst{valueLengthOpt} & \text{optional width of dictionary values in bytes} \\
      \end{array}\) \\
       
  \hline
  
  \bf{Serialized as} & \hyperref[sec:serialization:operation:CreateAvlTree]{\lst{CreateAvlTree}} \\
  \hline
       
\end{tabularx}

\subsubsection{\lst{treeLookup} method (Code 183)}
\label{sec:appendix:primops:TreeLookup}
\noindent
\begin{tabularx}{\textwidth}{| l | X |}
   \hline
   \bf{Description} &  \\
   \hline
   \bf{Signature} & \footnotesize \lst{def treeLookup}(\lst{tree}$:$~\lst{AvlTree}, \lst{key}$:$~\lst{Coll[Byte]}, \lst{proof}$:$~\lst{Coll[Byte]}): \lst{Option[Coll[Byte]]} \\
  
  \hline
  \bf{Parameters} &
      \(\begin{array}{l l}
         \lst{tree} & \text{tree to lookup the key} \\
\lst{key} & \text{a key of an item in the \lst{tree} to lookup} \\
\lst{proof} & \text{proof to perform verification of the operation} \\
      \end{array}\) \\
       
  \hline
  
  \bf{Serialized as} & \hyperref[sec:serialization:operation:TreeLookup]{\lst{TreeLookup}} \\
  \hline
       
\end{tabularx}

\subsubsection{\lst{blake2b256} method (Code 203)}
\label{sec:appendix:primops:CalcBlake2b256}
\noindent
\begin{tabularx}{\textwidth}{| l | X |}
   \hline
   \bf{Description} & Calculate Blake2b hash from \lst{input} bytes. \\
   \hline
   \bf{Signature} & \lst{def blake2b256}(\lst{input}$:$~\lst{Coll[Byte]}): \lst{Coll[Byte]} \\
  
  \hline
  \bf{Parameters} &
      \(\begin{array}{l l}
         \lst{input} & \text{collection of bytes} \\
      \end{array}\) \\
       
  \hline
  
  \bf{Serialized as} & \hyperref[sec:serialization:operation:CalcBlake2b256]{\lst{CalcBlake2b256}} \\
  \hline
       
\end{tabularx}

\subsubsection{\lst{sha256} method (Code 204)}
\label{sec:appendix:primops:CalcSha256}
\noindent
\begin{tabularx}{\textwidth}{| l | X |}
   \hline
   \bf{Description} & Calculate Sha256 hash from \lst{input} bytes. \\
   \hline
   \bf{Signature} & \lst{def sha256}(\lst{input}$:$~\lst{Coll[Byte]}): \lst{Coll[Byte]} \\
  
  \hline
  \bf{Parameters} &
      \(\begin{array}{l l}
         \lst{input} & \text{collection of bytes} \\
      \end{array}\) \\
       
  \hline
  
  \bf{Serialized as} & \hyperref[sec:serialization:operation:CalcSha256]{\lst{CalcSha256}} \\
  \hline
       
\end{tabularx}

\subsubsection{\lst{proveDlog} method (Code 205)}
\label{sec:appendix:primops:CreateProveDlog}
\noindent
\begin{tabularx}{\textwidth}{| l | X |}
   \hline
   \bf{Description} & ErgoTree operation to create a new \lst{SigmaProp} value representing public key
 of discrete logarithm signature protocol.
         \\
   \hline
   \bf{Signature} & \lst{def proveDlog}(\lst{value}$:$~\lst{GroupElement}): \lst{SigmaProp} \\
  
  \hline
  \bf{Parameters} &
      \(\begin{array}{l l}
         \lst{value} & \text{element of elliptic curve group} \\
      \end{array}\) \\
       
  \hline
  
  \bf{Serialized as} & \hyperref[sec:serialization:operation:CreateProveDlog]{\lst{CreateProveDlog}} \\
  \hline
       
\end{tabularx}

\subsubsection{\lst{proveDHTuple} method (Code 206)}
\label{sec:appendix:primops:CreateProveDHTuple}
\noindent
\begin{tabularx}{\textwidth}{| l | X |}
   \hline
   \bf{Description} &  ErgoTree operation to create a new SigmaProp value representing public key
 of Diffie Hellman signature protocol.
 Common input: (g,h,u,v)
         \\
   \hline
   \bf{Signature} & \footnotesize \lst{def proveDHTuple}(\lst{g}$:$~\lst{GroupElement}, \lst{h}$:$~\lst{GroupElement}, \lst{u}$:$~\lst{GroupElement}, \lst{v}$:$~\lst{GroupElement}): \lst{SigmaProp} \\
  
  \hline
  \bf{Parameters} &
      \(\begin{array}{l l}
         \lst{g} & \text{} \\
\lst{h} & \text{} \\
\lst{u} & \text{} \\
\lst{v} & \text{} \\
      \end{array}\) \\
       
  \hline
  
  \bf{Serialized as} & \hyperref[sec:serialization:operation:CreateProveDHTuple]{\lst{CreateProveDHTuple}} \\
  \hline
       
\end{tabularx}

\subsubsection{\lst{sigmaProp} method (Code 209)}
\label{sec:appendix:primops:BoolToSigmaProp}
\noindent
\begin{tabularx}{\textwidth}{| l | X |}
   \hline
   \bf{Description} & Embedding of \lst{Boolean} values to \lst{SigmaProp} values.
 As an example, this operation allows boolean experessions
 to be used as arguments of \lst{atLeast(..., sigmaProp(boolExpr), ...)} operation.
 During execution results to either \lst{TrueProp} or \lst{FalseProp} values of \lst{SigmaProp} type.
         \\
   \hline
   \bf{Signature} & \lst{def sigmaProp}(\lst{condition}$:$~\lst{Boolean}): \lst{SigmaProp} \\
  
  \hline
  \bf{Parameters} &
      \(\begin{array}{l l}
         \lst{condition} & \text{boolean value to embed in SigmaProp value} \\
      \end{array}\) \\
       
  \hline
  
  \bf{Serialized as} & \hyperref[sec:serialization:operation:BoolToSigmaProp]{\lst{BoolToSigmaProp}} \\
  \hline
       
\end{tabularx}

\subsubsection{\lst{executeFromVar} method (Code 212)}
\label{sec:appendix:primops:DeserializeContext}
\noindent
\begin{tabularx}{\textwidth}{| l | X |}
   \hline
   \bf{Description} & Extracts context variable as \lst{Coll[Byte]}, deserializes it to script
 and then executes this script in the current context.
 The original \lst{Coll[Byte]} of the script is available as \lst{getVar[Coll[Byte]](id)}.
 Type parameter \lst{V} result type of the deserialized script.
 Throws an exception if the actual script type doesn't conform to T.
 Returns a result of the script execution in the current context
         \\
   \hline
   \bf{Signature} & \lst{def executeFromVar}$[$\lst{T}$]$(\lst{id}$:$~\lst{Byte}): \lst{T} \\
  
  \hline
  \bf{Parameters} &
      \(\begin{array}{l l}
         \lst{id} & \text{identifier of the context variable} \\
      \end{array}\) \\
       
  \hline
  
  \bf{Serialized as} & \hyperref[sec:serialization:operation:DeserializeContext]{\lst{DeserializeContext}} \\
  \hline
       
\end{tabularx}

\subsubsection{\lst{executeFromSelfReg} method (Code 213)}
\label{sec:appendix:primops:DeserializeRegister}
\noindent
\begin{tabularx}{\textwidth}{| l | X |}
   \hline
   \bf{Description} & Extracts SELF register as \lst{Coll[Byte]}, deserializes it to script
 and then executes this script in the current context.
 The original \lst{Coll[Byte]} of the script is available as \lst{SELF.getReg[Coll[Byte]](id)}.
 Type parameter \lst{T} result type of the deserialized script.
 Throws an exception if the actual script type doesn't conform to \lst{T}.
 Returns a result of the script execution in the current context
         \\
   \hline
   \bf{Signature} & \footnotesize \lst{def executeFromSelfReg}$[$\lst{T}$]$(\lst{id}$:$~\lst{Byte}, \lst{default}$:$~\lst{Option[T]}): \lst{T} \\
  
  \hline
  \bf{Parameters} &
      \(\begin{array}{l l}
         \lst{id} & \text{identifier of the register} \\
\lst{default} & \text{optional default value, if register is not available} \\
      \end{array}\) \\
       
  \hline
  
  \bf{Serialized as} & \hyperref[sec:serialization:operation:DeserializeRegister]{\lst{DeserializeRegister}} \\
  \hline
       
\end{tabularx}

\subsubsection{\lst{apply} method (Code 218)}
\label{sec:appendix:primops:Apply}
\noindent
\begin{tabularx}{\textwidth}{| l | X |}
   \hline
   \bf{Description} & Apply the function to the arguments.  \\
   \hline
   \bf{Signature} & \lst{def apply}$[$\lst{T}, \lst{R}$]$(\lst{func}$:$~\lst{(T) => R}, \lst{args}$:$~\lst{T}): \lst{R} \\
  
  \hline
  \bf{Parameters} &
      \(\begin{array}{l l}
         \lst{func} & \text{function which is applied} \\
\lst{args} & \text{list of arguments} \\
      \end{array}\) \\
       
  \hline
  
  \bf{Serialized as} & \hyperref[sec:serialization:operation:Apply]{\lst{Apply}} \\
  \hline
       
\end{tabularx}

\subsubsection{\lst{getVar} method (Code 227)}
\label{sec:appendix:primops:GetVar}
\noindent
\begin{tabularx}{\textwidth}{| l | X |}
   \hline
   \bf{Description} & Get context variable with given \lst{varId} and type. \\
   \hline
   \bf{Signature} & \lst{def getVar}$[$\lst{T}$]$(\lst{varId}$:$~\lst{Byte}): \lst{Option[T]} \\
  
  \hline
  \bf{Parameters} &
      \(\begin{array}{l l}
         \lst{varId} & \text{\lst{Byte} identifier of context variable} \\
      \end{array}\) \\
       
  \hline
  
  \bf{Serialized as} & \hyperref[sec:serialization:operation:GetVar]{\lst{GetVar}} \\
  \hline
       
\end{tabularx}

\subsubsection{\lst{allZK} method (Code 234)}
\label{sec:appendix:primops:SigmaAnd}
\noindent
\begin{tabularx}{\textwidth}{| l | X |}
   \hline
   \bf{Description} & Returns sigma proposition which is proven when \emph{all} the elements in collection are proven. \\
   \hline
   \bf{Signature} & \lst{def allZK}(\lst{propositions}$:$~\lst{Coll[SigmaProp]}): \lst{SigmaProp} \\
  
  \hline
  \bf{Parameters} &
      \(\begin{array}{l l}
         \lst{propositions} & \text{a collection of propositions} \\
      \end{array}\) \\
       
  \hline
  
  \bf{Serialized as} & \hyperref[sec:serialization:operation:SigmaAnd]{\lst{SigmaAnd}} \\
  \hline
       
\end{tabularx}

\subsubsection{\lst{anyZK} method (Code 235)}
\label{sec:appendix:primops:SigmaOr}
\noindent
\begin{tabularx}{\textwidth}{| l | X |}
   \hline
   \bf{Description} & Returns sigma proposition which is proven when \emph{any} of the elements in collection is proven. \\
   \hline
   \bf{Signature} & \lst{def anyZK}(\lst{propositions}$:$~\lst{Coll[SigmaProp]}): \lst{SigmaProp} \\
  
  \hline
  \bf{Parameters} &
      \(\begin{array}{l l}
         \lst{propositions} & \text{a collection of propositions} \\
      \end{array}\) \\
       
  \hline
  
  \bf{Serialized as} & \hyperref[sec:serialization:operation:SigmaOr]{\lst{SigmaOr}} \\
  \hline
       
\end{tabularx}

\subsubsection{\lst{||} method (Code 236)}
\label{sec:appendix:primops:BinOr}
\noindent
\begin{tabularx}{\textwidth}{| l | X |}
   \hline
   \bf{Description} & Logical OR of two operands \\
   \hline
   \bf{Signature} & \lst{def ||}(\lst{left}$:$~\lst{Boolean}, \lst{right}$:$~\lst{Boolean}): \lst{Boolean} \\
  
  \hline
  \bf{Parameters} &
      \(\begin{array}{l l}
         \lst{left} & \text{left operand} \\
\lst{right} & \text{right operand} \\
      \end{array}\) \\
       
  \hline
  
  \bf{Serialized as} & \hyperref[sec:serialization:operation:BinOr]{\lst{BinOr}} \\
  \hline
       
\end{tabularx}

\subsubsection{\lst{&&} method (Code 237)}
\label{sec:appendix:primops:BinAnd}
\noindent
\begin{tabularx}{\textwidth}{| l | X |}
   \hline
   \bf{Description} & Logical AND of two operands \\
   \hline
   \bf{Signature} & \lst{def &&}(\lst{left}$:$~\lst{Boolean}, \lst{right}$:$~\lst{Boolean}): \lst{Boolean} \\
  
  \hline
  \bf{Parameters} &
      \(\begin{array}{l l}
         \lst{left} & \text{left operand} \\
\lst{right} & \text{right operand} \\
      \end{array}\) \\
       
  \hline
  
  \bf{Serialized as} & \hyperref[sec:serialization:operation:BinAnd]{\lst{BinAnd}} \\
  \hline
       
\end{tabularx}

\subsubsection{\lst{decodePoint} method (Code 238)}
\label{sec:appendix:primops:DecodePoint}
\noindent
\begin{tabularx}{\textwidth}{| l | X |}
   \hline
   \bf{Description} & Convert \lst{Coll[Byte]} to \lst{GroupElement} using \lst{GroupElementSerializer} \\
   \hline
   \bf{Signature} & \lst{def decodePoint}(\lst{input}$:$~\lst{Coll[Byte]}): \lst{GroupElement} \\
  
  \hline
  \bf{Parameters} &
      \(\begin{array}{l l}
         \lst{input} & \text{serialized bytes of some \lst{GroupElement} value} \\
      \end{array}\) \\
       
  \hline
  
  \bf{Serialized as} & \hyperref[sec:serialization:operation:DecodePoint]{\lst{DecodePoint}} \\
  \hline
       
\end{tabularx}

\subsubsection{\lst{unary_!} method (Code 239)}
\label{sec:appendix:primops:LogicalNot}
\noindent
\begin{tabularx}{\textwidth}{| l | X |}
   \hline
   \bf{Description} & Logical NOT operation. Returns \lst{true} if input is \lst{false} and \lst{false} if input is \lst{true}. \\
   \hline
   \bf{Signature} & \lst{def unary_!}(\lst{input}$:$~\lst{Boolean}): \lst{Boolean} \\
  
  \hline
  \bf{Parameters} &
      \(\begin{array}{l l}
         \lst{input} & \text{input \lst{Boolean} value} \\
      \end{array}\) \\
       
  \hline
  
  \bf{Serialized as} & \hyperref[sec:serialization:operation:LogicalNot]{\lst{LogicalNot}} \\
  \hline
       
\end{tabularx}

\subsubsection{\lst{unary_-} method (Code 240)}
\label{sec:appendix:primops:Negation}
\noindent
\begin{tabularx}{\textwidth}{| l | X |}
   \hline
   \bf{Description} & Negates numeric value \lst{x} by returning \lst{-x}. \\
   \hline
   \bf{Signature} & \lst{def unary_-}$[$\lst{T}$]$(\lst{input}$:$~\lst{T}): \lst{T} \\
  
  \hline
  \bf{Parameters} &
      \(\begin{array}{l l}
         \lst{input} & \text{value of numeric type} \\
      \end{array}\) \\
       
  \hline
  
  \bf{Serialized as} & \hyperref[sec:serialization:operation:Negation]{\lst{Negation}} \\
  \hline
       
\end{tabularx}

\subsubsection{\lst{unary_~} method (Code 241)}
\label{sec:appendix:primops:BitInversion}
\noindent
\begin{tabularx}{\textwidth}{| l | X |}
   \hline
   \bf{Description} & Invert every bit of the numeric value. \\
   \hline
   \bf{Signature} & \lst{def unary_~}$[$\lst{T}$]$(\lst{input}$:$~\lst{T}): \lst{T} \\
  
  \hline
  \bf{Parameters} &
      \(\begin{array}{l l}
         \lst{input} & \text{value of numeric type} \\
      \end{array}\) \\
       
  \hline
  
  \bf{Serialized as} & \hyperref[sec:serialization:operation:BitInversion]{\lst{BitInversion}} \\
  \hline
       
\end{tabularx}

\subsubsection{\lst{bit_|} method (Code 242)}
\label{sec:appendix:primops:BitOr}
\noindent
\begin{tabularx}{\textwidth}{| l | X |}
   \hline
   \bf{Description} & Bitwise OR of two numeric operands. \\
   \hline
   \bf{Signature} & \lst{def bit_|}$[$\lst{T}$]$(\lst{left}$:$~\lst{T}, \lst{right}$:$~\lst{T}): \lst{T} \\
  
  \hline
  \bf{Parameters} &
      \(\begin{array}{l l}
         \lst{left} & \text{left operand} \\
\lst{right} & \text{right operand} \\
      \end{array}\) \\
       
  \hline
  
  \bf{Serialized as} & \hyperref[sec:serialization:operation:BitOr]{\lst{BitOr}} \\
  \hline
       
\end{tabularx}

\subsubsection{\lst{bit_&} method (Code 243)}
\label{sec:appendix:primops:BitAnd}
\noindent
\begin{tabularx}{\textwidth}{| l | X |}
   \hline
   \bf{Description} & Bitwise AND of two numeric operands. \\
   \hline
   \bf{Signature} & \lst{def bit_&}$[$\lst{T}$]$(\lst{left}$:$~\lst{T}, \lst{right}$:$~\lst{T}): \lst{T} \\
  
  \hline
  \bf{Parameters} &
      \(\begin{array}{l l}
         \lst{left} & \text{left operand} \\
\lst{right} & \text{right operand} \\
      \end{array}\) \\
       
  \hline
  
  \bf{Serialized as} & \hyperref[sec:serialization:operation:BitAnd]{\lst{BitAnd}} \\
  \hline
       
\end{tabularx}

\subsubsection{\lst{^} method (Code 244)}
\label{sec:appendix:primops:BinXor}
\noindent
\begin{tabularx}{\textwidth}{| l | X |}
   \hline
   \bf{Description} & Logical XOR of two operands \\
   \hline
   \bf{Signature} & \lst{def ^}(\lst{left}$:$~\lst{Boolean}, \lst{right}$:$~\lst{Boolean}): \lst{Boolean} \\
  
  \hline
  \bf{Parameters} &
      \(\begin{array}{l l}
         \lst{left} & \text{left operand} \\
\lst{right} & \text{right operand} \\
      \end{array}\) \\
       
  \hline
  
  \bf{Serialized as} & \hyperref[sec:serialization:operation:BinXor]{\lst{BinXor}} \\
  \hline
       
\end{tabularx}

\subsubsection{\lst{bit_^} method (Code 245)}
\label{sec:appendix:primops:BitXor}
\noindent
\begin{tabularx}{\textwidth}{| l | X |}
   \hline
   \bf{Description} & Bitwise XOR of two numeric operands. \\
   \hline
   \bf{Signature} & \lst{def bit_^}$[$\lst{T}$]$(\lst{left}$:$~\lst{T}, \lst{right}$:$~\lst{T}): \lst{T} \\
  
  \hline
  \bf{Parameters} &
      \(\begin{array}{l l}
         \lst{left} & \text{left operand} \\
\lst{right} & \text{right operand} \\
      \end{array}\) \\
       
  \hline
  
  \bf{Serialized as} & \hyperref[sec:serialization:operation:BitXor]{\lst{BitXor}} \\
  \hline
       
\end{tabularx}

\subsubsection{\lst{bit_>>} method (Code 246)}
\label{sec:appendix:primops:BitShiftRight}
\noindent
\begin{tabularx}{\textwidth}{| l | X |}
   \hline
   \bf{Description} & Right shift of bits. \\
   \hline
   \bf{Signature} & \lst{def bit_>>}$[$\lst{T}$]$(\lst{left}$:$~\lst{T}, \lst{right}$:$~\lst{T}): \lst{T} \\
  
  \hline
  \bf{Parameters} &
      \(\begin{array}{l l}
         \lst{left} & \text{left operand} \\
\lst{right} & \text{right operand} \\
      \end{array}\) \\
       
  \hline
  
  \bf{Serialized as} & \hyperref[sec:serialization:operation:BitShiftRight]{\lst{BitShiftRight}} \\
  \hline
       
\end{tabularx}

\subsubsection{\lst{bit_<<} method (Code 247)}
\label{sec:appendix:primops:BitShiftLeft}
\noindent
\begin{tabularx}{\textwidth}{| l | X |}
   \hline
   \bf{Description} & Left shift of bits. \\
   \hline
   \bf{Signature} & \lst{def bit_<<}$[$\lst{T}$]$(\lst{left}$:$~\lst{T}, \lst{right}$:$~\lst{T}): \lst{T} \\
  
  \hline
  \bf{Parameters} &
      \(\begin{array}{l l}
         \lst{left} & \text{left operand} \\
\lst{right} & \text{right operand} \\
      \end{array}\) \\
       
  \hline
  
  \bf{Serialized as} & \hyperref[sec:serialization:operation:BitShiftLeft]{\lst{BitShiftLeft}} \\
  \hline
       
\end{tabularx}

\subsubsection{\lst{bit_>>>} method (Code 248)}
\label{sec:appendix:primops:BitShiftRightZeroed}
\noindent
\begin{tabularx}{\textwidth}{| l | X |}
   \hline
   \bf{Description} & Right shift of bits. \\
   \hline
   \bf{Signature} & \lst{def bit_>>>}$[$\lst{T}$]$(\lst{left}$:$~\lst{T}, \lst{right}$:$~\lst{T}): \lst{T} \\
  
  \hline
  \bf{Parameters} &
      \(\begin{array}{l l}
         \lst{left} & \text{left operand} \\
\lst{right} & \text{right operand} \\
      \end{array}\) \\
       
  \hline
  
  \bf{Serialized as} & \hyperref[sec:serialization:operation:BitShiftRightZeroed]{\lst{BitShiftRightZeroed}} \\
  \hline
       
\end{tabularx}

\subsubsection{\lst{xorOf} method (Code 255)}
\label{sec:appendix:primops:XorOf}
\noindent
\begin{tabularx}{\textwidth}{| l | X |}
   \hline
   \bf{Description} & Similar to \lst{allOf}, but performing logical XOR operation between all conditions instead of \lst{&&} \\
   \hline
   \bf{Signature} & \lst{def xorOf}(\lst{conditions}$:$~\lst{Coll[Boolean]}): \lst{Boolean} \\
  
  \hline
  \bf{Parameters} &
      \(\begin{array}{l l}
         \lst{conditions} & \text{a collection of conditions} \\
      \end{array}\) \\
       
  \hline
  
  \bf{Serialized as} & \hyperref[sec:serialization:operation:XorOf]{\lst{XorOf}} \\
  \hline
       
\end{tabularx}


\section{Serialization format of ErgoTree nodes}
\label{sec:appendix:ergotree_serialization}

\mnote{These subsections are autogenerated from instrumented ValueSerializers}


\subsubsection{\lst{ConcreteCollection} operation (OpCode 131)}
\label{sec:serialization:operation:ConcreteCollection}

 

\noindent
\(\begin{tabularx}{\textwidth}{| l | l | l | X |}
    \hline
    \bf{Slot} & \bf{Format} & \bf{\#bytes} & \bf{Description} \\
    \hline
         $ numItems $ & \lst{VLQ(UShort)} & [1, *] & number of item in a collection of expressions \\
    \hline
           $ elementType $ & \lst{Type} & [1, *] & type of each expression in the collection \\
    \hline
          \multicolumn{4}{l}{\lst{for}~$i=1$~\lst{to}~$numItems$} \\
    \hline
             ~~ $ item_i $ & \lst{Expr} & [1, *] & expression in i-th position \\
    \hline
          \multicolumn{4}{l}{\lst{end for}} \\
\end{tabularx}\)
       

\subsubsection{\lst{ConcreteCollectionBooleanConstant} operation (OpCode 133)}
\label{sec:serialization:operation:ConcreteCollectionBooleanConstant}

 

\noindent
\(\begin{tabularx}{\textwidth}{| l | l | l | X |}
    \hline
    \bf{Slot} & \bf{Format} & \bf{\#bytes} & \bf{Description} \\
    \hline
         $ numBits $ & \lst{VLQ(UShort)} & [1, *] & number of items in a collection of Boolean values \\
    \hline
           $ bits $ & \lst{Bits} & [1, 1024] & Boolean values encoded as as bits (right most byte is zero-padded on the right) \\
    \hline
      
\end{tabularx}\)
       

\subsubsection{\lst{Tuple} operation (OpCode 134)}
\label{sec:serialization:operation:Tuple}

 

\noindent
\(\begin{tabularx}{\textwidth}{| l | l | l | X |}
    \hline
    \bf{Slot} & \bf{Format} & \bf{\#bytes} & \bf{Description} \\
    \hline
         $ numItems $ & \lst{UByte} & 1 & number of items in the tuple \\
    \hline
          \multicolumn{4}{l}{\lst{for}~$i=1$~\lst{to}~$numItems$} \\
    \hline
             ~~ $ item_i $ & \lst{Expr} & [1, *] & tuple's item in i-th position \\
    \hline
          \multicolumn{4}{l}{\lst{end for}} \\
\end{tabularx}\)
       

\subsubsection{\lst{SelectField} operation (OpCode 140)}
\label{sec:serialization:operation:SelectField}

Select tuple field by its 1-based index. E.g. \lst{input._1} is transformed to \lst{SelectField(input, 1)} See~\hyperref[sec:appendix:primops:SelectField]{\lst{selectField}}

\noindent
\(\begin{tabularx}{\textwidth}{| l | l | l | X |}
    \hline
    \bf{Slot} & \bf{Format} & \bf{\#bytes} & \bf{Description} \\
    \hline
         $ input $ & \lst{Expr} & [1, *] & tuple of items \\
    \hline
           $ fieldIndex $ & \lst{Byte} & 1 & index of an item to select \\
    \hline
      
\end{tabularx}\)
       

\subsubsection{\lst{LT} operation (OpCode 143)}
\label{sec:serialization:operation:LT}

Returns \lst{true} is the left operand is less then the right operand, \lst{false} otherwise. See~\hyperref[sec:appendix:primops:LT]{\lst{<}}

\noindent
\(\begin{tabularx}{\textwidth}{| l | l | l | X |}
    \hline
    \bf{Slot} & \bf{Format} & \bf{\#bytes} & \bf{Description} \\
    \hline
        \multicolumn{4}{l}{\lst{match}~$ (left, right) $} \\
         
    \multicolumn{4}{l}{~~\lst{otherwise} } \\
    \hline
            ~~~~ $ left $ & \lst{Expr} & [1, *] & left operand \\
    \hline
          ~~~~ $ right $ & \lst{Expr} & [1, *] & right operand \\
    \hline
          \multicolumn{4}{l}{\lst{end match}} \\
\end{tabularx}\)
       

\subsubsection{\lst{LE} operation (OpCode 144)}
\label{sec:serialization:operation:LE}

Returns \lst{true} is the left operand is less then or equal to the right operand, \lst{false} otherwise. See~\hyperref[sec:appendix:primops:LE]{\lst{<=}}

\noindent
\(\begin{tabularx}{\textwidth}{| l | l | l | X |}
    \hline
    \bf{Slot} & \bf{Format} & \bf{\#bytes} & \bf{Description} \\
    \hline
        \multicolumn{4}{l}{\lst{match}~$ (left, right) $} \\
         
    \multicolumn{4}{l}{~~\lst{otherwise} } \\
    \hline
            ~~~~ $ left $ & \lst{Expr} & [1, *] & left operand \\
    \hline
          ~~~~ $ right $ & \lst{Expr} & [1, *] & right operand \\
    \hline
          \multicolumn{4}{l}{\lst{end match}} \\
\end{tabularx}\)
       

\subsubsection{\lst{GT} operation (OpCode 145)}
\label{sec:serialization:operation:GT}

Returns \lst{true} is the left operand is greater then the right operand, \lst{false} otherwise. See~\hyperref[sec:appendix:primops:GT]{\lst{>}}

\noindent
\(\begin{tabularx}{\textwidth}{| l | l | l | X |}
    \hline
    \bf{Slot} & \bf{Format} & \bf{\#bytes} & \bf{Description} \\
    \hline
        \multicolumn{4}{l}{\lst{match}~$ (left, right) $} \\
         
    \multicolumn{4}{l}{~~\lst{otherwise} } \\
    \hline
            ~~~~ $ left $ & \lst{Expr} & [1, *] & left operand \\
    \hline
          ~~~~ $ right $ & \lst{Expr} & [1, *] & right operand \\
    \hline
          \multicolumn{4}{l}{\lst{end match}} \\
\end{tabularx}\)
       

\subsubsection{\lst{GE} operation (OpCode 146)}
\label{sec:serialization:operation:GE}

Returns \lst{true} is the left operand is greater then or equal to the right operand, \lst{false} otherwise. See~\hyperref[sec:appendix:primops:GE]{\lst{>=}}

\noindent
\(\begin{tabularx}{\textwidth}{| l | l | l | X |}
    \hline
    \bf{Slot} & \bf{Format} & \bf{\#bytes} & \bf{Description} \\
    \hline
        \multicolumn{4}{l}{\lst{match}~$ (left, right) $} \\
         
    \multicolumn{4}{l}{~~\lst{otherwise} } \\
    \hline
            ~~~~ $ left $ & \lst{Expr} & [1, *] & left operand \\
    \hline
          ~~~~ $ right $ & \lst{Expr} & [1, *] & right operand \\
    \hline
          \multicolumn{4}{l}{\lst{end match}} \\
\end{tabularx}\)
       

\subsubsection{\lst{EQ} operation (OpCode 147)}
\label{sec:serialization:operation:EQ}

Compare equality of \lst{left} and \lst{right} arguments See~\hyperref[sec:appendix:primops:EQ]{\lst{==}}

\noindent
\(\begin{tabularx}{\textwidth}{| l | l | l | X |}
    \hline
    \bf{Slot} & \bf{Format} & \bf{\#bytes} & \bf{Description} \\
    \hline
        \multicolumn{4}{l}{\lst{match}~$ (left, right) $} \\
         
    \multicolumn{4}{l}{~~\lst{otherwise} } \\
    \hline
            ~~~~ $ left $ & \lst{Expr} & [1, *] & left operand \\
    \hline
          ~~~~ $ right $ & \lst{Expr} & [1, *] & right operand \\
    \hline
          \multicolumn{4}{l}{\lst{end match}} \\
\end{tabularx}\)
       

\subsubsection{\lst{NEQ} operation (OpCode 148)}
\label{sec:serialization:operation:NEQ}

Compare inequality of \lst{left} and \lst{right} arguments See~\hyperref[sec:appendix:primops:NEQ]{\lst{!=}}

\noindent
\(\begin{tabularx}{\textwidth}{| l | l | l | X |}
    \hline
    \bf{Slot} & \bf{Format} & \bf{\#bytes} & \bf{Description} \\
    \hline
        \multicolumn{4}{l}{\lst{match}~$ (left, right) $} \\
         
    \multicolumn{4}{l}{~~\lst{otherwise} } \\
    \hline
            ~~~~ $ left $ & \lst{Expr} & [1, *] & left operand \\
    \hline
          ~~~~ $ right $ & \lst{Expr} & [1, *] & right operand \\
    \hline
          \multicolumn{4}{l}{\lst{end match}} \\
\end{tabularx}\)
       

\subsubsection{\lst{If} operation (OpCode 149)}
\label{sec:serialization:operation:If}

Compute condition, if true then compute trueBranch else compute falseBranch See~\hyperref[sec:appendix:primops:If]{\lst{if}}

\noindent
\(\begin{tabularx}{\textwidth}{| l | l | l | X |}
    \hline
    \bf{Slot} & \bf{Format} & \bf{\#bytes} & \bf{Description} \\
    \hline
         $ condition $ & \lst{Expr} & [1, *] & condition expression \\
    \hline
           $ trueBranch $ & \lst{Expr} & [1, *] & expression to execute when \lst{condition == true} \\
    \hline
           $ falseBranch $ & \lst{Expr} & [1, *] & expression to execute when \lst{condition == false} \\
    \hline
      
\end{tabularx}\)
       

\subsubsection{\lst{AND} operation (OpCode 150)}
\label{sec:serialization:operation:AND}

Returns true if \emph{all} the elements in collection are \lst{true}. See~\hyperref[sec:appendix:primops:AND]{\lst{allOf}}

\noindent
\(\begin{tabularx}{\textwidth}{| l | l | l | X |}
    \hline
    \bf{Slot} & \bf{Format} & \bf{\#bytes} & \bf{Description} \\
    \hline
         $ conditions $ & \lst{Expr} & [1, *] & a collection of conditions \\
    \hline
      
\end{tabularx}\)
       

\subsubsection{\lst{OR} operation (OpCode 151)}
\label{sec:serialization:operation:OR}

Returns true if \emph{any} the elements in collection are \lst{true}. See~\hyperref[sec:appendix:primops:OR]{\lst{anyOf}}

\noindent
\(\begin{tabularx}{\textwidth}{| l | l | l | X |}
    \hline
    \bf{Slot} & \bf{Format} & \bf{\#bytes} & \bf{Description} \\
    \hline
         $ conditions $ & \lst{Expr} & [1, *] & a collection of conditions \\
    \hline
      
\end{tabularx}\)
       

\subsubsection{\lst{AtLeast} operation (OpCode 152)}
\label{sec:serialization:operation:AtLeast}

 Logical threshold.
 AtLeast has two inputs: integer \lst{bound} and \lst{children} same as in AND/OR.
 The result is true if at least \lst{bound} children are proven.
         See~\hyperref[sec:appendix:primops:AtLeast]{\lst{atLeast}}

\noindent
\(\begin{tabularx}{\textwidth}{| l | l | l | X |}
    \hline
    \bf{Slot} & \bf{Format} & \bf{\#bytes} & \bf{Description} \\
    \hline
         $ bound $ & \lst{Expr} & [1, *] & required minimum of proven children \\
    \hline
           $ children $ & \lst{Expr} & [1, *] & proposition to be proven/validated \\
    \hline
      
\end{tabularx}\)
       

\subsubsection{\lst{Minus} operation (OpCode 153)}
\label{sec:serialization:operation:Minus}

Returns a result of subtracting second numeric operand from the first. See~\hyperref[sec:appendix:primops:Minus]{\lst{-}}

\noindent
\(\begin{tabularx}{\textwidth}{| l | l | l | X |}
    \hline
    \bf{Slot} & \bf{Format} & \bf{\#bytes} & \bf{Description} \\
    \hline
         $ left $ & \lst{Expr} & [1, *] & left operand \\
    \hline
           $ right $ & \lst{Expr} & [1, *] & right operand \\
    \hline
      
\end{tabularx}\)
       

\subsubsection{\lst{Plus} operation (OpCode 154)}
\label{sec:serialization:operation:Plus}

Returns a sum of two numeric operands See~\hyperref[sec:appendix:primops:Plus]{\lst{+}}

\noindent
\(\begin{tabularx}{\textwidth}{| l | l | l | X |}
    \hline
    \bf{Slot} & \bf{Format} & \bf{\#bytes} & \bf{Description} \\
    \hline
         $ left $ & \lst{Expr} & [1, *] & left operand \\
    \hline
           $ right $ & \lst{Expr} & [1, *] & right operand \\
    \hline
      
\end{tabularx}\)
       

\subsubsection{\lst{Xor} operation (OpCode 155)}
\label{sec:serialization:operation:Xor}

Byte-wise XOR of two collections of bytes See~\hyperref[sec:appendix:primops:Xor]{\lst{binary_|}}

\noindent
\(\begin{tabularx}{\textwidth}{| l | l | l | X |}
    \hline
    \bf{Slot} & \bf{Format} & \bf{\#bytes} & \bf{Description} \\
    \hline
         $ left $ & \lst{Expr} & [1, *] & left operand \\
    \hline
           $ right $ & \lst{Expr} & [1, *] & right operand \\
    \hline
      
\end{tabularx}\)
       

\subsubsection{\lst{Multiply} operation (OpCode 156)}
\label{sec:serialization:operation:Multiply}

Returns a multiplication of two numeric operands See~\hyperref[sec:appendix:primops:Multiply]{\lst{*}}

\noindent
\(\begin{tabularx}{\textwidth}{| l | l | l | X |}
    \hline
    \bf{Slot} & \bf{Format} & \bf{\#bytes} & \bf{Description} \\
    \hline
         $ left $ & \lst{Expr} & [1, *] & left operand \\
    \hline
           $ right $ & \lst{Expr} & [1, *] & right operand \\
    \hline
      
\end{tabularx}\)
       

\subsubsection{\lst{Division} operation (OpCode 157)}
\label{sec:serialization:operation:Division}

Integer division of the first operand by the second operand. See~\hyperref[sec:appendix:primops:Division]{\lst{/}}

\noindent
\(\begin{tabularx}{\textwidth}{| l | l | l | X |}
    \hline
    \bf{Slot} & \bf{Format} & \bf{\#bytes} & \bf{Description} \\
    \hline
         $ left $ & \lst{Expr} & [1, *] & left operand \\
    \hline
           $ right $ & \lst{Expr} & [1, *] & right operand \\
    \hline
      
\end{tabularx}\)
       

\subsubsection{\lst{Modulo} operation (OpCode 158)}
\label{sec:serialization:operation:Modulo}

Reminder from division of the first operand by the second operand. See~\hyperref[sec:appendix:primops:Modulo]{\lst{\%}}

\noindent
\(\begin{tabularx}{\textwidth}{| l | l | l | X |}
    \hline
    \bf{Slot} & \bf{Format} & \bf{\#bytes} & \bf{Description} \\
    \hline
         $ left $ & \lst{Expr} & [1, *] & left operand \\
    \hline
           $ right $ & \lst{Expr} & [1, *] & right operand \\
    \hline
      
\end{tabularx}\)
       

\subsubsection{\lst{Exponentiate} operation (OpCode 159)}
\label{sec:serialization:operation:Exponentiate}

Exponentiate this \lst{GroupElement} to the given number. Returns this to the power of k See~\hyperref[sec:type:GroupElement:exp]{\lst{GroupElement.exp}}

\noindent
\(\begin{tabularx}{\textwidth}{| l | l | l | X |}
    \hline
    \bf{Slot} & \bf{Format} & \bf{\#bytes} & \bf{Description} \\
    \hline
         $ this $ & \lst{Expr} & [1, *] & this instance \\
    \hline
           $ k $ & \lst{Expr} & [1, *] & The power \\
    \hline
      
\end{tabularx}\)
       

\subsubsection{\lst{MultiplyGroup} operation (OpCode 160)}
\label{sec:serialization:operation:MultiplyGroup}

Group operation. See~\hyperref[sec:type:GroupElement:multiply]{\lst{GroupElement.multiply}}

\noindent
\(\begin{tabularx}{\textwidth}{| l | l | l | X |}
    \hline
    \bf{Slot} & \bf{Format} & \bf{\#bytes} & \bf{Description} \\
    \hline
         $ this $ & \lst{Expr} & [1, *] & this instance \\
    \hline
           $ other $ & \lst{Expr} & [1, *] & other element of the group \\
    \hline
      
\end{tabularx}\)
       

\subsubsection{\lst{Min} operation (OpCode 161)}
\label{sec:serialization:operation:Min}

Minimum value of two operands. See~\hyperref[sec:appendix:primops:Min]{\lst{min}}

\noindent
\(\begin{tabularx}{\textwidth}{| l | l | l | X |}
    \hline
    \bf{Slot} & \bf{Format} & \bf{\#bytes} & \bf{Description} \\
    \hline
         $ left $ & \lst{Expr} & [1, *] & left operand \\
    \hline
           $ right $ & \lst{Expr} & [1, *] & right operand \\
    \hline
      
\end{tabularx}\)
       

\subsubsection{\lst{Max} operation (OpCode 162)}
\label{sec:serialization:operation:Max}

Maximum value of two operands. See~\hyperref[sec:appendix:primops:Max]{\lst{max}}

\noindent
\(\begin{tabularx}{\textwidth}{| l | l | l | X |}
    \hline
    \bf{Slot} & \bf{Format} & \bf{\#bytes} & \bf{Description} \\
    \hline
         $ left $ & \lst{Expr} & [1, *] & left operand \\
    \hline
           $ right $ & \lst{Expr} & [1, *] & right operand \\
    \hline
      
\end{tabularx}\)
       

\subsubsection{\lst{MapCollection} operation (OpCode 173)}
\label{sec:serialization:operation:MapCollection}

 Builds a new collection by applying a function to all elements of this collection.
 Returns a new collection of type \lst{Coll[B]} resulting from applying the given function
 \lst{f} to each element of this collection and collecting the results.
         See~\hyperref[sec:type:SCollection:map]{\lst{SCollection.map}}

\noindent
\(\begin{tabularx}{\textwidth}{| l | l | l | X |}
    \hline
    \bf{Slot} & \bf{Format} & \bf{\#bytes} & \bf{Description} \\
    \hline
         $ this $ & \lst{Expr} & [1, *] & this instance \\
    \hline
           $ f $ & \lst{Expr} & [1, *] & the function to apply to each element \\
    \hline
      
\end{tabularx}\)
       

\subsubsection{\lst{Exists} operation (OpCode 174)}
\label{sec:serialization:operation:Exists}

Tests whether a predicate holds for at least one element of this collection.
Returns \lst{true} if the given predicate \lst{p} is satisfied by at least one element of this collection, otherwise \lst{false}
         See~\hyperref[sec:type:SCollection:exists]{\lst{SCollection.exists}}

\noindent
\(\begin{tabularx}{\textwidth}{| l | l | l | X |}
    \hline
    \bf{Slot} & \bf{Format} & \bf{\#bytes} & \bf{Description} \\
    \hline
         $ this $ & \lst{Expr} & [1, *] & this instance \\
    \hline
           $ p $ & \lst{Expr} & [1, *] & the predicate used to test elements \\
    \hline
      
\end{tabularx}\)
       

\subsubsection{\lst{ForAll} operation (OpCode 175)}
\label{sec:serialization:operation:ForAll}

Tests whether a predicate holds for all elements of this collection.
Returns \lst{true} if this collection is empty or the given predicate \lst{p}
holds for all elements of this collection, otherwise \lst{false}.
         See~\hyperref[sec:type:SCollection:forall]{\lst{SCollection.forall}}

\noindent
\(\begin{tabularx}{\textwidth}{| l | l | l | X |}
    \hline
    \bf{Slot} & \bf{Format} & \bf{\#bytes} & \bf{Description} \\
    \hline
         $ this $ & \lst{Expr} & [1, *] & this instance \\
    \hline
           $ p $ & \lst{Expr} & [1, *] & the predicate used to test elements \\
    \hline
      
\end{tabularx}\)
       

\subsubsection{\lst{Fold} operation (OpCode 176)}
\label{sec:serialization:operation:Fold}

Applies a binary operator to a start value and all elements of this collection, going left to right. See~\hyperref[sec:type:SCollection:fold]{\lst{SCollection.fold}}

\noindent
\(\begin{tabularx}{\textwidth}{| l | l | l | X |}
    \hline
    \bf{Slot} & \bf{Format} & \bf{\#bytes} & \bf{Description} \\
    \hline
         $ this $ & \lst{Expr} & [1, *] & this instance \\
    \hline
           $ zero $ & \lst{Expr} & [1, *] & a starting value \\
    \hline
           $ op $ & \lst{Expr} & [1, *] & the binary operator \\
    \hline
      
\end{tabularx}\)
       

\subsubsection{\lst{SizeOf} operation (OpCode 177)}
\label{sec:serialization:operation:SizeOf}

The size of the collection in elements. See~\hyperref[sec:type:SCollection:size]{\lst{SCollection.size}}

\noindent
\(\begin{tabularx}{\textwidth}{| l | l | l | X |}
    \hline
    \bf{Slot} & \bf{Format} & \bf{\#bytes} & \bf{Description} \\
    \hline
         $ this $ & \lst{Expr} & [1, *] & this instance \\
    \hline
      
\end{tabularx}\)
       

\subsubsection{\lst{ByIndex} operation (OpCode 178)}
\label{sec:serialization:operation:ByIndex}

Return the element of collection if \lst{index} is in range \lst{0 .. size-1} See~\hyperref[sec:type:SCollection:getOrElse]{\lst{SCollection.getOrElse}}

\noindent
\(\begin{tabularx}{\textwidth}{| l | l | l | X |}
    \hline
    \bf{Slot} & \bf{Format} & \bf{\#bytes} & \bf{Description} \\
    \hline
         $ this $ & \lst{Expr} & [1, *] & this instance \\
    \hline
           $ index $ & \lst{Expr} & [1, *] & index of the element of this collection \\
    \hline
          \multicolumn{4}{l}{\lst{optional}~$default$} \\
    \hline
    ~~$tag$ & \lst{Byte} & 1 & 0 - no value; 1 - has value \\
    \hline
    \multicolumn{4}{l}{~~\lst{when}~$tag == 1$} \\
    \hline
             ~~~~ $ default $ & \lst{Expr} & [1, *] & value to return when \lst{index} is out of range \\
    \hline
          \multicolumn{4}{l}{\lst{end optional}} \\
\end{tabularx}\)
       

\subsubsection{\lst{Append} operation (OpCode 179)}
\label{sec:serialization:operation:Append}

Puts the elements of other collection after the elements of this collection (concatenation of 2 collections) See~\hyperref[sec:type:SCollection:append]{\lst{SCollection.append}}

\noindent
\(\begin{tabularx}{\textwidth}{| l | l | l | X |}
    \hline
    \bf{Slot} & \bf{Format} & \bf{\#bytes} & \bf{Description} \\
    \hline
         $ this $ & \lst{Expr} & [1, *] & this instance \\
    \hline
           $ other $ & \lst{Expr} & [1, *] & the collection to append at the end of this \\
    \hline
      
\end{tabularx}\)
       

\subsubsection{\lst{Slice} operation (OpCode 180)}
\label{sec:serialization:operation:Slice}

Selects an interval of elements.  The returned collection is made up
  of all elements \lst{x} which satisfy the invariant:
  \lst{
     from <= indexOf(x) < until
  }
         See~\hyperref[sec:type:SCollection:slice]{\lst{SCollection.slice}}

\noindent
\(\begin{tabularx}{\textwidth}{| l | l | l | X |}
    \hline
    \bf{Slot} & \bf{Format} & \bf{\#bytes} & \bf{Description} \\
    \hline
         $ this $ & \lst{Expr} & [1, *] & this instance \\
    \hline
           $ from $ & \lst{Expr} & [1, *] & the lowest index to include from this collection \\
    \hline
           $ until $ & \lst{Expr} & [1, *] & the lowest index to EXCLUDE from this collection \\
    \hline
      
\end{tabularx}\)
       

\subsubsection{\lst{ExtractAmount} operation (OpCode 193)}
\label{sec:serialization:operation:ExtractAmount}

Mandatory: Monetary value, in Ergo tokens (NanoErg unit of measure) See~\hyperref[sec:type:Box:value]{\lst{Box.value}}

\noindent
\(\begin{tabularx}{\textwidth}{| l | l | l | X |}
    \hline
    \bf{Slot} & \bf{Format} & \bf{\#bytes} & \bf{Description} \\
    \hline
         $ this $ & \lst{Expr} & [1, *] & this instance \\
    \hline
      
\end{tabularx}\)
       

\subsubsection{\lst{ExtractScriptBytes} operation (OpCode 194)}
\label{sec:serialization:operation:ExtractScriptBytes}

Serialized bytes of guarding script, which should be evaluated to true in order to
 open this box. (aka spend it in a transaction) See~\hyperref[sec:type:Box:propositionBytes]{\lst{Box.propositionBytes}}

\noindent
\(\begin{tabularx}{\textwidth}{| l | l | l | X |}
    \hline
    \bf{Slot} & \bf{Format} & \bf{\#bytes} & \bf{Description} \\
    \hline
         $ this $ & \lst{Expr} & [1, *] & this instance \\
    \hline
      
\end{tabularx}\)
       

\subsubsection{\lst{ExtractBytes} operation (OpCode 195)}
\label{sec:serialization:operation:ExtractBytes}

Serialized bytes of this box's content, including proposition bytes. See~\hyperref[sec:type:Box:bytes]{\lst{Box.bytes}}

\noindent
\(\begin{tabularx}{\textwidth}{| l | l | l | X |}
    \hline
    \bf{Slot} & \bf{Format} & \bf{\#bytes} & \bf{Description} \\
    \hline
         $ this $ & \lst{Expr} & [1, *] & this instance \\
    \hline
      
\end{tabularx}\)
       

\subsubsection{\lst{ExtractBytesWithNoRef} operation (OpCode 196)}
\label{sec:serialization:operation:ExtractBytesWithNoRef}

Serialized bytes of this box's content, excluding transactionId and index of output. See~\hyperref[sec:type:Box:bytesWithoutRef]{\lst{Box.bytesWithoutRef}}

\noindent
\(\begin{tabularx}{\textwidth}{| l | l | l | X |}
    \hline
    \bf{Slot} & \bf{Format} & \bf{\#bytes} & \bf{Description} \\
    \hline
         $ this $ & \lst{Expr} & [1, *] & this instance \\
    \hline
      
\end{tabularx}\)
       

\subsubsection{\lst{ExtractId} operation (OpCode 197)}
\label{sec:serialization:operation:ExtractId}

Blake2b256 hash of this box's content, basically equals to \lst{blake2b256(bytes)} See~\hyperref[sec:type:Box:id]{\lst{Box.id}}

\noindent
\(\begin{tabularx}{\textwidth}{| l | l | l | X |}
    \hline
    \bf{Slot} & \bf{Format} & \bf{\#bytes} & \bf{Description} \\
    \hline
         $ this $ & \lst{Expr} & [1, *] & this instance \\
    \hline
      
\end{tabularx}\)
       

\subsubsection{\lst{ExtractRegisterAs} operation (OpCode 198)}
\label{sec:serialization:operation:ExtractRegisterAs}

 Extracts register by id and type.
 Type param \lst{T} expected type of the register.
 Returns \lst{Some(value)} if the register is defined and has given type and \lst{None} otherwise
         See~\hyperref[sec:type:Box:getReg]{\lst{Box.getReg}}

\noindent
\(\begin{tabularx}{\textwidth}{| l | l | l | X |}
    \hline
    \bf{Slot} & \bf{Format} & \bf{\#bytes} & \bf{Description} \\
    \hline
         $ this $ & \lst{Expr} & [1, *] & this instance \\
    \hline
           $ regId $ & \lst{Byte} & 1 & zero-based identifier of the register. \\
    \hline
           $ type $ & \lst{Type} & [1, *] & expected type of the value in register \\
    \hline
      
\end{tabularx}\)
       

\subsubsection{\lst{ExtractCreationInfo} operation (OpCode 199)}
\label{sec:serialization:operation:ExtractCreationInfo}

 If \lst{tx} is a transaction which generated this box, then \lst{creationInfo._1}
 is a height of the tx's block. The \lst{creationInfo._2} is a serialized transaction
 identifier followed by box index in the transaction outputs.
         See~\hyperref[sec:type:Box:creationInfo]{\lst{Box.creationInfo}}

\noindent
\(\begin{tabularx}{\textwidth}{| l | l | l | X |}
    \hline
    \bf{Slot} & \bf{Format} & \bf{\#bytes} & \bf{Description} \\
    \hline
         $ this $ & \lst{Expr} & [1, *] & this instance \\
    \hline
      
\end{tabularx}\)
       

\subsubsection{\lst{CalcBlake2b256} operation (OpCode 203)}
\label{sec:serialization:operation:CalcBlake2b256}

Calculate Blake2b hash from \lst{input} bytes. See~\hyperref[sec:appendix:primops:CalcBlake2b256]{\lst{blake2b256}}

\noindent
\(\begin{tabularx}{\textwidth}{| l | l | l | X |}
    \hline
    \bf{Slot} & \bf{Format} & \bf{\#bytes} & \bf{Description} \\
    \hline
         $ input $ & \lst{Expr} & [1, *] & collection of bytes \\
    \hline
      
\end{tabularx}\)
       

\subsubsection{\lst{CalcSha256} operation (OpCode 204)}
\label{sec:serialization:operation:CalcSha256}

Calculate Sha256 hash from \lst{input} bytes. See~\hyperref[sec:appendix:primops:CalcSha256]{\lst{sha256}}

\noindent
\(\begin{tabularx}{\textwidth}{| l | l | l | X |}
    \hline
    \bf{Slot} & \bf{Format} & \bf{\#bytes} & \bf{Description} \\
    \hline
         $ input $ & \lst{Expr} & [1, *] & collection of bytes \\
    \hline
      
\end{tabularx}\)
       

\subsubsection{\lst{CreateProveDlog} operation (OpCode 205)}
\label{sec:serialization:operation:CreateProveDlog}

ErgoTree operation to create a new \lst{SigmaProp} value representing public key
 of discrete logarithm signature protocol.
         See~\hyperref[sec:appendix:primops:CreateProveDlog]{\lst{proveDlog}}

\noindent
\(\begin{tabularx}{\textwidth}{| l | l | l | X |}
    \hline
    \bf{Slot} & \bf{Format} & \bf{\#bytes} & \bf{Description} \\
    \hline
         $ value $ & \lst{Expr} & [1, *] & element of elliptic curve group \\
    \hline
      
\end{tabularx}\)
       

\subsubsection{\lst{CreateProveDHTuple} operation (OpCode 206)}
\label{sec:serialization:operation:CreateProveDHTuple}

 ErgoTree operation to create a new SigmaProp value representing public key
 of Diffie Hellman signature protocol.
 Common input: (g,h,u,v)
         See~\hyperref[sec:appendix:primops:CreateProveDHTuple]{\lst{proveDHTuple}}

\noindent
\(\begin{tabularx}{\textwidth}{| l | l | l | X |}
    \hline
    \bf{Slot} & \bf{Format} & \bf{\#bytes} & \bf{Description} \\
    \hline
         $ g $ & \lst{Expr} & [1, *] &  \\
    \hline
           $ h $ & \lst{Expr} & [1, *] &  \\
    \hline
           $ u $ & \lst{Expr} & [1, *] &  \\
    \hline
           $ v $ & \lst{Expr} & [1, *] &  \\
    \hline
      
\end{tabularx}\)
       

\subsubsection{\lst{SigmaPropBytes} operation (OpCode 208)}
\label{sec:serialization:operation:SigmaPropBytes}

Serialized bytes of this sigma proposition taken as ErgoTree. See~\hyperref[sec:type:SigmaProp:propBytes]{\lst{SigmaProp.propBytes}}

\noindent
\(\begin{tabularx}{\textwidth}{| l | l | l | X |}
    \hline
    \bf{Slot} & \bf{Format} & \bf{\#bytes} & \bf{Description} \\
    \hline
         $ this $ & \lst{Expr} & [1, *] & this instance \\
    \hline
      
\end{tabularx}\)
       

\subsubsection{\lst{BoolToSigmaProp} operation (OpCode 209)}
\label{sec:serialization:operation:BoolToSigmaProp}

Embedding of \lst{Boolean} values to \lst{SigmaProp} values.
 As an example, this operation allows boolean experessions
 to be used as arguments of \lst{atLeast(..., sigmaProp(boolExpr), ...)} operation.
 During execution results to either \lst{TrueProp} or \lst{FalseProp} values of \lst{SigmaProp} type.
         See~\hyperref[sec:appendix:primops:BoolToSigmaProp]{\lst{sigmaProp}}

\noindent
\(\begin{tabularx}{\textwidth}{| l | l | l | X |}
    \hline
    \bf{Slot} & \bf{Format} & \bf{\#bytes} & \bf{Description} \\
    \hline
         $ condition $ & \lst{Expr} & [1, *] & boolean value to embed in SigmaProp value \\
    \hline
      
\end{tabularx}\)
       

\subsubsection{\lst{DeserializeContext} operation (OpCode 212)}
\label{sec:serialization:operation:DeserializeContext}

Extracts context variable as \lst{Coll[Byte]}, deserializes it to script
 and then executes this script in the current context.
 The original \lst{Coll[Byte]} of the script is available as \lst{getVar[Coll[Byte]](id)}.
 Type parameter \lst{V} result type of the deserialized script.
 Throws an exception if the actual script type doesn't conform to T.
 Returns a result of the script execution in the current context
         See~\hyperref[sec:appendix:primops:DeserializeContext]{\lst{executeFromVar}}

\noindent
\(\begin{tabularx}{\textwidth}{| l | l | l | X |}
    \hline
    \bf{Slot} & \bf{Format} & \bf{\#bytes} & \bf{Description} \\
    \hline
         $ type $ & \lst{Type} & [1, *] & expected type of the deserialized script \\
    \hline
           $ id $ & \lst{Byte} & 1 & identifier of the context variable \\
    \hline
      
\end{tabularx}\)
       

\subsubsection{\lst{DeserializeRegister} operation (OpCode 213)}
\label{sec:serialization:operation:DeserializeRegister}

Extracts SELF register as \lst{Coll[Byte]}, deserializes it to script
 and then executes this script in the current context.
 The original \lst{Coll[Byte]} of the script is available as \lst{SELF.getReg[Coll[Byte]](id)}.
 Type parameter \lst{T} result type of the deserialized script.
 Throws an exception if the actual script type doesn't conform to \lst{T}.
 Returns a result of the script execution in the current context
         See~\hyperref[sec:appendix:primops:DeserializeRegister]{\lst{executeFromSelfReg}}

\noindent
\(\begin{tabularx}{\textwidth}{| l | l | l | X |}
    \hline
    \bf{Slot} & \bf{Format} & \bf{\#bytes} & \bf{Description} \\
    \hline
         $ id $ & \lst{Byte} & 1 & identifier of the register \\
    \hline
           $ type $ & \lst{Type} & [1, *] & expected type of the deserialized script \\
    \hline
          \multicolumn{4}{l}{\lst{optional}~$default$} \\
    \hline
    ~~$tag$ & \lst{Byte} & 1 & 0 - no value; 1 - has value \\
    \hline
    \multicolumn{4}{l}{~~\lst{when}~$tag == 1$} \\
    \hline
             ~~~~ $ default $ & \lst{Expr} & [1, *] & optional default value, if register is not available \\
    \hline
          \multicolumn{4}{l}{\lst{end optional}} \\
\end{tabularx}\)
       

\subsubsection{\lst{ValDef} operation (OpCode 214)}
\label{sec:serialization:operation:ValDef}

 

\noindent
\(\begin{tabularx}{\textwidth}{| l | l | l | X |}
    \hline
    \bf{Slot} & \bf{Format} & \bf{\#bytes} & \bf{Description} \\
    \hline
    % skipped OptionalScope(type arguments, ArrayBuffer())

\end{tabularx}\)
       

\subsubsection{\lst{FunDef} operation (OpCode 215)}
\label{sec:serialization:operation:FunDef}

 

\noindent
\(\begin{tabularx}{\textwidth}{| l | l | l | X |}
    \hline
    \bf{Slot} & \bf{Format} & \bf{\#bytes} & \bf{Description} \\
    \hline
    % skipped OptionalScope(type arguments, ArrayBuffer())

\end{tabularx}\)
       

\subsubsection{\lst{BlockValue} operation (OpCode 216)}
\label{sec:serialization:operation:BlockValue}

 

\noindent
\(\begin{tabularx}{\textwidth}{| l | l | l | X |}
    \hline
    \bf{Slot} & \bf{Format} & \bf{\#bytes} & \bf{Description} \\
    \hline
         $ numItems $ & \lst{VLQ(UInt)} & [1, *] & number of block items \\
    \hline
          \multicolumn{4}{l}{\lst{for}~$i=1$~\lst{to}~$numItems$} \\
    \hline
             ~~ $ item_i $ & \lst{Expr} & [1, *] & block's item in i-th position \\
    \hline
          \multicolumn{4}{l}{\lst{end for}} \\\hline
     $ result $ & \lst{Expr} & [1, *] & result expression of the block \\
    \hline
      
\end{tabularx}\)
       

\subsubsection{\lst{FuncValue} operation (OpCode 217)}
\label{sec:serialization:operation:FuncValue}

 

\noindent
\(\begin{tabularx}{\textwidth}{| l | l | l | X |}
    \hline
    \bf{Slot} & \bf{Format} & \bf{\#bytes} & \bf{Description} \\
    \hline
         $ numArgs $ & \lst{VLQ(UInt)} & [1, *] & number of function arguments \\
    \hline
          \multicolumn{4}{l}{\lst{for}~$i=1$~\lst{to}~$numArgs$} \\
    \hline
             ~~ $ id_i $ & \lst{VLQ(UInt)} & [1, *] & identifier of the i-th argument \\
    \hline
          ~~ $ type_i $ & \lst{Type} & [1, *] & type of the i-th argument \\
    \hline
          \multicolumn{4}{l}{\lst{end for}} \\\hline
     $ body $ & \lst{Expr} & [1, *] & function body, which is parameterized by arguments \\
    \hline
      
\end{tabularx}\)
       

\subsubsection{\lst{Apply} operation (OpCode 218)}
\label{sec:serialization:operation:Apply}

Apply the function to the arguments.  See~\hyperref[sec:appendix:primops:Apply]{\lst{apply}}

\noindent
\(\begin{tabularx}{\textwidth}{| l | l | l | X |}
    \hline
    \bf{Slot} & \bf{Format} & \bf{\#bytes} & \bf{Description} \\
    \hline
         $ func $ & \lst{Expr} & [1, *] & function which is applied \\
    \hline
           $ \#items $ & \lst{VLQ(UInt)} & [1, *] & number of items in the collection \\
    \hline
          \multicolumn{4}{l}{\lst{for}~$i=1$~\lst{to}~$\#items$} \\
    \hline
             ~~ $ args_i $ & \lst{Expr} & [1, *] & i-th item in the list of arguments \\
    \hline
          \multicolumn{4}{l}{\lst{end for}} \\
\end{tabularx}\)
       

\subsubsection{\lst{PropertyCall} operation (OpCode 219)}
\label{sec:serialization:operation:PropertyCall}

 

\noindent
\(\begin{tabularx}{\textwidth}{| l | l | l | X |}
    \hline
    \bf{Slot} & \bf{Format} & \bf{\#bytes} & \bf{Description} \\
    \hline
         $ typeCode $ & \lst{Byte} & 1 & type of the method (see Table~\ref{table:predeftypes}) \\
    \hline
           $ methodCode $ & \lst{Byte} & 1 & a code of the property \\
    \hline
           $ obj $ & \lst{Expr} & [1, *] & receiver object of this property call \\
    \hline
      
\end{tabularx}\)
       

\subsubsection{\lst{MethodCall} operation (OpCode 220)}
\label{sec:serialization:operation:MethodCall}

 

\noindent
\(\begin{tabularx}{\textwidth}{| l | l | l | X |}
    \hline
    \bf{Slot} & \bf{Format} & \bf{\#bytes} & \bf{Description} \\
    \hline
         $ typeCode $ & \lst{Byte} & 1 & type of the method (see Table~\ref{table:predeftypes}) \\
    \hline
           $ methodCode $ & \lst{Byte} & 1 & a code of the method \\
    \hline
           $ obj $ & \lst{Expr} & [1, *] & receiver object of this method call \\
    \hline
           $ \#items $ & \lst{VLQ(UInt)} & [1, *] & number of items in the collection \\
    \hline
          \multicolumn{4}{l}{\lst{for}~$i=1$~\lst{to}~$\#items$} \\
    \hline
             ~~ $ args_i $ & \lst{Expr} & [1, *] & i-th item in the arguments of the method call \\
    \hline
          \multicolumn{4}{l}{\lst{end for}} \\
\end{tabularx}\)
       

\subsubsection{\lst{GetVar} operation (OpCode 227)}
\label{sec:serialization:operation:GetVar}

Get context variable with given \lst{varId} and type. See~\hyperref[sec:appendix:primops:GetVar]{\lst{getVar}}

\noindent
\(\begin{tabularx}{\textwidth}{| l | l | l | X |}
    \hline
    \bf{Slot} & \bf{Format} & \bf{\#bytes} & \bf{Description} \\
    \hline
         $ varId $ & \lst{Byte} & 1 & \lst{Byte} identifier of context variable \\
    \hline
           $ type $ & \lst{Type} & [1, *] & expected type of context variable \\
    \hline
      
\end{tabularx}\)
       

\subsubsection{\lst{OptionGet} operation (OpCode 228)}
\label{sec:serialization:operation:OptionGet}

Returns the option's value. The option must be nonempty. Throws exception if the option is empty. See~\hyperref[sec:type:SOption:get]{\lst{SOption.get}}

\noindent
\(\begin{tabularx}{\textwidth}{| l | l | l | X |}
    \hline
    \bf{Slot} & \bf{Format} & \bf{\#bytes} & \bf{Description} \\
    \hline
         $ this $ & \lst{Expr} & [1, *] & this instance \\
    \hline
      
\end{tabularx}\)
       

\subsubsection{\lst{OptionGetOrElse} operation (OpCode 229)}
\label{sec:serialization:operation:OptionGetOrElse}

Returns the option's value if the option is nonempty, otherwise
return the result of evaluating \lst{default}.
         See~\hyperref[sec:type:SOption:getOrElse]{\lst{SOption.getOrElse}}

\noindent
\(\begin{tabularx}{\textwidth}{| l | l | l | X |}
    \hline
    \bf{Slot} & \bf{Format} & \bf{\#bytes} & \bf{Description} \\
    \hline
         $ this $ & \lst{Expr} & [1, *] & this instance \\
    \hline
           $ default $ & \lst{Expr} & [1, *] & the default value \\
    \hline
      
\end{tabularx}\)
       

\subsubsection{\lst{OptionIsDefined} operation (OpCode 230)}
\label{sec:serialization:operation:OptionIsDefined}

Returns \lst{true} if the option is an instance of \lst{Some}, \lst{false} otherwise. See~\hyperref[sec:type:SOption:isDefined]{\lst{SOption.isDefined}}

\noindent
\(\begin{tabularx}{\textwidth}{| l | l | l | X |}
    \hline
    \bf{Slot} & \bf{Format} & \bf{\#bytes} & \bf{Description} \\
    \hline
         $ this $ & \lst{Expr} & [1, *] & this instance \\
    \hline
      
\end{tabularx}\)
       

\subsubsection{\lst{SigmaAnd} operation (OpCode 234)}
\label{sec:serialization:operation:SigmaAnd}

Returns sigma proposition which is proven when \emph{all} the elements in collection are proven. See~\hyperref[sec:appendix:primops:SigmaAnd]{\lst{allZK}}

\noindent
\(\begin{tabularx}{\textwidth}{| l | l | l | X |}
    \hline
    \bf{Slot} & \bf{Format} & \bf{\#bytes} & \bf{Description} \\
    \hline
         $ \#items $ & \lst{VLQ(UInt)} & [1, *] & number of items in the collection \\
    \hline
          \multicolumn{4}{l}{\lst{for}~$i=1$~\lst{to}~$\#items$} \\
    \hline
             ~~ $ propositions_i $ & \lst{Expr} & [1, *] & i-th item in the a collection of propositions \\
    \hline
          \multicolumn{4}{l}{\lst{end for}} \\
\end{tabularx}\)
       

\subsubsection{\lst{SigmaOr} operation (OpCode 235)}
\label{sec:serialization:operation:SigmaOr}

Returns sigma proposition which is proven when \emph{any} of the elements in collection is proven. See~\hyperref[sec:appendix:primops:SigmaOr]{\lst{anyZK}}

\noindent
\(\begin{tabularx}{\textwidth}{| l | l | l | X |}
    \hline
    \bf{Slot} & \bf{Format} & \bf{\#bytes} & \bf{Description} \\
    \hline
         $ \#items $ & \lst{VLQ(UInt)} & [1, *] & number of items in the collection \\
    \hline
          \multicolumn{4}{l}{\lst{for}~$i=1$~\lst{to}~$\#items$} \\
    \hline
             ~~ $ propositions_i $ & \lst{Expr} & [1, *] & i-th item in the a collection of propositions \\
    \hline
          \multicolumn{4}{l}{\lst{end for}} \\
\end{tabularx}\)
       

\subsubsection{\lst{BinOr} operation (OpCode 236)}
\label{sec:serialization:operation:BinOr}

Logical OR of two operands See~\hyperref[sec:appendix:primops:BinOr]{\lst{||}}

\noindent
\(\begin{tabularx}{\textwidth}{| l | l | l | X |}
    \hline
    \bf{Slot} & \bf{Format} & \bf{\#bytes} & \bf{Description} \\
    \hline
        \multicolumn{4}{l}{\lst{match}~$ (left, right) $} \\
         
    \multicolumn{4}{l}{~~\lst{otherwise} } \\
    \hline
            ~~~~ $ left $ & \lst{Expr} & [1, *] & left operand \\
    \hline
          ~~~~ $ right $ & \lst{Expr} & [1, *] & right operand \\
    \hline
          \multicolumn{4}{l}{\lst{end match}} \\
\end{tabularx}\)
       

\subsubsection{\lst{BinAnd} operation (OpCode 237)}
\label{sec:serialization:operation:BinAnd}

Logical AND of two operands See~\hyperref[sec:appendix:primops:BinAnd]{\lst{&&}}

\noindent
\(\begin{tabularx}{\textwidth}{| l | l | l | X |}
    \hline
    \bf{Slot} & \bf{Format} & \bf{\#bytes} & \bf{Description} \\
    \hline
        \multicolumn{4}{l}{\lst{match}~$ (left, right) $} \\
         
    \multicolumn{4}{l}{~~\lst{otherwise} } \\
    \hline
            ~~~~ $ left $ & \lst{Expr} & [1, *] & left operand \\
    \hline
          ~~~~ $ right $ & \lst{Expr} & [1, *] & right operand \\
    \hline
          \multicolumn{4}{l}{\lst{end match}} \\
\end{tabularx}\)
       

\subsubsection{\lst{DecodePoint} operation (OpCode 238)}
\label{sec:serialization:operation:DecodePoint}

Convert \lst{Coll[Byte]} to \lst{GroupElement} using \lst{GroupElementSerializer} See~\hyperref[sec:appendix:primops:DecodePoint]{\lst{decodePoint}}

\noindent
\(\begin{tabularx}{\textwidth}{| l | l | l | X |}
    \hline
    \bf{Slot} & \bf{Format} & \bf{\#bytes} & \bf{Description} \\
    \hline
         $ input $ & \lst{Expr} & [1, *] & serialized bytes of some \lst{GroupElement} value \\
    \hline
      
\end{tabularx}\)
       

\subsubsection{\lst{LogicalNot} operation (OpCode 239)}
\label{sec:serialization:operation:LogicalNot}

Logical NOT operation. Returns \lst{true} if input is \lst{false} and \lst{false} if input is \lst{true}. See~\hyperref[sec:appendix:primops:LogicalNot]{\lst{unary_!}}

\noindent
\(\begin{tabularx}{\textwidth}{| l | l | l | X |}
    \hline
    \bf{Slot} & \bf{Format} & \bf{\#bytes} & \bf{Description} \\
    \hline
         $ input $ & \lst{Expr} & [1, *] & input \lst{Boolean} value \\
    \hline
      
\end{tabularx}\)
       

\subsubsection{\lst{Negation} operation (OpCode 240)}
\label{sec:serialization:operation:Negation}

Negates numeric value \lst{x} by returning \lst{-x}. See~\hyperref[sec:appendix:primops:Negation]{\lst{unary_-}}

\noindent
\(\begin{tabularx}{\textwidth}{| l | l | l | X |}
    \hline
    \bf{Slot} & \bf{Format} & \bf{\#bytes} & \bf{Description} \\
    \hline
         $ input $ & \lst{Expr} & [1, *] & value of numeric type \\
    \hline
      
\end{tabularx}\)
       

\subsubsection{\lst{BinXor} operation (OpCode 244)}
\label{sec:serialization:operation:BinXor}

Logical XOR of two operands See~\hyperref[sec:appendix:primops:BinXor]{\lst{^}}

\noindent
\(\begin{tabularx}{\textwidth}{| l | l | l | X |}
    \hline
    \bf{Slot} & \bf{Format} & \bf{\#bytes} & \bf{Description} \\
    \hline
        \multicolumn{4}{l}{\lst{match}~$ (left, right) $} \\
         
    \multicolumn{4}{l}{~~\lst{otherwise} } \\
    \hline
            ~~~~ $ left $ & \lst{Expr} & [1, *] & left operand \\
    \hline
          ~~~~ $ right $ & \lst{Expr} & [1, *] & right operand \\
    \hline
          \multicolumn{4}{l}{\lst{end match}} \\
\end{tabularx}\)
       

\subsubsection{\lst{XorOf} operation (OpCode 255)}
\label{sec:serialization:operation:XorOf}

Similar to \lst{allOf}, but performing logical XOR operation between all conditions instead of \lst{&&} See~\hyperref[sec:appendix:primops:XorOf]{\lst{xorOf}}

\noindent
\(\begin{tabularx}{\textwidth}{| l | l | l | X |}
    \hline
    \bf{Slot} & \bf{Format} & \bf{\#bytes} & \bf{Description} \\
    \hline
         $ conditions $ & \lst{Expr} & [1, *] & a collection of conditions \\
    \hline
      
\end{tabularx}\)
       

\subsubsection{\lst{SubstConstants} operation (OpCode 116)}
\label{sec:serialization:operation:SubstConstants}

Transforms serialized bytes of ErgoTree with segregated constants by replacing constants
 at given positions with new values. This operation allow to use serialized scripts as
 pre-defined templates.
 The typical usage is "check that output box have proposition equal to given script bytes,
 where minerPk (constants(0)) is replaced with currentMinerPk".
 Each constant in original scriptBytes have SType serialized before actual data (see ConstantSerializer).
 During substitution each value from newValues is checked to be an instance of the corresponding type.
 This means, the constants during substitution cannot change their types.

 Returns original scriptBytes array where only specified constants are replaced and all other bytes remain exactly the same.
         See~\hyperref[sec:appendix:primops:SubstConstants]{\lst{substConstants}}

\noindent
\(\begin{tabularx}{\textwidth}{| l | l | l | X |}
    \hline
    \bf{Slot} & \bf{Format} & \bf{\#bytes} & \bf{Description} \\
    \hline
         $ scriptBytes $ & \lst{Expr} & [1, *] & serialized ErgoTree with ConstantSegregationFlag set to 1. \\
    \hline
           $ positions $ & \lst{Expr} & [1, *] & zero based indexes in ErgoTree.constants array which should be replaced with new values \\
    \hline
           $ newValues $ & \lst{Expr} & [1, *] & new values to be injected into the corresponding positions in ErgoTree.constants array \\
    \hline
      
\end{tabularx}\)
       

\subsubsection{\lst{LongToByteArray} operation (OpCode 122)}
\label{sec:serialization:operation:LongToByteArray}

Converts \lst{Long} value to big-endian bytes representation. See~\hyperref[sec:appendix:primops:LongToByteArray]{\lst{longToByteArray}}

\noindent
\(\begin{tabularx}{\textwidth}{| l | l | l | X |}
    \hline
    \bf{Slot} & \bf{Format} & \bf{\#bytes} & \bf{Description} \\
    \hline
         $ input $ & \lst{Expr} & [1, *] & value to convert \\
    \hline
      
\end{tabularx}\)
       

\subsubsection{\lst{ByteArrayToBigInt} operation (OpCode 123)}
\label{sec:serialization:operation:ByteArrayToBigInt}

Convert big-endian bytes representation (Coll[Byte]) to BigInt value. See~\hyperref[sec:appendix:primops:ByteArrayToBigInt]{\lst{byteArrayToBigInt}}

\noindent
\(\begin{tabularx}{\textwidth}{| l | l | l | X |}
    \hline
    \bf{Slot} & \bf{Format} & \bf{\#bytes} & \bf{Description} \\
    \hline
         $ input $ & \lst{Expr} & [1, *] & collection of bytes in big-endian format \\
    \hline
      
\end{tabularx}\)
       

\subsubsection{\lst{ByteArrayToLong} operation (OpCode 124)}
\label{sec:serialization:operation:ByteArrayToLong}

Convert big-endian bytes representation (Coll[Byte]) to Long value. See~\hyperref[sec:appendix:primops:ByteArrayToLong]{\lst{byteArrayToLong}}

\noindent
\(\begin{tabularx}{\textwidth}{| l | l | l | X |}
    \hline
    \bf{Slot} & \bf{Format} & \bf{\#bytes} & \bf{Description} \\
    \hline
         $ input $ & \lst{Expr} & [1, *] & collection of bytes in big-endian format \\
    \hline
      
\end{tabularx}\)
       

\subsubsection{\lst{Downcast} operation (OpCode 125)}
\label{sec:serialization:operation:Downcast}

Cast this numeric value to a smaller type (e.g. Long to Int). Throws exception if overflow. See~\hyperref[sec:appendix:primops:Downcast]{\lst{downcast}}

\noindent
\(\begin{tabularx}{\textwidth}{| l | l | l | X |}
    \hline
    \bf{Slot} & \bf{Format} & \bf{\#bytes} & \bf{Description} \\
    \hline
         $ input $ & \lst{Expr} & [1, *] & value to cast \\
    \hline
           $ type $ & \lst{Type} & [1, *] & resulting type of the cast operation \\
    \hline
      
\end{tabularx}\)
       

\subsubsection{\lst{Upcast} operation (OpCode 126)}
\label{sec:serialization:operation:Upcast}

Cast this numeric value to a bigger type (e.g. Int to Long) See~\hyperref[sec:appendix:primops:Upcast]{\lst{upcast}}

\noindent
\(\begin{tabularx}{\textwidth}{| l | l | l | X |}
    \hline
    \bf{Slot} & \bf{Format} & \bf{\#bytes} & \bf{Description} \\
    \hline
         $ input $ & \lst{Expr} & [1, *] & value to cast \\
    \hline
           $ type $ & \lst{Type} & [1, *] & resulting type of the cast operation \\
    \hline
      
\end{tabularx}\)
       

\section{Motivations}
\label{sec:appendix:motivation}

\subsection{Type Serialization format rationale}
\label{sec:appendix:motivation:type}

\langname types terms are serialized using special encoding designed for compact
storage yet fast deserialization. In this section we describe the motivation.

Some operations of \ASDag have type parameters, for which concrete types should be
specified (since \ASDag is monomorphic IR). When the operation (such as
\hyperref[sec:serialization:operation:ExtractRegisterAs]{\lst{ExtractRegisterAs}}) is
serialized those type parameters should also be serialized as part of the operation.
The following encoding is designed to minimize a number of bytes required to represent
type in the serialization format of \ASDag. Since most of the scripts will use simple
types so we want them the take a single byte of the storage.

In the intermediate representation of ErgoTree each type is represented by a tree of
nodes where leaves are primitive types and other nodes are type constructors. Simple
(but sub-optimal) way to serialize a type would be to give each primitive type and each
type constructor a unique type code. Then, to serialize a node, we whould need to emit
its code and then perform recursive descent to serialize all children.

However, to save storage space, we use special encoding schema to save bytes
for the types that are used more often.

We assume the most frequently used types are:
\begin{itemize}
    \item primitive types (\lst{Boolean}, \lst{Byte}, \lst{Short}, \lst{Int}, \lst{Long}, \lst{BigInt}, \lst{GroupElement},
    \lst{SigmaProp}, \lst{Box}, \lst{AvlTree})
    \item  Collections of primitive types (\lst{Coll[Byte]} etc)
    \item  Options of primitive types (\lst{Option[Int]} etc.)
    \item Nested arrays of primitive types (\lst{Coll[Coll[Int]]} etc.)
    \item Functions of primitive types (\lst{Box => Boolean} etc.)
    \item First biased pair of types (\lst{(_, Int)} when we know the first
    component is a primitive type).
    \item Second biased pair of types (\lst{(Int, _)} when we know the second
    component is a primitive type)
    \item Symmetric pair of types (\lst{(Int, Int)} when we know both types are
    the same)
\end{itemize}

All the types above should be represented in an optimized way preferably by a single
byte (see examples in Figure~\ref{fig:ser:type:examples}). For other types, we do
recursive descent down the type tree as it is defined in
section~\ref{sec:ser:type:recursive}.

\subsection{Constant Segregation rationale}

\subsubsection{Massive script validation}

Consider a transaction \lst{tx} which have \lst{INPUTS} collection of boxes to
spend. Every input box can have a script protecting it (\lst{propostionBytes}
property). This script should be executed in a context of the current
transaction. The simplest transaction have 1 input box. Thus if we want to
have a sustained block validation of 1000 transactions per second we need to
be able to validate 1000 scripts per second.

For every script (of input \lst{box}) the following is done in order to
validate it:
\begin{enumerate}
    \item Context is created with \lst{SELF} = box
    \item The script is deserialized into ErgoTree 
    \item ErgoTree is traversed to build costGraph and calcGraph, two graphs for
    cost estimation function and script calculation function.
    \item Cost estimation is computed by evaluating costGraph with current context data
    \item If cost and data size limits are not exceeded, calcGraph is
    evaluated using context data to obtain sigma proposition (see
    \hyperref[sec:type:SigmaProp]{\lst{SigmaProp}})
    \item Verification procedure is executed
\end{enumerate}

\subsubsection{Potential for Script processing optimization}

Before an \langname contract can be stored in a blockchain it should be first
compiled from its source text into ErgoTree and then serialized into byte
array.

Because the language is purely functional and IR is graph-based, the
compilation process has an effect of normalization/unification. This means
that different original scripts may have identical ErgoTrees and as the
result identical serialized bytes.

Because of normalization, and also because of script reusability, the number
of conceptually (or logically) different scripts is much less than the number
of individual scripts in a blockchain. For example we may have 1000s of
different scripts in a blockchain with millions of boxes.

The average reusability ratio is 1000 in this case. And even those different
scripts may have different usage frequency. Having big reusability ratio we
can optimize script evaluation by performing steps 1 - 4 only once per unique
script.

The compiled calcGraph can be cached in \lst{Map[Array[Byte], Context =>
SigmaBoolean]}. Every script extracted from an input box can be used as a key
in this map to obtain ready to execute graph.

However, we have a problem with constants embedded in contracts. There is one
obstacle to the optimization by caching. In many cases it is very natural to
embed constants in the script body, most notable scenario is when public keys
are embedded. As result two functionally identical scripts may serialize to
different byte arrays because they have different embedded constants.

\subsubsection{Constant-less ErgoTree}

The solution to the problem with embedded constants is simple, we don't need
to embed constants. Each constant in the body of \ASDag can be replaced
with indexed placeholder (see \hyperref[sec:appendix:primops:ConstantPlaceholder]{\lst{ConstantPlaceholder}}).
Each placeholder have an index field. The index of the placeholder is
assigned by breadth-first topological order of the graph traversal.

The transformation is part of compilation and is performed ahead of time.
Each \ASDag have an array of all the constants extracted from its body. Each
placeholder refers to the constant by the constant's index in the array.

Thus the format of serialized script is shown in Figure~\ref{fig:ser:ergotree} which contains:
\begin{enumerate}
    \item number of constants
    \item constants collection
    \item script expression with placeholders
\end{enumerate}

The constants collection contains serialized constant data (using
ConstantSerializer) one after another.
The script expression is a serialized ErgoTree with placeholders.

Using this new script format we can use script expression part as a key in
the cache. An observation is that after the constants are extracted, what
remains is a template. Thus instead of applying steps 1-4 to
\emph{constant-full} scripts we can apply them to \emph{constant-less}
templates. Before applying steps 4 and 5 we need to bind placeholders with
actual values taken from the cconstants collection.

\section{Compressed encoding of integer values}

\subsection{VLQ encoding}
\label{sec:vlq-encoding}

\begin{verbatim}
public final void putULong(long value) {
    while (true) {
        if ((value & ~0x7FL) == 0) {
            buffer[position++] = (byte) value;
            return;
        } else {
            buffer[position++] = (byte) (((int) value & 0x7F) | 0x80);
            value >>>= 7;
        }
    }
}
\end{verbatim}

\subsection{ZigZag encoding}
\label{sec:zigzag-encoding}

Encode a ZigZag-encoded 64-bit value.  ZigZag encodes signed integers
into values that can be efficiently encoded with varint.  (Otherwise,
negative values must be sign-extended to 64 bits to be varint encoded,
thus always taking 10 bytes in the buffer.

Parameter \lst{n} is a signed 64-bit integer.
This Java method returns an unsigned 64-bit integer, stored in a signed int because Java has no explicit unsigned support.

\begin{verbatim}
 public static long encodeZigZag64(final long n) {
   // Note:  the right-shift must be arithmetic
   return (n << 1) ^ (n >> 63);
 }    
\end{verbatim}

\end{document}