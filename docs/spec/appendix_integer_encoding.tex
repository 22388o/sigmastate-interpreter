\section{Compressed encoding of integer values}

\subsection{VLQ encoding}
\label{sec:vlq-encoding}

\begin{verbatim}
public final void putULong(long value) {
    while (true) {
        if ((value & ~0x7FL) == 0) {
            buffer[position++] = (byte) value;
            return;
        } else {
            buffer[position++] = (byte) (((int) value & 0x7F) | 0x80);
            value >>>= 7;
        }
    }
}
\end{verbatim}

\subsection{ZigZag encoding}
\label{sec:zigzag-encoding}

Encode a ZigZag-encoded 64-bit value.  ZigZag encodes signed integers
into values that can be efficiently encoded with varint.  (Otherwise,
negative values must be sign-extended to 64 bits to be varint encoded,
thus always taking 10 bytes in the buffer.

Parameter \lst{n} is a signed 64-bit integer.
This Java method returns an unsigned 64-bit integer, stored in a signed int because Java has no explicit unsigned support.

\begin{verbatim}
 public static long encodeZigZag64(final long n) {
   // Note:  the right-shift must be arithmetic
   return (n << 1) ^ (n >> 63);
 }    
\end{verbatim}