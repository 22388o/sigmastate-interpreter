\section{Motivations}
\label{sec:appendix:motivation}

\subsection{Type Serialization format rationale}
\label{sec:appendix:motivation:type}

\langname types terms are serialized using special encoding designed for compact
storage yet fast deserialization. In this section we describe the motivation.

Some operations of \ASDag have type parameters, for which concrete types should be
specified (since \ASDag is monomorphic IR). When the operation (such as
\hyperref[sec:serialization:operation:ExtractRegisterAs]{\lst{ExtractRegisterAs}}) is
serialized those type parameters should also be serialized as part of the operation.
The following encoding is designed to minimize a number of bytes required to represent
type in the serialization format of \ASDag. Since most of the scripts will use simple
types so we want them the take a single byte of the storage.

In the intermediate representation of ErgoTree each type is represented by a tree of
nodes where leaves are primitive types and other nodes are type constructors. Simple
(but sub-optimal) way to serialize a type would be to give each primitive type and each
type constructor a unique type code. Then, to serialize a node, we whould need to emit
its code and then perform recursive descent to serialize all children.

However, to save storage space, we use special encoding schema to save bytes
for the types that are used more often.

We assume the most frequently used types are:
\begin{itemize}
    \item primitive types (\lst{Boolean}, \lst{Byte}, \lst{Short}, \lst{Int}, \lst{Long}, \lst{BigInt}, \lst{GroupElement},
    \lst{SigmaProp}, \lst{Box}, \lst{AvlTree})
    \item  Collections of primitive types (\lst{Coll[Byte]} etc)
    \item  Options of primitive types (\lst{Option[Int]} etc.)
    \item Nested arrays of primitive types (\lst{Coll[Coll[Int]]} etc.)
    \item Functions of primitive types (\lst{Box => Boolean} etc.)
    \item First biased pair of types (\lst{(_, Int)} when we know the first
    component is a primitive type).
    \item Second biased pair of types (\lst{(Int, _)} when we know the second
    component is a primitive type)
    \item Symmetric pair of types (\lst{(Int, Int)} when we know both types are
    the same)
\end{itemize}

All the types above should be represented in an optimized way preferably by a single
byte (see examples in Figure~\ref{fig:ser:type:examples}). For other types, we do
recursive descent down the type tree as it is defined in
section~\ref{sec:ser:type:recursive}.

\subsection{Constant Segregation rationale}

\subsubsection{Massive script validation}

Consider a transaction \lst{tx} which have \lst{INPUTS} collection of boxes to
spend. Every input box can have a script protecting it (\lst{propostionBytes}
property). This script should be executed in a context of the current
transaction. The simplest transaction have 1 input box. Thus if we want to
have a sustained block validation of 1000 transactions per second we need to
be able to validate 1000 scripts per second.

For every script (of input \lst{box}) the following is done in order to
validate it:
\begin{enumerate}
    \item Context is created with \lst{SELF} = box
    \item The script is deserialized into ErgoTree 
    \item ErgoTree is traversed to build costGraph and calcGraph, two graphs for
    cost estimation function and script calculation function.
    \item Cost estimation is computed by evaluating costGraph with current context data
    \item If cost and data size limits are not exceeded, calcGraph is
    evaluated using context data to obtain sigma proposition (see
    \hyperref[sec:type:SigmaProp]{\lst{SigmaProp}})
    \item Verification procedure is executed
\end{enumerate}

\subsubsection{Potential for Script processing optimization}

Before an \langname contract can be stored in a blockchain it should be first
compiled from its source text into ErgoTree and then serialized into byte
array.

Because the language is purely functional and IR is graph-based, the
compilation process has an effect of normalization/unification. This means
that different original scripts may have identical ErgoTrees and as the
result identical serialized bytes.

Because of normalization, and also because of script reusability, the number
of conceptually (or logically) different scripts is much less than the number
of individual scripts in a blockchain. For example we may have 1000s of
different scripts in a blockchain with millions of boxes.

The average reusability ratio is 1000 in this case. And even those different
scripts may have different usage frequency. Having big reusability ratio we
can optimize script evaluation by performing steps 1 - 4 only once per unique
script.

The compiled calcGraph can be cached in \lst{Map[Array[Byte], Context =>
SigmaBoolean]}. Every script extracted from an input box can be used as a key
in this map to obtain ready to execute graph.

However, we have a problem with constants embedded in contracts. There is one
obstacle to the optimization by caching. In many cases it is very natural to
embed constants in the script body, most notable scenario is when public keys
are embedded. As result two functionally identical scripts may serialize to
different byte arrays because they have different embedded constants.

\subsubsection{Constant-less ErgoTree}

The solution to the problem with embedded constants is simple, we don't need
to embed constants. Each constant in the body of \ASDag can be replaced
with indexed placeholder (see \hyperref[sec:appendix:primops:ConstantPlaceholder]{\lst{ConstantPlaceholder}}).
Each placeholder have an index field. The index of the placeholder is
assigned by breadth-first topological order of the graph traversal.

The transformation is part of compilation and is performed ahead of time.
Each \ASDag have an array of all the constants extracted from its body. Each
placeholder refers to the constant by the constant's index in the array.

Thus the format of serialized script is shown in Figure~\ref{fig:ser:ergotree} which contains:
\begin{enumerate}
    \item number of constants
    \item constants collection
    \item script expression with placeholders
\end{enumerate}

The constants collection contains serialized constant data (using
ConstantSerializer) one after another.
The script expression is a serialized ErgoTree with placeholders.

Using this new script format we can use script expression part as a key in
the cache. An observation is that after the constants are extracted, what
remains is a template. Thus instead of applying steps 1-4 to
\emph{constant-full} scripts we can apply them to \emph{constant-less}
templates. Before applying steps 4 and 5 we need to bind placeholders with
actual values taken from the cconstants collection.
