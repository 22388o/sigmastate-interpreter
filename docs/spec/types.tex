\section{Typing}
\label{sec:typing}

\langname is a strictly typed language, in which every term should have a
type in order to be wellformed and evaluated. Typing judgement of the form
$\Der{\Gamma}{e : T}$ say that $e$ is a term of type $T$ in the typing
context $\Gamma$.

\begin{figure}[h]

\begin{center}
% var and consts
\(\begin{array}{c c c}
\frac{}{\Der{\Gamma,x : \tau}{x : \tau}}
 	 & 
\frac{}{\Der{\Gamma}{l : \text{\lst{Int}}}}
     &
\frac{}{\Der{\Gamma}{() : \text{\lst{Unit}}}}
	 \\
	 & & \\
\end{array}\) 

% primitive
\(
\frac{\oplus : (\tau_1\times\tau_2) \to \tau_3~~~\DerEnv{e_1 : \tau_1}~~~\DerEnv{e_2 : \tau_2}}{\DerEnv{e_1 \oplus e_2 : \tau_3}}
\)

% pairs
\(\begin{array}{c c c}
 & \\
\frac{\DerEnv{e : \TPair{\tau_1}{\tau_2}}}{\DerEnv{\Fst{e} : \tau_1}}
     &
\frac{\DerEnv{e : \TPair{\tau_1}{\tau_2}}}{\DerEnv{\Snd{e} : \tau_2}}
     &
\frac{\DerEnv{e_1 : \tau_1}~~~\DerEnv{e_2 : \tau_2}}{\DerEnv{\Tup{e_1,e_2} : \TPair{\tau_1}{\tau_2}}}
	 \\
\end{array}\) 

% sums
\(\begin{array}{c c}
 & \\
\frac{\DerEnv{e : \tau_1}}{\DerEnv{\Left{\tau_1}{\tau_2}{e} : \tau_1 + \tau_2}}
     &
\frac{\DerEnv{e : \tau_2}}{\DerEnv{\Right{\tau_1}{\tau_2}{e} : \tau_1 + \tau_2}}
	 \\
	& \\ 
\end{array}\) 

% case
\(\begin{array}{c}
\frac{\DerEnv{e : \tau_1 + \tau_2}~~~\Der{\Gamma,x_1 : \tau_1}{e_1 : \tau}~~~\Der{\Gamma,x_2 : \tau_2}{e_2 : \tau}}
     {\DerEnv{\CaseOfXX{e}{\Left{\tau_1}{\tau_2}{x_1} \to e_1;~~\Right{\tau_1}{\tau_2}{x_2} \to e_2} : \tau}} \\
  \\
\end{array}\) 

% case
\(\begin{array}{c}
\frac{\DerEnv{e : \text{\lst{Int}}}~~~\DerEnv{e_i : \tau}}
     {\DerEnv{\CaseOf{e}{l_i}{e_i} : \tau}} \\
  \\
\end{array}\)

% let
% \(
% \frac{\Der{\TEnv,x : \tau_1}{e_2 : \tau_2}}{\Der{\Gamma}{\Let{x}{e_1}{e_2} : \tau_2}}
% \)

% functions
\(\begin{array}{c c}
\frac{\Der{\TEnv,x:\tau_1}{e:\tau_2}}{\Der{\Gamma}{\TyLam{x}{\tau_1}{e} : \tau_1 \to \tau_2}}
 	 & 
\frac{\Der{\TEnv}{e_1 : \tau_2 \to \tau}~~~\Der{\TEnv}{e_2 : \tau_2}}{\Der{\Gamma}{e_1~e_2 : \tau}}
	 \\
\end{array}\) 

\end{center}



\caption{Typing rules of \langname}
\label{fig:typing}
\end{figure}

Note that each well-typed term has exactly one type hence we assume there
exists a funcion $termType: Term \to \mathcal{T}$ which relates each well-typed
term with the corresponding type.

Primitive operations can be parameterized with type variables, for example
addition (\ref{sec:appendix:primops:Plus}) has the signature \lst{def +}$[$\lst{T}$]$(\lst{left}$:$~\lst{T}, \lst{right}$:$~\lst{T}): \lst{T}
where \lst{T} is one of the numeric types (Table~\ref{table:predeftypes}). 
Function $ptype$ returns the type of a primitive operation specialized for the concrete
types of its arguments, for example
$ptype(+,\lst{Int}, \lst{Int}) = (\lst{Int}, \lst{Int}) \to \lst{Int}$.

Similarily, the function $mtype$ returns a type of method specialized for concrete types of the arguments of the \lst{MethodCall} term.

\lst{BlockValue} rule defines a type of well-formed block expression. It
assumes a total ordering on \lst{val} definitions. If a block expression is
not well-formed than it cannot be typed and evaluated.

The rest of the rules are standard for typed lambda calculus.