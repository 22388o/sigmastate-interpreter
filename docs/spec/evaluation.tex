\section{Evaluation Semantics}
\label{sec:evaluation}

Here we specifiy evaluation semantics of \corelang, which is based on
call-by-push-value (CBPV) lambda calculus. Evaluation of \corelang is
specified using denotational semantics. To do that we first specify types,
then typed terms and then equations of denotationsl semantics.

The types of \corelang are given by 

\begin{equation}
\begin{array}{l c l}
  A			& ::= & U\Un{B} \mid \mathcal{T}  \\
  \Un{B}	& ::= & FA \mid A \to \Un{B}
\end{array}
\end{equation}

Since identifiers can be bound only to values, they must have value type. So
we have the following:

\begin{definition}
 A \emph{context} $\Gamma$ is a finite sequence of identifiers with value
 types $x_1: A_1, \dots ,x_n: A_n$. Sometimes we omit the identifiers and
 write $\Gamma$ as a list of value types.
\end{definition}

We make a syntactic distinction between values and computations, so that
every term is either a value or a computation but not both. Thus we have two kinds of judgements

\begin{equation}
\begin{array}{c c}\DerEnvV{V : A} & \DerEnvC{M : \Un{B}} \end{array}
\end{equation}

\noindent which respectively say that $V$ is a value of type $A$ and that $M$
is a computation of type $\Un{B}$ (i.e. $M$ produces a value of type $B$). The
calculus that this leads to is called call-by-push-value (CBPV).

\begin{figure}[h]
    % \footnotesize
    \input{figures/core_terms.tex}
    \caption{Typed terms of \corelang}
    \label{fig:core-terms}
\end{figure}

% \begin{figure}[h]
%     \footnotesize
%     \begin{center}
\(\begin{array}{c}
\multicolumn{1}{l}{\text{Reduction contexts of \langname}}\\
\\
\begin{tabular}{@{}l c l l}
$\Ctx$ 	& ::= 		&	$\Hole$ 	&  hole \\
		& $\mid$ 	&	$\delta~\Ov{v}~\Ctx~\Ov{e}$	&  \\
		& $\mid$ 	&	$\Ctx~e$				&  \\
		& $\mid$ 	&	$(\Lam{x}{e})\Ctx$ 	&  \\
		& & \\
\end{tabular} \\
\multicolumn{1}{l}{\text{Call-by-value evaluation relation}}\\
\\
\begin{tabular}{@{}l c l r}
$[(\Lam{x}{e})~v]\Ctx$					& $\mapsto$	& 	$[[v/x]e]\Ctx$					    & (1)     \\
$[$ \lst{let} $x = v$ \lst{in} $e]\Ctx$	& $\mapsto$	& 	$[[v/x]e]\Ctx$					    & (2)     \\
$[$\lst{case} $k~\Ov{v}$ \lst{of} \{ $k_i~\overline{x_i}$ $\rightarrow$ $e_i$  \}$]\Ctx$   & $\mapsto$ 	& 	$[[\overline{v}/\overline{x_j}]e_j]\Ctx$, if $k = k_j$	& (2)    \\
$\Apply{\Ctx}{\IfThenElse{v}{e_1}{e_2}}$   & $\mapsto$ 	& \begin{tabular}{ l l}
													$\Apply{\Ctx}{e_1}$ & if $v = true$ \\
													$\Apply{\Ctx}{e_2}$ & otherwise
												  \end{tabular} & (3)    \\
$[\delta~\overline{v}]\Ctx$   & $\mapsto$ 	& 	$[l]\Ctx$, if $l = |[\Prim|]\overline{v}$					    & (4)     \\

\end{tabular}
\end{array}\)
\end{center}
% \begin{center}
% \small{Here $\mu\B{\alpha}.\B{T}$ stands for isorecursive type \cite{wiki_recursive_data_type}. 
% Note that types are described using \emph{phantom types} notation~\cite{JamesCheney2003,wiki_phantom_types}}
% \end{center} 

%     \caption{Evaluation semantics of \langname}
%     \label{fig:semantics}
% \end{figure}
    