\section{Evaluation}
\label{sec:evaluation}

In this section we describe evaluation semantics of the \langname language
and the corresponding reference implementation of the interpreter.

\subsection{Semantics}
\label{sec:evaluation:semantics}

Evaluation of \langname is specified by its translation to \corelang, whose
terms form a subset of \langname terms. Thus, typing rules of \corelang form
a subset of typing rules of \langname.

Here we specify evaluation semantics of \corelang, which is based on
call-by-value (CBV) lambda calculus. Evaluation of \corelang is specified
using denotational semantics. To do that, we first specify denotations of
types, then typed terms and then equations of denotational semantics.

\begin{definition}
  (values, producers)
  \begin{itemize}
    \item The following \corelang terms are called values:
    $$ V :== x \mid C(d, T) \mid \Lam{x}{M}$$
    \item All \corelang terms are called producers. (This is because, when evaluated, they produce a value.)
  \end{itemize}
\end{definition}

We now describe and explain a denotational semantics for the \corelang
language. The key principle is that each type $A$ denotes a set $\Denot{A}$
whose elements are the denotations of values of the type $A$.

Thus, the type \lst{Boolean} denotes the 2-element set
$\{\lst{true},\lst{false}\}$, because there are two values of type
\lst{Boolean}. Likewise the type $(T_1,\dots,T_n)$ denotes
$(\Denot{T_1},\dots,\Denot{T_n})$ because a value of type $(T_1,\dots,T_n)$
must be of the form $(V_1,\dots,V_n)$, where each $V_i$ is value of type
$T_i$.

Given a value $V$ of type $A$, we write $\Denot{V}$ for the element of $A$
that it denotes. Given a close term $M$ of type $A$, we recall that it
produces a value $V$ of type $A$. So $M$ will denote an element $\Denot{M}$
of $\Denot{A}$.

A value of type $A \to B$ is of the form $\Lam{x}{M}$. This, when
applied to a value of type $A$ gives a value of type $B$. So $A \to B$
denotes $\Denot{A} \to \Denot{B}$. It is true that the syntax appears to
allow us to apply $\Lam{x}{M}$ to any term $N$ of type $A$. But $N$ will be
evaluated before it interracts with $\Lam{x}{M}$, so $\Lam{x}{M}$ is really only
applied to the value that $N$ produces (hence the semantics is call-by-value).

\begin{definition}
 A \emph{context} $\Gamma$ is a finite sequence of identifiers with value
 types $x_1:A_1, \dots ,x_n:A_n$. Sometimes we omit the identifiers and
 write $\Gamma$ as a list of value types.
\end{definition}

Given a context $\Gamma = x_1:A_1,\dots,x_n:A_n$, an environment (list of
bindings for identifiers) associates to each $x_i$ as value of type $A_i$. So
the environment denotes an element of $(\Denot{A_1},\dots,\Denot{A_n})$, and
we write $\Denot{\Gamma}$ for this set.

Given a \corelang term $\DerEnv{M: B}$, we see that $M$, together with
environment, gives a closed term of type $B$. So $M$ denotes a function
$\Denot{M}$ from $\Denot{\Gamma}$ to $\Denot{B}$.

In summary, the denotational semantics is organized as follows.

\begin{itemize}
  \item A type $A$ denotes the set $\Denot{A}$
  \item A context $x_1:A_1,\dots,x_n:A_n$ denotes the set $(\Denot{A_1},\dots,\Denot{A_n})$
  \item A term $\DerEnv{M: B}$ denotes a function $\Denot{M}:
  \Denot{\Gamma} \to \Denot{B}$
\end{itemize}

The denotations of types and terms is given in Figure~\ref{fig:denotations}.

\begin{figure}[h]
\caption{Denotational semantics of \corelang}
\label{fig:denotations}
The denotations of \corelang types

\begin{center}
  \(\begin{array}{ l c l }
  \Denot{\lst{Boolean}} & = & \{ \lst{true}, \lst{false} \}  \\	
  \Denot{\lst{P}} & = & \text{see set of values in Appendix~\ref{sec:appendix:predeftypes}} \\	
  \Denot{(T_1,\dots,T_n)} & = & (\Denot{T_1},\dots,\Denot{T_n})  \\	
  \Denot{A \to B} & = & \Denot{A} \to \Denot{B}  \\	
  \end{array}\)
\end{center}

The denotations of \corelang terms which together specify the function 
$\Denot{\_}: \Denot{\Gamma} \to \Denot{T}$

\begin{center}
  \(\begin{array}{ l c l }
  \Apply{ \Denot{\lst{x}}			}{(\rho,\lst{x}\mapsto x, \rho')} & = & x \\	
  \Apply{ \Denot{C(d, T)} 			}{\rho} & = & d \\	
  \Apply{ \Denot{(\Ov{M_i})} 		}{\rho} & = & (\Ov{ \Apply{\Denot{M_i}}{\rho} }) \\	

  \Apply{ \Denot{\Apply{\delta}{N}} }{\rho} & = 
		& \Apply{ (\Apply{\Denot{\delta}}{\rho}) }{ v }~where~v = \Apply{\Denot{N}}{\rho} \\	

  \Apply{ \Denot{\Lam{\lst{x}}{M}}	}{\rho} & = 
		& \Lam{x}{ \Apply{\Denot{M}}{(\rho, \lst{x}\mapsto x)} } \\	

  \Apply{ \Denot{\Apply{M_f}{N}}	}{\rho} & = 
		& \Apply{ (\Apply{\Denot{M_f}}{\rho}) }{ v }~where~v = \Apply{\Denot{N}}{\rho} \\	

  \Apply{ \Denot{ \Apply{M_I.\lst{m}}{\Ov{N_i}} }	}{\rho} & = 
		& \Apply{ (\Apply{\Denot{M_I}}{\rho}).m }{ \Ov{v_i} }~where~\Ov{v_i = \Apply{\Denot{N_i}}{\rho}} \\	
  \end{array}\)
\end{center}
\end{figure}

% \subsection{Interpreter}
% \label{sec:evaluation:interpreter}