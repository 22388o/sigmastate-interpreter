\subsection{Type Serialization}
\label{sec:ser:type}

In this section we describe how the types (like \lst{Int}, \lst{Coll[Byte]},
etc.) are serialized, then we define serialization of typed data. This will
give us a basis to describe serialization of Constant nodes of \ASDag. From
that we proceed to serialization of arbitrary \ASDag trees.

For motivation behind this type encoding please see Appendix~\ref{sec:appendix:motivation:type}.

\subsubsection{Distribution of type codes}
\label{sec:ser:type:codedist}

The whole space of 256 codes is divided as the following:

\begin{figure}[h] \footnotesize
\(\begin{tabularx}{\textwidth}{| l | X |}
    \hline
    \bf{Interval} & \bf{Distribution} \\
    \hline
    \lst{0x00} & special value to represent undefined type (\lst{NoType} in \ASDag) \\
    \hline
    \lst{0x01 - 0x6F(111)} & data types including primitive types, arrays, options
    aka nullable types, classes (in future), 111 = 255 - 144 different codes \\
    \hline
    \lst{0x70(112) - 0xFF(255)} & function types \lst{T1 => T2}, 144 = 12 x 12
    different codes \\
    \hline 
\end{tabularx}\)
\caption{Distribution of type codes}
\label{fig:ser:type:codedist}
\end{figure}

\subsubsection{Encoding Data Types}

There are 9 different values for primitive types and 2 more are reserved for future extensions.
Each primitive type has an id in a range {1,...,11} as the following.

\begin{figure}[h] \footnotesize
    \(\begin{tabularx}{\textwidth}{| l | X |}
        \hline
        \bf{Id} & \bf{Type} \\ \hline
1     &   Boolean \\  \hline
2     &   Byte\\  \hline
3     &   Short (16 bit)\\  \hline
4     &   Int (32 bit)\\  \hline
5     &   Long (64 bit)\\  \hline
6     &   BigInt (java.math.BigInteger)\\  \hline
7     &   GroupElement (org.bouncycastle.math.ec.ECPoint)\\  \hline
8     &   SigmaProp \\  \hline
9     &   reserved for Char \\  \hline
10    &   reserved for Double \\  \hline
11    &   reserved \\ \hline 
\end{tabularx}\)
\label{fig:ser:type:primtypes}
\end{figure}

For each type constructor like \lst{Coll} or \lst{Option} we use the encoding
schema defined below. Type constructor has associated \emph{base code} (e.g.
12 for \lst{Coll[_]}, 24 for \lst{Coll[Coll[_]]} etc. ), which is multiple of
12.
Base code can be added to primitive type id to produce code of constructed
type, for example 12 + 1 = 13 is a code of \lst{Coll[Byte]}. The code of type
constructor (12 in this example) is used when type parameter is non-primitive
type (e.g. \lst{Coll[(Byte, Int)]}). In this case the code of type
constructor is read first, and then recursive descent is performed to read
bytes of the parameter type (in this case \lst{(Byte, Int)}) This encoding
allows very simple and quick decoding by using div and mod operations.

The interval of codes for data types is divided as the following:

\begin{figure}[h] \footnotesize
    \(\begin{tabularx}{\textwidth}{| l | l | X |}
\hline
\bf{Interval} & \bf{Type constructor} & \bf{Description} \\ \hline
0x01 - 0x0B(11)     &                    & primitive types (including 2 reserved) \\ \hline
0x0C(12)            & \lst{Coll[_]}          & Collection of non-primivite types (\lst{Coll[(Int,Boolean)]}) \\ \hline
0x0D(13) - 0x17(23) & \lst{Coll[_]}          & Collection of primitive types (\lst{Coll[Byte]}, \lst{Coll[Int]}, etc.) \\ \hline
0x18(24)            & \lst{Coll[Coll[_]]}    & Nested collection of non-primitive types (\lst{Coll[Coll[(Int,Boolean)]]}) \\ \hline
0x19(25) - 0x23(35) & \lst{Coll[Coll[_]]}    & Nested collection of primitive types (\lst{Coll[Coll[Byte]]}, \lst{Coll[Coll[Int]]}) \\ \hline
0x24(36)            & \lst{Option[_]}        & Option of non-primitive type (\lst{Option[(Int, Byte)]}) \\ \hline
0x25(37) - 0x2F(47) & \lst{Option[_]}        & Option of primitive type (\lst{Option[Int]}) \\ \hline
0x30(48)            & \lst{Option[Coll[_]]}  & Option of Coll of non-primitive type (\lst{Option[Coll[(Int, Boolean)]]}) \\ \hline
0x31(49) - 0x3B(59) & \lst{Option[Coll[_]]}  & Option of Coll of primitive type (\lst{Option[Coll[Int]]}) \\ \hline
0x3C(60)            & \lst{(_,_)}            & Pair of non-primitive types (\lst{((Int, Byte), (Boolean,Box))}, etc.) \\ \hline
0x3D(61) - 0x47(71) & \lst{(_, Int)}         & Pair of types where first is primitive (\lst{(_, Int)}) \\ \hline
0x48(72)            & \lst{(_,_,_)}          & Triple of types  \\ \hline
0x49(73) - 0x53(83) & \lst{(Int, _)}         & Pair of types where second is primitive (\lst{(Int, _)}) \\ \hline
0x54(84)            & \lst{(_,_,_,_)}        & Quadruple of types  \\ \hline
0x55(85) - 0x5F(95) & \lst{(_, _)}           & Symmetric pair of primitive types (\lst{(Int, Int)}, \lst{(Byte,Byte)}, etc.) \\ \hline
0x60(96)            & \lst{(_,...,_)}        & \lst{Tuple} type with more than 4 items \lst{(Int, Byte, Box, Boolean, Int)} \\ \hline
0x61(97)            & \lst{Any}             & Any type  \\ \hline
0x62(98)            & \lst{Unit}            & Unit type \\ \hline
0x63(99)            & \lst{Box}             & Box type  \\ \hline
0x64(100)           & \lst{AvlTree}         & AvlTree type  \\ \hline
0x65(101)           & \lst{Context}         & Context type  \\ \hline
0x65(102)           & \lst{String}          & String  \\ \hline
0x66(103)           & \lst{IV}              & TypeIdent  \\ \hline
0x67(104)- 0x6E(110)&                    & reserved for future use  \\ \hline
0x6F(111)           &                    & Reserved for future \lst{Class} type (e.g. user-defined types)  \\ \hline
\end{tabularx}\)
\label{fig:ser:type:primtypes}
\end{figure}

\subsubsection{Encoding Function Types}

We use $12$ different values for both domain and range types of functions. This
gives us $12 * 12 = 144$ function types in total and allows to represent $11 *
11 = 121$ functions over primitive types using just single byte.

Each code $F$ in a range of function types can be represented as
$F = D * 12 + R + 112$, where $D, R \in \{0,\dots,11\}$ - indices of domain and range types correspondingly, 
$112$ - is the first code in an interval of function types. 

If $D = 0$ then domain type is not primitive and recursive descent is necessary to write/read domain type.

If $R = 0$ then range type is not primitive and recursive descent is necessary to write/read range type.

\subsubsection{Recursive Descent}

When an argument of a type constructor is not a primitive type we fallback to
the simple encoding schema.

In such a case we emit the special code for the type constructor according to
the table above and descend recursively to every child node of the type tree.

We do this descend only for those children whose code cannot be embedded in
the parent code. For example, serialization of \lst{Coll[(Int,Boolean)]}
proceeds as the following:
\begin{enumerate}
\item emit \lst{0x0C} because element of collection is not primitive 
\item recursively serialize \lst{(Int, Boolean)}
\item emit \lst{0x3D} because first item in the pair is primitive
\item recursivley serialize \lst{Boolean}
\item emit \lst{0x02} - the code for primitive type \lst{Boolean}
\end{enumerate}

\noindent Examples

\begin{figure}[h] \footnotesize
\(\begin{tabularx}{\textwidth}{| l | c | c | l | c | X |}
\hline
\bf{Type}                &\bf{D} & \bf{R} & \bf{Bytes} & \bf{\#Bytes} &  \bf{Comments} \\ \hline
\lst{Byte}               &     &     &  1                   &  1     &    \\ \hline
\lst{Coll[Byte]}         &     &     &  12 + 1 = 13         &  1     &    \\ \hline
\lst{Coll[Coll[Byte]]}   &     &     &  24 + 1 = 25         &  1     &     \\ \hline
\lst{Option[Byte]}       &     &     &  36 + 1 = 37         &  1     & register    \\ \hline
\lst{Option[Coll[Byte]]} &     &     &  48 + 1 = 49         &  1     & register    \\ \hline
\lst{(Int,Int)}          &     &     &  84 + 3 = 87         &  1     & fold    \\ \hline
\lst{Box=>Boolean}       & 7   & 2   &  198 = 7*12+2+112    &  1     & exist, forall    \\ \hline
\lst{(Int,Int)=>Int}     & 0   & 3   &  115=0*12+3+112, 87  &  2     &  fold    \\ \hline
\lst{(Int,Boolean)}      &     &     &  60 + 3, 2           &  2     &      \\ \hline
\lst{(Int,Box)=>Boolean} & 0   & 2   &  0*12+2+112, 60+3, 7 &  3     &     \\ \hline
\end{tabularx}\)
\label{fig:ser:type:primtypes}
\end{figure}
